%% Generated by Sphinx.
\def\sphinxdocclass{report}
\documentclass[a4paper,10pt,english]{sphinxmanual}
\ifdefined\pdfpxdimen
   \let\sphinxpxdimen\pdfpxdimen\else\newdimen\sphinxpxdimen
\fi \sphinxpxdimen=.75bp\relax

\PassOptionsToPackage{warn}{textcomp}
\usepackage[utf8]{inputenc}
\ifdefined\DeclareUnicodeCharacter
 \ifdefined\DeclareUnicodeCharacterAsOptional
  \DeclareUnicodeCharacter{"00A0}{\nobreakspace}
  \DeclareUnicodeCharacter{"2500}{\sphinxunichar{2500}}
  \DeclareUnicodeCharacter{"2502}{\sphinxunichar{2502}}
  \DeclareUnicodeCharacter{"2514}{\sphinxunichar{2514}}
  \DeclareUnicodeCharacter{"251C}{\sphinxunichar{251C}}
  \DeclareUnicodeCharacter{"2572}{\textbackslash}
 \else
  \DeclareUnicodeCharacter{00A0}{\nobreakspace}
  \DeclareUnicodeCharacter{2500}{\sphinxunichar{2500}}
  \DeclareUnicodeCharacter{2502}{\sphinxunichar{2502}}
  \DeclareUnicodeCharacter{2514}{\sphinxunichar{2514}}
  \DeclareUnicodeCharacter{251C}{\sphinxunichar{251C}}
  \DeclareUnicodeCharacter{2572}{\textbackslash}
 \fi
\fi
\usepackage{cmap}
\usepackage[T1]{fontenc}
\usepackage{amsmath,amssymb,amstext}
\usepackage{babel}
\usepackage{times}
\usepackage[Bjarne]{fncychap}
\usepackage{sphinx}
\sphinxsetup{verbatimwithframe=false, VerbatimColor={rgb}{.98,.94,.94}}
\usepackage{geometry}

% Include hyperref last.
\usepackage{hyperref}
% Fix anchor placement for figures with captions.
\usepackage{hypcap}% it must be loaded after hyperref.
% Set up styles of URL: it should be placed after hyperref.
\urlstyle{same}

\addto\captionsenglish{\renewcommand{\figurename}{Fig.}}
\addto\captionsenglish{\renewcommand{\tablename}{Table}}
\addto\captionsenglish{\renewcommand{\literalblockname}{Listing}}

\addto\captionsenglish{\renewcommand{\literalblockcontinuedname}{continued from previous page}}
\addto\captionsenglish{\renewcommand{\literalblockcontinuesname}{continues on next page}}

\addto\extrasenglish{\def\pageautorefname{page}}

\setcounter{tocdepth}{1}



\title{salomeTools Documentation}
\date{May 02, 2018}
\release{5.0.0dev}
\author{CEA DEN/DANS/DM2S/STMF/LGLS}
\newcommand{\sphinxlogo}{\vbox{}}
\renewcommand{\releasename}{Release}
\makeindex

\begin{document}

\maketitle
\sphinxtableofcontents
\phantomsection\label{\detokenize{index::doc}}
\clearpage



\noindent{\hspace*{\fill}\sphinxincludegraphics[scale=1.0]{{sat_about}.png}\hspace*{\fill}}

\begin{sphinxadmonition}{warning}{Warning:}
This documentation is under construction.
\end{sphinxadmonition}

The \sphinxstylestrong{Sa}lome\sphinxstylestrong{T}ools (sat) is a suite of commands
that can be used to perform operations on \sphinxhref{http://www.salome-platform.org}{SALOME}%
\begin{footnote}[1]\sphinxAtStartFootnote
\sphinxnolinkurl{http://www.salome-platform.org}
%
\end{footnote}.

For example, sat allows you to compile SALOME’s codes
(prerequisites, products)
create application, run tests, create package, etc.

This utility code is a set of \sphinxhref{https://docs.python.org/2.7}{Python}%
\begin{footnote}[2]\sphinxAtStartFootnote
\sphinxnolinkurl{https://docs.python.org/2.7}
%
\end{footnote} scripts files.

Find a \sphinxhref{./../latex/salomeTools.pdf}{pdf version of this documentation}


\chapter{Quick start}
\label{\detokenize{index:quick-start}}\label{\detokenize{index:salome-tools}}

\section{Installation}
\label{\detokenize{installation_of_sat:installation}}\label{\detokenize{installation_of_sat::doc}}
Usually user could find (and use) command \sphinxstylestrong{sat} directly after a ‘detar’ installation of SALOME.

\fvset{hllines={, ,}}%
\begin{sphinxVerbatim}[commandchars=\\\{\}]
tar \PYGZhy{}xf .../SALOME\PYGZus{}xx.tgz
\PYG{n+nb}{cd} SALOME\PYGZus{}xx
ls \PYGZhy{}l sat      \PYG{c+c1}{\PYGZsh{} sat \PYGZhy{}\PYGZgt{} salomeTools/sat}
\end{sphinxVerbatim}

Python package (scripts of salomeTools) actually remains in directory \sphinxstyleemphasis{salomeTools}.


\section{Configuration}
\label{\detokenize{configuration:configuration}}\label{\detokenize{configuration::doc}}
\sphinxstyleemphasis{salomeTools} uses files to store its configuration parameters.

There are several configuration files which are loaded by salomeTools in a specific order.
When all the files are loaded a \sphinxstyleemphasis{config} object is created.
Then, this object is passed to all command scripts.


\subsection{Syntax}
\label{\detokenize{configuration:syntax}}
The configuration files use a python-like structure format
(see \sphinxhref{http://www.red-dove.com/config-doc/}{config module}%
\begin{footnote}[3]\sphinxAtStartFootnote
\sphinxnolinkurl{http://www.red-dove.com/config-doc/}
%
\end{footnote} for a complete description).
\begin{itemize}
\item {} 
\sphinxstylestrong{\{\}} define a dictionary,

\item {} 
\sphinxstylestrong{{[}{]}} define a list,

\item {} 
\sphinxstylestrong{@} can be used to include a file,

\item {} 
\sphinxstylestrong{\$prefix} reference to another parameter (ex: \sphinxcode{\sphinxupquote{\$PRODUCT.name}}),

\item {} 
\sphinxstylestrong{\#} comments.

\end{itemize}

\begin{sphinxadmonition}{note}{Note:}
in this documentation a reference to a configuration parameter will be noted \sphinxcode{\sphinxupquote{XXX.YYY}}.
\end{sphinxadmonition}


\subsection{Description}
\label{\detokenize{configuration:description}}

\subsubsection{VARS section}
\label{\detokenize{configuration:vars-section}}\label{\detokenize{configuration:id1}}
\begin{DUlineblock}{0em}
\item[] This section is dynamically created by salomeTools at run time.
\item[] It contains information about the environment: date, time, OS, architecture etc.
\end{DUlineblock}

\fvset{hllines={, ,}}%
\begin{sphinxVerbatim}[commandchars=\\\{\}]
\PYG{c+c1}{\PYGZsh{} to get the current setting}
\PYG{n}{sat} \PYG{n}{config} \PYG{o}{\PYGZhy{}}\PYG{o}{\PYGZhy{}}\PYG{n}{value} \PYG{n}{VARS}
\end{sphinxVerbatim}


\subsubsection{PRODUCTS section}
\label{\detokenize{configuration:products-section}}
\begin{DUlineblock}{0em}
\item[] This section is defined in the product file.
\item[] It contains instructions on how to build a version of SALOME (list of prerequisites-products and versions)
\end{DUlineblock}

\fvset{hllines={, ,}}%
\begin{sphinxVerbatim}[commandchars=\\\{\}]
\PYG{c+c1}{\PYGZsh{} to get the current setting}
\PYG{n}{sat} \PYG{n}{config} \PYG{n}{SALOME}\PYG{o}{\PYGZhy{}}\PYG{n}{xx} \PYG{o}{\PYGZhy{}}\PYG{o}{\PYGZhy{}}\PYG{n}{value} \PYG{n}{PRODUCTS}
\end{sphinxVerbatim}


\subsubsection{APPLICATION section}
\label{\detokenize{configuration:application-section}}
\begin{DUlineblock}{0em}
\item[] This section is optional, it is also defined in the product file.
\item[] It gives additional parameters to create an application based on SALOME, as versions of products to use.
\end{DUlineblock}

\fvset{hllines={, ,}}%
\begin{sphinxVerbatim}[commandchars=\\\{\}]
\PYG{c+c1}{\PYGZsh{} to get the current setting}
\PYG{n}{sat} \PYG{n}{config} \PYG{n}{SALOME}\PYG{o}{\PYGZhy{}}\PYG{n}{xx} \PYG{o}{\PYGZhy{}}\PYG{o}{\PYGZhy{}}\PYG{n}{value} \PYG{n}{APPLICATION}
\end{sphinxVerbatim}


\subsubsection{USER section}
\label{\detokenize{configuration:user-section}}\label{\detokenize{configuration:id2}}
This section is defined by the user configuration file,
\sphinxcode{\sphinxupquote{\textasciitilde{}/.salomeTools/salomeTools.pyconf}}.

The \sphinxcode{\sphinxupquote{USER}} section defines some parameters (not exhaustive):
\begin{itemize}
\item {} 
\sphinxstylestrong{workDir} :
\begin{quote}

\begin{DUlineblock}{0em}
\item[] The working directory.
\item[] Each product will be usually installed here (in sub-directories).
\end{DUlineblock}
\end{quote}

\item {} 
\sphinxstylestrong{browser} : The web browser to use (\sphinxstyleemphasis{firefox}).

\item {} 
\sphinxstylestrong{editor} : The editor to use (\sphinxstyleemphasis{vi, pluma}).

\item {} 
and other user preferences.

\end{itemize}

\fvset{hllines={, ,}}%
\begin{sphinxVerbatim}[commandchars=\\\{\}]
\PYG{c+c1}{\PYGZsh{} to get the current setting}
\PYG{n}{sat} \PYG{n}{config} \PYG{n}{SALOME}\PYG{o}{\PYGZhy{}}\PYG{n}{xx} \PYG{o}{\PYGZhy{}}\PYG{o}{\PYGZhy{}}\PYG{n}{value} \PYG{n}{USER}
\end{sphinxVerbatim}

\clearpage


\section{Usage of SAlomeTools}
\label{\detokenize{usage_of_sat:svn}}\label{\detokenize{usage_of_sat:usage-of-salometools}}\label{\detokenize{usage_of_sat::doc}}

\subsection{Usage}
\label{\detokenize{usage_of_sat:usage}}
sat usage is a Command Line Interface (\sphinxhref{https://en.wikipedia.org/wiki/Command-line\_interface}{CLI}%
\begin{footnote}[4]\sphinxAtStartFootnote
\sphinxnolinkurl{https://en.wikipedia.org/wiki/Command-line\_interface}
%
\end{footnote}).

\fvset{hllines={, ,}}%
\begin{sphinxVerbatim}[commandchars=\\\{\}]
sat \PYG{o}{[}generic\PYGZus{}options\PYG{o}{]} \PYG{o}{[}command\PYG{o}{]} \PYG{o}{[}product\PYG{o}{]} \PYG{o}{[}command\PYGZus{}options\PYG{o}{]}
\end{sphinxVerbatim}


\subsubsection{Options of sat}
\label{\detokenize{usage_of_sat:options-of-sat}}
Useful \sphinxstyleemphasis{not exhaustive} generic options of \sphinxstyleemphasis{sat} CLI.


\paragraph{\sphinxstyleemphasis{\textendash{}help or -h}}
\label{\detokenize{usage_of_sat:help-or-h}}
Get help as simple text.

\fvset{hllines={, ,}}%
\begin{sphinxVerbatim}[commandchars=\\\{\}]
sat \PYGZhy{}\PYGZhy{}help          \PYG{c+c1}{\PYGZsh{} get the list of existing commands}
sat \PYGZhy{}\PYGZhy{}help compile  \PYG{c+c1}{\PYGZsh{} get the help on a specific command \PYGZsq{}compile\PYGZsq{}}
\end{sphinxVerbatim}


\paragraph{\sphinxstyleemphasis{\textendash{}debug or -g}}
\label{\detokenize{usage_of_sat:debug-or-g}}
Execution in debug mode allows to see more trace and \sphinxstyleemphasis{stack} if an exception is raised.


\paragraph{\sphinxstyleemphasis{\textendash{}verbose or -v}}
\label{\detokenize{usage_of_sat:verbose-or-v}}
Change verbosity level (default is 3).

\fvset{hllines={, ,}}%
\begin{sphinxVerbatim}[commandchars=\\\{\}]
\PYG{c+c1}{\PYGZsh{} for product \PYGZsq{}SALOME\PYGZus{}xx\PYGZsq{} for example}
\PYG{c+c1}{\PYGZsh{} execute compile command in debug mode with trace level 4}
sat \PYGZhy{}g \PYGZhy{}v \PYG{l+m}{4} compile SALOME\PYGZus{}xx
\end{sphinxVerbatim}


\subsection{Build a SALOME product}
\label{\detokenize{usage_of_sat:build-a-salome-product}}

\subsubsection{Get the list of available products}
\label{\detokenize{usage_of_sat:get-the-list-of-available-products}}
To get the list of the current available products in your context:

\fvset{hllines={, ,}}%
\begin{sphinxVerbatim}[commandchars=\\\{\}]
sat config \PYGZhy{}\PYGZhy{}list
\end{sphinxVerbatim}


\subsubsection{Prepare sources of a product}
\label{\detokenize{usage_of_sat:prepare-sources-of-a-product}}
To prepare (get) \sphinxstyleemphasis{all} the sources of a product (\sphinxstyleemphasis{SALOME\_xx} for example):

\fvset{hllines={, ,}}%
\begin{sphinxVerbatim}[commandchars=\\\{\}]
sat prepare SALOME\PYGZus{}xx
\end{sphinxVerbatim}

\begin{DUlineblock}{0em}
\item[] The sources are usually copied in directories
\item[] \sphinxstyleemphasis{\$USER.workDir + SALOME\_xx… + SOURCES + \$PRODUCT.name}
\end{DUlineblock}


\subsubsection{Compile SALOME}
\label{\detokenize{usage_of_sat:compile-salome}}
To compile products:

\fvset{hllines={, ,}}%
\begin{sphinxVerbatim}[commandchars=\\\{\}]
\PYG{c+c1}{\PYGZsh{} compile all prerequisites/products}
sat compile SALOME\PYGZus{}xx

\PYG{c+c1}{\PYGZsh{} compile only 2 products (KERNEL and SAMPLES), if not done yet}
sat compile SALOME\PYGZus{}xx \PYGZhy{}\PYGZhy{}products KERNEL,SAMPLES

\PYG{c+c1}{\PYGZsh{} compile only 2 products, unconditionaly}
sat compile SALOME\PYGZus{}xx \PYGZhy{}\PYGZhy{}\PYGZhy{}products SAMPLES \PYGZhy{}\PYGZhy{}clean\PYGZus{}all
\end{sphinxVerbatim}

\begin{DUlineblock}{0em}
\item[] The products are usually build in the directories
\item[] \sphinxstyleemphasis{\$USER.workDir + SALOME\_xx… + BUILD + \$PRODUCT.name}
\item[] 
\item[] The products are usually installed in the directories
\item[] \sphinxstyleemphasis{\$USER.workDir + SALOME\_xx… + INSTALL + \$PRODUCT.name}
\end{DUlineblock}


\chapter{List of Commands}
\label{\detokenize{index:list-of-commands}}
\clearpage


\section{Command config}
\label{\detokenize{commands/config:svn}}\label{\detokenize{commands/config:command-config}}\label{\detokenize{commands/config::doc}}

\subsection{Description}
\label{\detokenize{commands/config:description}}
The \sphinxstylestrong{config} command manages sat configuration.
It allows display, manipulation and operation on configuration files


\subsection{Usage}
\label{\detokenize{commands/config:usage}}\begin{itemize}
\item {} 
Edit the user personal configuration file \sphinxcode{\sphinxupquote{\$HOME/.salomeTools/SAT.pyconf}}. It is used to store the user personal choices, like the favorite editor, browser, pdf viewer:

\fvset{hllines={, ,}}%
\begin{sphinxVerbatim}[commandchars=\\\{\}]
\PYG{n}{sat} \PYG{n}{config} \PYG{o}{\PYGZhy{}}\PYG{o}{\PYGZhy{}}\PYG{n}{edit}
\end{sphinxVerbatim}

\item {} 
List the available applications (they come from the sat projects defined in \sphinxcode{\sphinxupquote{data/local.pyconf}}:

\fvset{hllines={, ,}}%
\begin{sphinxVerbatim}[commandchars=\\\{\}]
\PYG{n}{sat} \PYG{n}{config} \PYG{o}{\PYGZhy{}}\PYG{o}{\PYGZhy{}}\PYG{n+nb}{list}
\end{sphinxVerbatim}

\item {} 
Edit the configuration of an application:

\fvset{hllines={, ,}}%
\begin{sphinxVerbatim}[commandchars=\\\{\}]
\PYG{n}{sat} \PYG{n}{config} \PYG{o}{\PYGZlt{}}\PYG{n}{application}\PYG{o}{\PYGZgt{}} \PYG{o}{\PYGZhy{}}\PYG{o}{\PYGZhy{}}\PYG{n}{edit}
\end{sphinxVerbatim}

\item {} 
Copy an application configuration file into the user personal directory:

\fvset{hllines={, ,}}%
\begin{sphinxVerbatim}[commandchars=\\\{\}]
\PYG{n}{sat} \PYG{n}{config} \PYG{o}{\PYGZlt{}}\PYG{n}{application}\PYG{o}{\PYGZgt{}} \PYG{o}{\PYGZhy{}}\PYG{o}{\PYGZhy{}}\PYG{n}{copy} \PYG{p}{[}\PYG{n}{new\PYGZus{}name}\PYG{p}{]}
\end{sphinxVerbatim}

\item {} 
\begin{DUlineblock}{0em}
\item[] Print the value of a configuration parameter.
\item[] Use the automatic completion to get recursively the parameter names.
\item[] Use \sphinxstyleemphasis{\textendash{}no\_label} option to get \sphinxstyleemphasis{only} the value, \sphinxstyleemphasis{without} label (useful in automatic scripts).
\item[] Examples (with \sphinxstyleemphasis{SALOME-xx} as \sphinxstyleemphasis{SALOME-8.4.0} ):
\end{DUlineblock}

\fvset{hllines={, ,}}%
\begin{sphinxVerbatim}[commandchars=\\\{\}]
\PYG{c+c1}{\PYGZsh{} sat config \PYGZhy{}\PYGZhy{}value \PYGZlt{}parameter\PYGZus{}path\PYGZgt{}}
sat config \PYGZhy{}\PYGZhy{}value .         \PYG{c+c1}{\PYGZsh{} all the configuration}
sat config \PYGZhy{}\PYGZhy{}value LOCAL
sat config \PYGZhy{}\PYGZhy{}value LOCAL.workdir

\PYG{c+c1}{\PYGZsh{} sat config \PYGZlt{}application\PYGZgt{} \PYGZhy{}\PYGZhy{}value \PYGZlt{}parameter\PYGZus{}path\PYGZgt{}}
sat config SALOME\PYGZhy{}xx \PYGZhy{}\PYGZhy{}value APPLICATION.workdir
sat config SALOME\PYGZhy{}xx \PYGZhy{}\PYGZhy{}no\PYGZus{}label \PYGZhy{}\PYGZhy{}value APPLICATION.workdir
\end{sphinxVerbatim}

\item {} 
\begin{DUlineblock}{0em}
\item[] Print in one-line-by-value mode the value of a configuration parameter,
\item[] with its source \sphinxstyleemphasis{expression}, if any.
\item[] This is a debug mode, useful for developers.
\item[] Prints the parameter path, the source expression if any, and the final value:
\end{DUlineblock}

\fvset{hllines={, ,}}%
\begin{sphinxVerbatim}[commandchars=\\\{\}]
\PYG{n}{sat} \PYG{n}{config} \PYG{n}{SALOME}\PYG{o}{\PYGZhy{}}\PYG{n}{xx} \PYG{o}{\PYGZhy{}}\PYG{n}{g} \PYG{n}{USER}
\end{sphinxVerbatim}

\begin{sphinxadmonition}{note}{Note:}
And so, \sphinxstyleemphasis{not only for fun}, to get \sphinxstylestrong{all expressions} of configuration

\fvset{hllines={, ,}}%
\begin{sphinxVerbatim}[commandchars=\\\{\}]
sat config SALOME\PYGZhy{}xx \PYGZhy{}g . \PYG{p}{\textbar{}} grep \PYGZhy{}e \PYG{l+s+s2}{\PYGZdq{}\PYGZhy{}\PYGZhy{}\PYGZgt{}\PYGZdq{}}
\end{sphinxVerbatim}
\end{sphinxadmonition}

\item {} 
Print the patches that are applied:

\fvset{hllines={, ,}}%
\begin{sphinxVerbatim}[commandchars=\\\{\}]
\PYG{n}{sat} \PYG{n}{config} \PYG{n}{SALOME}\PYG{o}{\PYGZhy{}}\PYG{n}{xx} \PYG{o}{\PYGZhy{}}\PYG{o}{\PYGZhy{}}\PYG{n}{show\PYGZus{}patchs}
\end{sphinxVerbatim}

\item {} 
Get information on a product configuration:

\end{itemize}

\fvset{hllines={, ,}}%
\begin{sphinxVerbatim}[commandchars=\\\{\}]
\PYG{c+c1}{\PYGZsh{} sat config \PYGZlt{}application\PYGZgt{} \PYGZhy{}\PYGZhy{}info \PYGZlt{}product\PYGZgt{}}
sat config SALOME\PYGZhy{}xx \PYGZhy{}\PYGZhy{}info KERNEL
sat config SALOME\PYGZhy{}xx \PYGZhy{}\PYGZhy{}info qt
\end{sphinxVerbatim}


\subsection{Some useful configuration pathes}
\label{\detokenize{commands/config:some-useful-configuration-pathes}}
Exploring a current configuration.
\begin{itemize}
\item {} 
\sphinxstylestrong{PATHS}: To get list of directories where to find files.

\item {} 
\sphinxstylestrong{USER}: To get user preferences (editor, pdf viewer, web browser, default working dir).

\end{itemize}

sat commands:

\fvset{hllines={, ,}}%
\begin{sphinxVerbatim}[commandchars=\\\{\}]
\PYG{n}{sat} \PYG{n}{config} \PYG{n}{SALOME}\PYG{o}{\PYGZhy{}}\PYG{n}{xx} \PYG{o}{\PYGZhy{}}\PYG{n}{v} \PYG{n}{PATHS}
\PYG{n}{sat} \PYG{n}{config} \PYG{n}{SALOME}\PYG{o}{\PYGZhy{}}\PYG{n}{xx} \PYG{o}{\PYGZhy{}}\PYG{n}{v} \PYG{n}{USERS}
\end{sphinxVerbatim}

\clearpage


\section{Command prepare}
\label{\detokenize{commands/prepare:svn}}\label{\detokenize{commands/prepare:command-prepare}}\label{\detokenize{commands/prepare::doc}}

\subsection{Description}
\label{\detokenize{commands/prepare:description}}
The \sphinxstylestrong{prepare} command brings the sources of an application in the \sphinxstyleemphasis{sources
application directory}, in order to compile them with the compile command.

The sources can be prepared from VCS software (\sphinxstyleemphasis{cvs, svn, git}), an archive or a directory.

\begin{sphinxadmonition}{warning}{Warning:}
When sat prepares a product, it first removes the
existing directory, except if the development mode is activated.
When you are working on a product, you need to declare in
the application configuration this product in \sphinxstylestrong{dev} mode.
\end{sphinxadmonition}


\subsection{Remarks}
\label{\detokenize{commands/prepare:remarks}}

\subsubsection{VCS bases (git, svn, cvs)}
\label{\detokenize{commands/prepare:vcs-bases-git-svn-cvs}}
The \sphinxstyleemphasis{prepare} command does not manage authentication on the cvs server.
For example, to prepare modules from a cvs server, you first need to login once.

To avoid typing a password for each product,
you may use a ssh key with passphrase, or store your password
(in .cvspass or .gitconfig files).
If you have security concerns, it is also possible to use
a bash agent and type your password only once.


\subsubsection{Dev mode}
\label{\detokenize{commands/prepare:dev-mode}}
By default \sphinxstyleemphasis{prepare} uses \sphinxstyleemphasis{export} mode: it creates an image
of the sources, corresponding to the tag or branch specified,
without any link to the VCS base.
To perform a \sphinxstyleemphasis{checkout} (svn, cvs) or a \sphinxstyleemphasis{git clone} (git),
you need to declare the product in dev mode in your application configuration:
edit the application configuration file (pyconf) and modify the product declaration:

\fvset{hllines={, ,}}%
\begin{sphinxVerbatim}[commandchars=\\\{\}]
sat config \PYGZlt{}application\PYGZgt{} \PYGZhy{}e
\PYG{c+c1}{\PYGZsh{} and edit the product section:}
\PYG{c+c1}{\PYGZsh{}   \PYGZlt{}product\PYGZgt{} : \PYGZob{}tag : \PYGZdq{}my\PYGZus{}tag\PYGZdq{}, dev : \PYGZdq{}yes\PYGZdq{}, debug : \PYGZdq{}yes\PYGZdq{}\PYGZcb{}}
\end{sphinxVerbatim}

The first time you will execute the \sphinxstyleemphasis{sat prepare} command,
your module will be downloaded in \sphinxstyleemphasis{checkout} mode
(inside the SOURCES directory of the application.
Then, you can develop in this repository, and finally push
them in the base when they are ready.
If you type during the development process by mistake
a \sphinxstyleemphasis{sat prepare} command, the sources in dev mode will
not be altered/removed (Unless you use -f option)


\subsection{Usage}
\label{\detokenize{commands/prepare:usage}}\begin{itemize}
\item {} 
Prepare the sources of a complete application in SOURCES directory (all products):

\fvset{hllines={, ,}}%
\begin{sphinxVerbatim}[commandchars=\\\{\}]
\PYG{n}{sat} \PYG{n}{prepare} \PYG{o}{\PYGZlt{}}\PYG{n}{application}\PYG{o}{\PYGZgt{}}
\end{sphinxVerbatim}

\item {} 
Prepare only some modules:

\fvset{hllines={, ,}}%
\begin{sphinxVerbatim}[commandchars=\\\{\}]
\PYG{n}{sat} \PYG{n}{prepare} \PYG{o}{\PYGZlt{}}\PYG{n}{application}\PYG{o}{\PYGZgt{}}  \PYG{o}{\PYGZhy{}}\PYG{o}{\PYGZhy{}}\PYG{n}{products} \PYG{o}{\PYGZlt{}}\PYG{n}{product1}\PYG{o}{\PYGZgt{}}\PYG{p}{,}\PYG{o}{\PYGZlt{}}\PYG{n}{product2}\PYG{o}{\PYGZgt{}} \PYG{o}{.}\PYG{o}{.}\PYG{o}{.}
\end{sphinxVerbatim}

\item {} 
Use \textendash{}force to force to prepare the products in development mode
(this will remove the sources and do a new clone/checkout):

\fvset{hllines={, ,}}%
\begin{sphinxVerbatim}[commandchars=\\\{\}]
\PYG{n}{sat} \PYG{n}{prepare} \PYG{o}{\PYGZlt{}}\PYG{n}{application}\PYG{o}{\PYGZgt{}} \PYG{o}{\PYGZhy{}}\PYG{o}{\PYGZhy{}}\PYG{n}{force}
\end{sphinxVerbatim}

\item {} 
Use \textendash{}force\_patch to force to apply patch to the products
in development mode (otherwise they are not applied):

\fvset{hllines={, ,}}%
\begin{sphinxVerbatim}[commandchars=\\\{\}]
\PYG{n}{sat} \PYG{n}{prepare} \PYG{o}{\PYGZlt{}}\PYG{n}{application}\PYG{o}{\PYGZgt{}} \PYG{o}{\PYGZhy{}}\PYG{o}{\PYGZhy{}}\PYG{n}{force\PYGZus{}patch}
\end{sphinxVerbatim}

\end{itemize}


\subsection{Some useful configuration pathes}
\label{\detokenize{commands/prepare:some-useful-configuration-pathes}}
Command \sphinxstyleemphasis{sat prepare} uses the \sphinxstyleemphasis{pyconf file configuration} of each product to know how to get the sources.

\begin{sphinxadmonition}{note}{Note:}
to verify configuration of a product, and get name of this \sphinxstyleemphasis{pyconf files configuration}

\fvset{hllines={, ,}}%
\begin{sphinxVerbatim}[commandchars=\\\{\}]
sat config \PYGZlt{}application\PYGZgt{} \PYGZhy{}\PYGZhy{}info \PYGZlt{}product\PYGZgt{}
\end{sphinxVerbatim}
\end{sphinxadmonition}
\begin{itemize}
\item {} 
\sphinxstylestrong{get\_method}: the method to use to prepare the module, possible values are cvs, git, archive, dir.

\item {} 
\sphinxstylestrong{git\_info} : (used if get\_method = git) information to prepare sources from git.

\item {} 
\sphinxstylestrong{svn\_info} : (used if get\_method = svn) information to prepare sources from cvs.

\item {} 
\sphinxstylestrong{cvs\_info} : (used if get\_method = cvs) information to prepare sources from cvs.

\item {} 
\sphinxstylestrong{archive\_info} : (used if get\_method = archive) the path to the archive.

\item {} 
\sphinxstylestrong{dir\_info} : (used if get\_method = dir) the directory with the sources.

\end{itemize}

\clearpage


\section{Command compile}
\label{\detokenize{commands/compile:svn}}\label{\detokenize{commands/compile:command-compile}}\label{\detokenize{commands/compile::doc}}

\subsection{Description}
\label{\detokenize{commands/compile:description}}
The \sphinxstylestrong{compile} command allows compiling the products of a \sphinxhref{http://www.salome-platform.org}{SALOME}%
\begin{footnote}[5]\sphinxAtStartFootnote
\sphinxnolinkurl{http://www.salome-platform.org}
%
\end{footnote} application.


\subsection{Usage}
\label{\detokenize{commands/compile:usage}}\begin{itemize}
\item {} 
Compile a complete application:

\fvset{hllines={, ,}}%
\begin{sphinxVerbatim}[commandchars=\\\{\}]
\PYG{n}{sat} \PYG{n+nb}{compile} \PYG{o}{\PYGZlt{}}\PYG{n}{application}\PYG{o}{\PYGZgt{}}
\end{sphinxVerbatim}

\item {} 
Compile only some products:

\fvset{hllines={, ,}}%
\begin{sphinxVerbatim}[commandchars=\\\{\}]
\PYG{n}{sat} \PYG{n+nb}{compile} \PYG{o}{\PYGZlt{}}\PYG{n}{application}\PYG{o}{\PYGZgt{}} \PYG{o}{\PYGZhy{}}\PYG{o}{\PYGZhy{}}\PYG{n}{products} \PYG{o}{\PYGZlt{}}\PYG{n}{product1}\PYG{o}{\PYGZgt{}}\PYG{p}{,}\PYG{o}{\PYGZlt{}}\PYG{n}{product2}\PYG{o}{\PYGZgt{}} \PYG{o}{.}\PYG{o}{.}\PYG{o}{.}
\end{sphinxVerbatim}

\item {} 
Use \sphinxstyleemphasis{sat -t} to duplicate the logs in the terminal (by default the log are stored and displayed with \sphinxstyleemphasis{sat log} command):

\fvset{hllines={, ,}}%
\begin{sphinxVerbatim}[commandchars=\\\{\}]
\PYG{n}{sat} \PYG{o}{\PYGZhy{}}\PYG{n}{t} \PYG{n+nb}{compile} \PYG{o}{\PYGZlt{}}\PYG{n}{application}\PYG{o}{\PYGZgt{}} \PYG{o}{\PYGZhy{}}\PYG{o}{\PYGZhy{}}\PYG{n}{products} \PYG{o}{\PYGZlt{}}\PYG{n}{product1}\PYG{o}{\PYGZgt{}}
\end{sphinxVerbatim}

\item {} 
Compile a module and its dependencies:

\fvset{hllines={, ,}}%
\begin{sphinxVerbatim}[commandchars=\\\{\}]
\PYG{n}{sat} \PYG{n+nb}{compile} \PYG{o}{\PYGZlt{}}\PYG{n}{application}\PYG{o}{\PYGZgt{}} \PYG{o}{\PYGZhy{}}\PYG{o}{\PYGZhy{}}\PYG{n}{products} \PYG{n}{med} \PYG{o}{\PYGZhy{}}\PYG{o}{\PYGZhy{}}\PYG{n}{with\PYGZus{}fathers}
\end{sphinxVerbatim}

\item {} 
Compile a module and the modules depending on it (for example plugins):

\fvset{hllines={, ,}}%
\begin{sphinxVerbatim}[commandchars=\\\{\}]
\PYG{n}{sat} \PYG{n+nb}{compile} \PYG{o}{\PYGZlt{}}\PYG{n}{application}\PYG{o}{\PYGZgt{}} \PYG{o}{\PYGZhy{}}\PYG{o}{\PYGZhy{}}\PYG{n}{products} \PYG{n}{med} \PYG{o}{\PYGZhy{}}\PYG{o}{\PYGZhy{}}\PYG{n}{with\PYGZus{}children}
\end{sphinxVerbatim}

\item {} 
Clean the build and install directories before starting compilation:

\fvset{hllines={, ,}}%
\begin{sphinxVerbatim}[commandchars=\\\{\}]
\PYG{n}{sat} \PYG{n+nb}{compile} \PYG{o}{\PYGZlt{}}\PYG{n}{application}\PYG{o}{\PYGZgt{}} \PYG{o}{\PYGZhy{}}\PYG{o}{\PYGZhy{}}\PYG{n}{products} \PYG{n}{GEOM}  \PYG{o}{\PYGZhy{}}\PYG{o}{\PYGZhy{}}\PYG{n}{clean\PYGZus{}all}
\end{sphinxVerbatim}

\begin{sphinxadmonition}{note}{Note:}
\begin{DUlineblock}{0em}
\item[] a warning will be shown if option \sphinxstyleemphasis{\textendash{}products} is missing
\item[] (as it will clean everything)
\end{DUlineblock}
\end{sphinxadmonition}

\item {} 
Clean only the install directories before starting compilation:

\fvset{hllines={, ,}}%
\begin{sphinxVerbatim}[commandchars=\\\{\}]
\PYG{n}{sat} \PYG{n+nb}{compile} \PYG{o}{\PYGZlt{}}\PYG{n}{application}\PYG{o}{\PYGZgt{}} \PYG{o}{\PYGZhy{}}\PYG{o}{\PYGZhy{}}\PYG{n}{clean\PYGZus{}install}
\end{sphinxVerbatim}

\item {} 
Add options for make:

\fvset{hllines={, ,}}%
\begin{sphinxVerbatim}[commandchars=\\\{\}]
\PYG{n}{sat} \PYG{n+nb}{compile} \PYG{o}{\PYGZlt{}}\PYG{n}{application}\PYG{o}{\PYGZgt{}} \PYG{o}{\PYGZhy{}}\PYG{o}{\PYGZhy{}}\PYG{n}{products} \PYG{o}{\PYGZlt{}}\PYG{n}{product}\PYG{o}{\PYGZgt{}} \PYG{o}{\PYGZhy{}}\PYG{o}{\PYGZhy{}}\PYG{n}{make\PYGZus{}flags} \PYG{o}{\PYGZlt{}}\PYG{n}{flags}\PYG{o}{\PYGZgt{}}
\end{sphinxVerbatim}

\item {} 
Use the \sphinxstyleemphasis{\textendash{}check} option to execute the unit tests after compilation:

\fvset{hllines={, ,}}%
\begin{sphinxVerbatim}[commandchars=\\\{\}]
\PYG{n}{sat} \PYG{n+nb}{compile} \PYG{o}{\PYGZlt{}}\PYG{n}{application}\PYG{o}{\PYGZgt{}} \PYG{o}{\PYGZhy{}}\PYG{o}{\PYGZhy{}}\PYG{n}{check}
\end{sphinxVerbatim}

\item {} 
Remove the build directory after successful compilation (some build directory like qt are big):

\fvset{hllines={, ,}}%
\begin{sphinxVerbatim}[commandchars=\\\{\}]
\PYG{n}{sat} \PYG{n+nb}{compile} \PYG{o}{\PYGZlt{}}\PYG{n}{application}\PYG{o}{\PYGZgt{}} \PYG{o}{\PYGZhy{}}\PYG{o}{\PYGZhy{}}\PYG{n}{products} \PYG{n}{qt} \PYG{o}{\PYGZhy{}}\PYG{o}{\PYGZhy{}}\PYG{n}{clean\PYGZus{}build\PYGZus{}after}
\end{sphinxVerbatim}

\item {} 
Stop the compilation as soon as the compilation of a module fails:

\fvset{hllines={, ,}}%
\begin{sphinxVerbatim}[commandchars=\\\{\}]
\PYG{n}{sat} \PYG{n+nb}{compile} \PYG{o}{\PYGZlt{}}\PYG{n}{product}\PYG{o}{\PYGZgt{}} \PYG{o}{\PYGZhy{}}\PYG{o}{\PYGZhy{}}\PYG{n}{stop\PYGZus{}first\PYGZus{}fail}
\end{sphinxVerbatim}

\item {} 
Do not compile, just show if products are installed or not, and where is the installation:

\fvset{hllines={, ,}}%
\begin{sphinxVerbatim}[commandchars=\\\{\}]
\PYG{n}{sat} \PYG{n+nb}{compile} \PYG{o}{\PYGZlt{}}\PYG{n}{application}\PYG{o}{\PYGZgt{}} \PYG{o}{\PYGZhy{}}\PYG{o}{\PYGZhy{}}\PYG{n}{show}
\end{sphinxVerbatim}

\end{itemize}


\subsection{Some useful configuration pathes}
\label{\detokenize{commands/compile:some-useful-configuration-pathes}}
The way to compile a product is defined in the \sphinxstyleemphasis{pyconf file configuration}.
The main options are:
\begin{itemize}
\item {} 
\sphinxstylestrong{build\_source} : the method used to build the product (cmake/autotools/script)

\item {} 
\sphinxstylestrong{compil\_script} : the compilation script if build\_source is equal to “script”

\item {} 
\sphinxstylestrong{cmake\_options} : additional options for cmake.

\item {} 
\sphinxstylestrong{nb\_proc} : number of jobs to use with make for this product.

\end{itemize}

\clearpage


\section{Command launcher}
\label{\detokenize{commands/launcher:svn}}\label{\detokenize{commands/launcher:command-launcher}}\label{\detokenize{commands/launcher::doc}}

\subsection{Description}
\label{\detokenize{commands/launcher:description}}
The \sphinxstylestrong{launcher} command creates a SALOME launcher, a python script file to start \sphinxhref{http://www.salome-platform.org}{SALOME}%
\begin{footnote}[6]\sphinxAtStartFootnote
\sphinxnolinkurl{http://www.salome-platform.org}
%
\end{footnote}.


\subsection{Usage}
\label{\detokenize{commands/launcher:usage}}\begin{itemize}
\item {} 
Create a launcher:

\fvset{hllines={, ,}}%
\begin{sphinxVerbatim}[commandchars=\\\{\}]
\PYG{n}{sat} \PYG{n}{launcher} \PYG{o}{\PYGZlt{}}\PYG{n}{application}\PYG{o}{\PYGZgt{}}
\end{sphinxVerbatim}

Generate a launcher in the application directory, i.e \sphinxcode{\sphinxupquote{\$APPLICATION.workdir}}.

\item {} 
Create a launcher with a given name (default name is \sphinxcode{\sphinxupquote{APPLICATION.profile.launcher\_name}})

\fvset{hllines={, ,}}%
\begin{sphinxVerbatim}[commandchars=\\\{\}]
\PYG{n}{sat} \PYG{n}{launcher} \PYG{o}{\PYGZlt{}}\PYG{n}{application}\PYG{o}{\PYGZgt{}} \PYG{o}{\PYGZhy{}}\PYG{o}{\PYGZhy{}}\PYG{n}{name} \PYG{n}{ZeLauncher}
\end{sphinxVerbatim}

The launcher will be called \sphinxstyleemphasis{ZeLauncher}.

\item {} 
Set a specific resources catalog:

\fvset{hllines={, ,}}%
\begin{sphinxVerbatim}[commandchars=\\\{\}]
\PYG{n}{sat} \PYG{n}{launcher} \PYG{o}{\PYGZlt{}}\PYG{n}{application}\PYG{o}{\PYGZgt{}}  \PYG{o}{\PYGZhy{}}\PYG{o}{\PYGZhy{}}\PYG{n}{catalog}  \PYG{o}{\PYGZlt{}}\PYG{n}{path} \PYG{n}{of} \PYG{n}{a} \PYG{n}{salome} \PYG{n}{resources} \PYG{n}{catalog}\PYG{o}{\PYGZgt{}}
\end{sphinxVerbatim}

Note that the catalog specified will be copied to the profile directory.

\item {} 
Generate the catalog for a list of machines:

\fvset{hllines={, ,}}%
\begin{sphinxVerbatim}[commandchars=\\\{\}]
\PYG{n}{sat} \PYG{n}{launcher} \PYG{o}{\PYGZlt{}}\PYG{n}{application}\PYG{o}{\PYGZgt{}} \PYG{o}{\PYGZhy{}}\PYG{o}{\PYGZhy{}}\PYG{n}{gencat} \PYG{o}{\PYGZlt{}}\PYG{n+nb}{list} \PYG{n}{of} \PYG{n}{machines}\PYG{o}{\PYGZgt{}}
\end{sphinxVerbatim}

This will create a catalog by querying each machine (memory, number of processor) with ssh.

\item {} 
Generate a mesa launcher (if mesa and llvm are parts of the application). Use this option only if you have to use salome through ssh and have problems with ssh X forwarding of OpengGL modules (like Paravis):

\fvset{hllines={, ,}}%
\begin{sphinxVerbatim}[commandchars=\\\{\}]
\PYG{n}{sat} \PYG{n}{launcher} \PYG{o}{\PYGZlt{}}\PYG{n}{application}\PYG{o}{\PYGZgt{}} \PYG{o}{\PYGZhy{}}\PYG{o}{\PYGZhy{}}\PYG{n}{use\PYGZus{}mesa}
\end{sphinxVerbatim}

\end{itemize}


\subsection{Configuration}
\label{\detokenize{commands/launcher:configuration}}
Some useful configuration pathes:
\begin{itemize}
\item {} 
\sphinxstylestrong{APPLICATION.profile}
\begin{itemize}
\item {} 
\sphinxstylestrong{product} : the name of the profile product (the product in charge of holding the application stuff, like logos, splashscreen)

\item {} 
\sphinxstylestrong{launcher\_name} : the name of the launcher.

\end{itemize}

\end{itemize}

\clearpage


\section{Command application}
\label{\detokenize{commands/application:svn}}\label{\detokenize{commands/application::doc}}\label{\detokenize{commands/application:command-application}}

\subsection{Description}
\label{\detokenize{commands/application:description}}
The \sphinxstylestrong{application} command creates a virtual \sphinxhref{http://www.salome-platform.org}{SALOME}%
\begin{footnote}[7]\sphinxAtStartFootnote
\sphinxnolinkurl{http://www.salome-platform.org}
%
\end{footnote} application.
Virtual SALOME applications are used to start SALOME when distribution is needed.


\subsection{Usage}
\label{\detokenize{commands/application:usage}}\begin{itemize}
\item {} 
Create an application:

\fvset{hllines={, ,}}%
\begin{sphinxVerbatim}[commandchars=\\\{\}]
\PYG{n}{sat} \PYG{n}{application} \PYG{o}{\PYGZlt{}}\PYG{n}{application}\PYG{o}{\PYGZgt{}}
\end{sphinxVerbatim}

Create the virtual application directory in the salomeTool application directory \sphinxcode{\sphinxupquote{\$APPLICATION.workdir}}.

\item {} 
Give a name to the application:

\fvset{hllines={, ,}}%
\begin{sphinxVerbatim}[commandchars=\\\{\}]
\PYG{n}{sat} \PYG{n}{application} \PYG{o}{\PYGZlt{}}\PYG{n}{application}\PYG{o}{\PYGZgt{}} \PYG{o}{\PYGZhy{}}\PYG{o}{\PYGZhy{}}\PYG{n}{name} \PYG{o}{\PYGZlt{}}\PYG{n}{my\PYGZus{}application\PYGZus{}name}\PYG{o}{\PYGZgt{}}
\end{sphinxVerbatim}

\sphinxstyleemphasis{Remark}: this option overrides the name given in the virtual\_app section of the configuration file \sphinxcode{\sphinxupquote{\$APPLICATION.virtual\_app.name}}.

\item {} 
Change the directory where the application is created:

\fvset{hllines={, ,}}%
\begin{sphinxVerbatim}[commandchars=\\\{\}]
\PYG{n}{sat} \PYG{n}{application} \PYG{o}{\PYGZlt{}}\PYG{n}{application}\PYG{o}{\PYGZgt{}} \PYG{o}{\PYGZhy{}}\PYG{o}{\PYGZhy{}}\PYG{n}{target} \PYG{o}{\PYGZlt{}}\PYG{n}{my\PYGZus{}application\PYGZus{}directory}\PYG{o}{\PYGZgt{}}
\end{sphinxVerbatim}

\item {} 
Set a specific \sphinxhref{http://www.salome-platform.org}{SALOME}%
\begin{footnote}[8]\sphinxAtStartFootnote
\sphinxnolinkurl{http://www.salome-platform.org}
%
\end{footnote} resources catalog (it will be used for the distribution of components on distant machines):

\fvset{hllines={, ,}}%
\begin{sphinxVerbatim}[commandchars=\\\{\}]
\PYG{n}{sat} \PYG{n}{application} \PYG{o}{\PYGZlt{}}\PYG{n}{application}\PYG{o}{\PYGZgt{}} \PYG{o}{\PYGZhy{}}\PYG{o}{\PYGZhy{}}\PYG{n}{catalog} \PYG{o}{\PYGZlt{}}\PYG{n}{path\PYGZus{}to\PYGZus{}catalog}\PYG{o}{\PYGZgt{}}
\end{sphinxVerbatim}

Note that the catalog specified will be copied to the application directory.

\item {} 
Generate the catalog for a list of machines:

\fvset{hllines={, ,}}%
\begin{sphinxVerbatim}[commandchars=\\\{\}]
\PYG{n}{sat} \PYG{n}{application} \PYG{o}{\PYGZlt{}}\PYG{n}{application}\PYG{o}{\PYGZgt{}} \PYG{o}{\PYGZhy{}}\PYG{o}{\PYGZhy{}}\PYG{n}{gencat} \PYG{n}{machine1}\PYG{p}{,}\PYG{n}{machine2}\PYG{p}{,}\PYG{n}{machine3}
\end{sphinxVerbatim}

This will create a catalog by querying each machine through ssh protocol (memory, number of processor) with ssh.

\item {} 
Generate a mesa application (if mesa and llvm are parts of the application). Use this option only if you have to use salome through ssh and have problems with ssh X forwarding of OpengGL modules (like Paravis):

\fvset{hllines={, ,}}%
\begin{sphinxVerbatim}[commandchars=\\\{\}]
\PYG{n}{sat} \PYG{n}{launcher} \PYG{o}{\PYGZlt{}}\PYG{n}{application}\PYG{o}{\PYGZgt{}} \PYG{o}{\PYGZhy{}}\PYG{o}{\PYGZhy{}}\PYG{n}{use\PYGZus{}mesa}
\end{sphinxVerbatim}

\end{itemize}


\subsection{Some useful configuration pathes}
\label{\detokenize{commands/application:some-useful-configuration-pathes}}
The virtual application can be configured with the virtual\_app section of the configutation file.
\begin{itemize}
\item {} 
\sphinxstylestrong{APPLICATION.virtual\_app}
\begin{itemize}
\item {} 
\sphinxstylestrong{name} : name of the launcher (to replace the default runAppli).

\item {} 
\sphinxstylestrong{application\_name} : (optional) the name of the virtual application directory, if missing the default value is \sphinxcode{\sphinxupquote{\$name + \_appli}}.

\end{itemize}

\end{itemize}

\clearpage


\section{Command log}
\label{\detokenize{commands/log:svn}}\label{\detokenize{commands/log:command-log}}\label{\detokenize{commands/log::doc}}

\subsection{Description}
\label{\detokenize{commands/log:description}}
The \sphinxstylestrong{log} command displays sat log in a web browser or in a terminal.


\subsection{Usage}
\label{\detokenize{commands/log:usage}}\begin{itemize}
\item {} 
Show (in a web browser) the log of the commands corresponding to an application:

\fvset{hllines={, ,}}%
\begin{sphinxVerbatim}[commandchars=\\\{\}]
\PYG{n}{sat} \PYG{n}{log} \PYG{o}{\PYGZlt{}}\PYG{n}{application}\PYG{o}{\PYGZgt{}}
\end{sphinxVerbatim}

\item {} 
Show the log for commands that do not use any application:

\fvset{hllines={, ,}}%
\begin{sphinxVerbatim}[commandchars=\\\{\}]
\PYG{n}{sat} \PYG{n}{log}
\end{sphinxVerbatim}

\item {} 
The \textendash{}terminal (or -t) display the log directly in the terminal, through a \sphinxhref{https://en.wikipedia.org/wiki/Command-line\_interface}{CLI}%
\begin{footnote}[9]\sphinxAtStartFootnote
\sphinxnolinkurl{https://en.wikipedia.org/wiki/Command-line\_interface}
%
\end{footnote} interactive menu:

\fvset{hllines={, ,}}%
\begin{sphinxVerbatim}[commandchars=\\\{\}]
\PYG{n}{sat} \PYG{n}{log} \PYG{o}{\PYGZlt{}}\PYG{n}{application}\PYG{o}{\PYGZgt{}} \PYG{o}{\PYGZhy{}}\PYG{o}{\PYGZhy{}}\PYG{n}{terminal}
\end{sphinxVerbatim}

\item {} 
The \textendash{}last option displays only the last command:

\fvset{hllines={, ,}}%
\begin{sphinxVerbatim}[commandchars=\\\{\}]
\PYG{n}{sat} \PYG{n}{log} \PYG{o}{\PYGZlt{}}\PYG{n}{application}\PYG{o}{\PYGZgt{}} \PYG{o}{\PYGZhy{}}\PYG{o}{\PYGZhy{}}\PYG{n}{last}
\end{sphinxVerbatim}

\item {} 
To access the last compilation log in terminal mode, use \textendash{}last\_terminal option:

\fvset{hllines={, ,}}%
\begin{sphinxVerbatim}[commandchars=\\\{\}]
\PYG{n}{sat} \PYG{n}{log} \PYG{o}{\PYGZlt{}}\PYG{n}{application}\PYG{o}{\PYGZgt{}} \PYG{o}{\PYGZhy{}}\PYG{o}{\PYGZhy{}}\PYG{n}{last\PYGZus{}terminal}
\end{sphinxVerbatim}

\item {} 
The \textendash{}clean (int) option erases the n older log files and print the number of remaining log files:

\fvset{hllines={, ,}}%
\begin{sphinxVerbatim}[commandchars=\\\{\}]
\PYG{n}{sat} \PYG{n}{log} \PYG{o}{\PYGZlt{}}\PYG{n}{application}\PYG{o}{\PYGZgt{}} \PYG{o}{\PYGZhy{}}\PYG{o}{\PYGZhy{}}\PYG{n}{clean} \PYG{l+m+mi}{50}
\end{sphinxVerbatim}

\end{itemize}


\subsection{Some useful configuration pathes}
\label{\detokenize{commands/log:some-useful-configuration-pathes}}\begin{itemize}
\item {} 
\sphinxstylestrong{USER}
\begin{itemize}
\item {} 
\sphinxstylestrong{browser} : The browser used to show the log (by default \sphinxstyleemphasis{firefox}).

\item {} 
\sphinxstylestrong{log\_dir} : The directory used to store the log files.

\end{itemize}

\end{itemize}

\clearpage


\section{Command environ}
\label{\detokenize{commands/environ:svn}}\label{\detokenize{commands/environ:command-environ}}\label{\detokenize{commands/environ::doc}}

\subsection{Description}
\label{\detokenize{commands/environ:description}}
The \sphinxstylestrong{environ} command generates the environment files used
to run and compile your application (as \sphinxhref{http://www.salome-platform.org}{SALOME}%
\begin{footnote}[10]\sphinxAtStartFootnote
\sphinxnolinkurl{http://www.salome-platform.org}
%
\end{footnote} is an example).

\begin{sphinxadmonition}{note}{Note:}
these files are \sphinxstylestrong{not} required,
salomeTool set the environment himself, when compiling.
And so does the salome launcher.

These files are useful when someone wants to check the environment.
They could be used in debug mode to set the environment for \sphinxstyleemphasis{gdb}.
\end{sphinxadmonition}

The configuration part at the end of this page explains how
to specify the environment used by sat (at build or run time),
and saved in some files by \sphinxstyleemphasis{sat environ} command.


\subsection{Usage}
\label{\detokenize{commands/environ:usage}}\begin{itemize}
\item {} 
Create the shell environment files of the application:

\fvset{hllines={, ,}}%
\begin{sphinxVerbatim}[commandchars=\\\{\}]
\PYG{n}{sat} \PYG{n}{environ} \PYG{o}{\PYGZlt{}}\PYG{n}{application}\PYG{o}{\PYGZgt{}}
\end{sphinxVerbatim}

\item {} 
Create the environment files of the application for a given shell.
Options are bash, bat (for windows) and cfg (the configuration format used by \sphinxhref{http://www.salome-platform.org}{SALOME}%
\begin{footnote}[11]\sphinxAtStartFootnote
\sphinxnolinkurl{http://www.salome-platform.org}
%
\end{footnote}):

\fvset{hllines={, ,}}%
\begin{sphinxVerbatim}[commandchars=\\\{\}]
\PYG{n}{sat} \PYG{n}{environ} \PYG{o}{\PYGZlt{}}\PYG{n}{application}\PYG{o}{\PYGZgt{}} \PYG{o}{\PYGZhy{}}\PYG{o}{\PYGZhy{}}\PYG{n}{shell} \PYG{p}{[}\PYG{n}{bash}\PYG{o}{\textbar{}}\PYG{n}{cfg}\PYG{o}{\textbar{}}\PYG{n+nb}{all}\PYG{p}{]}
\end{sphinxVerbatim}

\item {} 
Use a different prefix for the files (default is ‘env’):

\fvset{hllines={, ,}}%
\begin{sphinxVerbatim}[commandchars=\\\{\}]
\PYG{c+c1}{\PYGZsh{} This will create file \PYGZlt{}prefix\PYGZgt{}\PYGZus{}launch.sh, \PYGZlt{}prefix\PYGZgt{}\PYGZus{}build.sh}
sat environ \PYGZlt{}application\PYGZgt{} \PYGZhy{}\PYGZhy{}prefix \PYGZlt{}prefix\PYGZgt{}
\end{sphinxVerbatim}

\item {} 
Use a different target directory for the files:

\fvset{hllines={, ,}}%
\begin{sphinxVerbatim}[commandchars=\\\{\}]
\PYG{c+c1}{\PYGZsh{} This will create file env\PYGZus{}launch.sh, env\PYGZus{}build.sh}
\PYG{c+c1}{\PYGZsh{} in the directory corresponding to \PYGZlt{}path\PYGZgt{}}
sat environ \PYGZlt{}application\PYGZgt{} \PYGZhy{}\PYGZhy{}target \PYGZlt{}path\PYGZgt{}
\end{sphinxVerbatim}

\item {} 
Generate the environment files only with the given products:

\fvset{hllines={, ,}}%
\begin{sphinxVerbatim}[commandchars=\\\{\}]
\PYG{c+c1}{\PYGZsh{} This will create the environment files only for the given products}
\PYG{c+c1}{\PYGZsh{} and their prerequisites.}
\PYG{c+c1}{\PYGZsh{} It is useful when you want to visualise which environment uses}
\PYG{c+c1}{\PYGZsh{} sat to compile a given product.}
sat environ \PYGZlt{}application\PYGZgt{} \PYGZhy{}\PYGZhy{}product \PYGZlt{}product1\PYGZgt{},\PYGZlt{}product2\PYGZgt{}, ...
\end{sphinxVerbatim}

\end{itemize}


\subsection{Configuration}
\label{\detokenize{commands/environ:configuration}}
The specification of the environment can be done through several mechanisms.
\begin{enumerate}
\item {} 
For salome products (the products with the property \sphinxcode{\sphinxupquote{is\_SALOME\_module}} as \sphinxcode{\sphinxupquote{yes}}) the environment is set automatically by sat, in respect with \sphinxhref{http://www.salome-platform.org}{SALOME}%
\begin{footnote}[12]\sphinxAtStartFootnote
\sphinxnolinkurl{http://www.salome-platform.org}
%
\end{footnote} requirements.

\item {} 
For other products, the environment is set with the use of the environ section within the pyconf file of the product. The user has two possibilities, either set directly the environment within the section, or specify a python script which wil be used to set the environment programmatically.

\end{enumerate}

Within the section, the user can define environment variables. He can also modify PATH variables, by appending or prepending directories.
In the following example, we prepend \sphinxstyleemphasis{\textless{}install\_dir\textgreater{}/lib} to \sphinxcode{\sphinxupquote{LD\_LIBRARY\_PATH}} (note the \sphinxstyleemphasis{left first} underscore), append \sphinxstyleemphasis{\textless{}install\_dir\textgreater{}/lib} to \sphinxcode{\sphinxupquote{PYTHONPATH}} (note the \sphinxstyleemphasis{right last} underscore), and set \sphinxcode{\sphinxupquote{LAPACK\_ROOT\_DIR}} to \sphinxstyleemphasis{\textless{}install\_dir\textgreater{}}:

\fvset{hllines={, ,}}%
\begin{sphinxVerbatim}[commandchars=\\\{\}]
environ :
\PYG{o}{\PYGZob{}}
  \PYGZus{}LD\PYGZus{}LIBRARY\PYGZus{}PATH : \PYG{n+nv}{\PYGZdl{}install\PYGZus{}dir} + \PYG{n+nv}{\PYGZdl{}VARS}.sep + \PYG{l+s+s2}{\PYGZdq{}lib\PYGZdq{}}
  PYTHONPATH\PYGZus{} : \PYG{n+nv}{\PYGZdl{}install\PYGZus{}dir} + \PYG{n+nv}{\PYGZdl{}VARS}.sep + \PYG{l+s+s2}{\PYGZdq{}lib\PYGZdq{}}
  LAPACK\PYGZus{}ROOT\PYGZus{}DIR : \PYG{n+nv}{\PYGZdl{}install\PYGZus{}dir}
\PYG{o}{\PYGZcb{}}
\end{sphinxVerbatim}

It is possible to distinguish the build environment from the launch environment: use a subsection called \sphinxstyleemphasis{build} or \sphinxstyleemphasis{launch}. In the example below, \sphinxcode{\sphinxupquote{LD\_LIBRARY\_PATH}} and \sphinxcode{\sphinxupquote{PYTHONPATH}} are only modified at run time, not at compile time:

\fvset{hllines={, ,}}%
\begin{sphinxVerbatim}[commandchars=\\\{\}]
environ :
\PYG{o}{\PYGZob{}}
  build :
  \PYG{o}{\PYGZob{}}
    LAPACK\PYGZus{}ROOT\PYGZus{}DIR : \PYG{n+nv}{\PYGZdl{}install\PYGZus{}dir}
  \PYG{o}{\PYGZcb{}}
  launch :
  \PYG{o}{\PYGZob{}}
    LAPACK\PYGZus{}ROOT\PYGZus{}DIR : \PYG{n+nv}{\PYGZdl{}install\PYGZus{}dir}
    \PYGZus{}LD\PYGZus{}LIBRARY\PYGZus{}PATH : \PYG{n+nv}{\PYGZdl{}install\PYGZus{}dir} + \PYG{n+nv}{\PYGZdl{}VARS}.sep + \PYG{l+s+s2}{\PYGZdq{}lib\PYGZdq{}}
    PYTHONPATH\PYGZus{} : \PYG{n+nv}{\PYGZdl{}install\PYGZus{}dir} + \PYG{n+nv}{\PYGZdl{}VARS}.sep + \PYG{l+s+s2}{\PYGZdq{}lib\PYGZdq{}}
  \PYG{o}{\PYGZcb{}}
\PYG{o}{\PYGZcb{}}
\end{sphinxVerbatim}
\begin{enumerate}
\setcounter{enumi}{2}
\item {} 
The last possibility is to set the environment with a python script. The script should be provided in the \sphinxstyleemphasis{products/env\_scripts} directory of the sat project, and its name is specified in the environment section with the key \sphinxcode{\sphinxupquote{environ.env\_script}}:

\end{enumerate}

\fvset{hllines={, ,}}%
\begin{sphinxVerbatim}[commandchars=\\\{\}]
\PYG{n}{environ} \PYG{p}{:}
\PYG{p}{\PYGZob{}}
  \PYG{n}{env\PYGZus{}script} \PYG{p}{:} \PYG{l+s+s1}{\PYGZsq{}}\PYG{l+s+s1}{lapack.py}\PYG{l+s+s1}{\PYGZsq{}}
\PYG{p}{\PYGZcb{}}
\end{sphinxVerbatim}

Please note that the two modes are complementary and are both taken into account.
Most of the time, the first mode is sufficient.

The second mode can be used when the environment has to be set programmatically.
The developer implements a handle (as a python method)
which is called by sat to set the environment.
Here is an example:

\fvset{hllines={, ,}}%
\begin{sphinxVerbatim}[commandchars=\\\{\}]
\PYG{c+ch}{\PYGZsh{}!/usr/bin/env python}
\PYG{c+c1}{\PYGZsh{}\PYGZhy{}*\PYGZhy{} coding:utf\PYGZhy{}8 \PYGZhy{}*\PYGZhy{}}

\PYG{k+kn}{import} \PYG{n+nn}{os.path}
\PYG{k+kn}{import} \PYG{n+nn}{platform}

\PYG{k}{def} \PYG{n+nf}{set\PYGZus{}env}\PYG{p}{(}\PYG{n}{env}\PYG{p}{,} \PYG{n}{prereq\PYGZus{}dir}\PYG{p}{,} \PYG{n}{version}\PYG{p}{)}\PYG{p}{:}
    \PYG{n}{env}\PYG{o}{.}\PYG{n}{set}\PYG{p}{(}\PYG{l+s+s2}{\PYGZdq{}}\PYG{l+s+s2}{TRUST\PYGZus{}ROOT\PYGZus{}DIR}\PYG{l+s+s2}{\PYGZdq{}}\PYG{p}{,}\PYG{n}{prereq\PYGZus{}dir}\PYG{p}{)}
    \PYG{n}{env}\PYG{o}{.}\PYG{n}{prepend}\PYG{p}{(}\PYG{l+s+s1}{\PYGZsq{}}\PYG{l+s+s1}{PATH}\PYG{l+s+s1}{\PYGZsq{}}\PYG{p}{,} \PYG{n}{os}\PYG{o}{.}\PYG{n}{path}\PYG{o}{.}\PYG{n}{join}\PYG{p}{(}\PYG{n}{prereq\PYGZus{}dir}\PYG{p}{,} \PYG{l+s+s1}{\PYGZsq{}}\PYG{l+s+s1}{bin}\PYG{l+s+s1}{\PYGZsq{}}\PYG{p}{)}\PYG{p}{)}
    \PYG{n}{env}\PYG{o}{.}\PYG{n}{prepend}\PYG{p}{(}\PYG{l+s+s1}{\PYGZsq{}}\PYG{l+s+s1}{PATH}\PYG{l+s+s1}{\PYGZsq{}}\PYG{p}{,} \PYG{n}{os}\PYG{o}{.}\PYG{n}{path}\PYG{o}{.}\PYG{n}{join}\PYG{p}{(}\PYG{n}{prereq\PYGZus{}dir}\PYG{p}{,} \PYG{l+s+s1}{\PYGZsq{}}\PYG{l+s+s1}{include}\PYG{l+s+s1}{\PYGZsq{}}\PYG{p}{)}\PYG{p}{)}
    \PYG{n}{env}\PYG{o}{.}\PYG{n}{prepend}\PYG{p}{(}\PYG{l+s+s1}{\PYGZsq{}}\PYG{l+s+s1}{LD\PYGZus{}LIBRARY\PYGZus{}PATH}\PYG{l+s+s1}{\PYGZsq{}}\PYG{p}{,} \PYG{n}{os}\PYG{o}{.}\PYG{n}{path}\PYG{o}{.}\PYG{n}{join}\PYG{p}{(}\PYG{n}{prereq\PYGZus{}dir}\PYG{p}{,} \PYG{l+s+s1}{\PYGZsq{}}\PYG{l+s+s1}{lib}\PYG{l+s+s1}{\PYGZsq{}}\PYG{p}{)}\PYG{p}{)}
    \PYG{k}{return}
\end{sphinxVerbatim}

SalomeTools defines four handles:
\begin{itemize}
\item {} 
\sphinxstylestrong{set\_env(env, prereq\_dir, version)} : used at build and run time.

\item {} 
\sphinxstylestrong{set\_env\_launch(env, prereq\_dir, version)} : used only at run time (if defined!)

\item {} 
\sphinxstylestrong{set\_env\_build(env, prereq\_dir, version)} : used only at build time (if defined!)

\item {} 
\sphinxstylestrong{set\_native\_env(env)} : used only for native products, at build and run time.

\end{itemize}

\clearpage


\section{Command clean}
\label{\detokenize{commands/clean:svn}}\label{\detokenize{commands/clean:command-clean}}\label{\detokenize{commands/clean::doc}}

\subsection{Description}
\label{\detokenize{commands/clean:description}}
The \sphinxstylestrong{clean} command removes products in the \sphinxstyleemphasis{source, build, or install} directories of an application. Theses directories are usually named \sphinxcode{\sphinxupquote{SOURCES, BUILD, INSTALL}}.

Use the options to define what directories you want to suppress and to set the list of products


\subsection{Usage}
\label{\detokenize{commands/clean:usage}}\begin{itemize}
\item {} 
Clean all previously created \sphinxstyleemphasis{build} and \sphinxstyleemphasis{install} directories (example application as \sphinxstyleemphasis{SALOME\_xx}):

\fvset{hllines={, ,}}%
\begin{sphinxVerbatim}[commandchars=\\\{\}]
\PYG{c+c1}{\PYGZsh{} take care, is long time to restore, sometimes}
sat clean SALOME\PYGZhy{}xx \PYGZhy{}\PYGZhy{}build \PYGZhy{}\PYGZhy{}install
\end{sphinxVerbatim}

\item {} 
Clean previously created \sphinxstyleemphasis{build} and \sphinxstyleemphasis{install} directories, only for products with property \sphinxstyleemphasis{is\_salome\_module}:

\fvset{hllines={, ,}}%
\begin{sphinxVerbatim}[commandchars=\\\{\}]
sat clean SALOME\PYGZhy{}xxx \PYGZhy{}\PYGZhy{}build \PYGZhy{}\PYGZhy{}install \PYG{l+s+se}{\PYGZbs{}}
                     \PYGZhy{}\PYGZhy{}properties is\PYGZus{}salome\PYGZus{}module:yes
\end{sphinxVerbatim}

\end{itemize}


\subsection{Availables options}
\label{\detokenize{commands/clean:availables-options}}\begin{itemize}
\item {} 
\sphinxstylestrong{\textendash{}products} : Products to clean.

\item {} 
\sphinxstylestrong{\textendash{}properties} :

\begin{DUlineblock}{0em}
\item[] Filter the products by their properties.
\item[] Syntax: \sphinxstyleemphasis{\textendash{}properties \textless{}property\textgreater{}:\textless{}value\textgreater{}}
\end{DUlineblock}

\item {} 
\sphinxstylestrong{\textendash{}sources} : Clean the product source directories.

\item {} 
\sphinxstylestrong{\textendash{}build} : Clean the product build directories.

\item {} 
\sphinxstylestrong{\textendash{}install} : Clean the product install directories.

\item {} 
\sphinxstylestrong{\textendash{}all} : Clean the product source, build and install directories.

\item {} 
\sphinxstylestrong{\textendash{}sources\_without\_dev} :

\begin{DUlineblock}{0em}
\item[] Do not clean the products in development mode,
\item[] (they could have \sphinxhref{https://en.wikipedia.org/wiki/Version\_control}{VCS}%
\begin{footnote}[13]\sphinxAtStartFootnote
\sphinxnolinkurl{https://en.wikipedia.org/wiki/Version\_control}
%
\end{footnote} commits pending).
\end{DUlineblock}

\end{itemize}


\subsection{Some useful configuration pathes}
\label{\detokenize{commands/clean:some-useful-configuration-pathes}}
No specific configuration.

\clearpage


\section{Command package}
\label{\detokenize{commands/package:svn}}\label{\detokenize{commands/package:command-package}}\label{\detokenize{commands/package::doc}}

\subsection{Description}
\label{\detokenize{commands/package:description}}
The \sphinxstylestrong{package} command creates a \sphinxhref{http://www.salome-platform.org}{SALOME}%
\begin{footnote}[14]\sphinxAtStartFootnote
\sphinxnolinkurl{http://www.salome-platform.org}
%
\end{footnote} archive (usually a compressed \sphinxhref{https://en.wikipedia.org/wiki/Tar\_(computing)}{Tar}%
\begin{footnote}[15]\sphinxAtStartFootnote
\sphinxnolinkurl{https://en.wikipedia.org/wiki/Tar\_(computing)}
%
\end{footnote} file .tgz).
This tar file is used later to intall SALOME on other remote computer.

Depending on the selected options, the archive includes sources and binaries
of SALOME products and prerequisites.

Usually utility \sphinxstyleemphasis{salomeTools} is included in the archive.

\begin{sphinxadmonition}{note}{Note:}
By default the package includes the sources of prerequisites and products.
To select a subset use the \sphinxstyleemphasis{\textendash{}without\_property} or \sphinxstyleemphasis{\textendash{}with\_vcs} options.
\end{sphinxadmonition}


\subsection{Usage}
\label{\detokenize{commands/package:usage}}\begin{itemize}
\item {} 
Create a package for a product (example as \sphinxstyleemphasis{SALOME\_xx}):

\fvset{hllines={, ,}}%
\begin{sphinxVerbatim}[commandchars=\\\{\}]
\PYG{n}{sat} \PYG{n}{package} \PYG{n}{SALOME\PYGZus{}xx}
\end{sphinxVerbatim}

This command will create an archive named \sphinxcode{\sphinxupquote{SALOME\_xx.tgz}}
in the working directory (\sphinxcode{\sphinxupquote{USER.workDir}}).
If the archive already exists, do nothing.

\item {} 
Create a package with a specific name:

\fvset{hllines={, ,}}%
\begin{sphinxVerbatim}[commandchars=\\\{\}]
\PYG{n}{sat} \PYG{n}{package} \PYG{n}{SALOME\PYGZus{}xx} \PYG{o}{\PYGZhy{}}\PYG{o}{\PYGZhy{}}\PYG{n}{name} \PYG{n}{YourSpecificName}
\end{sphinxVerbatim}

\end{itemize}

\begin{sphinxadmonition}{note}{Note:}
By default, the archive is created in the working directory of the user (\sphinxcode{\sphinxupquote{USER.workDir}}).

If the option \sphinxstyleemphasis{\textendash{}name} is used with a path (relative or absolute) it will be used.

If the option \sphinxstyleemphasis{\textendash{}name} is not used and binaries (prerequisites and products)
are included in the package, the \sphinxhref{https://en.wikipedia.org/wiki/Operating\_system}{OS}%
\begin{footnote}[16]\sphinxAtStartFootnote
\sphinxnolinkurl{https://en.wikipedia.org/wiki/Operating\_system}
%
\end{footnote} architecture
will be appended to the name (example: \sphinxcode{\sphinxupquote{SALOME\_xx-CO7.tgz}}).

Examples:

\fvset{hllines={, ,}}%
\begin{sphinxVerbatim}[commandchars=\\\{\}]
\PYG{c+c1}{\PYGZsh{} Creates SALOME\PYGZus{}xx.tgz in \PYGZdl{}USER.workDir}
\PYG{n}{sat} \PYG{n}{package} \PYG{n}{SALOME\PYGZus{}xx}

\PYG{c+c1}{\PYGZsh{} Creates SALOME\PYGZus{}xx\PYGZus{}\PYGZlt{}arch\PYGZgt{}.tgz in \PYGZdl{}USER.workDir}
\PYG{n}{sat} \PYG{n}{package} \PYG{n}{SALOME\PYGZus{}xx} \PYG{o}{\PYGZhy{}}\PYG{o}{\PYGZhy{}}\PYG{n}{binaries}

\PYG{c+c1}{\PYGZsh{} Creates MySpecificName.tgz in \PYGZdl{}USER.workDir}
\PYG{n}{sat} \PYG{n}{package} \PYG{n}{SALOME\PYGZus{}xx} \PYG{o}{\PYGZhy{}}\PYG{o}{\PYGZhy{}}\PYG{n}{name} \PYG{n}{MySpecificName}
\end{sphinxVerbatim}
\end{sphinxadmonition}
\begin{itemize}
\item {} 
Force the creation of the archive (if it already exists):

\fvset{hllines={, ,}}%
\begin{sphinxVerbatim}[commandchars=\\\{\}]
\PYG{n}{sat} \PYG{n}{package} \PYG{n}{SALOME\PYGZus{}xx} \PYG{o}{\PYGZhy{}}\PYG{o}{\PYGZhy{}}\PYG{n}{force}
\end{sphinxVerbatim}

\item {} 
Include the binaries in the archive (products and prerequisites):

\fvset{hllines={, ,}}%
\begin{sphinxVerbatim}[commandchars=\\\{\}]
\PYG{n}{sat} \PYG{n}{package} \PYG{n}{SALOME\PYGZus{}xx} \PYG{o}{\PYGZhy{}}\PYG{o}{\PYGZhy{}}\PYG{n}{binaries}
\end{sphinxVerbatim}

This command will create an archive named \sphinxcode{\sphinxupquote{SALOME\_xx \_\textless{}arch\textgreater{}.tgz}}
where \textless{}arch\textgreater{} is the \sphinxhref{https://en.wikipedia.org/wiki/Operating\_system}{OS}%
\begin{footnote}[17]\sphinxAtStartFootnote
\sphinxnolinkurl{https://en.wikipedia.org/wiki/Operating\_system}
%
\end{footnote} architecture of the machine.

\item {} 
Do not delete Version Control System (\sphinxhref{https://en.wikipedia.org/wiki/Version\_control}{VCS}%
\begin{footnote}[18]\sphinxAtStartFootnote
\sphinxnolinkurl{https://en.wikipedia.org/wiki/Version\_control}
%
\end{footnote}) informations from the configurations files of the embedded salomeTools:

\fvset{hllines={, ,}}%
\begin{sphinxVerbatim}[commandchars=\\\{\}]
\PYG{n}{sat} \PYG{n}{package} \PYG{n}{SALOME\PYGZus{}xx} \PYG{o}{\PYGZhy{}}\PYG{o}{\PYGZhy{}}\PYG{n}{with\PYGZus{}vcs}
\end{sphinxVerbatim}

The version control systems known by this option are \sphinxhref{https://fr.wikipedia.org/wiki/Concurrent\_versions\_system}{CVS}%
\begin{footnote}[19]\sphinxAtStartFootnote
\sphinxnolinkurl{https://fr.wikipedia.org/wiki/Concurrent\_versions\_system}
%
\end{footnote}, \sphinxhref{https://en.wikipedia.org/wiki/Apache\_Subversion}{SVN}%
\begin{footnote}[20]\sphinxAtStartFootnote
\sphinxnolinkurl{https://en.wikipedia.org/wiki/Apache\_Subversion}
%
\end{footnote} and \sphinxhref{https://git-scm.com}{Git}%
\begin{footnote}[21]\sphinxAtStartFootnote
\sphinxnolinkurl{https://git-scm.com}
%
\end{footnote}.

\end{itemize}


\subsection{Some useful configuration pathes}
\label{\detokenize{commands/package:some-useful-configuration-pathes}}
No specific configuration.

\clearpage


\section{Command generate}
\label{\detokenize{commands/generate:svn}}\label{\detokenize{commands/generate:command-generate}}\label{\detokenize{commands/generate::doc}}

\subsection{Description}
\label{\detokenize{commands/generate:description}}
The \sphinxstylestrong{generate} command generates and compile SALOME modules from cpp modules using YACSGEN.

\begin{sphinxadmonition}{note}{Note:}
This command uses YACSGEN to generate the module. It needs to be specified with \sphinxstyleemphasis{\textendash{}yacsgen} option, or defined in the product or by the environment variable \sphinxcode{\sphinxupquote{\$YACSGEN\_ROOT\_DIR}}.
\end{sphinxadmonition}


\subsection{Remarks}
\label{\detokenize{commands/generate:remarks}}\begin{itemize}
\item {} 
This command will only apply on the CPP modules of the application, those who have both properties:

\fvset{hllines={, ,}}%
\begin{sphinxVerbatim}[commandchars=\\\{\}]
\PYG{n}{cpp} \PYG{p}{:} \PYG{l+s+s2}{\PYGZdq{}}\PYG{l+s+s2}{yes}\PYG{l+s+s2}{\PYGZdq{}}
\PYG{n}{generate} \PYG{p}{:} \PYG{l+s+s2}{\PYGZdq{}}\PYG{l+s+s2}{yes}\PYG{l+s+s2}{\PYGZdq{}}
\end{sphinxVerbatim}

\item {} 
The cpp module are usually computational components, and the generated module brings the CORBA layer which allows distributing the compononent on remore machines. cpp modules should conform to YACSGEN/hxx2salome requirements (please refer to YACSGEN documentation)

\end{itemize}


\subsection{Usage}
\label{\detokenize{commands/generate:usage}}\begin{itemize}
\item {} 
Generate all the modules of a product:

\fvset{hllines={, ,}}%
\begin{sphinxVerbatim}[commandchars=\\\{\}]
\PYG{n}{sat} \PYG{n}{generate} \PYG{o}{\PYGZlt{}}\PYG{n}{application}\PYG{o}{\PYGZgt{}}
\end{sphinxVerbatim}

\item {} 
Generate only specific modules:

\fvset{hllines={, ,}}%
\begin{sphinxVerbatim}[commandchars=\\\{\}]
\PYG{n}{sat} \PYG{n}{generate} \PYG{o}{\PYGZlt{}}\PYG{n}{application}\PYG{o}{\PYGZgt{}} \PYG{o}{\PYGZhy{}}\PYG{o}{\PYGZhy{}}\PYG{n}{products} \PYG{o}{\PYGZlt{}}\PYG{n}{list\PYGZus{}of\PYGZus{}products}\PYG{o}{\PYGZgt{}}
\end{sphinxVerbatim}

Remark: modules which don’t have the \sphinxstyleemphasis{generate} property are ignored.

\item {} 
Use a specific version of YACSGEN:

\fvset{hllines={, ,}}%
\begin{sphinxVerbatim}[commandchars=\\\{\}]
\PYG{n}{sat} \PYG{n}{generate} \PYG{o}{\PYGZlt{}}\PYG{n}{application}\PYG{o}{\PYGZgt{}} \PYG{o}{\PYGZhy{}}\PYG{o}{\PYGZhy{}}\PYG{n}{yacsgen} \PYG{o}{\PYGZlt{}}\PYG{n}{path\PYGZus{}to\PYGZus{}yacsgen}\PYG{o}{\PYGZgt{}}
\end{sphinxVerbatim}

\end{itemize}


\chapter{Developer documentation}
\label{\detokenize{index:developer-documentation}}
\clearpage


\section{Add a user custom command}
\label{\detokenize{write_command:svn}}\label{\detokenize{write_command:add-a-user-custom-command}}\label{\detokenize{write_command::doc}}

\subsection{Introduction}
\label{\detokenize{write_command:introduction}}
\begin{sphinxadmonition}{note}{Note:}
This documentation is for \sphinxhref{https://docs.python.org/2.7}{Python}%
\begin{footnote}[22]\sphinxAtStartFootnote
\sphinxnolinkurl{https://docs.python.org/2.7}
%
\end{footnote} developers.
\end{sphinxadmonition}

The salomeTools product provides a simple way to develop commands.
The first thing to do is to add a file with \sphinxstyleemphasis{.py} extension in the \sphinxcode{\sphinxupquote{commands}} directory of salomeTools.

Here are the basic requirements that must be followed in this file in order to add a command.


\subsection{Basic requirements}
\label{\detokenize{write_command:basic-requirements}}
\begin{sphinxadmonition}{warning}{Warning:}
ALL THIS IS OBSOLETE FOR SAT 5.1
\end{sphinxadmonition}

By adding a file \sphinxstyleemphasis{mycommand.py} in the \sphinxcode{\sphinxupquote{commands}} directory, salomeTools will define a new command named \sphinxcode{\sphinxupquote{mycommand}}.

In \sphinxstyleemphasis{mycommand.py}, there must be the following method:

\fvset{hllines={, ,}}%
\begin{sphinxVerbatim}[commandchars=\\\{\}]
\PYG{k}{def} \PYG{n+nf}{run}\PYG{p}{(}\PYG{n}{args}\PYG{p}{,} \PYG{n}{runner}\PYG{p}{,} \PYG{n}{logger}\PYG{p}{)}\PYG{p}{:}
    \PYG{c+c1}{\PYGZsh{} your algorithm ...}
    \PYG{k}{pass}
\end{sphinxVerbatim}

In fact, at this point, the command will already be functional.
But there are some useful services provided by salomeTools :
\begin{itemize}
\item {} 
You can give some options to your command:

\end{itemize}

\fvset{hllines={, ,}}%
\begin{sphinxVerbatim}[commandchars=\\\{\}]
\PYG{k+kn}{import} \PYG{n+nn}{src}

\PYG{c+c1}{\PYGZsh{} Define all possible option for mycommand command :  \PYGZsq{}sat mycommand \PYGZlt{}options\PYGZgt{}\PYGZsq{}}
\PYG{n}{parser} \PYG{o}{=} \PYG{n}{src}\PYG{o}{.}\PYG{n}{options}\PYG{o}{.}\PYG{n}{Options}\PYG{p}{(}\PYG{p}{)}
\PYG{n}{parser}\PYG{o}{.}\PYG{n}{add\PYGZus{}option}\PYG{p}{(}\PYG{l+s+s1}{\PYGZsq{}}\PYG{l+s+s1}{m}\PYG{l+s+s1}{\PYGZsq{}}\PYG{p}{,} \PYG{l+s+s1}{\PYGZsq{}}\PYG{l+s+s1}{myoption}\PYG{l+s+s1}{\PYGZsq{}}\PYG{p}{,} \PYGZbs{}
                  \PYG{l+s+s1}{\PYGZsq{}}\PYG{l+s+s1}{boolean}\PYG{l+s+s1}{\PYGZsq{}}\PYG{p}{,} \PYG{l+s+s1}{\PYGZsq{}}\PYG{l+s+s1}{myoption}\PYG{l+s+s1}{\PYGZsq{}}\PYG{p}{,} \PYGZbs{}
                  \PYG{l+s+s1}{\PYGZsq{}}\PYG{l+s+s1}{My option changes the behavior of my command.}\PYG{l+s+s1}{\PYGZsq{}}\PYG{p}{)}

\PYG{k}{def} \PYG{n+nf}{run}\PYG{p}{(}\PYG{n}{args}\PYG{p}{,} \PYG{n}{runner}\PYG{p}{,} \PYG{n}{logger}\PYG{p}{)}\PYG{p}{:}
    \PYG{c+c1}{\PYGZsh{} Parse the options}
    \PYG{p}{(}\PYG{n}{options}\PYG{p}{,} \PYG{n}{args}\PYG{p}{)} \PYG{o}{=} \PYG{n}{parser}\PYG{o}{.}\PYG{n}{parse\PYGZus{}args}\PYG{p}{(}\PYG{n}{args}\PYG{p}{)}
    \PYG{c+c1}{\PYGZsh{} algorithm}
\end{sphinxVerbatim}
\begin{itemize}
\item {} 
You can add a \sphinxstyleemphasis{description} method that will display a message when the user will call the help:

\end{itemize}

\fvset{hllines={, 9, 10,}}%
\begin{sphinxVerbatim}[commandchars=\\\{\}]
 \PYG{k+kn}{import} \PYG{n+nn}{src}

 \PYG{c+c1}{\PYGZsh{} Define all possible option for mycommand command : \PYGZsq{}sat mycommand \PYGZlt{}options\PYGZgt{}\PYGZsq{}}
 \PYG{n}{parser} \PYG{o}{=} \PYG{n}{src}\PYG{o}{.}\PYG{n}{options}\PYG{o}{.}\PYG{n}{Options}\PYG{p}{(}\PYG{p}{)}
 \PYG{n}{parser}\PYG{o}{.}\PYG{n}{add\PYGZus{}option}\PYG{p}{(}\PYG{l+s+s1}{\PYGZsq{}}\PYG{l+s+s1}{m}\PYG{l+s+s1}{\PYGZsq{}}\PYG{p}{,} \PYG{l+s+s1}{\PYGZsq{}}\PYG{l+s+s1}{myoption}\PYG{l+s+s1}{\PYGZsq{}}\PYG{p}{,} \PYGZbs{}
                   \PYG{l+s+s1}{\PYGZsq{}}\PYG{l+s+s1}{boolean}\PYG{l+s+s1}{\PYGZsq{}}\PYG{p}{,} \PYG{l+s+s1}{\PYGZsq{}}\PYG{l+s+s1}{myoption}\PYG{l+s+s1}{\PYGZsq{}}\PYG{p}{,} \PYGZbs{}
                   \PYG{l+s+s1}{\PYGZsq{}}\PYG{l+s+s1}{My option changes the behavior of my command.}\PYG{l+s+s1}{\PYGZsq{}}\PYG{p}{)}

 \PYG{k}{def} \PYG{n+nf}{description}\PYG{p}{(}\PYG{p}{)}\PYG{p}{:}
     \PYG{k}{return} \PYG{n}{\PYGZus{}}\PYG{p}{(}\PYG{l+s+s2}{\PYGZdq{}}\PYG{l+s+s2}{The help of mycommand.}\PYG{l+s+s2}{\PYGZdq{}}\PYG{p}{)}

 \PYG{k}{def} \PYG{n+nf}{run}\PYG{p}{(}\PYG{n}{args}\PYG{p}{,} \PYG{n}{runner}\PYG{p}{,} \PYG{n}{logger}\PYG{p}{)}\PYG{p}{:}
     \PYG{c+c1}{\PYGZsh{} Parse the options}
     \PYG{p}{(}\PYG{n}{options}\PYG{p}{,} \PYG{n}{args}\PYG{p}{)} \PYG{o}{=} \PYG{n}{parser}\PYG{o}{.}\PYG{n}{parse\PYGZus{}args}\PYG{p}{(}\PYG{n}{args}\PYG{p}{)}
     \PYG{c+c1}{\PYGZsh{} algorithm}
\end{sphinxVerbatim}


\subsection{HowTo access salomeTools config and other commands}
\label{\detokenize{write_command:howto-access-salometools-config-and-other-commands}}
The \sphinxstyleemphasis{runner} variable is an python instance of \sphinxstyleemphasis{Sat} class.
It gives access to \sphinxstyleemphasis{runner.getConfig()} which is the data model defined from all
\sphinxstyleemphasis{configuration pyconf files} of salomeTools
For example, \sphinxstyleemphasis{runner.cfg.APPLICATION.workdir}
contains the root directory of the current application.

The \sphinxstyleemphasis{runner} variable gives also access to other commands of salomeTools:

\fvset{hllines={, ,}}%
\begin{sphinxVerbatim}[commandchars=\\\{\}]
\PYG{c+c1}{\PYGZsh{} as CLI\PYGZus{} \PYGZsq{}sat prepare ...\PYGZsq{}}
\PYG{n}{runner}\PYG{o}{.}\PYG{n}{prepare}\PYG{p}{(}\PYG{n}{runner}\PYG{o}{.}\PYG{n}{cfg}\PYG{o}{.}\PYG{n}{VARS}\PYG{o}{.}\PYG{n}{application}\PYG{p}{)}
\end{sphinxVerbatim}


\subsection{HowTo logger}
\label{\detokenize{write_command:howto-logger}}
The logger variable is an instance of the python \sphinxcode{\sphinxupquote{logging}} package class.
It gives access to \sphinxcode{\sphinxupquote{debug, info, warning, error, critical}} methods.

Using these methods, the message passed as parameter
will be displayed in the terminal and written in an xml log file.

\fvset{hllines={, ,}}%
\begin{sphinxVerbatim}[commandchars=\\\{\}]
\PYG{n}{logger}\PYG{o}{.}\PYG{n}{info}\PYG{p}{(}\PYG{l+s+s2}{\PYGZdq{}}\PYG{l+s+s2}{My message}\PYG{l+s+s2}{\PYGZdq{}}\PYG{p}{)}
\end{sphinxVerbatim}


\subsection{HELLO example}
\label{\detokenize{write_command:hello-example}}
Here is a \sphinxstyleemphasis{hello} command, file \sphinxstyleemphasis{commands/hello.py}:

\fvset{hllines={, ,}}%
\begin{sphinxVerbatim}[commandchars=\\\{\}]
\PYG{k+kn}{import} \PYG{n+nn}{src}

\PYG{l+s+sd}{\PYGZdq{}\PYGZdq{}\PYGZdq{}}
\PYG{l+s+sd}{hello.py}
\PYG{l+s+sd}{Define all possible options for hello command:}
\PYG{l+s+sd}{sat hello \PYGZlt{}options\PYGZgt{}}
\PYG{l+s+sd}{\PYGZdq{}\PYGZdq{}\PYGZdq{}}

\PYG{n}{parser} \PYG{o}{=} \PYG{n}{src}\PYG{o}{.}\PYG{n}{options}\PYG{o}{.}\PYG{n}{Options}\PYG{p}{(}\PYG{p}{)}
\PYG{n}{parser}\PYG{o}{.}\PYG{n}{add\PYGZus{}option}\PYG{p}{(}\PYG{l+s+s1}{\PYGZsq{}}\PYG{l+s+s1}{f}\PYG{l+s+s1}{\PYGZsq{}}\PYG{p}{,} \PYG{l+s+s1}{\PYGZsq{}}\PYG{l+s+s1}{french}\PYG{l+s+s1}{\PYGZsq{}}\PYG{p}{,} \PYG{l+s+s1}{\PYGZsq{}}\PYG{l+s+s1}{boolean}\PYG{l+s+s1}{\PYGZsq{}}\PYG{p}{,} \PYG{l+s+s1}{\PYGZsq{}}\PYG{l+s+s1}{french}\PYG{l+s+s1}{\PYGZsq{}}\PYG{p}{,} \PYG{l+s+s2}{\PYGZdq{}}\PYG{l+s+s2}{french set hello message in french.}\PYG{l+s+s2}{\PYGZdq{}}\PYG{p}{)}

\PYG{k}{def} \PYG{n+nf}{description}\PYG{p}{(}\PYG{p}{)}\PYG{p}{:}
    \PYG{k}{return} \PYG{n}{\PYGZus{}}\PYG{p}{(}\PYG{l+s+s2}{\PYGZdq{}}\PYG{l+s+s2}{The help of hello.}\PYG{l+s+s2}{\PYGZdq{}}\PYG{p}{)}

\PYG{k}{def} \PYG{n+nf}{run}\PYG{p}{(}\PYG{n}{args}\PYG{p}{,} \PYG{n}{runner}\PYG{p}{,} \PYG{n}{logger}\PYG{p}{)}\PYG{p}{:}
    \PYG{c+c1}{\PYGZsh{} Parse the options}
    \PYG{p}{(}\PYG{n}{options}\PYG{p}{,} \PYG{n}{args}\PYG{p}{)} \PYG{o}{=} \PYG{n}{parser}\PYG{o}{.}\PYG{n}{parse\PYGZus{}args}\PYG{p}{(}\PYG{n}{args}\PYG{p}{)}
    \PYG{c+c1}{\PYGZsh{} algorithm}
    \PYG{k}{if} \PYG{o+ow}{not} \PYG{n}{options}\PYG{o}{.}\PYG{n}{french}\PYG{p}{:}
        \PYG{n}{logger}\PYG{o}{.}\PYG{n}{info}\PYG{p}{(}\PYG{l+s+s1}{\PYGZsq{}}\PYG{l+s+s1}{HELLO! WORLD!}\PYG{l+s+se}{\PYGZbs{}n}\PYG{l+s+s1}{\PYGZsq{}}\PYG{p}{)}
    \PYG{k}{else}\PYG{p}{:}
        \PYG{n}{logger}\PYG{o}{.}\PYG{n}{writeinfo}\PYG{p}{(}\PYG{l+s+s1}{\PYGZsq{}}\PYG{l+s+s1}{Bonjour tout le monde!}\PYG{l+s+se}{\PYGZbs{}n}\PYG{l+s+s1}{\PYGZsq{}}\PYG{p}{)}
\end{sphinxVerbatim}

A first call of hello:

\fvset{hllines={, ,}}%
\begin{sphinxVerbatim}[commandchars=\\\{\}]
\PYG{c+c1}{\PYGZsh{} Get the help of hello:}
./sat \PYGZhy{}\PYGZhy{}help hello

\PYG{c+c1}{\PYGZsh{} To get bonjour}
./sat hello \PYGZhy{}\PYGZhy{}french
Bonjour tout le monde!

\PYG{c+c1}{\PYGZsh{} To get hello}
./sat hello
HELLO! WORLD!

\PYG{c+c1}{\PYGZsh{} To get the log}
./sat log
\end{sphinxVerbatim}


\chapter{Code documentation}
\label{\detokenize{index:code-documentation}}

\section{src}
\label{\detokenize{apidoc_src/modules:src}}\label{\detokenize{apidoc_src/modules::doc}}

\subsection{src package}
\label{\detokenize{apidoc_src/src::doc}}\label{\detokenize{apidoc_src/src:src-package}}

\subsubsection{Subpackages}
\label{\detokenize{apidoc_src/src:subpackages}}

\paragraph{src.colorama package}
\label{\detokenize{apidoc_src/src.colorama:src-colorama-package}}\label{\detokenize{apidoc_src/src.colorama::doc}}

\subparagraph{Submodules}
\label{\detokenize{apidoc_src/src.colorama:submodules}}

\subparagraph{src.colorama.ansi module}
\label{\detokenize{apidoc_src/src.colorama:module-src.colorama.ansi}}\label{\detokenize{apidoc_src/src.colorama:src-colorama-ansi-module}}\index{src.colorama.ansi (module)}
This module generates ANSI character codes to printing colors to terminals.
See: \sphinxurl{http://en.wikipedia.org/wiki/ANSI\_escape\_code}
\index{AnsiBack (class in src.colorama.ansi)}

\begin{fulllineitems}
\phantomsection\label{\detokenize{apidoc_src/src.colorama:src.colorama.ansi.AnsiBack}}\pysigline{\sphinxbfcode{\sphinxupquote{class }}\sphinxcode{\sphinxupquote{src.colorama.ansi.}}\sphinxbfcode{\sphinxupquote{AnsiBack}}}
Bases: {\hyperref[\detokenize{apidoc_src/src.colorama:src.colorama.ansi.AnsiCodes}]{\sphinxcrossref{\sphinxcode{\sphinxupquote{src.colorama.ansi.AnsiCodes}}}}} (\autopageref*{\detokenize{apidoc_src/src.colorama:src.colorama.ansi.AnsiCodes}})
\index{BLACK (src.colorama.ansi.AnsiBack attribute)}

\begin{fulllineitems}
\phantomsection\label{\detokenize{apidoc_src/src.colorama:src.colorama.ansi.AnsiBack.BLACK}}\pysigline{\sphinxbfcode{\sphinxupquote{BLACK}}\sphinxbfcode{\sphinxupquote{ = 40}}}
\end{fulllineitems}

\index{BLUE (src.colorama.ansi.AnsiBack attribute)}

\begin{fulllineitems}
\phantomsection\label{\detokenize{apidoc_src/src.colorama:src.colorama.ansi.AnsiBack.BLUE}}\pysigline{\sphinxbfcode{\sphinxupquote{BLUE}}\sphinxbfcode{\sphinxupquote{ = 44}}}
\end{fulllineitems}

\index{CYAN (src.colorama.ansi.AnsiBack attribute)}

\begin{fulllineitems}
\phantomsection\label{\detokenize{apidoc_src/src.colorama:src.colorama.ansi.AnsiBack.CYAN}}\pysigline{\sphinxbfcode{\sphinxupquote{CYAN}}\sphinxbfcode{\sphinxupquote{ = 46}}}
\end{fulllineitems}

\index{GREEN (src.colorama.ansi.AnsiBack attribute)}

\begin{fulllineitems}
\phantomsection\label{\detokenize{apidoc_src/src.colorama:src.colorama.ansi.AnsiBack.GREEN}}\pysigline{\sphinxbfcode{\sphinxupquote{GREEN}}\sphinxbfcode{\sphinxupquote{ = 42}}}
\end{fulllineitems}

\index{LIGHTBLACK\_EX (src.colorama.ansi.AnsiBack attribute)}

\begin{fulllineitems}
\phantomsection\label{\detokenize{apidoc_src/src.colorama:src.colorama.ansi.AnsiBack.LIGHTBLACK_EX}}\pysigline{\sphinxbfcode{\sphinxupquote{LIGHTBLACK\_EX}}\sphinxbfcode{\sphinxupquote{ = 100}}}
\end{fulllineitems}

\index{LIGHTBLUE\_EX (src.colorama.ansi.AnsiBack attribute)}

\begin{fulllineitems}
\phantomsection\label{\detokenize{apidoc_src/src.colorama:src.colorama.ansi.AnsiBack.LIGHTBLUE_EX}}\pysigline{\sphinxbfcode{\sphinxupquote{LIGHTBLUE\_EX}}\sphinxbfcode{\sphinxupquote{ = 104}}}
\end{fulllineitems}

\index{LIGHTCYAN\_EX (src.colorama.ansi.AnsiBack attribute)}

\begin{fulllineitems}
\phantomsection\label{\detokenize{apidoc_src/src.colorama:src.colorama.ansi.AnsiBack.LIGHTCYAN_EX}}\pysigline{\sphinxbfcode{\sphinxupquote{LIGHTCYAN\_EX}}\sphinxbfcode{\sphinxupquote{ = 106}}}
\end{fulllineitems}

\index{LIGHTGREEN\_EX (src.colorama.ansi.AnsiBack attribute)}

\begin{fulllineitems}
\phantomsection\label{\detokenize{apidoc_src/src.colorama:src.colorama.ansi.AnsiBack.LIGHTGREEN_EX}}\pysigline{\sphinxbfcode{\sphinxupquote{LIGHTGREEN\_EX}}\sphinxbfcode{\sphinxupquote{ = 102}}}
\end{fulllineitems}

\index{LIGHTMAGENTA\_EX (src.colorama.ansi.AnsiBack attribute)}

\begin{fulllineitems}
\phantomsection\label{\detokenize{apidoc_src/src.colorama:src.colorama.ansi.AnsiBack.LIGHTMAGENTA_EX}}\pysigline{\sphinxbfcode{\sphinxupquote{LIGHTMAGENTA\_EX}}\sphinxbfcode{\sphinxupquote{ = 105}}}
\end{fulllineitems}

\index{LIGHTRED\_EX (src.colorama.ansi.AnsiBack attribute)}

\begin{fulllineitems}
\phantomsection\label{\detokenize{apidoc_src/src.colorama:src.colorama.ansi.AnsiBack.LIGHTRED_EX}}\pysigline{\sphinxbfcode{\sphinxupquote{LIGHTRED\_EX}}\sphinxbfcode{\sphinxupquote{ = 101}}}
\end{fulllineitems}

\index{LIGHTWHITE\_EX (src.colorama.ansi.AnsiBack attribute)}

\begin{fulllineitems}
\phantomsection\label{\detokenize{apidoc_src/src.colorama:src.colorama.ansi.AnsiBack.LIGHTWHITE_EX}}\pysigline{\sphinxbfcode{\sphinxupquote{LIGHTWHITE\_EX}}\sphinxbfcode{\sphinxupquote{ = 107}}}
\end{fulllineitems}

\index{LIGHTYELLOW\_EX (src.colorama.ansi.AnsiBack attribute)}

\begin{fulllineitems}
\phantomsection\label{\detokenize{apidoc_src/src.colorama:src.colorama.ansi.AnsiBack.LIGHTYELLOW_EX}}\pysigline{\sphinxbfcode{\sphinxupquote{LIGHTYELLOW\_EX}}\sphinxbfcode{\sphinxupquote{ = 103}}}
\end{fulllineitems}

\index{MAGENTA (src.colorama.ansi.AnsiBack attribute)}

\begin{fulllineitems}
\phantomsection\label{\detokenize{apidoc_src/src.colorama:src.colorama.ansi.AnsiBack.MAGENTA}}\pysigline{\sphinxbfcode{\sphinxupquote{MAGENTA}}\sphinxbfcode{\sphinxupquote{ = 45}}}
\end{fulllineitems}

\index{RED (src.colorama.ansi.AnsiBack attribute)}

\begin{fulllineitems}
\phantomsection\label{\detokenize{apidoc_src/src.colorama:src.colorama.ansi.AnsiBack.RED}}\pysigline{\sphinxbfcode{\sphinxupquote{RED}}\sphinxbfcode{\sphinxupquote{ = 41}}}
\end{fulllineitems}

\index{RESET (src.colorama.ansi.AnsiBack attribute)}

\begin{fulllineitems}
\phantomsection\label{\detokenize{apidoc_src/src.colorama:src.colorama.ansi.AnsiBack.RESET}}\pysigline{\sphinxbfcode{\sphinxupquote{RESET}}\sphinxbfcode{\sphinxupquote{ = 49}}}
\end{fulllineitems}

\index{WHITE (src.colorama.ansi.AnsiBack attribute)}

\begin{fulllineitems}
\phantomsection\label{\detokenize{apidoc_src/src.colorama:src.colorama.ansi.AnsiBack.WHITE}}\pysigline{\sphinxbfcode{\sphinxupquote{WHITE}}\sphinxbfcode{\sphinxupquote{ = 47}}}
\end{fulllineitems}

\index{YELLOW (src.colorama.ansi.AnsiBack attribute)}

\begin{fulllineitems}
\phantomsection\label{\detokenize{apidoc_src/src.colorama:src.colorama.ansi.AnsiBack.YELLOW}}\pysigline{\sphinxbfcode{\sphinxupquote{YELLOW}}\sphinxbfcode{\sphinxupquote{ = 43}}}
\end{fulllineitems}


\end{fulllineitems}

\index{AnsiCodes (class in src.colorama.ansi)}

\begin{fulllineitems}
\phantomsection\label{\detokenize{apidoc_src/src.colorama:src.colorama.ansi.AnsiCodes}}\pysigline{\sphinxbfcode{\sphinxupquote{class }}\sphinxcode{\sphinxupquote{src.colorama.ansi.}}\sphinxbfcode{\sphinxupquote{AnsiCodes}}}
Bases: \sphinxcode{\sphinxupquote{object}}

\end{fulllineitems}

\index{AnsiCursor (class in src.colorama.ansi)}

\begin{fulllineitems}
\phantomsection\label{\detokenize{apidoc_src/src.colorama:src.colorama.ansi.AnsiCursor}}\pysigline{\sphinxbfcode{\sphinxupquote{class }}\sphinxcode{\sphinxupquote{src.colorama.ansi.}}\sphinxbfcode{\sphinxupquote{AnsiCursor}}}
Bases: \sphinxcode{\sphinxupquote{object}}
\index{BACK() (src.colorama.ansi.AnsiCursor method)}

\begin{fulllineitems}
\phantomsection\label{\detokenize{apidoc_src/src.colorama:src.colorama.ansi.AnsiCursor.BACK}}\pysiglinewithargsret{\sphinxbfcode{\sphinxupquote{BACK}}}{\emph{n=1}}{}
\end{fulllineitems}

\index{DOWN() (src.colorama.ansi.AnsiCursor method)}

\begin{fulllineitems}
\phantomsection\label{\detokenize{apidoc_src/src.colorama:src.colorama.ansi.AnsiCursor.DOWN}}\pysiglinewithargsret{\sphinxbfcode{\sphinxupquote{DOWN}}}{\emph{n=1}}{}
\end{fulllineitems}

\index{FORWARD() (src.colorama.ansi.AnsiCursor method)}

\begin{fulllineitems}
\phantomsection\label{\detokenize{apidoc_src/src.colorama:src.colorama.ansi.AnsiCursor.FORWARD}}\pysiglinewithargsret{\sphinxbfcode{\sphinxupquote{FORWARD}}}{\emph{n=1}}{}
\end{fulllineitems}

\index{POS() (src.colorama.ansi.AnsiCursor method)}

\begin{fulllineitems}
\phantomsection\label{\detokenize{apidoc_src/src.colorama:src.colorama.ansi.AnsiCursor.POS}}\pysiglinewithargsret{\sphinxbfcode{\sphinxupquote{POS}}}{\emph{x=1}, \emph{y=1}}{}
\end{fulllineitems}

\index{UP() (src.colorama.ansi.AnsiCursor method)}

\begin{fulllineitems}
\phantomsection\label{\detokenize{apidoc_src/src.colorama:src.colorama.ansi.AnsiCursor.UP}}\pysiglinewithargsret{\sphinxbfcode{\sphinxupquote{UP}}}{\emph{n=1}}{}
\end{fulllineitems}


\end{fulllineitems}

\index{AnsiFore (class in src.colorama.ansi)}

\begin{fulllineitems}
\phantomsection\label{\detokenize{apidoc_src/src.colorama:src.colorama.ansi.AnsiFore}}\pysigline{\sphinxbfcode{\sphinxupquote{class }}\sphinxcode{\sphinxupquote{src.colorama.ansi.}}\sphinxbfcode{\sphinxupquote{AnsiFore}}}
Bases: {\hyperref[\detokenize{apidoc_src/src.colorama:src.colorama.ansi.AnsiCodes}]{\sphinxcrossref{\sphinxcode{\sphinxupquote{src.colorama.ansi.AnsiCodes}}}}} (\autopageref*{\detokenize{apidoc_src/src.colorama:src.colorama.ansi.AnsiCodes}})
\index{BLACK (src.colorama.ansi.AnsiFore attribute)}

\begin{fulllineitems}
\phantomsection\label{\detokenize{apidoc_src/src.colorama:src.colorama.ansi.AnsiFore.BLACK}}\pysigline{\sphinxbfcode{\sphinxupquote{BLACK}}\sphinxbfcode{\sphinxupquote{ = 30}}}
\end{fulllineitems}

\index{BLUE (src.colorama.ansi.AnsiFore attribute)}

\begin{fulllineitems}
\phantomsection\label{\detokenize{apidoc_src/src.colorama:src.colorama.ansi.AnsiFore.BLUE}}\pysigline{\sphinxbfcode{\sphinxupquote{BLUE}}\sphinxbfcode{\sphinxupquote{ = 34}}}
\end{fulllineitems}

\index{CYAN (src.colorama.ansi.AnsiFore attribute)}

\begin{fulllineitems}
\phantomsection\label{\detokenize{apidoc_src/src.colorama:src.colorama.ansi.AnsiFore.CYAN}}\pysigline{\sphinxbfcode{\sphinxupquote{CYAN}}\sphinxbfcode{\sphinxupquote{ = 36}}}
\end{fulllineitems}

\index{GREEN (src.colorama.ansi.AnsiFore attribute)}

\begin{fulllineitems}
\phantomsection\label{\detokenize{apidoc_src/src.colorama:src.colorama.ansi.AnsiFore.GREEN}}\pysigline{\sphinxbfcode{\sphinxupquote{GREEN}}\sphinxbfcode{\sphinxupquote{ = 32}}}
\end{fulllineitems}

\index{LIGHTBLACK\_EX (src.colorama.ansi.AnsiFore attribute)}

\begin{fulllineitems}
\phantomsection\label{\detokenize{apidoc_src/src.colorama:src.colorama.ansi.AnsiFore.LIGHTBLACK_EX}}\pysigline{\sphinxbfcode{\sphinxupquote{LIGHTBLACK\_EX}}\sphinxbfcode{\sphinxupquote{ = 90}}}
\end{fulllineitems}

\index{LIGHTBLUE\_EX (src.colorama.ansi.AnsiFore attribute)}

\begin{fulllineitems}
\phantomsection\label{\detokenize{apidoc_src/src.colorama:src.colorama.ansi.AnsiFore.LIGHTBLUE_EX}}\pysigline{\sphinxbfcode{\sphinxupquote{LIGHTBLUE\_EX}}\sphinxbfcode{\sphinxupquote{ = 94}}}
\end{fulllineitems}

\index{LIGHTCYAN\_EX (src.colorama.ansi.AnsiFore attribute)}

\begin{fulllineitems}
\phantomsection\label{\detokenize{apidoc_src/src.colorama:src.colorama.ansi.AnsiFore.LIGHTCYAN_EX}}\pysigline{\sphinxbfcode{\sphinxupquote{LIGHTCYAN\_EX}}\sphinxbfcode{\sphinxupquote{ = 96}}}
\end{fulllineitems}

\index{LIGHTGREEN\_EX (src.colorama.ansi.AnsiFore attribute)}

\begin{fulllineitems}
\phantomsection\label{\detokenize{apidoc_src/src.colorama:src.colorama.ansi.AnsiFore.LIGHTGREEN_EX}}\pysigline{\sphinxbfcode{\sphinxupquote{LIGHTGREEN\_EX}}\sphinxbfcode{\sphinxupquote{ = 92}}}
\end{fulllineitems}

\index{LIGHTMAGENTA\_EX (src.colorama.ansi.AnsiFore attribute)}

\begin{fulllineitems}
\phantomsection\label{\detokenize{apidoc_src/src.colorama:src.colorama.ansi.AnsiFore.LIGHTMAGENTA_EX}}\pysigline{\sphinxbfcode{\sphinxupquote{LIGHTMAGENTA\_EX}}\sphinxbfcode{\sphinxupquote{ = 95}}}
\end{fulllineitems}

\index{LIGHTRED\_EX (src.colorama.ansi.AnsiFore attribute)}

\begin{fulllineitems}
\phantomsection\label{\detokenize{apidoc_src/src.colorama:src.colorama.ansi.AnsiFore.LIGHTRED_EX}}\pysigline{\sphinxbfcode{\sphinxupquote{LIGHTRED\_EX}}\sphinxbfcode{\sphinxupquote{ = 91}}}
\end{fulllineitems}

\index{LIGHTWHITE\_EX (src.colorama.ansi.AnsiFore attribute)}

\begin{fulllineitems}
\phantomsection\label{\detokenize{apidoc_src/src.colorama:src.colorama.ansi.AnsiFore.LIGHTWHITE_EX}}\pysigline{\sphinxbfcode{\sphinxupquote{LIGHTWHITE\_EX}}\sphinxbfcode{\sphinxupquote{ = 97}}}
\end{fulllineitems}

\index{LIGHTYELLOW\_EX (src.colorama.ansi.AnsiFore attribute)}

\begin{fulllineitems}
\phantomsection\label{\detokenize{apidoc_src/src.colorama:src.colorama.ansi.AnsiFore.LIGHTYELLOW_EX}}\pysigline{\sphinxbfcode{\sphinxupquote{LIGHTYELLOW\_EX}}\sphinxbfcode{\sphinxupquote{ = 93}}}
\end{fulllineitems}

\index{MAGENTA (src.colorama.ansi.AnsiFore attribute)}

\begin{fulllineitems}
\phantomsection\label{\detokenize{apidoc_src/src.colorama:src.colorama.ansi.AnsiFore.MAGENTA}}\pysigline{\sphinxbfcode{\sphinxupquote{MAGENTA}}\sphinxbfcode{\sphinxupquote{ = 35}}}
\end{fulllineitems}

\index{RED (src.colorama.ansi.AnsiFore attribute)}

\begin{fulllineitems}
\phantomsection\label{\detokenize{apidoc_src/src.colorama:src.colorama.ansi.AnsiFore.RED}}\pysigline{\sphinxbfcode{\sphinxupquote{RED}}\sphinxbfcode{\sphinxupquote{ = 31}}}
\end{fulllineitems}

\index{RESET (src.colorama.ansi.AnsiFore attribute)}

\begin{fulllineitems}
\phantomsection\label{\detokenize{apidoc_src/src.colorama:src.colorama.ansi.AnsiFore.RESET}}\pysigline{\sphinxbfcode{\sphinxupquote{RESET}}\sphinxbfcode{\sphinxupquote{ = 39}}}
\end{fulllineitems}

\index{WHITE (src.colorama.ansi.AnsiFore attribute)}

\begin{fulllineitems}
\phantomsection\label{\detokenize{apidoc_src/src.colorama:src.colorama.ansi.AnsiFore.WHITE}}\pysigline{\sphinxbfcode{\sphinxupquote{WHITE}}\sphinxbfcode{\sphinxupquote{ = 37}}}
\end{fulllineitems}

\index{YELLOW (src.colorama.ansi.AnsiFore attribute)}

\begin{fulllineitems}
\phantomsection\label{\detokenize{apidoc_src/src.colorama:src.colorama.ansi.AnsiFore.YELLOW}}\pysigline{\sphinxbfcode{\sphinxupquote{YELLOW}}\sphinxbfcode{\sphinxupquote{ = 33}}}
\end{fulllineitems}


\end{fulllineitems}

\index{AnsiStyle (class in src.colorama.ansi)}

\begin{fulllineitems}
\phantomsection\label{\detokenize{apidoc_src/src.colorama:src.colorama.ansi.AnsiStyle}}\pysigline{\sphinxbfcode{\sphinxupquote{class }}\sphinxcode{\sphinxupquote{src.colorama.ansi.}}\sphinxbfcode{\sphinxupquote{AnsiStyle}}}
Bases: {\hyperref[\detokenize{apidoc_src/src.colorama:src.colorama.ansi.AnsiCodes}]{\sphinxcrossref{\sphinxcode{\sphinxupquote{src.colorama.ansi.AnsiCodes}}}}} (\autopageref*{\detokenize{apidoc_src/src.colorama:src.colorama.ansi.AnsiCodes}})
\index{BRIGHT (src.colorama.ansi.AnsiStyle attribute)}

\begin{fulllineitems}
\phantomsection\label{\detokenize{apidoc_src/src.colorama:src.colorama.ansi.AnsiStyle.BRIGHT}}\pysigline{\sphinxbfcode{\sphinxupquote{BRIGHT}}\sphinxbfcode{\sphinxupquote{ = 1}}}
\end{fulllineitems}

\index{DIM (src.colorama.ansi.AnsiStyle attribute)}

\begin{fulllineitems}
\phantomsection\label{\detokenize{apidoc_src/src.colorama:src.colorama.ansi.AnsiStyle.DIM}}\pysigline{\sphinxbfcode{\sphinxupquote{DIM}}\sphinxbfcode{\sphinxupquote{ = 2}}}
\end{fulllineitems}

\index{NORMAL (src.colorama.ansi.AnsiStyle attribute)}

\begin{fulllineitems}
\phantomsection\label{\detokenize{apidoc_src/src.colorama:src.colorama.ansi.AnsiStyle.NORMAL}}\pysigline{\sphinxbfcode{\sphinxupquote{NORMAL}}\sphinxbfcode{\sphinxupquote{ = 22}}}
\end{fulllineitems}

\index{RESET\_ALL (src.colorama.ansi.AnsiStyle attribute)}

\begin{fulllineitems}
\phantomsection\label{\detokenize{apidoc_src/src.colorama:src.colorama.ansi.AnsiStyle.RESET_ALL}}\pysigline{\sphinxbfcode{\sphinxupquote{RESET\_ALL}}\sphinxbfcode{\sphinxupquote{ = 0}}}
\end{fulllineitems}


\end{fulllineitems}

\index{clear\_line() (in module src.colorama.ansi)}

\begin{fulllineitems}
\phantomsection\label{\detokenize{apidoc_src/src.colorama:src.colorama.ansi.clear_line}}\pysiglinewithargsret{\sphinxcode{\sphinxupquote{src.colorama.ansi.}}\sphinxbfcode{\sphinxupquote{clear\_line}}}{\emph{mode=2}}{}
\end{fulllineitems}

\index{clear\_screen() (in module src.colorama.ansi)}

\begin{fulllineitems}
\phantomsection\label{\detokenize{apidoc_src/src.colorama:src.colorama.ansi.clear_screen}}\pysiglinewithargsret{\sphinxcode{\sphinxupquote{src.colorama.ansi.}}\sphinxbfcode{\sphinxupquote{clear\_screen}}}{\emph{mode=2}}{}
\end{fulllineitems}

\index{code\_to\_chars() (in module src.colorama.ansi)}

\begin{fulllineitems}
\phantomsection\label{\detokenize{apidoc_src/src.colorama:src.colorama.ansi.code_to_chars}}\pysiglinewithargsret{\sphinxcode{\sphinxupquote{src.colorama.ansi.}}\sphinxbfcode{\sphinxupquote{code\_to\_chars}}}{\emph{code}}{}
\end{fulllineitems}

\index{set\_title() (in module src.colorama.ansi)}

\begin{fulllineitems}
\phantomsection\label{\detokenize{apidoc_src/src.colorama:src.colorama.ansi.set_title}}\pysiglinewithargsret{\sphinxcode{\sphinxupquote{src.colorama.ansi.}}\sphinxbfcode{\sphinxupquote{set\_title}}}{\emph{title}}{}
\end{fulllineitems}



\subparagraph{src.colorama.ansitowin32 module}
\label{\detokenize{apidoc_src/src.colorama:module-src.colorama.ansitowin32}}\label{\detokenize{apidoc_src/src.colorama:src-colorama-ansitowin32-module}}\index{src.colorama.ansitowin32 (module)}\index{AnsiToWin32 (class in src.colorama.ansitowin32)}

\begin{fulllineitems}
\phantomsection\label{\detokenize{apidoc_src/src.colorama:src.colorama.ansitowin32.AnsiToWin32}}\pysiglinewithargsret{\sphinxbfcode{\sphinxupquote{class }}\sphinxcode{\sphinxupquote{src.colorama.ansitowin32.}}\sphinxbfcode{\sphinxupquote{AnsiToWin32}}}{\emph{wrapped}, \emph{convert=None}, \emph{strip=None}, \emph{autoreset=False}}{}
Bases: \sphinxcode{\sphinxupquote{object}}

Implements a ‘write()’ method which, on Windows, will strip ANSI character
sequences from the text, and if outputting to a tty, will convert them into
win32 function calls.
\index{ANSI\_CSI\_RE (src.colorama.ansitowin32.AnsiToWin32 attribute)}

\begin{fulllineitems}
\phantomsection\label{\detokenize{apidoc_src/src.colorama:src.colorama.ansitowin32.AnsiToWin32.ANSI_CSI_RE}}\pysigline{\sphinxbfcode{\sphinxupquote{ANSI\_CSI\_RE}}\sphinxbfcode{\sphinxupquote{ = \textless{}\_sre.SRE\_Pattern object at 0x3504970\textgreater{}}}}
\end{fulllineitems}

\index{ANSI\_OSC\_RE (src.colorama.ansitowin32.AnsiToWin32 attribute)}

\begin{fulllineitems}
\phantomsection\label{\detokenize{apidoc_src/src.colorama:src.colorama.ansitowin32.AnsiToWin32.ANSI_OSC_RE}}\pysigline{\sphinxbfcode{\sphinxupquote{ANSI\_OSC\_RE}}\sphinxbfcode{\sphinxupquote{ = \textless{}\_sre.SRE\_Pattern object at 0x350dc10\textgreater{}}}}
\end{fulllineitems}

\index{call\_win32() (src.colorama.ansitowin32.AnsiToWin32 method)}

\begin{fulllineitems}
\phantomsection\label{\detokenize{apidoc_src/src.colorama:src.colorama.ansitowin32.AnsiToWin32.call_win32}}\pysiglinewithargsret{\sphinxbfcode{\sphinxupquote{call\_win32}}}{\emph{command}, \emph{params}}{}
\end{fulllineitems}

\index{convert\_ansi() (src.colorama.ansitowin32.AnsiToWin32 method)}

\begin{fulllineitems}
\phantomsection\label{\detokenize{apidoc_src/src.colorama:src.colorama.ansitowin32.AnsiToWin32.convert_ansi}}\pysiglinewithargsret{\sphinxbfcode{\sphinxupquote{convert\_ansi}}}{\emph{paramstring}, \emph{command}}{}
\end{fulllineitems}

\index{convert\_osc() (src.colorama.ansitowin32.AnsiToWin32 method)}

\begin{fulllineitems}
\phantomsection\label{\detokenize{apidoc_src/src.colorama:src.colorama.ansitowin32.AnsiToWin32.convert_osc}}\pysiglinewithargsret{\sphinxbfcode{\sphinxupquote{convert\_osc}}}{\emph{text}}{}
\end{fulllineitems}

\index{extract\_params() (src.colorama.ansitowin32.AnsiToWin32 method)}

\begin{fulllineitems}
\phantomsection\label{\detokenize{apidoc_src/src.colorama:src.colorama.ansitowin32.AnsiToWin32.extract_params}}\pysiglinewithargsret{\sphinxbfcode{\sphinxupquote{extract\_params}}}{\emph{command}, \emph{paramstring}}{}
\end{fulllineitems}

\index{get\_win32\_calls() (src.colorama.ansitowin32.AnsiToWin32 method)}

\begin{fulllineitems}
\phantomsection\label{\detokenize{apidoc_src/src.colorama:src.colorama.ansitowin32.AnsiToWin32.get_win32_calls}}\pysiglinewithargsret{\sphinxbfcode{\sphinxupquote{get\_win32\_calls}}}{}{}
\end{fulllineitems}

\index{reset\_all() (src.colorama.ansitowin32.AnsiToWin32 method)}

\begin{fulllineitems}
\phantomsection\label{\detokenize{apidoc_src/src.colorama:src.colorama.ansitowin32.AnsiToWin32.reset_all}}\pysiglinewithargsret{\sphinxbfcode{\sphinxupquote{reset\_all}}}{}{}
\end{fulllineitems}

\index{should\_wrap() (src.colorama.ansitowin32.AnsiToWin32 method)}

\begin{fulllineitems}
\phantomsection\label{\detokenize{apidoc_src/src.colorama:src.colorama.ansitowin32.AnsiToWin32.should_wrap}}\pysiglinewithargsret{\sphinxbfcode{\sphinxupquote{should\_wrap}}}{}{}
True if this class is actually needed. If false, then the output
stream will not be affected, nor will win32 calls be issued, so
wrapping stdout is not actually required. This will generally be
False on non-Windows platforms, unless optional functionality like
autoreset has been requested using kwargs to init()

\end{fulllineitems}

\index{write() (src.colorama.ansitowin32.AnsiToWin32 method)}

\begin{fulllineitems}
\phantomsection\label{\detokenize{apidoc_src/src.colorama:src.colorama.ansitowin32.AnsiToWin32.write}}\pysiglinewithargsret{\sphinxbfcode{\sphinxupquote{write}}}{\emph{text}}{}
\end{fulllineitems}

\index{write\_and\_convert() (src.colorama.ansitowin32.AnsiToWin32 method)}

\begin{fulllineitems}
\phantomsection\label{\detokenize{apidoc_src/src.colorama:src.colorama.ansitowin32.AnsiToWin32.write_and_convert}}\pysiglinewithargsret{\sphinxbfcode{\sphinxupquote{write\_and\_convert}}}{\emph{text}}{}
Write the given text to our wrapped stream, stripping any ANSI
sequences from the text, and optionally converting them into win32
calls.

\end{fulllineitems}

\index{write\_plain\_text() (src.colorama.ansitowin32.AnsiToWin32 method)}

\begin{fulllineitems}
\phantomsection\label{\detokenize{apidoc_src/src.colorama:src.colorama.ansitowin32.AnsiToWin32.write_plain_text}}\pysiglinewithargsret{\sphinxbfcode{\sphinxupquote{write\_plain\_text}}}{\emph{text}, \emph{start}, \emph{end}}{}
\end{fulllineitems}


\end{fulllineitems}

\index{StreamWrapper (class in src.colorama.ansitowin32)}

\begin{fulllineitems}
\phantomsection\label{\detokenize{apidoc_src/src.colorama:src.colorama.ansitowin32.StreamWrapper}}\pysiglinewithargsret{\sphinxbfcode{\sphinxupquote{class }}\sphinxcode{\sphinxupquote{src.colorama.ansitowin32.}}\sphinxbfcode{\sphinxupquote{StreamWrapper}}}{\emph{wrapped}, \emph{converter}}{}
Bases: \sphinxcode{\sphinxupquote{object}}

Wraps a stream (such as stdout), acting as a transparent proxy for all
attribute access apart from method ‘write()’, which is delegated to our
Converter instance.
\index{write() (src.colorama.ansitowin32.StreamWrapper method)}

\begin{fulllineitems}
\phantomsection\label{\detokenize{apidoc_src/src.colorama:src.colorama.ansitowin32.StreamWrapper.write}}\pysiglinewithargsret{\sphinxbfcode{\sphinxupquote{write}}}{\emph{text}}{}
\end{fulllineitems}


\end{fulllineitems}

\index{is\_a\_tty() (in module src.colorama.ansitowin32)}

\begin{fulllineitems}
\phantomsection\label{\detokenize{apidoc_src/src.colorama:src.colorama.ansitowin32.is_a_tty}}\pysiglinewithargsret{\sphinxcode{\sphinxupquote{src.colorama.ansitowin32.}}\sphinxbfcode{\sphinxupquote{is\_a\_tty}}}{\emph{stream}}{}
\end{fulllineitems}

\index{is\_stream\_closed() (in module src.colorama.ansitowin32)}

\begin{fulllineitems}
\phantomsection\label{\detokenize{apidoc_src/src.colorama:src.colorama.ansitowin32.is_stream_closed}}\pysiglinewithargsret{\sphinxcode{\sphinxupquote{src.colorama.ansitowin32.}}\sphinxbfcode{\sphinxupquote{is\_stream\_closed}}}{\emph{stream}}{}
\end{fulllineitems}



\subparagraph{src.colorama.initialise module}
\label{\detokenize{apidoc_src/src.colorama:src-colorama-initialise-module}}\label{\detokenize{apidoc_src/src.colorama:module-src.colorama.initialise}}\index{src.colorama.initialise (module)}\index{colorama\_text() (in module src.colorama.initialise)}

\begin{fulllineitems}
\phantomsection\label{\detokenize{apidoc_src/src.colorama:src.colorama.initialise.colorama_text}}\pysiglinewithargsret{\sphinxcode{\sphinxupquote{src.colorama.initialise.}}\sphinxbfcode{\sphinxupquote{colorama\_text}}}{\emph{*args}, \emph{**kwds}}{}
\end{fulllineitems}

\index{deinit() (in module src.colorama.initialise)}

\begin{fulllineitems}
\phantomsection\label{\detokenize{apidoc_src/src.colorama:src.colorama.initialise.deinit}}\pysiglinewithargsret{\sphinxcode{\sphinxupquote{src.colorama.initialise.}}\sphinxbfcode{\sphinxupquote{deinit}}}{}{}
\end{fulllineitems}

\index{init() (in module src.colorama.initialise)}

\begin{fulllineitems}
\phantomsection\label{\detokenize{apidoc_src/src.colorama:src.colorama.initialise.init}}\pysiglinewithargsret{\sphinxcode{\sphinxupquote{src.colorama.initialise.}}\sphinxbfcode{\sphinxupquote{init}}}{\emph{autoreset=False}, \emph{convert=None}, \emph{strip=None}, \emph{wrap=True}}{}
\end{fulllineitems}

\index{reinit() (in module src.colorama.initialise)}

\begin{fulllineitems}
\phantomsection\label{\detokenize{apidoc_src/src.colorama:src.colorama.initialise.reinit}}\pysiglinewithargsret{\sphinxcode{\sphinxupquote{src.colorama.initialise.}}\sphinxbfcode{\sphinxupquote{reinit}}}{}{}
\end{fulllineitems}

\index{reset\_all() (in module src.colorama.initialise)}

\begin{fulllineitems}
\phantomsection\label{\detokenize{apidoc_src/src.colorama:src.colorama.initialise.reset_all}}\pysiglinewithargsret{\sphinxcode{\sphinxupquote{src.colorama.initialise.}}\sphinxbfcode{\sphinxupquote{reset\_all}}}{}{}
\end{fulllineitems}

\index{wrap\_stream() (in module src.colorama.initialise)}

\begin{fulllineitems}
\phantomsection\label{\detokenize{apidoc_src/src.colorama:src.colorama.initialise.wrap_stream}}\pysiglinewithargsret{\sphinxcode{\sphinxupquote{src.colorama.initialise.}}\sphinxbfcode{\sphinxupquote{wrap\_stream}}}{\emph{stream}, \emph{convert}, \emph{strip}, \emph{autoreset}, \emph{wrap}}{}
\end{fulllineitems}



\subparagraph{src.colorama.win32 module}
\label{\detokenize{apidoc_src/src.colorama:src-colorama-win32-module}}\label{\detokenize{apidoc_src/src.colorama:module-src.colorama.win32}}\index{src.colorama.win32 (module)}\index{SetConsoleTextAttribute() (in module src.colorama.win32)}

\begin{fulllineitems}
\phantomsection\label{\detokenize{apidoc_src/src.colorama:src.colorama.win32.SetConsoleTextAttribute}}\pysiglinewithargsret{\sphinxcode{\sphinxupquote{src.colorama.win32.}}\sphinxbfcode{\sphinxupquote{SetConsoleTextAttribute}}}{\emph{*\_}}{}
\end{fulllineitems}

\index{winapi\_test() (in module src.colorama.win32)}

\begin{fulllineitems}
\phantomsection\label{\detokenize{apidoc_src/src.colorama:src.colorama.win32.winapi_test}}\pysiglinewithargsret{\sphinxcode{\sphinxupquote{src.colorama.win32.}}\sphinxbfcode{\sphinxupquote{winapi\_test}}}{\emph{*\_}}{}
\end{fulllineitems}



\subparagraph{src.colorama.winterm module}
\label{\detokenize{apidoc_src/src.colorama:module-src.colorama.winterm}}\label{\detokenize{apidoc_src/src.colorama:src-colorama-winterm-module}}\index{src.colorama.winterm (module)}\index{WinColor (class in src.colorama.winterm)}

\begin{fulllineitems}
\phantomsection\label{\detokenize{apidoc_src/src.colorama:src.colorama.winterm.WinColor}}\pysigline{\sphinxbfcode{\sphinxupquote{class }}\sphinxcode{\sphinxupquote{src.colorama.winterm.}}\sphinxbfcode{\sphinxupquote{WinColor}}}
Bases: \sphinxcode{\sphinxupquote{object}}
\index{BLACK (src.colorama.winterm.WinColor attribute)}

\begin{fulllineitems}
\phantomsection\label{\detokenize{apidoc_src/src.colorama:src.colorama.winterm.WinColor.BLACK}}\pysigline{\sphinxbfcode{\sphinxupquote{BLACK}}\sphinxbfcode{\sphinxupquote{ = 0}}}
\end{fulllineitems}

\index{BLUE (src.colorama.winterm.WinColor attribute)}

\begin{fulllineitems}
\phantomsection\label{\detokenize{apidoc_src/src.colorama:src.colorama.winterm.WinColor.BLUE}}\pysigline{\sphinxbfcode{\sphinxupquote{BLUE}}\sphinxbfcode{\sphinxupquote{ = 1}}}
\end{fulllineitems}

\index{CYAN (src.colorama.winterm.WinColor attribute)}

\begin{fulllineitems}
\phantomsection\label{\detokenize{apidoc_src/src.colorama:src.colorama.winterm.WinColor.CYAN}}\pysigline{\sphinxbfcode{\sphinxupquote{CYAN}}\sphinxbfcode{\sphinxupquote{ = 3}}}
\end{fulllineitems}

\index{GREEN (src.colorama.winterm.WinColor attribute)}

\begin{fulllineitems}
\phantomsection\label{\detokenize{apidoc_src/src.colorama:src.colorama.winterm.WinColor.GREEN}}\pysigline{\sphinxbfcode{\sphinxupquote{GREEN}}\sphinxbfcode{\sphinxupquote{ = 2}}}
\end{fulllineitems}

\index{GREY (src.colorama.winterm.WinColor attribute)}

\begin{fulllineitems}
\phantomsection\label{\detokenize{apidoc_src/src.colorama:src.colorama.winterm.WinColor.GREY}}\pysigline{\sphinxbfcode{\sphinxupquote{GREY}}\sphinxbfcode{\sphinxupquote{ = 7}}}
\end{fulllineitems}

\index{MAGENTA (src.colorama.winterm.WinColor attribute)}

\begin{fulllineitems}
\phantomsection\label{\detokenize{apidoc_src/src.colorama:src.colorama.winterm.WinColor.MAGENTA}}\pysigline{\sphinxbfcode{\sphinxupquote{MAGENTA}}\sphinxbfcode{\sphinxupquote{ = 5}}}
\end{fulllineitems}

\index{RED (src.colorama.winterm.WinColor attribute)}

\begin{fulllineitems}
\phantomsection\label{\detokenize{apidoc_src/src.colorama:src.colorama.winterm.WinColor.RED}}\pysigline{\sphinxbfcode{\sphinxupquote{RED}}\sphinxbfcode{\sphinxupquote{ = 4}}}
\end{fulllineitems}

\index{YELLOW (src.colorama.winterm.WinColor attribute)}

\begin{fulllineitems}
\phantomsection\label{\detokenize{apidoc_src/src.colorama:src.colorama.winterm.WinColor.YELLOW}}\pysigline{\sphinxbfcode{\sphinxupquote{YELLOW}}\sphinxbfcode{\sphinxupquote{ = 6}}}
\end{fulllineitems}


\end{fulllineitems}

\index{WinStyle (class in src.colorama.winterm)}

\begin{fulllineitems}
\phantomsection\label{\detokenize{apidoc_src/src.colorama:src.colorama.winterm.WinStyle}}\pysigline{\sphinxbfcode{\sphinxupquote{class }}\sphinxcode{\sphinxupquote{src.colorama.winterm.}}\sphinxbfcode{\sphinxupquote{WinStyle}}}
Bases: \sphinxcode{\sphinxupquote{object}}
\index{BRIGHT (src.colorama.winterm.WinStyle attribute)}

\begin{fulllineitems}
\phantomsection\label{\detokenize{apidoc_src/src.colorama:src.colorama.winterm.WinStyle.BRIGHT}}\pysigline{\sphinxbfcode{\sphinxupquote{BRIGHT}}\sphinxbfcode{\sphinxupquote{ = 8}}}
\end{fulllineitems}

\index{BRIGHT\_BACKGROUND (src.colorama.winterm.WinStyle attribute)}

\begin{fulllineitems}
\phantomsection\label{\detokenize{apidoc_src/src.colorama:src.colorama.winterm.WinStyle.BRIGHT_BACKGROUND}}\pysigline{\sphinxbfcode{\sphinxupquote{BRIGHT\_BACKGROUND}}\sphinxbfcode{\sphinxupquote{ = 128}}}
\end{fulllineitems}

\index{NORMAL (src.colorama.winterm.WinStyle attribute)}

\begin{fulllineitems}
\phantomsection\label{\detokenize{apidoc_src/src.colorama:src.colorama.winterm.WinStyle.NORMAL}}\pysigline{\sphinxbfcode{\sphinxupquote{NORMAL}}\sphinxbfcode{\sphinxupquote{ = 0}}}
\end{fulllineitems}


\end{fulllineitems}

\index{WinTerm (class in src.colorama.winterm)}

\begin{fulllineitems}
\phantomsection\label{\detokenize{apidoc_src/src.colorama:src.colorama.winterm.WinTerm}}\pysigline{\sphinxbfcode{\sphinxupquote{class }}\sphinxcode{\sphinxupquote{src.colorama.winterm.}}\sphinxbfcode{\sphinxupquote{WinTerm}}}
Bases: \sphinxcode{\sphinxupquote{object}}
\index{back() (src.colorama.winterm.WinTerm method)}

\begin{fulllineitems}
\phantomsection\label{\detokenize{apidoc_src/src.colorama:src.colorama.winterm.WinTerm.back}}\pysiglinewithargsret{\sphinxbfcode{\sphinxupquote{back}}}{\emph{back=None}, \emph{light=False}, \emph{on\_stderr=False}}{}
\end{fulllineitems}

\index{cursor\_adjust() (src.colorama.winterm.WinTerm method)}

\begin{fulllineitems}
\phantomsection\label{\detokenize{apidoc_src/src.colorama:src.colorama.winterm.WinTerm.cursor_adjust}}\pysiglinewithargsret{\sphinxbfcode{\sphinxupquote{cursor\_adjust}}}{\emph{x}, \emph{y}, \emph{on\_stderr=False}}{}
\end{fulllineitems}

\index{erase\_line() (src.colorama.winterm.WinTerm method)}

\begin{fulllineitems}
\phantomsection\label{\detokenize{apidoc_src/src.colorama:src.colorama.winterm.WinTerm.erase_line}}\pysiglinewithargsret{\sphinxbfcode{\sphinxupquote{erase\_line}}}{\emph{mode=0}, \emph{on\_stderr=False}}{}
\end{fulllineitems}

\index{erase\_screen() (src.colorama.winterm.WinTerm method)}

\begin{fulllineitems}
\phantomsection\label{\detokenize{apidoc_src/src.colorama:src.colorama.winterm.WinTerm.erase_screen}}\pysiglinewithargsret{\sphinxbfcode{\sphinxupquote{erase\_screen}}}{\emph{mode=0}, \emph{on\_stderr=False}}{}
\end{fulllineitems}

\index{fore() (src.colorama.winterm.WinTerm method)}

\begin{fulllineitems}
\phantomsection\label{\detokenize{apidoc_src/src.colorama:src.colorama.winterm.WinTerm.fore}}\pysiglinewithargsret{\sphinxbfcode{\sphinxupquote{fore}}}{\emph{fore=None}, \emph{light=False}, \emph{on\_stderr=False}}{}
\end{fulllineitems}

\index{get\_attrs() (src.colorama.winterm.WinTerm method)}

\begin{fulllineitems}
\phantomsection\label{\detokenize{apidoc_src/src.colorama:src.colorama.winterm.WinTerm.get_attrs}}\pysiglinewithargsret{\sphinxbfcode{\sphinxupquote{get\_attrs}}}{}{}
\end{fulllineitems}

\index{get\_position() (src.colorama.winterm.WinTerm method)}

\begin{fulllineitems}
\phantomsection\label{\detokenize{apidoc_src/src.colorama:src.colorama.winterm.WinTerm.get_position}}\pysiglinewithargsret{\sphinxbfcode{\sphinxupquote{get\_position}}}{\emph{handle}}{}
\end{fulllineitems}

\index{reset\_all() (src.colorama.winterm.WinTerm method)}

\begin{fulllineitems}
\phantomsection\label{\detokenize{apidoc_src/src.colorama:src.colorama.winterm.WinTerm.reset_all}}\pysiglinewithargsret{\sphinxbfcode{\sphinxupquote{reset\_all}}}{\emph{on\_stderr=None}}{}
\end{fulllineitems}

\index{set\_attrs() (src.colorama.winterm.WinTerm method)}

\begin{fulllineitems}
\phantomsection\label{\detokenize{apidoc_src/src.colorama:src.colorama.winterm.WinTerm.set_attrs}}\pysiglinewithargsret{\sphinxbfcode{\sphinxupquote{set\_attrs}}}{\emph{value}}{}
\end{fulllineitems}

\index{set\_console() (src.colorama.winterm.WinTerm method)}

\begin{fulllineitems}
\phantomsection\label{\detokenize{apidoc_src/src.colorama:src.colorama.winterm.WinTerm.set_console}}\pysiglinewithargsret{\sphinxbfcode{\sphinxupquote{set\_console}}}{\emph{attrs=None}, \emph{on\_stderr=False}}{}
\end{fulllineitems}

\index{set\_cursor\_position() (src.colorama.winterm.WinTerm method)}

\begin{fulllineitems}
\phantomsection\label{\detokenize{apidoc_src/src.colorama:src.colorama.winterm.WinTerm.set_cursor_position}}\pysiglinewithargsret{\sphinxbfcode{\sphinxupquote{set\_cursor\_position}}}{\emph{position=None}, \emph{on\_stderr=False}}{}
\end{fulllineitems}

\index{set\_title() (src.colorama.winterm.WinTerm method)}

\begin{fulllineitems}
\phantomsection\label{\detokenize{apidoc_src/src.colorama:src.colorama.winterm.WinTerm.set_title}}\pysiglinewithargsret{\sphinxbfcode{\sphinxupquote{set\_title}}}{\emph{title}}{}
\end{fulllineitems}

\index{style() (src.colorama.winterm.WinTerm method)}

\begin{fulllineitems}
\phantomsection\label{\detokenize{apidoc_src/src.colorama:src.colorama.winterm.WinTerm.style}}\pysiglinewithargsret{\sphinxbfcode{\sphinxupquote{style}}}{\emph{style=None}, \emph{on\_stderr=False}}{}
\end{fulllineitems}


\end{fulllineitems}



\subparagraph{Module contents}
\label{\detokenize{apidoc_src/src.colorama:module-src.colorama}}\label{\detokenize{apidoc_src/src.colorama:module-contents}}\index{src.colorama (module)}

\paragraph{src.example package}
\label{\detokenize{apidoc_src/src.example:src-example-package}}\label{\detokenize{apidoc_src/src.example::doc}}

\subparagraph{Submodules}
\label{\detokenize{apidoc_src/src.example:submodules}}

\subparagraph{src.example.essai\_logging\_1 module}
\label{\detokenize{apidoc_src/src.example:module-src.example.essai_logging_1}}\label{\detokenize{apidoc_src/src.example:src-example-essai-logging-1-module}}\index{src.example.essai\_logging\_1 (module)}
\sphinxurl{http://sametmax.com/ecrire-des-logs-en-python/}
\sphinxurl{https://docs.python.org/3/library/time.html\#time.strftime}

essai utilisation logger plusieurs handler format different
\begin{quote}

/usr/lib/python2.7/logging/\_\_init\_\_.pyc

init MyLogger, fmt=’\%(asctime)s :: \%(levelname)-8s :: \%(message)s’, level=‘20’

2018-03-11 18:51:21 :: INFO     :: test logger info
2018-03-11 18:51:21 :: WARNING  :: test logger warning
2018-03-11 18:51:21 :: ERROR    :: test logger error
2018-03-11 18:51:21 :: CRITICAL :: test logger critical

init MyLogger, fmt=’None’, level=‘10’

2018-03-11 18:51:21 :: DEBUG    :: test logger debug
test logger debug
2018-03-11 18:51:21 :: INFO     :: test logger info
test logger info
2018-03-11 18:51:21 :: WARNING  :: test logger warning
test logger warning
2018-03-11 18:51:21 :: ERROR    :: test logger error
test logger error
2018-03-11 18:51:21 :: CRITICAL :: test logger critical
test logger critical
\end{quote}
\index{getMyLogger() (in module src.example.essai\_logging\_1)}

\begin{fulllineitems}
\phantomsection\label{\detokenize{apidoc_src/src.example:src.example.essai_logging_1.getMyLogger}}\pysiglinewithargsret{\sphinxcode{\sphinxupquote{src.example.essai\_logging\_1.}}\sphinxbfcode{\sphinxupquote{getMyLogger}}}{}{}
\end{fulllineitems}

\index{initMyLogger() (in module src.example.essai\_logging\_1)}

\begin{fulllineitems}
\phantomsection\label{\detokenize{apidoc_src/src.example:src.example.essai_logging_1.initMyLogger}}\pysiglinewithargsret{\sphinxcode{\sphinxupquote{src.example.essai\_logging\_1.}}\sphinxbfcode{\sphinxupquote{initMyLogger}}}{\emph{fmt=None}, \emph{level=None}}{}
\end{fulllineitems}

\index{testLogger1() (in module src.example.essai\_logging\_1)}

\begin{fulllineitems}
\phantomsection\label{\detokenize{apidoc_src/src.example:src.example.essai_logging_1.testLogger1}}\pysiglinewithargsret{\sphinxcode{\sphinxupquote{src.example.essai\_logging\_1.}}\sphinxbfcode{\sphinxupquote{testLogger1}}}{}{}
\end{fulllineitems}



\subparagraph{src.example.essai\_logging\_2 module}
\label{\detokenize{apidoc_src/src.example:module-src.example.essai_logging_2}}\label{\detokenize{apidoc_src/src.example:src-example-essai-logging-2-module}}\index{src.example.essai\_logging\_2 (module)}
\sphinxurl{http://sametmax.com/ecrire-des-logs-en-python/}
\sphinxurl{https://docs.python.org/3/library/time.html\#time.strftime}

essai utilisation logger un handler format different 
sur info() pas de format et su other format
\begin{quote}

/usr/lib/python2.7/logging/\_\_init\_\_.pyc

init MyLogger, fmt=’\%(asctime)s :: \%(levelname)-8s :: \%(message)s’, level=‘20’

test logger info
2018-03-11 18:51:51 :: WARNING  :: test logger warning
2018-03-11 18:51:51 :: ERROR    :: test logger error
2018-03-11 18:51:51 :: CRITICAL :: test logger critical
\end{quote}
\index{MyFormatter (class in src.example.essai\_logging\_2)}

\begin{fulllineitems}
\phantomsection\label{\detokenize{apidoc_src/src.example:src.example.essai_logging_2.MyFormatter}}\pysiglinewithargsret{\sphinxbfcode{\sphinxupquote{class }}\sphinxcode{\sphinxupquote{src.example.essai\_logging\_2.}}\sphinxbfcode{\sphinxupquote{MyFormatter}}}{\emph{fmt=None}, \emph{datefmt=None}}{}
Bases: \sphinxcode{\sphinxupquote{logging.Formatter}}
\index{format() (src.example.essai\_logging\_2.MyFormatter method)}

\begin{fulllineitems}
\phantomsection\label{\detokenize{apidoc_src/src.example:src.example.essai_logging_2.MyFormatter.format}}\pysiglinewithargsret{\sphinxbfcode{\sphinxupquote{format}}}{\emph{record}}{}
Format the specified record as text.

The record’s attribute dictionary is used as the operand to a
string formatting operation which yields the returned string.
Before formatting the dictionary, a couple of preparatory steps
are carried out. The message attribute of the record is computed
using LogRecord.getMessage(). If the formatting string uses the
time (as determined by a call to usesTime(), formatTime() is
called to format the event time. If there is exception information,
it is formatted using formatException() and appended to the message.

\end{fulllineitems}


\end{fulllineitems}

\index{getMyLogger() (in module src.example.essai\_logging\_2)}

\begin{fulllineitems}
\phantomsection\label{\detokenize{apidoc_src/src.example:src.example.essai_logging_2.getMyLogger}}\pysiglinewithargsret{\sphinxcode{\sphinxupquote{src.example.essai\_logging\_2.}}\sphinxbfcode{\sphinxupquote{getMyLogger}}}{}{}
\end{fulllineitems}

\index{initMyLogger() (in module src.example.essai\_logging\_2)}

\begin{fulllineitems}
\phantomsection\label{\detokenize{apidoc_src/src.example:src.example.essai_logging_2.initMyLogger}}\pysiglinewithargsret{\sphinxcode{\sphinxupquote{src.example.essai\_logging\_2.}}\sphinxbfcode{\sphinxupquote{initMyLogger}}}{\emph{fmt=None}, \emph{level=None}}{}
\end{fulllineitems}

\index{testLogger1() (in module src.example.essai\_logging\_2)}

\begin{fulllineitems}
\phantomsection\label{\detokenize{apidoc_src/src.example:src.example.essai_logging_2.testLogger1}}\pysiglinewithargsret{\sphinxcode{\sphinxupquote{src.example.essai\_logging\_2.}}\sphinxbfcode{\sphinxupquote{testLogger1}}}{}{}
\end{fulllineitems}



\subparagraph{Module contents}
\label{\detokenize{apidoc_src/src.example:module-contents}}\label{\detokenize{apidoc_src/src.example:module-src.example}}\index{src.example (module)}

\subsubsection{Submodules}
\label{\detokenize{apidoc_src/src:submodules}}

\subsubsection{src.ElementTree module}
\label{\detokenize{apidoc_src/src:src-elementtree-module}}\label{\detokenize{apidoc_src/src:module-src.ElementTree}}\index{src.ElementTree (module)}\index{Comment() (in module src.ElementTree)}

\begin{fulllineitems}
\phantomsection\label{\detokenize{apidoc_src/src:src.ElementTree.Comment}}\pysiglinewithargsret{\sphinxcode{\sphinxupquote{src.ElementTree.}}\sphinxbfcode{\sphinxupquote{Comment}}}{\emph{text=None}}{}
\end{fulllineitems}

\index{dump() (in module src.ElementTree)}

\begin{fulllineitems}
\phantomsection\label{\detokenize{apidoc_src/src:src.ElementTree.dump}}\pysiglinewithargsret{\sphinxcode{\sphinxupquote{src.ElementTree.}}\sphinxbfcode{\sphinxupquote{dump}}}{\emph{elem}}{}
\end{fulllineitems}

\index{Element() (in module src.ElementTree)}

\begin{fulllineitems}
\phantomsection\label{\detokenize{apidoc_src/src:src.ElementTree.Element}}\pysiglinewithargsret{\sphinxcode{\sphinxupquote{src.ElementTree.}}\sphinxbfcode{\sphinxupquote{Element}}}{\emph{tag}, \emph{attrib=\{\}}, \emph{**extra}}{}
\end{fulllineitems}

\index{ElementTree (class in src.ElementTree)}

\begin{fulllineitems}
\phantomsection\label{\detokenize{apidoc_src/src:src.ElementTree.ElementTree}}\pysiglinewithargsret{\sphinxbfcode{\sphinxupquote{class }}\sphinxcode{\sphinxupquote{src.ElementTree.}}\sphinxbfcode{\sphinxupquote{ElementTree}}}{\emph{element=None}, \emph{file=None}}{}~\index{find() (src.ElementTree.ElementTree method)}

\begin{fulllineitems}
\phantomsection\label{\detokenize{apidoc_src/src:src.ElementTree.ElementTree.find}}\pysiglinewithargsret{\sphinxbfcode{\sphinxupquote{find}}}{\emph{path}}{}
\end{fulllineitems}

\index{findall() (src.ElementTree.ElementTree method)}

\begin{fulllineitems}
\phantomsection\label{\detokenize{apidoc_src/src:src.ElementTree.ElementTree.findall}}\pysiglinewithargsret{\sphinxbfcode{\sphinxupquote{findall}}}{\emph{path}}{}
\end{fulllineitems}

\index{findtext() (src.ElementTree.ElementTree method)}

\begin{fulllineitems}
\phantomsection\label{\detokenize{apidoc_src/src:src.ElementTree.ElementTree.findtext}}\pysiglinewithargsret{\sphinxbfcode{\sphinxupquote{findtext}}}{\emph{path}, \emph{default=None}}{}
\end{fulllineitems}

\index{getiterator() (src.ElementTree.ElementTree method)}

\begin{fulllineitems}
\phantomsection\label{\detokenize{apidoc_src/src:src.ElementTree.ElementTree.getiterator}}\pysiglinewithargsret{\sphinxbfcode{\sphinxupquote{getiterator}}}{\emph{tag=None}}{}
\end{fulllineitems}

\index{getroot() (src.ElementTree.ElementTree method)}

\begin{fulllineitems}
\phantomsection\label{\detokenize{apidoc_src/src:src.ElementTree.ElementTree.getroot}}\pysiglinewithargsret{\sphinxbfcode{\sphinxupquote{getroot}}}{}{}
\end{fulllineitems}

\index{parse() (src.ElementTree.ElementTree method)}

\begin{fulllineitems}
\phantomsection\label{\detokenize{apidoc_src/src:src.ElementTree.ElementTree.parse}}\pysiglinewithargsret{\sphinxbfcode{\sphinxupquote{parse}}}{\emph{source}, \emph{parser=None}}{}
\end{fulllineitems}

\index{write() (src.ElementTree.ElementTree method)}

\begin{fulllineitems}
\phantomsection\label{\detokenize{apidoc_src/src:src.ElementTree.ElementTree.write}}\pysiglinewithargsret{\sphinxbfcode{\sphinxupquote{write}}}{\emph{file}, \emph{encoding='us-ascii'}}{}
\end{fulllineitems}


\end{fulllineitems}

\index{fromstring() (in module src.ElementTree)}

\begin{fulllineitems}
\phantomsection\label{\detokenize{apidoc_src/src:src.ElementTree.fromstring}}\pysiglinewithargsret{\sphinxcode{\sphinxupquote{src.ElementTree.}}\sphinxbfcode{\sphinxupquote{fromstring}}}{\emph{text}}{}
\end{fulllineitems}

\index{iselement() (in module src.ElementTree)}

\begin{fulllineitems}
\phantomsection\label{\detokenize{apidoc_src/src:src.ElementTree.iselement}}\pysiglinewithargsret{\sphinxcode{\sphinxupquote{src.ElementTree.}}\sphinxbfcode{\sphinxupquote{iselement}}}{\emph{element}}{}
\end{fulllineitems}

\index{iterparse (class in src.ElementTree)}

\begin{fulllineitems}
\phantomsection\label{\detokenize{apidoc_src/src:src.ElementTree.iterparse}}\pysiglinewithargsret{\sphinxbfcode{\sphinxupquote{class }}\sphinxcode{\sphinxupquote{src.ElementTree.}}\sphinxbfcode{\sphinxupquote{iterparse}}}{\emph{source}, \emph{events=None}}{}~\index{next() (src.ElementTree.iterparse method)}

\begin{fulllineitems}
\phantomsection\label{\detokenize{apidoc_src/src:src.ElementTree.iterparse.next}}\pysiglinewithargsret{\sphinxbfcode{\sphinxupquote{next}}}{}{}
\end{fulllineitems}


\end{fulllineitems}

\index{parse() (in module src.ElementTree)}

\begin{fulllineitems}
\phantomsection\label{\detokenize{apidoc_src/src:src.ElementTree.parse}}\pysiglinewithargsret{\sphinxcode{\sphinxupquote{src.ElementTree.}}\sphinxbfcode{\sphinxupquote{parse}}}{\emph{source}, \emph{parser=None}}{}
\end{fulllineitems}

\index{PI() (in module src.ElementTree)}

\begin{fulllineitems}
\phantomsection\label{\detokenize{apidoc_src/src:src.ElementTree.PI}}\pysiglinewithargsret{\sphinxcode{\sphinxupquote{src.ElementTree.}}\sphinxbfcode{\sphinxupquote{PI}}}{\emph{target}, \emph{text=None}}{}
\end{fulllineitems}

\index{ProcessingInstruction() (in module src.ElementTree)}

\begin{fulllineitems}
\phantomsection\label{\detokenize{apidoc_src/src:src.ElementTree.ProcessingInstruction}}\pysiglinewithargsret{\sphinxcode{\sphinxupquote{src.ElementTree.}}\sphinxbfcode{\sphinxupquote{ProcessingInstruction}}}{\emph{target}, \emph{text=None}}{}
\end{fulllineitems}

\index{QName (class in src.ElementTree)}

\begin{fulllineitems}
\phantomsection\label{\detokenize{apidoc_src/src:src.ElementTree.QName}}\pysiglinewithargsret{\sphinxbfcode{\sphinxupquote{class }}\sphinxcode{\sphinxupquote{src.ElementTree.}}\sphinxbfcode{\sphinxupquote{QName}}}{\emph{text\_or\_uri}, \emph{tag=None}}{}
\end{fulllineitems}

\index{SubElement() (in module src.ElementTree)}

\begin{fulllineitems}
\phantomsection\label{\detokenize{apidoc_src/src:src.ElementTree.SubElement}}\pysiglinewithargsret{\sphinxcode{\sphinxupquote{src.ElementTree.}}\sphinxbfcode{\sphinxupquote{SubElement}}}{\emph{parent}, \emph{tag}, \emph{attrib=\{\}}, \emph{**extra}}{}
\end{fulllineitems}

\index{tostring() (in module src.ElementTree)}

\begin{fulllineitems}
\phantomsection\label{\detokenize{apidoc_src/src:src.ElementTree.tostring}}\pysiglinewithargsret{\sphinxcode{\sphinxupquote{src.ElementTree.}}\sphinxbfcode{\sphinxupquote{tostring}}}{\emph{element}, \emph{encoding=None}}{}
\end{fulllineitems}

\index{TreeBuilder (class in src.ElementTree)}

\begin{fulllineitems}
\phantomsection\label{\detokenize{apidoc_src/src:src.ElementTree.TreeBuilder}}\pysiglinewithargsret{\sphinxbfcode{\sphinxupquote{class }}\sphinxcode{\sphinxupquote{src.ElementTree.}}\sphinxbfcode{\sphinxupquote{TreeBuilder}}}{\emph{element\_factory=None}}{}~\index{close() (src.ElementTree.TreeBuilder method)}

\begin{fulllineitems}
\phantomsection\label{\detokenize{apidoc_src/src:src.ElementTree.TreeBuilder.close}}\pysiglinewithargsret{\sphinxbfcode{\sphinxupquote{close}}}{}{}
\end{fulllineitems}

\index{data() (src.ElementTree.TreeBuilder method)}

\begin{fulllineitems}
\phantomsection\label{\detokenize{apidoc_src/src:src.ElementTree.TreeBuilder.data}}\pysiglinewithargsret{\sphinxbfcode{\sphinxupquote{data}}}{\emph{data}}{}
\end{fulllineitems}

\index{end() (src.ElementTree.TreeBuilder method)}

\begin{fulllineitems}
\phantomsection\label{\detokenize{apidoc_src/src:src.ElementTree.TreeBuilder.end}}\pysiglinewithargsret{\sphinxbfcode{\sphinxupquote{end}}}{\emph{tag}}{}
\end{fulllineitems}

\index{start() (src.ElementTree.TreeBuilder method)}

\begin{fulllineitems}
\phantomsection\label{\detokenize{apidoc_src/src:src.ElementTree.TreeBuilder.start}}\pysiglinewithargsret{\sphinxbfcode{\sphinxupquote{start}}}{\emph{tag}, \emph{attrs}}{}
\end{fulllineitems}


\end{fulllineitems}

\index{XML() (in module src.ElementTree)}

\begin{fulllineitems}
\phantomsection\label{\detokenize{apidoc_src/src:src.ElementTree.XML}}\pysiglinewithargsret{\sphinxcode{\sphinxupquote{src.ElementTree.}}\sphinxbfcode{\sphinxupquote{XML}}}{\emph{text}}{}
\end{fulllineitems}

\index{XMLTreeBuilder (class in src.ElementTree)}

\begin{fulllineitems}
\phantomsection\label{\detokenize{apidoc_src/src:src.ElementTree.XMLTreeBuilder}}\pysiglinewithargsret{\sphinxbfcode{\sphinxupquote{class }}\sphinxcode{\sphinxupquote{src.ElementTree.}}\sphinxbfcode{\sphinxupquote{XMLTreeBuilder}}}{\emph{html=0}, \emph{target=None}}{}~\index{close() (src.ElementTree.XMLTreeBuilder method)}

\begin{fulllineitems}
\phantomsection\label{\detokenize{apidoc_src/src:src.ElementTree.XMLTreeBuilder.close}}\pysiglinewithargsret{\sphinxbfcode{\sphinxupquote{close}}}{}{}
\end{fulllineitems}

\index{doctype() (src.ElementTree.XMLTreeBuilder method)}

\begin{fulllineitems}
\phantomsection\label{\detokenize{apidoc_src/src:src.ElementTree.XMLTreeBuilder.doctype}}\pysiglinewithargsret{\sphinxbfcode{\sphinxupquote{doctype}}}{\emph{name}, \emph{pubid}, \emph{system}}{}
\end{fulllineitems}

\index{feed() (src.ElementTree.XMLTreeBuilder method)}

\begin{fulllineitems}
\phantomsection\label{\detokenize{apidoc_src/src:src.ElementTree.XMLTreeBuilder.feed}}\pysiglinewithargsret{\sphinxbfcode{\sphinxupquote{feed}}}{\emph{data}}{}
\end{fulllineitems}


\end{fulllineitems}



\subsubsection{src.architecture module}
\label{\detokenize{apidoc_src/src:module-src.architecture}}\label{\detokenize{apidoc_src/src:src-architecture-module}}\index{src.architecture (module)}
Contains all the stuff that can change with the architecture 
on which SAT is running
\index{get\_distrib\_version() (in module src.architecture)}

\begin{fulllineitems}
\phantomsection\label{\detokenize{apidoc_src/src:src.architecture.get_distrib_version}}\pysiglinewithargsret{\sphinxcode{\sphinxupquote{src.architecture.}}\sphinxbfcode{\sphinxupquote{get\_distrib\_version}}}{\emph{distrib}, \emph{codes}}{}
Gets the version of the distribution
\begin{quote}\begin{description}
\item[{Parameters}] \leavevmode\begin{itemize}
\item {} 
\sphinxstyleliteralstrong{\sphinxupquote{distrib}} \textendash{} (str) 
The distribution on which the version will be found.

\item {} 
\sphinxstyleliteralstrong{\sphinxupquote{codes}} \textendash{} (L\{Mapping\}) 
The map containing distribution correlation table.

\end{itemize}

\item[{Returns}] \leavevmode
(str)
The version of the distribution on which 
salomeTools is running, regarding the distribution 
correlation table contained in codes variable.

\end{description}\end{quote}

\end{fulllineitems}

\index{get\_distribution() (in module src.architecture)}

\begin{fulllineitems}
\phantomsection\label{\detokenize{apidoc_src/src:src.architecture.get_distribution}}\pysiglinewithargsret{\sphinxcode{\sphinxupquote{src.architecture.}}\sphinxbfcode{\sphinxupquote{get\_distribution}}}{\emph{codes}}{}
Gets the code for the distribution
\begin{quote}\begin{description}
\item[{Parameters}] \leavevmode
\sphinxstyleliteralstrong{\sphinxupquote{codes}} \textendash{} (L\{Mapping\}) 
The map containing distribution correlation table.

\item[{Returns}] \leavevmode
(str)
The distribution on which salomeTools is running, regarding the 
distribution correlation table contained in codes variable.

\end{description}\end{quote}

\end{fulllineitems}

\index{get\_nb\_proc() (in module src.architecture)}

\begin{fulllineitems}
\phantomsection\label{\detokenize{apidoc_src/src:src.architecture.get_nb_proc}}\pysiglinewithargsret{\sphinxcode{\sphinxupquote{src.architecture.}}\sphinxbfcode{\sphinxupquote{get\_nb\_proc}}}{}{}
Gets the number of processors of the machine 
on which salomeTools is running.
\begin{quote}\begin{description}
\item[{Returns}] \leavevmode
(str) The number of processors.

\end{description}\end{quote}

\end{fulllineitems}

\index{get\_python\_version() (in module src.architecture)}

\begin{fulllineitems}
\phantomsection\label{\detokenize{apidoc_src/src:src.architecture.get_python_version}}\pysiglinewithargsret{\sphinxcode{\sphinxupquote{src.architecture.}}\sphinxbfcode{\sphinxupquote{get\_python\_version}}}{}{}
Gets the version of the running python.
\begin{quote}\begin{description}
\item[{Returns}] \leavevmode
(str) The version of the running python.

\end{description}\end{quote}

\end{fulllineitems}

\index{get\_user() (in module src.architecture)}

\begin{fulllineitems}
\phantomsection\label{\detokenize{apidoc_src/src:src.architecture.get_user}}\pysiglinewithargsret{\sphinxcode{\sphinxupquote{src.architecture.}}\sphinxbfcode{\sphinxupquote{get\_user}}}{}{}
Gets the username that launched sat
\begin{quote}\begin{description}
\item[{Returns}] \leavevmode
(str) environ var USERNAME

\end{description}\end{quote}

\end{fulllineitems}

\index{is\_windows() (in module src.architecture)}

\begin{fulllineitems}
\phantomsection\label{\detokenize{apidoc_src/src:src.architecture.is_windows}}\pysiglinewithargsret{\sphinxcode{\sphinxupquote{src.architecture.}}\sphinxbfcode{\sphinxupquote{is\_windows}}}{}{}
Checks windows OS
\begin{quote}\begin{description}
\item[{Returns}] \leavevmode
(bool) True if system is Windows

\end{description}\end{quote}

\end{fulllineitems}



\subsubsection{src.catchAll module}
\label{\detokenize{apidoc_src/src:module-src.catchAll}}\label{\detokenize{apidoc_src/src:src-catchall-module}}\index{src.catchAll (module)}
define class as a simple dictionary with keys
with pretty print \_\_str\_\_ and \_\_repr\_\_ (indented as recursive)
and jsonDumps()
\begin{description}
\item[{usage:}] \leavevmode
\textgreater{}\textgreater{} import catchAll as CAA
\textgreater{}\textgreater{} a = CAA.CatchAll()
\textgreater{}\textgreater{} a.tintin = “reporter”
\textgreater{}\textgreater{} a.milou = “dog”
\textgreater{}\textgreater{} print(“a=\%s” \% a)
\textgreater{}\textgreater{} print(“tintin: \%s” \% a.tintin)

\end{description}
\index{CatchAll (class in src.catchAll)}

\begin{fulllineitems}
\phantomsection\label{\detokenize{apidoc_src/src:src.catchAll.CatchAll}}\pysigline{\sphinxbfcode{\sphinxupquote{class }}\sphinxcode{\sphinxupquote{src.catchAll.}}\sphinxbfcode{\sphinxupquote{CatchAll}}}
Bases: \sphinxcode{\sphinxupquote{object}}

class as simple dynamic dictionary 
with predefined keys as properties in
inherited classes through \_\_init\_\_ method. Or NOT. 
with pretty print \_\_str\_\_ and \_\_repr\_\_ (indented as recursive)
with jsonDumps()

usage:

\textgreater{}\textgreater{} import catchAll as CAA
\textgreater{}\textgreater{} a = CAA.CatchAll()
\textgreater{}\textgreater{} a.tintin = “reporter”
\textgreater{}\textgreater{} a.milou = “dog”
\textgreater{}\textgreater{} print(“a=\%s” \% a)
\textgreater{}\textgreater{} print(“tintin: \%s” \% a.tintin)

as

\textgreater{}\textgreater{} a = \{\}
\textgreater{}\textgreater{} a{[}“tintin”{]} = “reporter”
\textgreater{}\textgreater{} a{[}“milou”{]} = “dog”
\textgreater{}\textgreater{} print(“tintin: \%s” \% a{[}“tintin”{]}
\index{jsonDumps() (src.catchAll.CatchAll method)}

\begin{fulllineitems}
\phantomsection\label{\detokenize{apidoc_src/src:src.catchAll.CatchAll.jsonDumps}}\pysiglinewithargsret{\sphinxbfcode{\sphinxupquote{jsonDumps}}}{}{}
\end{fulllineitems}


\end{fulllineitems}

\index{dumper() (in module src.catchAll)}

\begin{fulllineitems}
\phantomsection\label{\detokenize{apidoc_src/src:src.catchAll.dumper}}\pysiglinewithargsret{\sphinxcode{\sphinxupquote{src.catchAll.}}\sphinxbfcode{\sphinxupquote{dumper}}}{\emph{obj}}{}
to json explore subclass object as dict

\end{fulllineitems}

\index{dumperType() (in module src.catchAll)}

\begin{fulllineitems}
\phantomsection\label{\detokenize{apidoc_src/src:src.catchAll.dumperType}}\pysiglinewithargsret{\sphinxcode{\sphinxupquote{src.catchAll.}}\sphinxbfcode{\sphinxupquote{dumperType}}}{\emph{obj}}{}
to get a “\_type” to trace json subclass object,
but ignore all attributes begining with ‘\_’

\end{fulllineitems}

\index{jsonDumps() (in module src.catchAll)}

\begin{fulllineitems}
\phantomsection\label{\detokenize{apidoc_src/src:src.catchAll.jsonDumps}}\pysiglinewithargsret{\sphinxcode{\sphinxupquote{src.catchAll.}}\sphinxbfcode{\sphinxupquote{jsonDumps}}}{\emph{obj}}{}
to get direct default jsonDumps method

\end{fulllineitems}



\subsubsection{src.coloringSat module}
\label{\detokenize{apidoc_src/src:module-src.coloringSat}}\label{\detokenize{apidoc_src/src:src-coloringsat-module}}\index{src.coloringSat (module)}
simple tagging as ‘\textless{}color\textgreater{}’ for simple coloring log messages on terminal(s)
window or unix or ios using backend colorama

using ‘\textless{}color\textgreater{}’ because EZ human readable
so ‘\textless{}color\textgreater{}’ are not supposed existing in log message
“\{\}”.format() is not choosen because “\{\}” are present
in log messages of contents of python dict (as JSON) etc.

usage:
\textgreater{}\textgreater{} import src.coloringSat as COLS

example:
\textgreater{}\textgreater{} log(“this is in \textless{}green\textgreater{}color green\textless{}reset\textgreater{}, OK is in blue: \textless{}blue\textgreater{}OK?”)
\index{ColoringStream (class in src.coloringSat)}

\begin{fulllineitems}
\phantomsection\label{\detokenize{apidoc_src/src:src.coloringSat.ColoringStream}}\pysigline{\sphinxbfcode{\sphinxupquote{class }}\sphinxcode{\sphinxupquote{src.coloringSat.}}\sphinxbfcode{\sphinxupquote{ColoringStream}}}
Bases: \sphinxcode{\sphinxupquote{object}}

write my stream class
only write and flush are used for the streaming
\sphinxurl{https://docs.python.org/2/library/logging.handlers.html}
\sphinxurl{https://stackoverflow.com/questions/31999627/storing-logger-messages-in-a-string}
\index{flush() (src.coloringSat.ColoringStream method)}

\begin{fulllineitems}
\phantomsection\label{\detokenize{apidoc_src/src:src.coloringSat.ColoringStream.flush}}\pysiglinewithargsret{\sphinxbfcode{\sphinxupquote{flush}}}{}{}
\end{fulllineitems}

\index{write() (src.coloringSat.ColoringStream method)}

\begin{fulllineitems}
\phantomsection\label{\detokenize{apidoc_src/src:src.coloringSat.ColoringStream.write}}\pysiglinewithargsret{\sphinxbfcode{\sphinxupquote{write}}}{\emph{astr}}{}
\end{fulllineitems}


\end{fulllineitems}

\index{cleanColors() (in module src.coloringSat)}

\begin{fulllineitems}
\phantomsection\label{\detokenize{apidoc_src/src:src.coloringSat.cleanColors}}\pysiglinewithargsret{\sphinxcode{\sphinxupquote{src.coloringSat.}}\sphinxbfcode{\sphinxupquote{cleanColors}}}{\emph{msg}}{}
clean the message of color tags ‘\textless{}red\textgreater{} …

\end{fulllineitems}

\index{indent() (in module src.coloringSat)}

\begin{fulllineitems}
\phantomsection\label{\detokenize{apidoc_src/src:src.coloringSat.indent}}\pysiglinewithargsret{\sphinxcode{\sphinxupquote{src.coloringSat.}}\sphinxbfcode{\sphinxupquote{indent}}}{\emph{msg}, \emph{nb}, \emph{car=' '}}{}
indent nb car (spaces) multi lines message except first one

\end{fulllineitems}

\index{log() (in module src.coloringSat)}

\begin{fulllineitems}
\phantomsection\label{\detokenize{apidoc_src/src:src.coloringSat.log}}\pysiglinewithargsret{\sphinxcode{\sphinxupquote{src.coloringSat.}}\sphinxbfcode{\sphinxupquote{log}}}{\emph{msg}}{}
elementary log stdout for debug if \_verbose

\end{fulllineitems}

\index{replace() (in module src.coloringSat)}

\begin{fulllineitems}
\phantomsection\label{\detokenize{apidoc_src/src:src.coloringSat.replace}}\pysiglinewithargsret{\sphinxcode{\sphinxupquote{src.coloringSat.}}\sphinxbfcode{\sphinxupquote{replace}}}{\emph{msg}, \emph{tags}}{}
\end{fulllineitems}

\index{toColor() (in module src.coloringSat)}

\begin{fulllineitems}
\phantomsection\label{\detokenize{apidoc_src/src:src.coloringSat.toColor}}\pysiglinewithargsret{\sphinxcode{\sphinxupquote{src.coloringSat.}}\sphinxbfcode{\sphinxupquote{toColor}}}{\emph{msg}}{}
automatically clean the message of color tags ‘\textless{}red\textgreater{} … 
if the terminal output stdout is redirected by user
if not, replace tags with ansi color codes

example:
\textgreater{}\textgreater{} sat compile SALOME \textgreater{} log.txt

\end{fulllineitems}

\index{toColor\_AnsiToWin32() (in module src.coloringSat)}

\begin{fulllineitems}
\phantomsection\label{\detokenize{apidoc_src/src:src.coloringSat.toColor_AnsiToWin32}}\pysiglinewithargsret{\sphinxcode{\sphinxupquote{src.coloringSat.}}\sphinxbfcode{\sphinxupquote{toColor\_AnsiToWin32}}}{\emph{msg}}{}
for test debug no wrapping

\end{fulllineitems}



\subsubsection{src.compilation module}
\label{\detokenize{apidoc_src/src:module-src.compilation}}\label{\detokenize{apidoc_src/src:src-compilation-module}}\index{src.compilation (module)}\index{Builder (class in src.compilation)}

\begin{fulllineitems}
\phantomsection\label{\detokenize{apidoc_src/src:src.compilation.Builder}}\pysiglinewithargsret{\sphinxbfcode{\sphinxupquote{class }}\sphinxcode{\sphinxupquote{src.compilation.}}\sphinxbfcode{\sphinxupquote{Builder}}}{\emph{config}, \emph{logger}, \emph{product\_info}, \emph{options=OptResult(   )}, \emph{check\_src=True}}{}
Class to handle all construction steps, like cmake, configure, make, …
\index{build\_configure() (src.compilation.Builder method)}

\begin{fulllineitems}
\phantomsection\label{\detokenize{apidoc_src/src:src.compilation.Builder.build_configure}}\pysiglinewithargsret{\sphinxbfcode{\sphinxupquote{build\_configure}}}{\emph{options=''}}{}
\end{fulllineitems}

\index{check() (src.compilation.Builder method)}

\begin{fulllineitems}
\phantomsection\label{\detokenize{apidoc_src/src:src.compilation.Builder.check}}\pysiglinewithargsret{\sphinxbfcode{\sphinxupquote{check}}}{\emph{command=''}}{}
\end{fulllineitems}

\index{cmake() (src.compilation.Builder method)}

\begin{fulllineitems}
\phantomsection\label{\detokenize{apidoc_src/src:src.compilation.Builder.cmake}}\pysiglinewithargsret{\sphinxbfcode{\sphinxupquote{cmake}}}{\emph{options=''}}{}
\end{fulllineitems}

\index{complete\_environment() (src.compilation.Builder method)}

\begin{fulllineitems}
\phantomsection\label{\detokenize{apidoc_src/src:src.compilation.Builder.complete_environment}}\pysiglinewithargsret{\sphinxbfcode{\sphinxupquote{complete\_environment}}}{\emph{make\_options}}{}
\end{fulllineitems}

\index{configure() (src.compilation.Builder method)}

\begin{fulllineitems}
\phantomsection\label{\detokenize{apidoc_src/src:src.compilation.Builder.configure}}\pysiglinewithargsret{\sphinxbfcode{\sphinxupquote{configure}}}{\emph{options=''}}{}
\end{fulllineitems}

\index{do\_batch\_script\_build() (src.compilation.Builder method)}

\begin{fulllineitems}
\phantomsection\label{\detokenize{apidoc_src/src:src.compilation.Builder.do_batch_script_build}}\pysiglinewithargsret{\sphinxbfcode{\sphinxupquote{do\_batch\_script\_build}}}{\emph{script}, \emph{nb\_proc}}{}
\end{fulllineitems}

\index{do\_default\_build() (src.compilation.Builder method)}

\begin{fulllineitems}
\phantomsection\label{\detokenize{apidoc_src/src:src.compilation.Builder.do_default_build}}\pysiglinewithargsret{\sphinxbfcode{\sphinxupquote{do\_default\_build}}}{\emph{build\_conf\_options=''}, \emph{configure\_options=''}, \emph{show\_warning=True}}{}
\end{fulllineitems}

\index{do\_python\_script\_build() (src.compilation.Builder method)}

\begin{fulllineitems}
\phantomsection\label{\detokenize{apidoc_src/src:src.compilation.Builder.do_python_script_build}}\pysiglinewithargsret{\sphinxbfcode{\sphinxupquote{do\_python\_script\_build}}}{\emph{script}, \emph{nb\_proc}}{}
Performs a build with a script.

\end{fulllineitems}

\index{do\_script\_build() (src.compilation.Builder method)}

\begin{fulllineitems}
\phantomsection\label{\detokenize{apidoc_src/src:src.compilation.Builder.do_script_build}}\pysiglinewithargsret{\sphinxbfcode{\sphinxupquote{do\_script\_build}}}{\emph{script}, \emph{number\_of\_proc=0}}{}
\end{fulllineitems}

\index{hack\_libtool() (src.compilation.Builder method)}

\begin{fulllineitems}
\phantomsection\label{\detokenize{apidoc_src/src:src.compilation.Builder.hack_libtool}}\pysiglinewithargsret{\sphinxbfcode{\sphinxupquote{hack\_libtool}}}{}{}
\end{fulllineitems}

\index{install() (src.compilation.Builder method)}

\begin{fulllineitems}
\phantomsection\label{\detokenize{apidoc_src/src:src.compilation.Builder.install}}\pysiglinewithargsret{\sphinxbfcode{\sphinxupquote{install}}}{}{}
\end{fulllineitems}

\index{log() (src.compilation.Builder method)}

\begin{fulllineitems}
\phantomsection\label{\detokenize{apidoc_src/src:src.compilation.Builder.log}}\pysiglinewithargsret{\sphinxbfcode{\sphinxupquote{log}}}{\emph{text}, \emph{level}, \emph{showInfo=True}}{}
Shortcut method to log in log file.

\end{fulllineitems}

\index{log\_command() (src.compilation.Builder method)}

\begin{fulllineitems}
\phantomsection\label{\detokenize{apidoc_src/src:src.compilation.Builder.log_command}}\pysiglinewithargsret{\sphinxbfcode{\sphinxupquote{log\_command}}}{\emph{command}}{}
Shortcut method to log a command.

\end{fulllineitems}

\index{make() (src.compilation.Builder method)}

\begin{fulllineitems}
\phantomsection\label{\detokenize{apidoc_src/src:src.compilation.Builder.make}}\pysiglinewithargsret{\sphinxbfcode{\sphinxupquote{make}}}{\emph{nb\_proc}, \emph{make\_opt=''}}{}
\end{fulllineitems}

\index{prepare() (src.compilation.Builder method)}

\begin{fulllineitems}
\phantomsection\label{\detokenize{apidoc_src/src:src.compilation.Builder.prepare}}\pysiglinewithargsret{\sphinxbfcode{\sphinxupquote{prepare}}}{}{}
Prepares the environment.
Build two environment: one for building and one for testing (launch).

\end{fulllineitems}

\index{put\_txt\_log\_in\_appli\_log\_dir() (src.compilation.Builder method)}

\begin{fulllineitems}
\phantomsection\label{\detokenize{apidoc_src/src:src.compilation.Builder.put_txt_log_in_appli_log_dir}}\pysiglinewithargsret{\sphinxbfcode{\sphinxupquote{put\_txt\_log\_in\_appli\_log\_dir}}}{\emph{file\_name}}{}
Put the txt log (that contain the system logs, like make command output)
in the directory \textless{}APPLICATION DIR\textgreater{}/LOGS/\textless{}product\_name\textgreater{}/
\begin{quote}\begin{description}
\item[{Parameters}] \leavevmode
\sphinxstyleliteralstrong{\sphinxupquote{file\_name}} \textendash{} (str) The name of the file to write

\end{description}\end{quote}

\end{fulllineitems}

\index{wmake() (src.compilation.Builder method)}

\begin{fulllineitems}
\phantomsection\label{\detokenize{apidoc_src/src:src.compilation.Builder.wmake}}\pysiglinewithargsret{\sphinxbfcode{\sphinxupquote{wmake}}}{\emph{nb\_proc}, \emph{opt\_nb\_proc=None}}{}
\end{fulllineitems}


\end{fulllineitems}



\subsubsection{src.configManager module}
\label{\detokenize{apidoc_src/src:module-src.configManager}}\label{\detokenize{apidoc_src/src:src-configmanager-module}}\index{src.configManager (module)}\index{ConfigManager (class in src.configManager)}

\begin{fulllineitems}
\phantomsection\label{\detokenize{apidoc_src/src:src.configManager.ConfigManager}}\pysiglinewithargsret{\sphinxbfcode{\sphinxupquote{class }}\sphinxcode{\sphinxupquote{src.configManager.}}\sphinxbfcode{\sphinxupquote{ConfigManager}}}{\emph{runner}}{}
Class that manages the read of all the config .pyconf files of salomeTools
\index{create\_config\_file() (src.configManager.ConfigManager method)}

\begin{fulllineitems}
\phantomsection\label{\detokenize{apidoc_src/src:src.configManager.ConfigManager.create_config_file}}\pysiglinewithargsret{\sphinxbfcode{\sphinxupquote{create\_config\_file}}}{\emph{config}}{}
This method is called when there are no user config file. 
It build it from scratch.
\begin{quote}\begin{description}
\item[{Parameters}] \leavevmode
\sphinxstyleliteralstrong{\sphinxupquote{config}} \textendash{} (Config) The global config.

\item[{Returns}] \leavevmode
(Config) 
The config corresponding to the file created.

\end{description}\end{quote}

\end{fulllineitems}

\index{get\_command\_line\_overrides() (src.configManager.ConfigManager method)}

\begin{fulllineitems}
\phantomsection\label{\detokenize{apidoc_src/src:src.configManager.ConfigManager.get_command_line_overrides}}\pysiglinewithargsret{\sphinxbfcode{\sphinxupquote{get\_command\_line\_overrides}}}{\emph{options}, \emph{sections}}{}
get all the overwrites that are in the command line
\begin{quote}\begin{description}
\item[{Parameters}] \leavevmode\begin{itemize}
\item {} 
\sphinxstyleliteralstrong{\sphinxupquote{options}} \textendash{} The options from salomeTools class initialization 
(as ‘-l5’ or ‘\textendash{}overwrite’)

\item {} 
\sphinxstyleliteralstrong{\sphinxupquote{sections}} \textendash{} (str) The config section to overwrite.

\end{itemize}

\item[{Returns}] \leavevmode
(list) The list of all the overwrites to apply.

\end{description}\end{quote}

\end{fulllineitems}

\index{get\_config() (src.configManager.ConfigManager method)}

\begin{fulllineitems}
\phantomsection\label{\detokenize{apidoc_src/src:src.configManager.ConfigManager.get_config}}\pysiglinewithargsret{\sphinxbfcode{\sphinxupquote{get\_config}}}{\emph{application=None}, \emph{options=None}, \emph{command=None}, \emph{datadir=None}}{}
get the config from all the configuration files.
\begin{quote}\begin{description}
\item[{Parameters}] \leavevmode\begin{itemize}
\item {} 
\sphinxstyleliteralstrong{\sphinxupquote{application}} \textendash{} (str) 
The application for which salomeTools is called.

\item {} 
\sphinxstyleliteralstrong{\sphinxupquote{options}} \textendash{} (Options) 
The general salomeTools options 
(as ‘\textendash{}overwrite’ or ‘-v5’)

\item {} 
\sphinxstyleliteralstrong{\sphinxupquote{command}} \textendash{} (str) The command that is called.

\item {} 
\sphinxstyleliteralstrong{\sphinxupquote{datadir}} \textendash{} (str) 
The repository that contain external data for salomeTools.

\end{itemize}

\item[{Returns}] \leavevmode
(Config) The final config.

\end{description}\end{quote}

\end{fulllineitems}

\index{get\_user\_config\_file() (src.configManager.ConfigManager method)}

\begin{fulllineitems}
\phantomsection\label{\detokenize{apidoc_src/src:src.configManager.ConfigManager.get_user_config_file}}\pysiglinewithargsret{\sphinxbfcode{\sphinxupquote{get\_user\_config\_file}}}{}{}
Get the user config file
\begin{quote}\begin{description}
\item[{Returns}] \leavevmode
(str) path to the user config file.

\end{description}\end{quote}

\end{fulllineitems}

\index{set\_user\_config\_file() (src.configManager.ConfigManager method)}

\begin{fulllineitems}
\phantomsection\label{\detokenize{apidoc_src/src:src.configManager.ConfigManager.set_user_config_file}}\pysiglinewithargsret{\sphinxbfcode{\sphinxupquote{set\_user\_config\_file}}}{\emph{config}}{}
Set the user config file name and path.
If necessary, build it from another one or create it from scratch.
\begin{quote}\begin{description}
\item[{Parameters}] \leavevmode
\sphinxstyleliteralstrong{\sphinxupquote{config}} \textendash{} (Config) 
The global config (containing all pyconf).

\end{description}\end{quote}

\end{fulllineitems}


\end{fulllineitems}

\index{ConfigOpener (class in src.configManager)}

\begin{fulllineitems}
\phantomsection\label{\detokenize{apidoc_src/src:src.configManager.ConfigOpener}}\pysiglinewithargsret{\sphinxbfcode{\sphinxupquote{class }}\sphinxcode{\sphinxupquote{src.configManager.}}\sphinxbfcode{\sphinxupquote{ConfigOpener}}}{\emph{pathList}}{}
Class that helps to find an application pyconf 
in all the possible directories (pathList)
\index{get\_path() (src.configManager.ConfigOpener method)}

\begin{fulllineitems}
\phantomsection\label{\detokenize{apidoc_src/src:src.configManager.ConfigOpener.get_path}}\pysiglinewithargsret{\sphinxbfcode{\sphinxupquote{get\_path}}}{\emph{name}}{}
The method that returns the entire path of the pyconf searched
\begin{quote}\begin{description}
\item[{Parameters}] \leavevmode
\sphinxstyleliteralstrong{\sphinxupquote{name}} \textendash{} (str) The name of the searched pyconf.

\end{description}\end{quote}

\end{fulllineitems}


\end{fulllineitems}

\index{check\_path() (in module src.configManager)}

\begin{fulllineitems}
\phantomsection\label{\detokenize{apidoc_src/src:src.configManager.check_path}}\pysiglinewithargsret{\sphinxcode{\sphinxupquote{src.configManager.}}\sphinxbfcode{\sphinxupquote{check\_path}}}{\emph{path}, \emph{ext={[}{]}}}{}
Construct a text with the input path and “not found” if it does not exist.
\begin{quote}\begin{description}
\item[{Parameters}] \leavevmode\begin{itemize}
\item {} 
\sphinxstyleliteralstrong{\sphinxupquote{path}} \textendash{} (str) The path to check.

\item {} 
\sphinxstyleliteralstrong{\sphinxupquote{ext}} \textendash{} (list) 
An extension. Verify that the path extension is in the list

\end{itemize}

\item[{Returns}] \leavevmode
(str) The string of the path with information

\end{description}\end{quote}

\end{fulllineitems}

\index{getConfigColored() (in module src.configManager)}

\begin{fulllineitems}
\phantomsection\label{\detokenize{apidoc_src/src:src.configManager.getConfigColored}}\pysiglinewithargsret{\sphinxcode{\sphinxupquote{src.configManager.}}\sphinxbfcode{\sphinxupquote{getConfigColored}}}{\emph{config}, \emph{path}, \emph{stream}, \emph{show\_label=False}, \emph{level=0}, \emph{show\_full\_path=False}}{}
Get a colored representation value from a config pyconf instance.
used recursively from the initial path.
\begin{quote}\begin{description}
\item[{Parameters}] \leavevmode\begin{itemize}
\item {} 
\sphinxstyleliteralstrong{\sphinxupquote{config}} \textendash{} (Config) 
The configuration from which the value is displayed.

\item {} 
\sphinxstyleliteralstrong{\sphinxupquote{path}} \textendash{} (str) The path in the configuration of the value to print.

\item {} 
\sphinxstyleliteralstrong{\sphinxupquote{show\_label}} \textendash{} (bool) 
If True, do a basic display. (useful for bash completion)

\item {} 
\sphinxstyleliteralstrong{\sphinxupquote{stream}} \textendash{} The output stream used

\item {} 
\sphinxstyleliteralstrong{\sphinxupquote{level}} \textendash{} (int) The number of spaces to add before display.

\item {} 
\sphinxstyleliteralstrong{\sphinxupquote{show\_full\_path}} \textendash{} (bool) Display full path, else relative

\end{itemize}

\end{description}\end{quote}

\end{fulllineitems}

\index{get\_config\_children() (in module src.configManager)}

\begin{fulllineitems}
\phantomsection\label{\detokenize{apidoc_src/src:src.configManager.get_config_children}}\pysiglinewithargsret{\sphinxcode{\sphinxupquote{src.configManager.}}\sphinxbfcode{\sphinxupquote{get\_config\_children}}}{\emph{config}, \emph{args}}{}
Gets the names of the children of the given parameter.
Useful only for completion mechanism
\begin{quote}\begin{description}
\item[{Parameters}] \leavevmode\begin{itemize}
\item {} 
\sphinxstyleliteralstrong{\sphinxupquote{config}} \textendash{} (Config) The configuration where to read the values

\item {} 
\sphinxstyleliteralstrong{\sphinxupquote{args}} \textendash{} The path in the config from which get the keys

\end{itemize}

\end{description}\end{quote}

\end{fulllineitems}

\index{get\_products\_list() (in module src.configManager)}

\begin{fulllineitems}
\phantomsection\label{\detokenize{apidoc_src/src:src.configManager.get_products_list}}\pysiglinewithargsret{\sphinxcode{\sphinxupquote{src.configManager.}}\sphinxbfcode{\sphinxupquote{get\_products\_list}}}{\emph{self}, \emph{options}, \emph{cfg}, \emph{logger}}{}
Gives the product list with their informations from 
configuration regarding the passed options.
\begin{quote}\begin{description}
\item[{Parameters}] \leavevmode\begin{itemize}
\item {} 
\sphinxstyleliteralstrong{\sphinxupquote{options}} \textendash{} (Options) 
The Options instance that stores the commands arguments

\item {} 
\sphinxstyleliteralstrong{\sphinxupquote{config}} \textendash{} (Config) The global configuration

\item {} 
\sphinxstyleliteralstrong{\sphinxupquote{logger}} \textendash{} (Logger) 
The logger instance to use for the display and logging

\end{itemize}

\item[{Returns}] \leavevmode
(list) The list of (product name, product\_informations).

\end{description}\end{quote}

\end{fulllineitems}

\index{print\_debug() (in module src.configManager)}

\begin{fulllineitems}
\phantomsection\label{\detokenize{apidoc_src/src:src.configManager.print_debug}}\pysiglinewithargsret{\sphinxcode{\sphinxupquote{src.configManager.}}\sphinxbfcode{\sphinxupquote{print\_debug}}}{\emph{config}, \emph{aPath}, \emph{logger}, \emph{show\_label=False}, \emph{level=0}, \emph{show\_full\_path=False}}{}
logger output for debugging a config/pyconf
lines contains: path : expression \textendash{}\textgreater{} ‘evaluation’

example:
PROJECTS.projects.salome.project\_path : \$PWD \textendash{}\textgreater{} ‘/tmp/SALOME’

\end{fulllineitems}

\index{print\_value() (in module src.configManager)}

\begin{fulllineitems}
\phantomsection\label{\detokenize{apidoc_src/src:src.configManager.print_value}}\pysiglinewithargsret{\sphinxcode{\sphinxupquote{src.configManager.}}\sphinxbfcode{\sphinxupquote{print\_value}}}{\emph{config}, \emph{path}, \emph{logger}, \emph{show\_label=False}, \emph{level=0}, \emph{show\_full\_path=False}}{}
print a colored representation value from a config pyconf instance.
used recursively from the initial path.
\begin{quote}\begin{description}
\item[{Param}] \leavevmode
as getConfigColored

\end{description}\end{quote}

\end{fulllineitems}

\index{show\_patchs() (in module src.configManager)}

\begin{fulllineitems}
\phantomsection\label{\detokenize{apidoc_src/src:src.configManager.show_patchs}}\pysiglinewithargsret{\sphinxcode{\sphinxupquote{src.configManager.}}\sphinxbfcode{\sphinxupquote{show\_patchs}}}{\emph{config}, \emph{logger}}{}
Prints all the used patchs in the application.
\begin{quote}\begin{description}
\item[{Parameters}] \leavevmode\begin{itemize}
\item {} 
\sphinxstyleliteralstrong{\sphinxupquote{config}} \textendash{} (Config) the global configuration.

\item {} 
\sphinxstyleliteralstrong{\sphinxupquote{logger}} \textendash{} (Logger) 
The logger instance to use for the display

\end{itemize}

\end{description}\end{quote}

\end{fulllineitems}

\index{show\_product\_info() (in module src.configManager)}

\begin{fulllineitems}
\phantomsection\label{\detokenize{apidoc_src/src:src.configManager.show_product_info}}\pysiglinewithargsret{\sphinxcode{\sphinxupquote{src.configManager.}}\sphinxbfcode{\sphinxupquote{show\_product\_info}}}{\emph{config}, \emph{name}, \emph{logger}}{}
Display on the terminal and logger information about a product.
\begin{quote}\begin{description}
\item[{Parameters}] \leavevmode\begin{itemize}
\item {} 
\sphinxstyleliteralstrong{\sphinxupquote{config}} \textendash{} (Config) the global configuration.

\item {} 
\sphinxstyleliteralstrong{\sphinxupquote{name}} \textendash{} (str) The name of the product

\item {} 
\sphinxstyleliteralstrong{\sphinxupquote{logger}} \textendash{} (Logger) The logger instance to use for the display

\end{itemize}

\end{description}\end{quote}

\end{fulllineitems}



\subsubsection{src.debug module}
\label{\detokenize{apidoc_src/src:module-src.debug}}\label{\detokenize{apidoc_src/src:src-debug-module}}\index{src.debug (module)}
This file assume DEBUG functionalities use
Print salomeTools debug messages in sys.stderr.
Show pretty print debug representation from instances of SAT classes 
(pretty print src.pyconf.Config)

WARNING: supposedly show messages in SAT development phase, not production

usage:
\textgreater{}\textgreater{} import debug as DBG
\textgreater{}\textgreater{} DBG.write(“aTitle”, aVariable)        \# not shown in production 
\textgreater{}\textgreater{} DBG.write(“aTitle”, aVariable, True)  \# unconditionaly shown
\index{InStream (class in src.debug)}

\begin{fulllineitems}
\phantomsection\label{\detokenize{apidoc_src/src:src.debug.InStream}}\pysiglinewithargsret{\sphinxbfcode{\sphinxupquote{class }}\sphinxcode{\sphinxupquote{src.debug.}}\sphinxbfcode{\sphinxupquote{InStream}}}{\emph{buf=''}}{}
Bases: \sphinxcode{\sphinxupquote{StringIO.StringIO}}

utility class for pyconf.Config input iostream

\end{fulllineitems}

\index{OutStream (class in src.debug)}

\begin{fulllineitems}
\phantomsection\label{\detokenize{apidoc_src/src:src.debug.OutStream}}\pysiglinewithargsret{\sphinxbfcode{\sphinxupquote{class }}\sphinxcode{\sphinxupquote{src.debug.}}\sphinxbfcode{\sphinxupquote{OutStream}}}{\emph{buf=''}}{}
Bases: \sphinxcode{\sphinxupquote{StringIO.StringIO}}

utility class for pyconf.Config output iostream
\index{close() (src.debug.OutStream method)}

\begin{fulllineitems}
\phantomsection\label{\detokenize{apidoc_src/src:src.debug.OutStream.close}}\pysiglinewithargsret{\sphinxbfcode{\sphinxupquote{close}}}{}{}
because Config.\_\_save\_\_ calls close() stream as file
keep value before lost as self.value

\end{fulllineitems}


\end{fulllineitems}

\index{format\_color\_exception() (in module src.debug)}

\begin{fulllineitems}
\phantomsection\label{\detokenize{apidoc_src/src:src.debug.format_color_exception}}\pysiglinewithargsret{\sphinxcode{\sphinxupquote{src.debug.}}\sphinxbfcode{\sphinxupquote{format\_color\_exception}}}{\emph{msg}, \emph{limit=None}, \emph{trace=None}}{}
Format a stack trace and the exception information.
as traceback.format\_exception(), with color
with traceback only if (\_debug) or (DBG.\_user in DBG.\_developpers)

\end{fulllineitems}

\index{getLocalEnv() (in module src.debug)}

\begin{fulllineitems}
\phantomsection\label{\detokenize{apidoc_src/src:src.debug.getLocalEnv}}\pysiglinewithargsret{\sphinxcode{\sphinxupquote{src.debug.}}\sphinxbfcode{\sphinxupquote{getLocalEnv}}}{}{}
get string for environment variables representation

\end{fulllineitems}

\index{getStrConfigDbg() (in module src.debug)}

\begin{fulllineitems}
\phantomsection\label{\detokenize{apidoc_src/src:src.debug.getStrConfigDbg}}\pysiglinewithargsret{\sphinxcode{\sphinxupquote{src.debug.}}\sphinxbfcode{\sphinxupquote{getStrConfigDbg}}}{\emph{config}}{}
set string as saveConfigDbg, 
as (path expression evaluation) for debug

\end{fulllineitems}

\index{getStrConfigStd() (in module src.debug)}

\begin{fulllineitems}
\phantomsection\label{\detokenize{apidoc_src/src:src.debug.getStrConfigStd}}\pysiglinewithargsret{\sphinxcode{\sphinxupquote{src.debug.}}\sphinxbfcode{\sphinxupquote{getStrConfigStd}}}{\emph{config}}{}
set string as saveConfigStd, 
as file .pyconf

\end{fulllineitems}

\index{indent() (in module src.debug)}

\begin{fulllineitems}
\phantomsection\label{\detokenize{apidoc_src/src:src.debug.indent}}\pysiglinewithargsret{\sphinxcode{\sphinxupquote{src.debug.}}\sphinxbfcode{\sphinxupquote{indent}}}{\emph{text}, \emph{amount=2}, \emph{ch=' '}}{}
indent multi lines message

\end{fulllineitems}

\index{isTypeConfig() (in module src.debug)}

\begin{fulllineitems}
\phantomsection\label{\detokenize{apidoc_src/src:src.debug.isTypeConfig}}\pysiglinewithargsret{\sphinxcode{\sphinxupquote{src.debug.}}\sphinxbfcode{\sphinxupquote{isTypeConfig}}}{\emph{var}}{}
To know if var is instance from Config/pyconf

\end{fulllineitems}

\index{pop\_debug() (in module src.debug)}

\begin{fulllineitems}
\phantomsection\label{\detokenize{apidoc_src/src:src.debug.pop_debug}}\pysiglinewithargsret{\sphinxcode{\sphinxupquote{src.debug.}}\sphinxbfcode{\sphinxupquote{pop\_debug}}}{}{}
restore previous debug outputs status

\end{fulllineitems}

\index{push\_debug() (in module src.debug)}

\begin{fulllineitems}
\phantomsection\label{\detokenize{apidoc_src/src:src.debug.push_debug}}\pysiglinewithargsret{\sphinxcode{\sphinxupquote{src.debug.}}\sphinxbfcode{\sphinxupquote{push\_debug}}}{\emph{aBool}}{}
set debug outputs activated, or not

\end{fulllineitems}

\index{saveConfigDbg() (in module src.debug)}

\begin{fulllineitems}
\phantomsection\label{\detokenize{apidoc_src/src:src.debug.saveConfigDbg}}\pysiglinewithargsret{\sphinxcode{\sphinxupquote{src.debug.}}\sphinxbfcode{\sphinxupquote{saveConfigDbg}}}{\emph{config}, \emph{aStream}, \emph{indent=0}, \emph{path=''}}{}
pyconf returns multilines (path expression evaluation) for debug

\end{fulllineitems}

\index{saveConfigStd() (in module src.debug)}

\begin{fulllineitems}
\phantomsection\label{\detokenize{apidoc_src/src:src.debug.saveConfigStd}}\pysiglinewithargsret{\sphinxcode{\sphinxupquote{src.debug.}}\sphinxbfcode{\sphinxupquote{saveConfigStd}}}{\emph{config}, \emph{aStream}}{}
returns as file .pyconf

\end{fulllineitems}

\index{tofix() (in module src.debug)}

\begin{fulllineitems}
\phantomsection\label{\detokenize{apidoc_src/src:src.debug.tofix}}\pysiglinewithargsret{\sphinxcode{\sphinxupquote{src.debug.}}\sphinxbfcode{\sphinxupquote{tofix}}}{\emph{title}, \emph{var=''}, \emph{force=None}}{}
write sys.stderr a message if \_debug{[}-1{]}==True or optionaly force=True
use this only if no logger accessible for classic logger.warning(message)

\end{fulllineitems}

\index{write() (in module src.debug)}

\begin{fulllineitems}
\phantomsection\label{\detokenize{apidoc_src/src:src.debug.write}}\pysiglinewithargsret{\sphinxcode{\sphinxupquote{src.debug.}}\sphinxbfcode{\sphinxupquote{write}}}{\emph{title}, \emph{var=''}, \emph{force=None}, \emph{fmt='\textbackslash{}n\#\#\#\# DEBUG: \%s:\textbackslash{}n\%s\textbackslash{}n'}}{}
write sys.stderr a message if \_debug{[}-1{]}==True or optionaly force=True

\end{fulllineitems}



\subsubsection{src.environment module}
\label{\detokenize{apidoc_src/src:src-environment-module}}\label{\detokenize{apidoc_src/src:module-src.environment}}\index{src.environment (module)}\index{Environ (class in src.environment)}

\begin{fulllineitems}
\phantomsection\label{\detokenize{apidoc_src/src:src.environment.Environ}}\pysiglinewithargsret{\sphinxbfcode{\sphinxupquote{class }}\sphinxcode{\sphinxupquote{src.environment.}}\sphinxbfcode{\sphinxupquote{Environ}}}{\emph{environ=None}}{}
Class to manage an environment context
\index{append() (src.environment.Environ method)}

\begin{fulllineitems}
\phantomsection\label{\detokenize{apidoc_src/src:src.environment.Environ.append}}\pysiglinewithargsret{\sphinxbfcode{\sphinxupquote{append}}}{\emph{key}, \emph{value}, \emph{sep=':'}}{}
Same as append\_value but the value argument can be a list
\begin{quote}\begin{description}
\item[{Parameters}] \leavevmode\begin{itemize}
\item {} 
\sphinxstyleliteralstrong{\sphinxupquote{key}} \textendash{} (str) the environment variable to append

\item {} 
\sphinxstyleliteralstrong{\sphinxupquote{value}} \textendash{} (str or list) the value(s) to append to key

\item {} 
\sphinxstyleliteralstrong{\sphinxupquote{sep}} \textendash{} (str) the separator string (usually ‘:’)

\end{itemize}

\end{description}\end{quote}

\end{fulllineitems}

\index{append\_value() (src.environment.Environ method)}

\begin{fulllineitems}
\phantomsection\label{\detokenize{apidoc_src/src:src.environment.Environ.append_value}}\pysiglinewithargsret{\sphinxbfcode{\sphinxupquote{append\_value}}}{\emph{key}, \emph{value}, \emph{sep=':'}}{}
append value to key using sep
\begin{quote}\begin{description}
\item[{Parameters}] \leavevmode\begin{itemize}
\item {} 
\sphinxstyleliteralstrong{\sphinxupquote{key}} \textendash{} (str) the environment variable to append

\item {} 
\sphinxstyleliteralstrong{\sphinxupquote{value}} \textendash{} (str) the value to append to key

\item {} 
\sphinxstyleliteralstrong{\sphinxupquote{sep}} \textendash{} (str) the separator string (usually ‘:’)

\end{itemize}

\end{description}\end{quote}

\end{fulllineitems}

\index{command\_value() (src.environment.Environ method)}

\begin{fulllineitems}
\phantomsection\label{\detokenize{apidoc_src/src:src.environment.Environ.command_value}}\pysiglinewithargsret{\sphinxbfcode{\sphinxupquote{command\_value}}}{\emph{key}, \emph{command}}{}
Get the value given by the system command “command” 
and put it in the environment variable key
\begin{quote}\begin{description}
\item[{Parameters}] \leavevmode\begin{itemize}
\item {} 
\sphinxstyleliteralstrong{\sphinxupquote{key}} \textendash{} (str) the environment variable

\item {} 
\sphinxstyleliteralstrong{\sphinxupquote{command}} \textendash{} (str) the command to execute

\end{itemize}

\end{description}\end{quote}

\end{fulllineitems}

\index{get() (src.environment.Environ method)}

\begin{fulllineitems}
\phantomsection\label{\detokenize{apidoc_src/src:src.environment.Environ.get}}\pysiglinewithargsret{\sphinxbfcode{\sphinxupquote{get}}}{\emph{key}}{}
Get the value of the environment variable “key”
\begin{quote}\begin{description}
\item[{Parameters}] \leavevmode
\sphinxstyleliteralstrong{\sphinxupquote{key}} \textendash{} (str) the environment variable

\end{description}\end{quote}

\end{fulllineitems}

\index{is\_defined() (src.environment.Environ method)}

\begin{fulllineitems}
\phantomsection\label{\detokenize{apidoc_src/src:src.environment.Environ.is_defined}}\pysiglinewithargsret{\sphinxbfcode{\sphinxupquote{is\_defined}}}{\emph{key}}{}
Check if the key exists in the environment
\begin{quote}\begin{description}
\item[{Parameters}] \leavevmode
\sphinxstyleliteralstrong{\sphinxupquote{key}} \textendash{} (str) the environment variable to check

\end{description}\end{quote}

\end{fulllineitems}

\index{prepend() (src.environment.Environ method)}

\begin{fulllineitems}
\phantomsection\label{\detokenize{apidoc_src/src:src.environment.Environ.prepend}}\pysiglinewithargsret{\sphinxbfcode{\sphinxupquote{prepend}}}{\emph{key}, \emph{value}, \emph{sep=':'}}{}
Same as prepend\_value but the value argument can be a list
\begin{quote}\begin{description}
\item[{Parameters}] \leavevmode\begin{itemize}
\item {} 
\sphinxstyleliteralstrong{\sphinxupquote{key}} \textendash{} (str) the environment variable to prepend

\item {} 
\sphinxstyleliteralstrong{\sphinxupquote{value}} \textendash{} (str or list) the value(s) to prepend to key

\item {} 
\sphinxstyleliteralstrong{\sphinxupquote{sep}} \textendash{} (str) the separator string (usually ‘:’)

\end{itemize}

\end{description}\end{quote}

\end{fulllineitems}

\index{prepend\_value() (src.environment.Environ method)}

\begin{fulllineitems}
\phantomsection\label{\detokenize{apidoc_src/src:src.environment.Environ.prepend_value}}\pysiglinewithargsret{\sphinxbfcode{\sphinxupquote{prepend\_value}}}{\emph{key}, \emph{value}, \emph{sep=':'}}{}
prepend value to key using sep
\begin{quote}\begin{description}
\item[{Parameters}] \leavevmode\begin{itemize}
\item {} 
\sphinxstyleliteralstrong{\sphinxupquote{key}} \textendash{} (str) the environment variable to prepend

\item {} 
\sphinxstyleliteralstrong{\sphinxupquote{value}} \textendash{} (str) the value to prepend to key

\item {} 
\sphinxstyleliteralstrong{\sphinxupquote{sep}} \textendash{} (str) the separator string (usually ‘:’)

\end{itemize}

\end{description}\end{quote}

\end{fulllineitems}

\index{set() (src.environment.Environ method)}

\begin{fulllineitems}
\phantomsection\label{\detokenize{apidoc_src/src:src.environment.Environ.set}}\pysiglinewithargsret{\sphinxbfcode{\sphinxupquote{set}}}{\emph{key}, \emph{value}}{}
Set the environment variable “key” to value “value”
\begin{quote}\begin{description}
\item[{Parameters}] \leavevmode\begin{itemize}
\item {} 
\sphinxstyleliteralstrong{\sphinxupquote{key}} \textendash{} (str) the environment variable to set

\item {} 
\sphinxstyleliteralstrong{\sphinxupquote{value}} \textendash{} (str) the value

\end{itemize}

\end{description}\end{quote}

\end{fulllineitems}


\end{fulllineitems}

\index{FileEnvWriter (class in src.environment)}

\begin{fulllineitems}
\phantomsection\label{\detokenize{apidoc_src/src:src.environment.FileEnvWriter}}\pysiglinewithargsret{\sphinxbfcode{\sphinxupquote{class }}\sphinxcode{\sphinxupquote{src.environment.}}\sphinxbfcode{\sphinxupquote{FileEnvWriter}}}{\emph{config}, \emph{logger}, \emph{out\_dir}, \emph{src\_root}, \emph{env\_info=None}}{}
Class to dump the environment to a file.
\index{write\_cfgForPy\_file() (src.environment.FileEnvWriter method)}

\begin{fulllineitems}
\phantomsection\label{\detokenize{apidoc_src/src:src.environment.FileEnvWriter.write_cfgForPy_file}}\pysiglinewithargsret{\sphinxbfcode{\sphinxupquote{write\_cfgForPy\_file}}}{\emph{filename}, \emph{additional\_env=\{\}}, \emph{for\_package=None}, \emph{with\_commercial=True}}{}
Append to current opened aFile a cfgForPy 
environment (SALOME python launcher).
\begin{quote}\begin{description}
\item[{Parameters}] \leavevmode\begin{itemize}
\item {} 
\sphinxstyleliteralstrong{\sphinxupquote{filename}} \textendash{} (str) the file path

\item {} 
\sphinxstyleliteralstrong{\sphinxupquote{additional\_env}} \textendash{} (dict) 
a dictionary of additional variables to add to the environment

\item {} 
\sphinxstyleliteralstrong{\sphinxupquote{for\_package}} \textendash{} (str) 
If not None, produce a relative environment 
(designed for a package)

\end{itemize}

\end{description}\end{quote}

\end{fulllineitems}

\index{write\_env\_file() (src.environment.FileEnvWriter method)}

\begin{fulllineitems}
\phantomsection\label{\detokenize{apidoc_src/src:src.environment.FileEnvWriter.write_env_file}}\pysiglinewithargsret{\sphinxbfcode{\sphinxupquote{write\_env\_file}}}{\emph{filename}, \emph{forBuild}, \emph{shell}, \emph{for\_package=None}}{}
Create an environment file.
\begin{quote}\begin{description}
\item[{Parameters}] \leavevmode\begin{itemize}
\item {} 
\sphinxstyleliteralstrong{\sphinxupquote{filename}} \textendash{} (str) the file path

\item {} 
\sphinxstyleliteralstrong{\sphinxupquote{forBuild}} \textendash{} (bool) if true, the build environment

\item {} 
\sphinxstyleliteralstrong{\sphinxupquote{shell}} \textendash{} (str) the type of file wanted (.sh, .bat)

\end{itemize}

\item[{Returns}] \leavevmode
(str) The path to the generated file

\end{description}\end{quote}

\end{fulllineitems}


\end{fulllineitems}

\index{SalomeEnviron (class in src.environment)}

\begin{fulllineitems}
\phantomsection\label{\detokenize{apidoc_src/src:src.environment.SalomeEnviron}}\pysiglinewithargsret{\sphinxbfcode{\sphinxupquote{class }}\sphinxcode{\sphinxupquote{src.environment.}}\sphinxbfcode{\sphinxupquote{SalomeEnviron}}}{\emph{cfg}, \emph{environ}, \emph{forBuild=False}, \emph{for\_package=None}, \emph{enable\_simple\_env\_script=True}}{}
Class to manage the environment of SALOME.
\index{add\_comment() (src.environment.SalomeEnviron method)}

\begin{fulllineitems}
\phantomsection\label{\detokenize{apidoc_src/src:src.environment.SalomeEnviron.add_comment}}\pysiglinewithargsret{\sphinxbfcode{\sphinxupquote{add\_comment}}}{\emph{comment}}{}
Add a commentary to the out stream (in case of file generation)
\begin{quote}\begin{description}
\item[{Parameters}] \leavevmode
\sphinxstyleliteralstrong{\sphinxupquote{comment}} \textendash{} (str) the commentary to add

\end{description}\end{quote}

\end{fulllineitems}

\index{add\_line() (src.environment.SalomeEnviron method)}

\begin{fulllineitems}
\phantomsection\label{\detokenize{apidoc_src/src:src.environment.SalomeEnviron.add_line}}\pysiglinewithargsret{\sphinxbfcode{\sphinxupquote{add\_line}}}{\emph{nb\_line}}{}
Add empty lines to the out stream (in case of file generation)
\begin{quote}\begin{description}
\item[{Parameters}] \leavevmode
\sphinxstyleliteralstrong{\sphinxupquote{nb\_line}} \textendash{} (int) the number of empty lines to add

\end{description}\end{quote}

\end{fulllineitems}

\index{add\_warning() (src.environment.SalomeEnviron method)}

\begin{fulllineitems}
\phantomsection\label{\detokenize{apidoc_src/src:src.environment.SalomeEnviron.add_warning}}\pysiglinewithargsret{\sphinxbfcode{\sphinxupquote{add\_warning}}}{\emph{warning}}{}
Add a warning to the out stream (in case of file generation)
\begin{quote}\begin{description}
\item[{Parameters}] \leavevmode
\sphinxstyleliteralstrong{\sphinxupquote{warning}} \textendash{} (str) the warning to add

\end{description}\end{quote}

\end{fulllineitems}

\index{append() (src.environment.SalomeEnviron method)}

\begin{fulllineitems}
\phantomsection\label{\detokenize{apidoc_src/src:src.environment.SalomeEnviron.append}}\pysiglinewithargsret{\sphinxbfcode{\sphinxupquote{append}}}{\emph{key}, \emph{value}, \emph{sep=':'}}{}
append value to key using sep
\begin{quote}\begin{description}
\item[{Parameters}] \leavevmode\begin{itemize}
\item {} 
\sphinxstyleliteralstrong{\sphinxupquote{key}} \textendash{} (str) the environment variable to append

\item {} 
\sphinxstyleliteralstrong{\sphinxupquote{value}} \textendash{} (str) the value to append to key

\item {} 
\sphinxstyleliteralstrong{\sphinxupquote{sep}} \textendash{} (str) the separator string

\end{itemize}

\end{description}\end{quote}

\end{fulllineitems}

\index{dump() (src.environment.SalomeEnviron method)}

\begin{fulllineitems}
\phantomsection\label{\detokenize{apidoc_src/src:src.environment.SalomeEnviron.dump}}\pysiglinewithargsret{\sphinxbfcode{\sphinxupquote{dump}}}{\emph{out}}{}
Write the environment to out
\begin{quote}\begin{description}
\item[{Parameters}] \leavevmode
\sphinxstyleliteralstrong{\sphinxupquote{out}} \textendash{} (file) the stream where to write the environment

\end{description}\end{quote}

\end{fulllineitems}

\index{finish() (src.environment.SalomeEnviron method)}

\begin{fulllineitems}
\phantomsection\label{\detokenize{apidoc_src/src:src.environment.SalomeEnviron.finish}}\pysiglinewithargsret{\sphinxbfcode{\sphinxupquote{finish}}}{\emph{required}}{}
Add a final instruction in the out file (in case of file generation)
\begin{quote}\begin{description}
\item[{Parameters}] \leavevmode
\sphinxstyleliteralstrong{\sphinxupquote{required}} \textendash{} (bool) Do nothing if required is False

\end{description}\end{quote}

\end{fulllineitems}

\index{get() (src.environment.SalomeEnviron method)}

\begin{fulllineitems}
\phantomsection\label{\detokenize{apidoc_src/src:src.environment.SalomeEnviron.get}}\pysiglinewithargsret{\sphinxbfcode{\sphinxupquote{get}}}{\emph{key}}{}
Get the value of the environment variable “key”
\begin{quote}\begin{description}
\item[{Parameters}] \leavevmode
\sphinxstyleliteralstrong{\sphinxupquote{key}} \textendash{} (str) the environment variable

\end{description}\end{quote}

\end{fulllineitems}

\index{get\_names() (src.environment.SalomeEnviron method)}

\begin{fulllineitems}
\phantomsection\label{\detokenize{apidoc_src/src:src.environment.SalomeEnviron.get_names}}\pysiglinewithargsret{\sphinxbfcode{\sphinxupquote{get\_names}}}{\emph{lProducts}}{}
Get the products name to add in SALOME\_MODULES environment variable
It is the name of the product, except in the case where the is a 
component name. And it has to be in SALOME\_MODULES variable only 
if the product has the property has\_salome\_hui = “yes”
\begin{quote}\begin{description}
\item[{Parameters}] \leavevmode
\sphinxstyleliteralstrong{\sphinxupquote{lProducts}} \textendash{} (list) List of products to potentially add

\end{description}\end{quote}

\end{fulllineitems}

\index{is\_defined() (src.environment.SalomeEnviron method)}

\begin{fulllineitems}
\phantomsection\label{\detokenize{apidoc_src/src:src.environment.SalomeEnviron.is_defined}}\pysiglinewithargsret{\sphinxbfcode{\sphinxupquote{is\_defined}}}{\emph{key}}{}
Check if the key exists in the environment
\begin{quote}\begin{description}
\item[{Parameters}] \leavevmode
\sphinxstyleliteralstrong{\sphinxupquote{key}} \textendash{} (str) the environment variable to check

\end{description}\end{quote}

\end{fulllineitems}

\index{load\_cfg\_environment() (src.environment.SalomeEnviron method)}

\begin{fulllineitems}
\phantomsection\label{\detokenize{apidoc_src/src:src.environment.SalomeEnviron.load_cfg_environment}}\pysiglinewithargsret{\sphinxbfcode{\sphinxupquote{load\_cfg\_environment}}}{\emph{cfg\_env}}{}
Loads environment defined in cfg\_env
\begin{quote}\begin{description}
\item[{Parameters}] \leavevmode
\sphinxstyleliteralstrong{\sphinxupquote{cfg\_env}} \textendash{} (Config) A config containing an environment

\end{description}\end{quote}

\end{fulllineitems}

\index{prepend() (src.environment.SalomeEnviron method)}

\begin{fulllineitems}
\phantomsection\label{\detokenize{apidoc_src/src:src.environment.SalomeEnviron.prepend}}\pysiglinewithargsret{\sphinxbfcode{\sphinxupquote{prepend}}}{\emph{key}, \emph{value}, \emph{sep=':'}}{}
prepend value to key using sep
\begin{quote}\begin{description}
\item[{Parameters}] \leavevmode\begin{itemize}
\item {} 
\sphinxstyleliteralstrong{\sphinxupquote{key}} \textendash{} (str) the environment variable to prepend

\item {} 
\sphinxstyleliteralstrong{\sphinxupquote{value}} \textendash{} (str) the value to prepend to key

\item {} 
\sphinxstyleliteralstrong{\sphinxupquote{sep}} \textendash{} (str) the separator string

\end{itemize}

\end{description}\end{quote}

\end{fulllineitems}

\index{run\_env\_script() (src.environment.SalomeEnviron method)}

\begin{fulllineitems}
\phantomsection\label{\detokenize{apidoc_src/src:src.environment.SalomeEnviron.run_env_script}}\pysiglinewithargsret{\sphinxbfcode{\sphinxupquote{run\_env\_script}}}{\emph{product\_info}, \emph{logger=None}, \emph{native=False}}{}
Runs an environment script.
\begin{quote}\begin{description}
\item[{Parameters}] \leavevmode\begin{itemize}
\item {} 
\sphinxstyleliteralstrong{\sphinxupquote{product\_info}} \textendash{} (Config) The product description

\item {} 
\sphinxstyleliteralstrong{\sphinxupquote{logger}} \textendash{} (Logger) The logger instance to display messages

\item {} 
\sphinxstyleliteralstrong{\sphinxupquote{native}} \textendash{} (bool) If True load set\_native\_env instead of set\_env

\end{itemize}

\end{description}\end{quote}

\end{fulllineitems}

\index{run\_simple\_env\_script() (src.environment.SalomeEnviron method)}

\begin{fulllineitems}
\phantomsection\label{\detokenize{apidoc_src/src:src.environment.SalomeEnviron.run_simple_env_script}}\pysiglinewithargsret{\sphinxbfcode{\sphinxupquote{run\_simple\_env\_script}}}{\emph{script\_path}, \emph{logger=None}}{}~\begin{description}
\item[{Runs an environment script. Same as run\_env\_script, but with a }] \leavevmode
script path as parameter.

\end{description}
\begin{quote}\begin{description}
\item[{Parameters}] \leavevmode\begin{itemize}
\item {} 
\sphinxstyleliteralstrong{\sphinxupquote{script\_path}} \textendash{} (str) A path to an environment script

\item {} 
\sphinxstyleliteralstrong{\sphinxupquote{logger}} \textendash{} (Logger) The logger instance to display messages

\end{itemize}

\end{description}\end{quote}

\end{fulllineitems}

\index{set() (src.environment.SalomeEnviron method)}

\begin{fulllineitems}
\phantomsection\label{\detokenize{apidoc_src/src:src.environment.SalomeEnviron.set}}\pysiglinewithargsret{\sphinxbfcode{\sphinxupquote{set}}}{\emph{key}, \emph{value}}{}
Set the environment variable “key” to value “value”
\begin{quote}\begin{description}
\item[{Parameters}] \leavevmode\begin{itemize}
\item {} 
\sphinxstyleliteralstrong{\sphinxupquote{key}} \textendash{} (str) the environment variable to set

\item {} 
\sphinxstyleliteralstrong{\sphinxupquote{value}} \textendash{} (str) the value

\end{itemize}

\end{description}\end{quote}

\end{fulllineitems}

\index{set\_a\_product() (src.environment.SalomeEnviron method)}

\begin{fulllineitems}
\phantomsection\label{\detokenize{apidoc_src/src:src.environment.SalomeEnviron.set_a_product}}\pysiglinewithargsret{\sphinxbfcode{\sphinxupquote{set\_a\_product}}}{\emph{product}, \emph{logger}}{}
Sets the environment of a product.
\begin{quote}\begin{description}
\item[{Parameters}] \leavevmode\begin{itemize}
\item {} 
\sphinxstyleliteralstrong{\sphinxupquote{product}} \textendash{} (str) The product name

\item {} 
\sphinxstyleliteralstrong{\sphinxupquote{logger}} \textendash{} (Logger) The logger instance to display messages

\end{itemize}

\end{description}\end{quote}

\end{fulllineitems}

\index{set\_application\_env() (src.environment.SalomeEnviron method)}

\begin{fulllineitems}
\phantomsection\label{\detokenize{apidoc_src/src:src.environment.SalomeEnviron.set_application_env}}\pysiglinewithargsret{\sphinxbfcode{\sphinxupquote{set\_application\_env}}}{\emph{logger}}{}
Sets the environment defined in the APPLICATION file.
\begin{quote}\begin{description}
\item[{Parameters}] \leavevmode
\sphinxstyleliteralstrong{\sphinxupquote{logger}} \textendash{} (Logger) The logger instance to display messages

\end{description}\end{quote}

\end{fulllineitems}

\index{set\_cpp\_env() (src.environment.SalomeEnviron method)}

\begin{fulllineitems}
\phantomsection\label{\detokenize{apidoc_src/src:src.environment.SalomeEnviron.set_cpp_env}}\pysiglinewithargsret{\sphinxbfcode{\sphinxupquote{set\_cpp\_env}}}{\emph{product\_info}}{}
Sets the generic environment for a SALOME cpp product.
\begin{quote}\begin{description}
\item[{Parameters}] \leavevmode
\sphinxstyleliteralstrong{\sphinxupquote{product\_info}} \textendash{} (Config) The product description

\end{description}\end{quote}

\end{fulllineitems}

\index{set\_full\_environ() (src.environment.SalomeEnviron method)}

\begin{fulllineitems}
\phantomsection\label{\detokenize{apidoc_src/src:src.environment.SalomeEnviron.set_full_environ}}\pysiglinewithargsret{\sphinxbfcode{\sphinxupquote{set\_full\_environ}}}{\emph{logger}, \emph{env\_info}}{}
Sets the full environment for products 
specified in env\_info dictionary.
\begin{quote}\begin{description}
\item[{Parameters}] \leavevmode\begin{itemize}
\item {} 
\sphinxstyleliteralstrong{\sphinxupquote{logger}} \textendash{} (Logger) The logger instance to display messages

\item {} 
\sphinxstyleliteralstrong{\sphinxupquote{env\_info}} \textendash{} (list) the list of products

\end{itemize}

\end{description}\end{quote}

\end{fulllineitems}

\index{set\_products() (src.environment.SalomeEnviron method)}

\begin{fulllineitems}
\phantomsection\label{\detokenize{apidoc_src/src:src.environment.SalomeEnviron.set_products}}\pysiglinewithargsret{\sphinxbfcode{\sphinxupquote{set\_products}}}{\emph{logger}, \emph{src\_root=None}}{}
Sets the environment for all the products.
\begin{quote}\begin{description}
\item[{Parameters}] \leavevmode\begin{itemize}
\item {} 
\sphinxstyleliteralstrong{\sphinxupquote{logger}} \textendash{} (Logger) The logger instance to display messages

\item {} 
\sphinxstyleliteralstrong{\sphinxupquote{src\_root}} \textendash{} the application working directory

\end{itemize}

\end{description}\end{quote}

\end{fulllineitems}

\index{set\_python\_libdirs() (src.environment.SalomeEnviron method)}

\begin{fulllineitems}
\phantomsection\label{\detokenize{apidoc_src/src:src.environment.SalomeEnviron.set_python_libdirs}}\pysiglinewithargsret{\sphinxbfcode{\sphinxupquote{set\_python\_libdirs}}}{}{}
Set some generic variables for python library paths

\end{fulllineitems}

\index{set\_salome\_generic\_product\_env() (src.environment.SalomeEnviron method)}

\begin{fulllineitems}
\phantomsection\label{\detokenize{apidoc_src/src:src.environment.SalomeEnviron.set_salome_generic_product_env}}\pysiglinewithargsret{\sphinxbfcode{\sphinxupquote{set\_salome\_generic\_product\_env}}}{\emph{pi}}{}
Sets the generic environment for a SALOME product.
\begin{quote}\begin{description}
\item[{Parameters}] \leavevmode
\sphinxstyleliteralstrong{\sphinxupquote{pi}} \textendash{} (Config) The product description

\end{description}\end{quote}

\end{fulllineitems}

\index{set\_salome\_minimal\_product\_env() (src.environment.SalomeEnviron method)}

\begin{fulllineitems}
\phantomsection\label{\detokenize{apidoc_src/src:src.environment.SalomeEnviron.set_salome_minimal_product_env}}\pysiglinewithargsret{\sphinxbfcode{\sphinxupquote{set\_salome\_minimal\_product\_env}}}{\emph{product\_info}, \emph{logger}}{}
Sets the minimal environment for a SALOME product.
xxx\_ROOT\_DIR and xxx\_SRC\_DIR
\begin{quote}\begin{description}
\item[{Parameters}] \leavevmode\begin{itemize}
\item {} 
\sphinxstyleliteralstrong{\sphinxupquote{product\_info}} \textendash{} (Config) The product description

\item {} 
\sphinxstyleliteralstrong{\sphinxupquote{logger}} \textendash{} (Logger) The logger instance to display messages

\end{itemize}

\end{description}\end{quote}

\end{fulllineitems}


\end{fulllineitems}

\index{Shell (class in src.environment)}

\begin{fulllineitems}
\phantomsection\label{\detokenize{apidoc_src/src:src.environment.Shell}}\pysiglinewithargsret{\sphinxbfcode{\sphinxupquote{class }}\sphinxcode{\sphinxupquote{src.environment.}}\sphinxbfcode{\sphinxupquote{Shell}}}{\emph{name}, \emph{extension}}{}
Definition of a Shell.

\end{fulllineitems}

\index{load\_environment() (in module src.environment)}

\begin{fulllineitems}
\phantomsection\label{\detokenize{apidoc_src/src:src.environment.load_environment}}\pysiglinewithargsret{\sphinxcode{\sphinxupquote{src.environment.}}\sphinxbfcode{\sphinxupquote{load\_environment}}}{\emph{config}, \emph{build}, \emph{logger}}{}
Loads the environment (used to run the tests, for example).
\begin{quote}\begin{description}
\item[{Parameters}] \leavevmode\begin{itemize}
\item {} 
\sphinxstyleliteralstrong{\sphinxupquote{config}} \textendash{} (Config) the global config

\item {} 
\sphinxstyleliteralstrong{\sphinxupquote{build}} \textendash{} (bool) build environement if True

\item {} 
\sphinxstyleliteralstrong{\sphinxupquote{logger}} \textendash{} (Logger) The logger instance to display messages

\end{itemize}

\end{description}\end{quote}

\end{fulllineitems}



\subsubsection{src.environs module}
\label{\detokenize{apidoc_src/src:module-src.environs}}\label{\detokenize{apidoc_src/src:src-environs-module}}\index{src.environs (module)}
Utility for print environment variables
\begin{description}
\item[{examples: }] \leavevmode\begin{itemize}
\item {} 
split all or specific environment variables \$XXX(s)…
\textgreater{}\textgreater{} environs.py -\textgreater{} all
\textgreater{}\textgreater{} environs.py SHELL PATH -\textgreater{} specific \$SHELL \$PATH

\item {} 
split all or specific environment variables on pattern \$*XXX*(s)…
\textgreater{}\textgreater{} environs.py \textendash{}pat ROOT -\textgreater{} specific \$*ROOT*

\item {} 
split search specific substrings in contents of environment variables \$XXX(s)…
\textgreater{}\textgreater{} environs.py \textendash{}grep usr  -\textgreater{} all specific environment variables containing usr

\end{itemize}

\item[{tips:}] \leavevmode\begin{itemize}
\item {} 
create unix alias as shortcut for bash console
\textgreater{}\textgreater{} alias envs=”…/environs.py”

\end{itemize}

\end{description}
\index{print\_grep\_environs() (in module src.environs)}

\begin{fulllineitems}
\phantomsection\label{\detokenize{apidoc_src/src:src.environs.print_grep_environs}}\pysiglinewithargsret{\sphinxcode{\sphinxupquote{src.environs.}}\sphinxbfcode{\sphinxupquote{print\_grep\_environs}}}{\emph{args={[}{]}}}{}
\end{fulllineitems}

\index{print\_split\_environs() (in module src.environs)}

\begin{fulllineitems}
\phantomsection\label{\detokenize{apidoc_src/src:src.environs.print_split_environs}}\pysiglinewithargsret{\sphinxcode{\sphinxupquote{src.environs.}}\sphinxbfcode{\sphinxupquote{print\_split\_environs}}}{\emph{args={[}{]}}}{}
\end{fulllineitems}

\index{print\_split\_pattern\_environs() (in module src.environs)}

\begin{fulllineitems}
\phantomsection\label{\detokenize{apidoc_src/src:src.environs.print_split_pattern_environs}}\pysiglinewithargsret{\sphinxcode{\sphinxupquote{src.environs.}}\sphinxbfcode{\sphinxupquote{print\_split\_pattern\_environs}}}{\emph{args={[}{]}}}{}
\end{fulllineitems}



\subsubsection{src.exceptionSat module}
\label{\detokenize{apidoc_src/src:module-src.exceptionSat}}\label{\detokenize{apidoc_src/src:src-exceptionsat-module}}\index{src.exceptionSat (module)}\index{ExceptionSat}

\begin{fulllineitems}
\phantomsection\label{\detokenize{apidoc_src/src:src.exceptionSat.ExceptionSat}}\pysigline{\sphinxbfcode{\sphinxupquote{exception }}\sphinxcode{\sphinxupquote{src.exceptionSat.}}\sphinxbfcode{\sphinxupquote{ExceptionSat}}}
Bases: \sphinxcode{\sphinxupquote{exceptions.Exception}}

rename Exception Class for sat convenience (for future…)

\end{fulllineitems}



\subsubsection{src.fileEnviron module}
\label{\detokenize{apidoc_src/src:src-fileenviron-module}}\label{\detokenize{apidoc_src/src:module-src.fileEnviron}}\index{src.fileEnviron (module)}\index{BashFileEnviron (class in src.fileEnviron)}

\begin{fulllineitems}
\phantomsection\label{\detokenize{apidoc_src/src:src.fileEnviron.BashFileEnviron}}\pysiglinewithargsret{\sphinxbfcode{\sphinxupquote{class }}\sphinxcode{\sphinxupquote{src.fileEnviron.}}\sphinxbfcode{\sphinxupquote{BashFileEnviron}}}{\emph{output}, \emph{environ=None}}{}
Bases: {\hyperref[\detokenize{apidoc_src/src:src.fileEnviron.FileEnviron}]{\sphinxcrossref{\sphinxcode{\sphinxupquote{src.fileEnviron.FileEnviron}}}}} (\autopageref*{\detokenize{apidoc_src/src:src.fileEnviron.FileEnviron}})

Class for bash shell.
\index{command\_value() (src.fileEnviron.BashFileEnviron method)}

\begin{fulllineitems}
\phantomsection\label{\detokenize{apidoc_src/src:src.fileEnviron.BashFileEnviron.command_value}}\pysiglinewithargsret{\sphinxbfcode{\sphinxupquote{command\_value}}}{\emph{key}, \emph{command}}{}
Get the value given by the system command “command” 
and put it in the environment variable key.
Has to be overwritten in the derived classes
This can be seen as a virtual method
\begin{quote}\begin{description}
\item[{Parameters}] \leavevmode\begin{itemize}
\item {} 
\sphinxstyleliteralstrong{\sphinxupquote{key}} \textendash{} (str) the environment variable

\item {} 
\sphinxstyleliteralstrong{\sphinxupquote{command}} \textendash{} (str) the command to execute

\end{itemize}

\end{description}\end{quote}

\end{fulllineitems}

\index{finish() (src.fileEnviron.BashFileEnviron method)}

\begin{fulllineitems}
\phantomsection\label{\detokenize{apidoc_src/src:src.fileEnviron.BashFileEnviron.finish}}\pysiglinewithargsret{\sphinxbfcode{\sphinxupquote{finish}}}{\emph{required=True}}{}
Add a final instruction in the out file (in case of file generation)
\begin{quote}\begin{description}
\item[{Parameters}] \leavevmode
\sphinxstyleliteralstrong{\sphinxupquote{required}} \textendash{} (bool) Do nothing if required is False

\end{description}\end{quote}

\end{fulllineitems}

\index{set() (src.fileEnviron.BashFileEnviron method)}

\begin{fulllineitems}
\phantomsection\label{\detokenize{apidoc_src/src:src.fileEnviron.BashFileEnviron.set}}\pysiglinewithargsret{\sphinxbfcode{\sphinxupquote{set}}}{\emph{key}, \emph{value}}{}
Set the environment variable “key” to value “value”
\begin{quote}\begin{description}
\item[{Parameters}] \leavevmode\begin{itemize}
\item {} 
\sphinxstyleliteralstrong{\sphinxupquote{key}} \textendash{} (str) the environment variable to set

\item {} 
\sphinxstyleliteralstrong{\sphinxupquote{value}} \textendash{} (str) the value

\end{itemize}

\end{description}\end{quote}

\end{fulllineitems}


\end{fulllineitems}

\index{BatFileEnviron (class in src.fileEnviron)}

\begin{fulllineitems}
\phantomsection\label{\detokenize{apidoc_src/src:src.fileEnviron.BatFileEnviron}}\pysiglinewithargsret{\sphinxbfcode{\sphinxupquote{class }}\sphinxcode{\sphinxupquote{src.fileEnviron.}}\sphinxbfcode{\sphinxupquote{BatFileEnviron}}}{\emph{output}, \emph{environ=None}}{}
Bases: {\hyperref[\detokenize{apidoc_src/src:src.fileEnviron.FileEnviron}]{\sphinxcrossref{\sphinxcode{\sphinxupquote{src.fileEnviron.FileEnviron}}}}} (\autopageref*{\detokenize{apidoc_src/src:src.fileEnviron.FileEnviron}})

for Windows batch shell.
\index{add\_comment() (src.fileEnviron.BatFileEnviron method)}

\begin{fulllineitems}
\phantomsection\label{\detokenize{apidoc_src/src:src.fileEnviron.BatFileEnviron.add_comment}}\pysiglinewithargsret{\sphinxbfcode{\sphinxupquote{add\_comment}}}{\emph{comment}}{}
Add a comment in the shell file
\begin{quote}\begin{description}
\item[{Parameters}] \leavevmode
\sphinxstyleliteralstrong{\sphinxupquote{comment}} \textendash{} (str) the comment to add

\end{description}\end{quote}

\end{fulllineitems}

\index{command\_value() (src.fileEnviron.BatFileEnviron method)}

\begin{fulllineitems}
\phantomsection\label{\detokenize{apidoc_src/src:src.fileEnviron.BatFileEnviron.command_value}}\pysiglinewithargsret{\sphinxbfcode{\sphinxupquote{command\_value}}}{\emph{key}, \emph{command}}{}
Get the value given by the system command “command” 
and put it in the environment variable key.
Has to be overwritten in the derived classes
This can be seen as a virtual method
\begin{quote}\begin{description}
\item[{Parameters}] \leavevmode\begin{itemize}
\item {} 
\sphinxstyleliteralstrong{\sphinxupquote{key}} \textendash{} (str) the environment variable

\item {} 
\sphinxstyleliteralstrong{\sphinxupquote{command}} \textendash{} (str) the command to execute

\end{itemize}

\end{description}\end{quote}

\end{fulllineitems}

\index{finish() (src.fileEnviron.BatFileEnviron method)}

\begin{fulllineitems}
\phantomsection\label{\detokenize{apidoc_src/src:src.fileEnviron.BatFileEnviron.finish}}\pysiglinewithargsret{\sphinxbfcode{\sphinxupquote{finish}}}{\emph{required=True}}{}
Add a final instruction in the out file (in case of file generation)
In the particular windows case, do nothing
\begin{quote}\begin{description}
\item[{Parameters}] \leavevmode
\sphinxstyleliteralstrong{\sphinxupquote{required}} \textendash{} (bool) Do nothing if required is False

\end{description}\end{quote}

\end{fulllineitems}

\index{get() (src.fileEnviron.BatFileEnviron method)}

\begin{fulllineitems}
\phantomsection\label{\detokenize{apidoc_src/src:src.fileEnviron.BatFileEnviron.get}}\pysiglinewithargsret{\sphinxbfcode{\sphinxupquote{get}}}{\emph{key}}{}
Get the value of the environment variable “key”
\begin{quote}\begin{description}
\item[{Parameters}] \leavevmode
\sphinxstyleliteralstrong{\sphinxupquote{key}} \textendash{} (str) the environment variable

\end{description}\end{quote}

\end{fulllineitems}

\index{set() (src.fileEnviron.BatFileEnviron method)}

\begin{fulllineitems}
\phantomsection\label{\detokenize{apidoc_src/src:src.fileEnviron.BatFileEnviron.set}}\pysiglinewithargsret{\sphinxbfcode{\sphinxupquote{set}}}{\emph{key}, \emph{value}}{}
Set the environment variable “key” to value “value”
\begin{quote}\begin{description}
\item[{Parameters}] \leavevmode\begin{itemize}
\item {} 
\sphinxstyleliteralstrong{\sphinxupquote{key}} \textendash{} (str) the environment variable to set

\item {} 
\sphinxstyleliteralstrong{\sphinxupquote{value}} \textendash{} (str) the value

\end{itemize}

\end{description}\end{quote}

\end{fulllineitems}


\end{fulllineitems}

\index{ContextFileEnviron (class in src.fileEnviron)}

\begin{fulllineitems}
\phantomsection\label{\detokenize{apidoc_src/src:src.fileEnviron.ContextFileEnviron}}\pysiglinewithargsret{\sphinxbfcode{\sphinxupquote{class }}\sphinxcode{\sphinxupquote{src.fileEnviron.}}\sphinxbfcode{\sphinxupquote{ContextFileEnviron}}}{\emph{output}, \emph{environ=None}}{}
Bases: {\hyperref[\detokenize{apidoc_src/src:src.fileEnviron.FileEnviron}]{\sphinxcrossref{\sphinxcode{\sphinxupquote{src.fileEnviron.FileEnviron}}}}} (\autopageref*{\detokenize{apidoc_src/src:src.fileEnviron.FileEnviron}})

Class for a salome context configuration file.
\index{add\_echo() (src.fileEnviron.ContextFileEnviron method)}

\begin{fulllineitems}
\phantomsection\label{\detokenize{apidoc_src/src:src.fileEnviron.ContextFileEnviron.add_echo}}\pysiglinewithargsret{\sphinxbfcode{\sphinxupquote{add\_echo}}}{\emph{text}}{}
Add a comment
\begin{quote}\begin{description}
\item[{Parameters}] \leavevmode
\sphinxstyleliteralstrong{\sphinxupquote{text}} \textendash{} (str) the comment to add

\end{description}\end{quote}

\end{fulllineitems}

\index{add\_warning() (src.fileEnviron.ContextFileEnviron method)}

\begin{fulllineitems}
\phantomsection\label{\detokenize{apidoc_src/src:src.fileEnviron.ContextFileEnviron.add_warning}}\pysiglinewithargsret{\sphinxbfcode{\sphinxupquote{add\_warning}}}{\emph{warning}}{}
Add a warning
\begin{quote}\begin{description}
\item[{Parameters}] \leavevmode
\sphinxstyleliteralstrong{\sphinxupquote{text}} \textendash{} (str) the warning to add

\end{description}\end{quote}

\end{fulllineitems}

\index{append\_value() (src.fileEnviron.ContextFileEnviron method)}

\begin{fulllineitems}
\phantomsection\label{\detokenize{apidoc_src/src:src.fileEnviron.ContextFileEnviron.append_value}}\pysiglinewithargsret{\sphinxbfcode{\sphinxupquote{append\_value}}}{\emph{key}, \emph{value}, \emph{sep=':'}}{}
append value to key using sep
\begin{quote}\begin{description}
\item[{Parameters}] \leavevmode\begin{itemize}
\item {} 
\sphinxstyleliteralstrong{\sphinxupquote{key}} \textendash{} (str) the environment variable to append

\item {} 
\sphinxstyleliteralstrong{\sphinxupquote{value}} \textendash{} (str) the value to append to key

\item {} 
\sphinxstyleliteralstrong{\sphinxupquote{sep}} \textendash{} (str) the separator string

\end{itemize}

\end{description}\end{quote}

\end{fulllineitems}

\index{command\_value() (src.fileEnviron.ContextFileEnviron method)}

\begin{fulllineitems}
\phantomsection\label{\detokenize{apidoc_src/src:src.fileEnviron.ContextFileEnviron.command_value}}\pysiglinewithargsret{\sphinxbfcode{\sphinxupquote{command\_value}}}{\emph{key}, \emph{command}}{}
Get the value given by the system command “command” 
and put it in the environment variable key.
Has to be overwritten in the derived classes
This can be seen as a virtual method
\begin{quote}\begin{description}
\item[{Parameters}] \leavevmode\begin{itemize}
\item {} 
\sphinxstyleliteralstrong{\sphinxupquote{key}} \textendash{} (str) the environment variable

\item {} 
\sphinxstyleliteralstrong{\sphinxupquote{command}} \textendash{} (str) the command to execute

\end{itemize}

\end{description}\end{quote}

\end{fulllineitems}

\index{finish() (src.fileEnviron.ContextFileEnviron method)}

\begin{fulllineitems}
\phantomsection\label{\detokenize{apidoc_src/src:src.fileEnviron.ContextFileEnviron.finish}}\pysiglinewithargsret{\sphinxbfcode{\sphinxupquote{finish}}}{\emph{required=True}}{}
Add a final instruction in the out file (in case of file generation)
\begin{quote}\begin{description}
\item[{Parameters}] \leavevmode
\sphinxstyleliteralstrong{\sphinxupquote{required}} \textendash{} (bool) Do nothing if required is False

\end{description}\end{quote}

\end{fulllineitems}

\index{get() (src.fileEnviron.ContextFileEnviron method)}

\begin{fulllineitems}
\phantomsection\label{\detokenize{apidoc_src/src:src.fileEnviron.ContextFileEnviron.get}}\pysiglinewithargsret{\sphinxbfcode{\sphinxupquote{get}}}{\emph{key}}{}
Get the value of the environment variable “key”
\begin{quote}\begin{description}
\item[{Parameters}] \leavevmode
\sphinxstyleliteralstrong{\sphinxupquote{key}} \textendash{} (str) the environment variable

\end{description}\end{quote}

\end{fulllineitems}

\index{prepend\_value() (src.fileEnviron.ContextFileEnviron method)}

\begin{fulllineitems}
\phantomsection\label{\detokenize{apidoc_src/src:src.fileEnviron.ContextFileEnviron.prepend_value}}\pysiglinewithargsret{\sphinxbfcode{\sphinxupquote{prepend\_value}}}{\emph{key}, \emph{value}, \emph{sep=':'}}{}
prepend value to key using sep
\begin{quote}\begin{description}
\item[{Parameters}] \leavevmode\begin{itemize}
\item {} 
\sphinxstyleliteralstrong{\sphinxupquote{key}} \textendash{} (str) the environment variable to prepend

\item {} 
\sphinxstyleliteralstrong{\sphinxupquote{value}} \textendash{} (str) the value to prepend to key

\item {} 
\sphinxstyleliteralstrong{\sphinxupquote{sep}} \textendash{} (str) the separator string

\end{itemize}

\end{description}\end{quote}

\end{fulllineitems}

\index{set() (src.fileEnviron.ContextFileEnviron method)}

\begin{fulllineitems}
\phantomsection\label{\detokenize{apidoc_src/src:src.fileEnviron.ContextFileEnviron.set}}\pysiglinewithargsret{\sphinxbfcode{\sphinxupquote{set}}}{\emph{key}, \emph{value}}{}
Set the environment variable “key” to value “value”
\begin{quote}\begin{description}
\item[{Parameters}] \leavevmode\begin{itemize}
\item {} 
\sphinxstyleliteralstrong{\sphinxupquote{key}} \textendash{} (str) the environment variable to set

\item {} 
\sphinxstyleliteralstrong{\sphinxupquote{value}} \textendash{} (str) the value

\end{itemize}

\end{description}\end{quote}

\end{fulllineitems}


\end{fulllineitems}

\index{FileEnviron (class in src.fileEnviron)}

\begin{fulllineitems}
\phantomsection\label{\detokenize{apidoc_src/src:src.fileEnviron.FileEnviron}}\pysiglinewithargsret{\sphinxbfcode{\sphinxupquote{class }}\sphinxcode{\sphinxupquote{src.fileEnviron.}}\sphinxbfcode{\sphinxupquote{FileEnviron}}}{\emph{output}, \emph{environ=None}}{}
Base class for shell environment
\index{add\_comment() (src.fileEnviron.FileEnviron method)}

\begin{fulllineitems}
\phantomsection\label{\detokenize{apidoc_src/src:src.fileEnviron.FileEnviron.add_comment}}\pysiglinewithargsret{\sphinxbfcode{\sphinxupquote{add\_comment}}}{\emph{comment}}{}
Add a comment in the shell file
\begin{quote}\begin{description}
\item[{Parameters}] \leavevmode
\sphinxstyleliteralstrong{\sphinxupquote{comment}} \textendash{} (str) the comment to add

\end{description}\end{quote}

\end{fulllineitems}

\index{add\_echo() (src.fileEnviron.FileEnviron method)}

\begin{fulllineitems}
\phantomsection\label{\detokenize{apidoc_src/src:src.fileEnviron.FileEnviron.add_echo}}\pysiglinewithargsret{\sphinxbfcode{\sphinxupquote{add\_echo}}}{\emph{text}}{}
Add a ‘echo’ in the shell file
\begin{quote}\begin{description}
\item[{Parameters}] \leavevmode
\sphinxstyleliteralstrong{\sphinxupquote{text}} \textendash{} (str) the text to echo

\end{description}\end{quote}

\end{fulllineitems}

\index{add\_line() (src.fileEnviron.FileEnviron method)}

\begin{fulllineitems}
\phantomsection\label{\detokenize{apidoc_src/src:src.fileEnviron.FileEnviron.add_line}}\pysiglinewithargsret{\sphinxbfcode{\sphinxupquote{add\_line}}}{\emph{number}}{}
Add some empty lines in the shell file
\begin{quote}\begin{description}
\item[{Parameters}] \leavevmode
\sphinxstyleliteralstrong{\sphinxupquote{number}} \textendash{} (int) the number of lines to add

\end{description}\end{quote}

\end{fulllineitems}

\index{add\_warning() (src.fileEnviron.FileEnviron method)}

\begin{fulllineitems}
\phantomsection\label{\detokenize{apidoc_src/src:src.fileEnviron.FileEnviron.add_warning}}\pysiglinewithargsret{\sphinxbfcode{\sphinxupquote{add\_warning}}}{\emph{warning}}{}
Add a warning “echo” in the shell file
\begin{quote}\begin{description}
\item[{Parameters}] \leavevmode
\sphinxstyleliteralstrong{\sphinxupquote{warning}} \textendash{} (str) the text to echo

\end{description}\end{quote}

\end{fulllineitems}

\index{append() (src.fileEnviron.FileEnviron method)}

\begin{fulllineitems}
\phantomsection\label{\detokenize{apidoc_src/src:src.fileEnviron.FileEnviron.append}}\pysiglinewithargsret{\sphinxbfcode{\sphinxupquote{append}}}{\emph{key}, \emph{value}, \emph{sep=':'}}{}
Same as append\_value but the value argument can be a list
\begin{quote}\begin{description}
\item[{Parameters}] \leavevmode\begin{itemize}
\item {} 
\sphinxstyleliteralstrong{\sphinxupquote{key}} \textendash{} (str) the environment variable to append

\item {} 
\sphinxstyleliteralstrong{\sphinxupquote{value}} \textendash{} (str or list) the value(s) to append to key

\item {} 
\sphinxstyleliteralstrong{\sphinxupquote{sep}} \textendash{} (str) the separator string

\end{itemize}

\end{description}\end{quote}

\end{fulllineitems}

\index{append\_value() (src.fileEnviron.FileEnviron method)}

\begin{fulllineitems}
\phantomsection\label{\detokenize{apidoc_src/src:src.fileEnviron.FileEnviron.append_value}}\pysiglinewithargsret{\sphinxbfcode{\sphinxupquote{append\_value}}}{\emph{key}, \emph{value}, \emph{sep=':'}}{}
append value to key using sep
\begin{quote}\begin{description}
\item[{Parameters}] \leavevmode\begin{itemize}
\item {} 
\sphinxstyleliteralstrong{\sphinxupquote{key}} \textendash{} (str) the environment variable to append

\item {} 
\sphinxstyleliteralstrong{\sphinxupquote{value}} \textendash{} (str) the value to append to key

\item {} 
\sphinxstyleliteralstrong{\sphinxupquote{sep}} \textendash{} (str) the separator string

\end{itemize}

\end{description}\end{quote}

\end{fulllineitems}

\index{command\_value() (src.fileEnviron.FileEnviron method)}

\begin{fulllineitems}
\phantomsection\label{\detokenize{apidoc_src/src:src.fileEnviron.FileEnviron.command_value}}\pysiglinewithargsret{\sphinxbfcode{\sphinxupquote{command\_value}}}{\emph{key}, \emph{command}}{}
Get the value given by the system command “command” 
and put it in the environment variable key.
Has to be overwritten in the derived classes
This can be seen as a virtual method
\begin{quote}\begin{description}
\item[{Parameters}] \leavevmode\begin{itemize}
\item {} 
\sphinxstyleliteralstrong{\sphinxupquote{key}} \textendash{} (str) the environment variable

\item {} 
\sphinxstyleliteralstrong{\sphinxupquote{command}} \textendash{} (str) the command to execute

\end{itemize}

\end{description}\end{quote}

\end{fulllineitems}

\index{finish() (src.fileEnviron.FileEnviron method)}

\begin{fulllineitems}
\phantomsection\label{\detokenize{apidoc_src/src:src.fileEnviron.FileEnviron.finish}}\pysiglinewithargsret{\sphinxbfcode{\sphinxupquote{finish}}}{\emph{required=True}}{}
Add a final instruction in the out file (in case of file generation)
\begin{quote}\begin{description}
\item[{Parameters}] \leavevmode
\sphinxstyleliteralstrong{\sphinxupquote{required}} \textendash{} (bool) Do nothing if required is False

\end{description}\end{quote}

\end{fulllineitems}

\index{get() (src.fileEnviron.FileEnviron method)}

\begin{fulllineitems}
\phantomsection\label{\detokenize{apidoc_src/src:src.fileEnviron.FileEnviron.get}}\pysiglinewithargsret{\sphinxbfcode{\sphinxupquote{get}}}{\emph{key}}{}
Get the value of the environment variable “key”
\begin{quote}\begin{description}
\item[{Parameters}] \leavevmode
\sphinxstyleliteralstrong{\sphinxupquote{key}} \textendash{} (str) the environment variable

\end{description}\end{quote}

\end{fulllineitems}

\index{is\_defined() (src.fileEnviron.FileEnviron method)}

\begin{fulllineitems}
\phantomsection\label{\detokenize{apidoc_src/src:src.fileEnviron.FileEnviron.is_defined}}\pysiglinewithargsret{\sphinxbfcode{\sphinxupquote{is\_defined}}}{\emph{key}}{}
Check if the key exists in the environment
\begin{quote}\begin{description}
\item[{Parameters}] \leavevmode
\sphinxstyleliteralstrong{\sphinxupquote{key}} \textendash{} (str) the environment variable to check

\end{description}\end{quote}

\end{fulllineitems}

\index{prepend() (src.fileEnviron.FileEnviron method)}

\begin{fulllineitems}
\phantomsection\label{\detokenize{apidoc_src/src:src.fileEnviron.FileEnviron.prepend}}\pysiglinewithargsret{\sphinxbfcode{\sphinxupquote{prepend}}}{\emph{key}, \emph{value}, \emph{sep=':'}}{}
Same as prepend\_value but the value argument can be a list
\begin{quote}\begin{description}
\item[{Parameters}] \leavevmode\begin{itemize}
\item {} 
\sphinxstyleliteralstrong{\sphinxupquote{key}} \textendash{} (str) the environment variable to prepend

\item {} 
\sphinxstyleliteralstrong{\sphinxupquote{value}} \textendash{} (str or list) the value(s) to prepend to key

\item {} 
\sphinxstyleliteralstrong{\sphinxupquote{sep}} \textendash{} (str) the separator string

\end{itemize}

\end{description}\end{quote}

\end{fulllineitems}

\index{prepend\_value() (src.fileEnviron.FileEnviron method)}

\begin{fulllineitems}
\phantomsection\label{\detokenize{apidoc_src/src:src.fileEnviron.FileEnviron.prepend_value}}\pysiglinewithargsret{\sphinxbfcode{\sphinxupquote{prepend\_value}}}{\emph{key}, \emph{value}, \emph{sep=':'}}{}
prepend value to key using sep
\begin{quote}\begin{description}
\item[{Parameters}] \leavevmode\begin{itemize}
\item {} 
\sphinxstyleliteralstrong{\sphinxupquote{key}} \textendash{} (str) the environment variable to prepend

\item {} 
\sphinxstyleliteralstrong{\sphinxupquote{value}} \textendash{} str) the value to prepend to key

\item {} 
\sphinxstyleliteralstrong{\sphinxupquote{sep}} \textendash{} (str) the separator string

\end{itemize}

\end{description}\end{quote}

\end{fulllineitems}

\index{set() (src.fileEnviron.FileEnviron method)}

\begin{fulllineitems}
\phantomsection\label{\detokenize{apidoc_src/src:src.fileEnviron.FileEnviron.set}}\pysiglinewithargsret{\sphinxbfcode{\sphinxupquote{set}}}{\emph{key}, \emph{value}}{}
Set the environment variable “key” to value “value”
\begin{quote}\begin{description}
\item[{Parameters}] \leavevmode\begin{itemize}
\item {} 
\sphinxstyleliteralstrong{\sphinxupquote{key}} \textendash{} (str) the environment variable to set

\item {} 
\sphinxstyleliteralstrong{\sphinxupquote{value}} \textendash{} (str) the value

\end{itemize}

\end{description}\end{quote}

\end{fulllineitems}


\end{fulllineitems}

\index{LauncherFileEnviron (class in src.fileEnviron)}

\begin{fulllineitems}
\phantomsection\label{\detokenize{apidoc_src/src:src.fileEnviron.LauncherFileEnviron}}\pysiglinewithargsret{\sphinxbfcode{\sphinxupquote{class }}\sphinxcode{\sphinxupquote{src.fileEnviron.}}\sphinxbfcode{\sphinxupquote{LauncherFileEnviron}}}{\emph{output}, \emph{environ=None}}{}
Class to generate a launcher file script 
(in python syntax) SalomeContext API
\index{add() (src.fileEnviron.LauncherFileEnviron method)}

\begin{fulllineitems}
\phantomsection\label{\detokenize{apidoc_src/src:src.fileEnviron.LauncherFileEnviron.add}}\pysiglinewithargsret{\sphinxbfcode{\sphinxupquote{add}}}{\emph{key}, \emph{value}}{}
prepend value to key using sep
\begin{quote}\begin{description}
\item[{Parameters}] \leavevmode\begin{itemize}
\item {} 
\sphinxstyleliteralstrong{\sphinxupquote{key}} \textendash{} (str) the environment variable to prepend

\item {} 
\sphinxstyleliteralstrong{\sphinxupquote{value}} \textendash{} (str) the value to prepend to key

\end{itemize}

\end{description}\end{quote}

\end{fulllineitems}

\index{add\_comment() (src.fileEnviron.LauncherFileEnviron method)}

\begin{fulllineitems}
\phantomsection\label{\detokenize{apidoc_src/src:src.fileEnviron.LauncherFileEnviron.add_comment}}\pysiglinewithargsret{\sphinxbfcode{\sphinxupquote{add\_comment}}}{\emph{comment}}{}
\end{fulllineitems}

\index{add\_echo() (src.fileEnviron.LauncherFileEnviron method)}

\begin{fulllineitems}
\phantomsection\label{\detokenize{apidoc_src/src:src.fileEnviron.LauncherFileEnviron.add_echo}}\pysiglinewithargsret{\sphinxbfcode{\sphinxupquote{add\_echo}}}{\emph{text}}{}
Add a comment
\begin{quote}\begin{description}
\item[{Parameters}] \leavevmode
\sphinxstyleliteralstrong{\sphinxupquote{text}} \textendash{} (str) the comment to add

\end{description}\end{quote}

\end{fulllineitems}

\index{add\_line() (src.fileEnviron.LauncherFileEnviron method)}

\begin{fulllineitems}
\phantomsection\label{\detokenize{apidoc_src/src:src.fileEnviron.LauncherFileEnviron.add_line}}\pysiglinewithargsret{\sphinxbfcode{\sphinxupquote{add\_line}}}{\emph{number}}{}
Add some empty lines in the launcher file
\begin{quote}\begin{description}
\item[{Parameters}] \leavevmode
\sphinxstyleliteralstrong{\sphinxupquote{number}} \textendash{} (int) the number of lines to add

\end{description}\end{quote}

\end{fulllineitems}

\index{add\_warning() (src.fileEnviron.LauncherFileEnviron method)}

\begin{fulllineitems}
\phantomsection\label{\detokenize{apidoc_src/src:src.fileEnviron.LauncherFileEnviron.add_warning}}\pysiglinewithargsret{\sphinxbfcode{\sphinxupquote{add\_warning}}}{\emph{warning}}{}
Add a warning
\begin{quote}\begin{description}
\item[{Parameters}] \leavevmode
\sphinxstyleliteralstrong{\sphinxupquote{text}} \textendash{} (str) the warning to add

\end{description}\end{quote}

\end{fulllineitems}

\index{append() (src.fileEnviron.LauncherFileEnviron method)}

\begin{fulllineitems}
\phantomsection\label{\detokenize{apidoc_src/src:src.fileEnviron.LauncherFileEnviron.append}}\pysiglinewithargsret{\sphinxbfcode{\sphinxupquote{append}}}{\emph{key}, \emph{value}, \emph{sep=':'}}{}
Same as append\_value but the value argument can be a list
\begin{quote}\begin{description}
\item[{Parameters}] \leavevmode\begin{itemize}
\item {} 
\sphinxstyleliteralstrong{\sphinxupquote{key}} \textendash{} (str) the environment variable to append

\item {} 
\sphinxstyleliteralstrong{\sphinxupquote{value}} \textendash{} (str or list) the value(s) to append to key

\item {} 
\sphinxstyleliteralstrong{\sphinxupquote{sep}} \textendash{} (str) the separator string

\end{itemize}

\end{description}\end{quote}

\end{fulllineitems}

\index{append\_value() (src.fileEnviron.LauncherFileEnviron method)}

\begin{fulllineitems}
\phantomsection\label{\detokenize{apidoc_src/src:src.fileEnviron.LauncherFileEnviron.append_value}}\pysiglinewithargsret{\sphinxbfcode{\sphinxupquote{append\_value}}}{\emph{key}, \emph{value}, \emph{sep=':'}}{}
append value to key using sep
\begin{quote}\begin{description}
\item[{Parameters}] \leavevmode\begin{itemize}
\item {} 
\sphinxstyleliteralstrong{\sphinxupquote{key}} \textendash{} (str) the environment variable to append

\item {} 
\sphinxstyleliteralstrong{\sphinxupquote{value}} \textendash{} (str) the value to append to key

\item {} 
\sphinxstyleliteralstrong{\sphinxupquote{sep}} \textendash{} (str) the separator string

\end{itemize}

\end{description}\end{quote}

\end{fulllineitems}

\index{change\_to\_launcher() (src.fileEnviron.LauncherFileEnviron method)}

\begin{fulllineitems}
\phantomsection\label{\detokenize{apidoc_src/src:src.fileEnviron.LauncherFileEnviron.change_to_launcher}}\pysiglinewithargsret{\sphinxbfcode{\sphinxupquote{change\_to\_launcher}}}{\emph{value}}{}
obsolete? do nothing

\end{fulllineitems}

\index{command\_value() (src.fileEnviron.LauncherFileEnviron method)}

\begin{fulllineitems}
\phantomsection\label{\detokenize{apidoc_src/src:src.fileEnviron.LauncherFileEnviron.command_value}}\pysiglinewithargsret{\sphinxbfcode{\sphinxupquote{command\_value}}}{\emph{key}, \emph{command}}{}
Get the value given by the system command “command” 
and put it in the environment variable key.
\begin{quote}\begin{description}
\item[{Parameters}] \leavevmode\begin{itemize}
\item {} 
\sphinxstyleliteralstrong{\sphinxupquote{key}} \textendash{} (str) the environment variable

\item {} 
\sphinxstyleliteralstrong{\sphinxupquote{command}} \textendash{} (str) the command to execute

\end{itemize}

\end{description}\end{quote}

\end{fulllineitems}

\index{finish() (src.fileEnviron.LauncherFileEnviron method)}

\begin{fulllineitems}
\phantomsection\label{\detokenize{apidoc_src/src:src.fileEnviron.LauncherFileEnviron.finish}}\pysiglinewithargsret{\sphinxbfcode{\sphinxupquote{finish}}}{\emph{required=True}}{}
Add a final instruction in the out file (in case of file generation)
In the particular launcher case, do nothing
\begin{quote}\begin{description}
\item[{Parameters}] \leavevmode
\sphinxstyleliteralstrong{\sphinxupquote{required}} \textendash{} (bool) Do nothing if required is False

\end{description}\end{quote}

\end{fulllineitems}

\index{get() (src.fileEnviron.LauncherFileEnviron method)}

\begin{fulllineitems}
\phantomsection\label{\detokenize{apidoc_src/src:src.fileEnviron.LauncherFileEnviron.get}}\pysiglinewithargsret{\sphinxbfcode{\sphinxupquote{get}}}{\emph{key}}{}
Get the value of the environment variable “key”
\begin{quote}\begin{description}
\item[{Parameters}] \leavevmode
\sphinxstyleliteralstrong{\sphinxupquote{key}} \textendash{} (str) the environment variable

\end{description}\end{quote}

\end{fulllineitems}

\index{is\_defined() (src.fileEnviron.LauncherFileEnviron method)}

\begin{fulllineitems}
\phantomsection\label{\detokenize{apidoc_src/src:src.fileEnviron.LauncherFileEnviron.is_defined}}\pysiglinewithargsret{\sphinxbfcode{\sphinxupquote{is\_defined}}}{\emph{key}}{}
Check if the key exists in the environment
\begin{quote}\begin{description}
\item[{Parameters}] \leavevmode
\sphinxstyleliteralstrong{\sphinxupquote{key}} \textendash{} (str) the environment variable to check

\end{description}\end{quote}

\end{fulllineitems}

\index{prepend() (src.fileEnviron.LauncherFileEnviron method)}

\begin{fulllineitems}
\phantomsection\label{\detokenize{apidoc_src/src:src.fileEnviron.LauncherFileEnviron.prepend}}\pysiglinewithargsret{\sphinxbfcode{\sphinxupquote{prepend}}}{\emph{key}, \emph{value}, \emph{sep=':'}}{}
Same as prepend\_value but the value argument can be a list
\begin{quote}\begin{description}
\item[{Parameters}] \leavevmode\begin{itemize}
\item {} 
\sphinxstyleliteralstrong{\sphinxupquote{key}} \textendash{} (str) the environment variable to prepend

\item {} 
\sphinxstyleliteralstrong{\sphinxupquote{value}} \textendash{} (str or list) the value(s) to prepend to key

\item {} 
\sphinxstyleliteralstrong{\sphinxupquote{sep}} \textendash{} (str) the separator string

\end{itemize}

\end{description}\end{quote}

\end{fulllineitems}

\index{prepend\_value() (src.fileEnviron.LauncherFileEnviron method)}

\begin{fulllineitems}
\phantomsection\label{\detokenize{apidoc_src/src:src.fileEnviron.LauncherFileEnviron.prepend_value}}\pysiglinewithargsret{\sphinxbfcode{\sphinxupquote{prepend\_value}}}{\emph{key}, \emph{value}, \emph{sep=':'}}{}
prepend value to key using sep
\begin{quote}\begin{description}
\item[{Parameters}] \leavevmode\begin{itemize}
\item {} 
\sphinxstyleliteralstrong{\sphinxupquote{key}} \textendash{} (str) the environment variable to prepend

\item {} 
\sphinxstyleliteralstrong{\sphinxupquote{value}} \textendash{} (str) the value to prepend to key

\item {} 
\sphinxstyleliteralstrong{\sphinxupquote{sep}} \textendash{} (str) the separator string

\end{itemize}

\end{description}\end{quote}

\end{fulllineitems}

\index{set() (src.fileEnviron.LauncherFileEnviron method)}

\begin{fulllineitems}
\phantomsection\label{\detokenize{apidoc_src/src:src.fileEnviron.LauncherFileEnviron.set}}\pysiglinewithargsret{\sphinxbfcode{\sphinxupquote{set}}}{\emph{key}, \emph{value}}{}
Set the environment variable “key” to value “value”
\begin{quote}\begin{description}
\item[{Parameters}] \leavevmode\begin{itemize}
\item {} 
\sphinxstyleliteralstrong{\sphinxupquote{key}} \textendash{} (str) the environment variable to set

\item {} 
\sphinxstyleliteralstrong{\sphinxupquote{value}} \textendash{} (str) the value

\end{itemize}

\end{description}\end{quote}

\end{fulllineitems}


\end{fulllineitems}

\index{ScreenEnviron (class in src.fileEnviron)}

\begin{fulllineitems}
\phantomsection\label{\detokenize{apidoc_src/src:src.fileEnviron.ScreenEnviron}}\pysiglinewithargsret{\sphinxbfcode{\sphinxupquote{class }}\sphinxcode{\sphinxupquote{src.fileEnviron.}}\sphinxbfcode{\sphinxupquote{ScreenEnviron}}}{\emph{output}, \emph{environ=None}}{}
Bases: {\hyperref[\detokenize{apidoc_src/src:src.fileEnviron.FileEnviron}]{\sphinxcrossref{\sphinxcode{\sphinxupquote{src.fileEnviron.FileEnviron}}}}} (\autopageref*{\detokenize{apidoc_src/src:src.fileEnviron.FileEnviron}})
\index{add\_comment() (src.fileEnviron.ScreenEnviron method)}

\begin{fulllineitems}
\phantomsection\label{\detokenize{apidoc_src/src:src.fileEnviron.ScreenEnviron.add_comment}}\pysiglinewithargsret{\sphinxbfcode{\sphinxupquote{add\_comment}}}{\emph{comment}}{}
\end{fulllineitems}

\index{add\_echo() (src.fileEnviron.ScreenEnviron method)}

\begin{fulllineitems}
\phantomsection\label{\detokenize{apidoc_src/src:src.fileEnviron.ScreenEnviron.add_echo}}\pysiglinewithargsret{\sphinxbfcode{\sphinxupquote{add\_echo}}}{\emph{text}}{}
\end{fulllineitems}

\index{add\_line() (src.fileEnviron.ScreenEnviron method)}

\begin{fulllineitems}
\phantomsection\label{\detokenize{apidoc_src/src:src.fileEnviron.ScreenEnviron.add_line}}\pysiglinewithargsret{\sphinxbfcode{\sphinxupquote{add\_line}}}{\emph{number}}{}
\end{fulllineitems}

\index{add\_warning() (src.fileEnviron.ScreenEnviron method)}

\begin{fulllineitems}
\phantomsection\label{\detokenize{apidoc_src/src:src.fileEnviron.ScreenEnviron.add_warning}}\pysiglinewithargsret{\sphinxbfcode{\sphinxupquote{add\_warning}}}{\emph{warning}}{}
\end{fulllineitems}

\index{append() (src.fileEnviron.ScreenEnviron method)}

\begin{fulllineitems}
\phantomsection\label{\detokenize{apidoc_src/src:src.fileEnviron.ScreenEnviron.append}}\pysiglinewithargsret{\sphinxbfcode{\sphinxupquote{append}}}{\emph{name}, \emph{value}, \emph{sep=':'}}{}
\end{fulllineitems}

\index{command\_value() (src.fileEnviron.ScreenEnviron method)}

\begin{fulllineitems}
\phantomsection\label{\detokenize{apidoc_src/src:src.fileEnviron.ScreenEnviron.command_value}}\pysiglinewithargsret{\sphinxbfcode{\sphinxupquote{command\_value}}}{\emph{key}, \emph{command}}{}
\end{fulllineitems}

\index{get() (src.fileEnviron.ScreenEnviron method)}

\begin{fulllineitems}
\phantomsection\label{\detokenize{apidoc_src/src:src.fileEnviron.ScreenEnviron.get}}\pysiglinewithargsret{\sphinxbfcode{\sphinxupquote{get}}}{\emph{name}}{}
\end{fulllineitems}

\index{is\_defined() (src.fileEnviron.ScreenEnviron method)}

\begin{fulllineitems}
\phantomsection\label{\detokenize{apidoc_src/src:src.fileEnviron.ScreenEnviron.is_defined}}\pysiglinewithargsret{\sphinxbfcode{\sphinxupquote{is\_defined}}}{\emph{name}}{}
\end{fulllineitems}

\index{prepend() (src.fileEnviron.ScreenEnviron method)}

\begin{fulllineitems}
\phantomsection\label{\detokenize{apidoc_src/src:src.fileEnviron.ScreenEnviron.prepend}}\pysiglinewithargsret{\sphinxbfcode{\sphinxupquote{prepend}}}{\emph{name}, \emph{value}, \emph{sep=':'}}{}
\end{fulllineitems}

\index{run\_env\_script() (src.fileEnviron.ScreenEnviron method)}

\begin{fulllineitems}
\phantomsection\label{\detokenize{apidoc_src/src:src.fileEnviron.ScreenEnviron.run_env_script}}\pysiglinewithargsret{\sphinxbfcode{\sphinxupquote{run\_env\_script}}}{\emph{module}, \emph{script}}{}
\end{fulllineitems}

\index{set() (src.fileEnviron.ScreenEnviron method)}

\begin{fulllineitems}
\phantomsection\label{\detokenize{apidoc_src/src:src.fileEnviron.ScreenEnviron.set}}\pysiglinewithargsret{\sphinxbfcode{\sphinxupquote{set}}}{\emph{name}, \emph{value}}{}
\end{fulllineitems}

\index{write() (src.fileEnviron.ScreenEnviron method)}

\begin{fulllineitems}
\phantomsection\label{\detokenize{apidoc_src/src:src.fileEnviron.ScreenEnviron.write}}\pysiglinewithargsret{\sphinxbfcode{\sphinxupquote{write}}}{\emph{command}, \emph{name}, \emph{value}, \emph{sign='='}}{}
\end{fulllineitems}


\end{fulllineitems}

\index{get\_file\_environ() (in module src.fileEnviron)}

\begin{fulllineitems}
\phantomsection\label{\detokenize{apidoc_src/src:src.fileEnviron.get_file_environ}}\pysiglinewithargsret{\sphinxcode{\sphinxupquote{src.fileEnviron.}}\sphinxbfcode{\sphinxupquote{get\_file\_environ}}}{\emph{output}, \emph{shell}, \emph{environ=None}}{}
Instantiate correct FileEnvironment sub-class.
\begin{quote}\begin{description}
\item[{Parameters}] \leavevmode\begin{itemize}
\item {} 
\sphinxstyleliteralstrong{\sphinxupquote{output}} \textendash{} (file) the output file stream.

\item {} 
\sphinxstyleliteralstrong{\sphinxupquote{shell}} \textendash{} (str) the type of shell syntax to use.

\item {} 
\sphinxstyleliteralstrong{\sphinxupquote{environ}} \textendash{} (dict) a potential additional environment.

\end{itemize}

\end{description}\end{quote}

\end{fulllineitems}

\index{special\_path\_separator() (in module src.fileEnviron)}

\begin{fulllineitems}
\phantomsection\label{\detokenize{apidoc_src/src:src.fileEnviron.special_path_separator}}\pysiglinewithargsret{\sphinxcode{\sphinxupquote{src.fileEnviron.}}\sphinxbfcode{\sphinxupquote{special\_path\_separator}}}{\emph{name}}{}
TCLLIBPATH, TKLIBPATH, PV\_PLUGIN\_PATH environments variables 
need some exotic path separator.
This function gives the separator regarding the name of the variable
to append or prepend.
\begin{quote}\begin{description}
\item[{Parameters}] \leavevmode
\sphinxstyleliteralstrong{\sphinxupquote{name}} \textendash{} (str) The name of the variable to find the separator

\end{description}\end{quote}

\end{fulllineitems}



\subsubsection{src.fork module}
\label{\detokenize{apidoc_src/src:module-src.fork}}\label{\detokenize{apidoc_src/src:src-fork-module}}\index{src.fork (module)}\index{batch() (in module src.fork)}

\begin{fulllineitems}
\phantomsection\label{\detokenize{apidoc_src/src:src.fork.batch}}\pysiglinewithargsret{\sphinxcode{\sphinxupquote{src.fork.}}\sphinxbfcode{\sphinxupquote{batch}}}{\emph{cmd}, \emph{logger}, \emph{cwd}, \emph{args={[}{]}}, \emph{log=None}, \emph{delai=20}, \emph{sommeil=1}}{}
Launch a batch

\end{fulllineitems}

\index{batch\_salome() (in module src.fork)}

\begin{fulllineitems}
\phantomsection\label{\detokenize{apidoc_src/src:src.fork.batch_salome}}\pysiglinewithargsret{\sphinxcode{\sphinxupquote{src.fork.}}\sphinxbfcode{\sphinxupquote{batch\_salome}}}{\emph{cmd}, \emph{logger}, \emph{cwd}, \emph{args}, \emph{getTmpDir}, \emph{pendant='SALOME\_Session\_Server'}, \emph{fin='killSalome.py'}, \emph{log=None}, \emph{delai=20}, \emph{sommeil=1}, \emph{delaiapp=0}}{}
Launch a salome process

\end{fulllineitems}

\index{launch\_command() (in module src.fork)}

\begin{fulllineitems}
\phantomsection\label{\detokenize{apidoc_src/src:src.fork.launch_command}}\pysiglinewithargsret{\sphinxcode{\sphinxupquote{src.fork.}}\sphinxbfcode{\sphinxupquote{launch\_command}}}{\emph{cmd}, \emph{logger}, \emph{cwd}, \emph{args={[}{]}}, \emph{log=None}}{}
Launch command

\end{fulllineitems}

\index{show\_progress() (in module src.fork)}

\begin{fulllineitems}
\phantomsection\label{\detokenize{apidoc_src/src:src.fork.show_progress}}\pysiglinewithargsret{\sphinxcode{\sphinxupquote{src.fork.}}\sphinxbfcode{\sphinxupquote{show\_progress}}}{\emph{logger}, \emph{top}, \emph{delai}, \emph{ss=''}}{}
shortcut function to display the progression
\begin{quote}\begin{description}
\item[{Parameters}] \leavevmode\begin{itemize}
\item {} 
\sphinxstyleliteralstrong{\sphinxupquote{logger}} \textendash{} (Logger) The logging instance

\item {} 
\sphinxstyleliteralstrong{\sphinxupquote{top}} \textendash{} (int) the number to display

\item {} 
\sphinxstyleliteralstrong{\sphinxupquote{delai}} \textendash{} (int) the number max

\item {} 
\sphinxstyleliteralstrong{\sphinxupquote{ss}} \textendash{} (str) the string to display

\end{itemize}

\end{description}\end{quote}

\end{fulllineitems}

\index{write\_back() (in module src.fork)}

\begin{fulllineitems}
\phantomsection\label{\detokenize{apidoc_src/src:src.fork.write_back}}\pysiglinewithargsret{\sphinxcode{\sphinxupquote{src.fork.}}\sphinxbfcode{\sphinxupquote{write\_back}}}{\emph{logger}, \emph{message}}{}
shortcut function to write at the begin of the line
\begin{quote}\begin{description}
\item[{Parameters}] \leavevmode\begin{itemize}
\item {} 
\sphinxstyleliteralstrong{\sphinxupquote{logger}} \textendash{} (Logger) The logging instance

\item {} 
\sphinxstyleliteralstrong{\sphinxupquote{message}} \textendash{} (str) the text to display

\item {} 
\sphinxstyleliteralstrong{\sphinxupquote{level}} \textendash{} (int) the level of verbosity

\end{itemize}

\end{description}\end{quote}

\end{fulllineitems}



\subsubsection{src.loggingSat module}
\label{\detokenize{apidoc_src/src:module-src.loggingSat}}\label{\detokenize{apidoc_src/src:src-loggingsat-module}}\index{src.loggingSat (module)}
\sphinxurl{http://sametmax.com/ecrire-des-logs-en-python/}
\sphinxurl{https://docs.python.org/3/library/time.html\#time.strftime}

use logging package for salometools
\begin{description}
\item[{handler:}] \leavevmode
on info() no format
on other formatted indented on multi lines messages

\end{description}
\index{DefaultFormatter (class in src.loggingSat)}

\begin{fulllineitems}
\phantomsection\label{\detokenize{apidoc_src/src:src.loggingSat.DefaultFormatter}}\pysiglinewithargsret{\sphinxbfcode{\sphinxupquote{class }}\sphinxcode{\sphinxupquote{src.loggingSat.}}\sphinxbfcode{\sphinxupquote{DefaultFormatter}}}{\emph{fmt=None}, \emph{datefmt=None}}{}
Bases: \sphinxcode{\sphinxupquote{logging.Formatter}}
\index{format() (src.loggingSat.DefaultFormatter method)}

\begin{fulllineitems}
\phantomsection\label{\detokenize{apidoc_src/src:src.loggingSat.DefaultFormatter.format}}\pysiglinewithargsret{\sphinxbfcode{\sphinxupquote{format}}}{\emph{record}}{}
Format the specified record as text.

The record’s attribute dictionary is used as the operand to a
string formatting operation which yields the returned string.
Before formatting the dictionary, a couple of preparatory steps
are carried out. The message attribute of the record is computed
using LogRecord.getMessage(). If the formatting string uses the
time (as determined by a call to usesTime(), formatTime() is
called to format the event time. If there is exception information,
it is formatted using formatException() and appended to the message.

\end{fulllineitems}

\index{setColorLevelname() (src.loggingSat.DefaultFormatter method)}

\begin{fulllineitems}
\phantomsection\label{\detokenize{apidoc_src/src:src.loggingSat.DefaultFormatter.setColorLevelname}}\pysiglinewithargsret{\sphinxbfcode{\sphinxupquote{setColorLevelname}}}{\emph{levelname}}{}
\end{fulllineitems}


\end{fulllineitems}

\index{UnittestFormatter (class in src.loggingSat)}

\begin{fulllineitems}
\phantomsection\label{\detokenize{apidoc_src/src:src.loggingSat.UnittestFormatter}}\pysiglinewithargsret{\sphinxbfcode{\sphinxupquote{class }}\sphinxcode{\sphinxupquote{src.loggingSat.}}\sphinxbfcode{\sphinxupquote{UnittestFormatter}}}{\emph{fmt=None}, \emph{datefmt=None}}{}
Bases: \sphinxcode{\sphinxupquote{logging.Formatter}}
\index{format() (src.loggingSat.UnittestFormatter method)}

\begin{fulllineitems}
\phantomsection\label{\detokenize{apidoc_src/src:src.loggingSat.UnittestFormatter.format}}\pysiglinewithargsret{\sphinxbfcode{\sphinxupquote{format}}}{\emph{record}}{}
Format the specified record as text.

The record’s attribute dictionary is used as the operand to a
string formatting operation which yields the returned string.
Before formatting the dictionary, a couple of preparatory steps
are carried out. The message attribute of the record is computed
using LogRecord.getMessage(). If the formatting string uses the
time (as determined by a call to usesTime(), formatTime() is
called to format the event time. If there is exception information,
it is formatted using formatException() and appended to the message.

\end{fulllineitems}


\end{fulllineitems}

\index{UnittestStream (class in src.loggingSat)}

\begin{fulllineitems}
\phantomsection\label{\detokenize{apidoc_src/src:src.loggingSat.UnittestStream}}\pysigline{\sphinxbfcode{\sphinxupquote{class }}\sphinxcode{\sphinxupquote{src.loggingSat.}}\sphinxbfcode{\sphinxupquote{UnittestStream}}}
Bases: \sphinxcode{\sphinxupquote{object}}

write my stream class
only write and flush are used for the streaming
\sphinxurl{https://docs.python.org/2/library/logging.handlers.html}
\sphinxurl{https://stackoverflow.com/questions/31999627/storing-logger-messages-in-a-string}
\index{flush() (src.loggingSat.UnittestStream method)}

\begin{fulllineitems}
\phantomsection\label{\detokenize{apidoc_src/src:src.loggingSat.UnittestStream.flush}}\pysiglinewithargsret{\sphinxbfcode{\sphinxupquote{flush}}}{}{}
\end{fulllineitems}

\index{getLogs() (src.loggingSat.UnittestStream method)}

\begin{fulllineitems}
\phantomsection\label{\detokenize{apidoc_src/src:src.loggingSat.UnittestStream.getLogs}}\pysiglinewithargsret{\sphinxbfcode{\sphinxupquote{getLogs}}}{}{}
\end{fulllineitems}

\index{getLogsAndClear() (src.loggingSat.UnittestStream method)}

\begin{fulllineitems}
\phantomsection\label{\detokenize{apidoc_src/src:src.loggingSat.UnittestStream.getLogsAndClear}}\pysiglinewithargsret{\sphinxbfcode{\sphinxupquote{getLogsAndClear}}}{}{}
\end{fulllineitems}

\index{write() (src.loggingSat.UnittestStream method)}

\begin{fulllineitems}
\phantomsection\label{\detokenize{apidoc_src/src:src.loggingSat.UnittestStream.write}}\pysiglinewithargsret{\sphinxbfcode{\sphinxupquote{write}}}{\emph{astr}}{}
\end{fulllineitems}


\end{fulllineitems}

\index{dirLogger() (in module src.loggingSat)}

\begin{fulllineitems}
\phantomsection\label{\detokenize{apidoc_src/src:src.loggingSat.dirLogger}}\pysiglinewithargsret{\sphinxcode{\sphinxupquote{src.loggingSat.}}\sphinxbfcode{\sphinxupquote{dirLogger}}}{\emph{logger}}{}
\end{fulllineitems}

\index{getDefaultLogger() (in module src.loggingSat)}

\begin{fulllineitems}
\phantomsection\label{\detokenize{apidoc_src/src:src.loggingSat.getDefaultLogger}}\pysiglinewithargsret{\sphinxcode{\sphinxupquote{src.loggingSat.}}\sphinxbfcode{\sphinxupquote{getDefaultLogger}}}{}{}
\end{fulllineitems}

\index{getUnittestLogger() (in module src.loggingSat)}

\begin{fulllineitems}
\phantomsection\label{\detokenize{apidoc_src/src:src.loggingSat.getUnittestLogger}}\pysiglinewithargsret{\sphinxcode{\sphinxupquote{src.loggingSat.}}\sphinxbfcode{\sphinxupquote{getUnittestLogger}}}{}{}
\end{fulllineitems}

\index{indent() (in module src.loggingSat)}

\begin{fulllineitems}
\phantomsection\label{\detokenize{apidoc_src/src:src.loggingSat.indent}}\pysiglinewithargsret{\sphinxcode{\sphinxupquote{src.loggingSat.}}\sphinxbfcode{\sphinxupquote{indent}}}{\emph{msg}, \emph{nb}, \emph{car=' '}}{}
indent nb car (spaces) multi lines message except first one

\end{fulllineitems}

\index{indentUnittest() (in module src.loggingSat)}

\begin{fulllineitems}
\phantomsection\label{\detokenize{apidoc_src/src:src.loggingSat.indentUnittest}}\pysiglinewithargsret{\sphinxcode{\sphinxupquote{src.loggingSat.}}\sphinxbfcode{\sphinxupquote{indentUnittest}}}{\emph{msg}, \emph{prefix=' \textbar{} '}}{}
indent car multi lines message except first one
car default is less spaces for size logs files
keep human readable

\end{fulllineitems}

\index{initLoggerAsDefault() (in module src.loggingSat)}

\begin{fulllineitems}
\phantomsection\label{\detokenize{apidoc_src/src:src.loggingSat.initLoggerAsDefault}}\pysiglinewithargsret{\sphinxcode{\sphinxupquote{src.loggingSat.}}\sphinxbfcode{\sphinxupquote{initLoggerAsDefault}}}{\emph{logger}, \emph{fmt=None}, \emph{level=None}}{}
init logger as prefixed message and indented message if multi line
exept info() outed ‘as it’ without any format

\end{fulllineitems}

\index{initLoggerAsUnittest() (in module src.loggingSat)}

\begin{fulllineitems}
\phantomsection\label{\detokenize{apidoc_src/src:src.loggingSat.initLoggerAsUnittest}}\pysiglinewithargsret{\sphinxcode{\sphinxupquote{src.loggingSat.}}\sphinxbfcode{\sphinxupquote{initLoggerAsUnittest}}}{\emph{logger}, \emph{fmt=None}, \emph{level=None}}{}
init logger as silent on stdout/stderr
used for retrieve messages in memory for post execution unittest
\sphinxurl{https://docs.python.org/2/library/logging.handlers.html}

\end{fulllineitems}

\index{log() (in module src.loggingSat)}

\begin{fulllineitems}
\phantomsection\label{\detokenize{apidoc_src/src:src.loggingSat.log}}\pysiglinewithargsret{\sphinxcode{\sphinxupquote{src.loggingSat.}}\sphinxbfcode{\sphinxupquote{log}}}{\emph{msg}}{}
elementary log when no logger yet

\end{fulllineitems}

\index{testLogger\_1() (in module src.loggingSat)}

\begin{fulllineitems}
\phantomsection\label{\detokenize{apidoc_src/src:src.loggingSat.testLogger_1}}\pysiglinewithargsret{\sphinxcode{\sphinxupquote{src.loggingSat.}}\sphinxbfcode{\sphinxupquote{testLogger\_1}}}{\emph{logger}}{}
small test

\end{fulllineitems}



\subsubsection{src.options module}
\label{\detokenize{apidoc_src/src:module-src.options}}\label{\detokenize{apidoc_src/src:src-options-module}}\index{src.options (module)}
The Options class that manages the access to all options passed as 
parameters in salomeTools command lines
\index{OptResult (class in src.options)}

\begin{fulllineitems}
\phantomsection\label{\detokenize{apidoc_src/src:src.options.OptResult}}\pysigline{\sphinxbfcode{\sphinxupquote{class }}\sphinxcode{\sphinxupquote{src.options.}}\sphinxbfcode{\sphinxupquote{OptResult}}}
Bases: \sphinxcode{\sphinxupquote{object}}

An instance of this class will be the object manipulated
in code of all salomeTools commands
The aim of this class is to have an elegant syntax to manipulate the options.

example: 
\textgreater{}\textgreater{} print(options.level)
\textgreater{}\textgreater{} 5

\end{fulllineitems}

\index{Options (class in src.options)}

\begin{fulllineitems}
\phantomsection\label{\detokenize{apidoc_src/src:src.options.Options}}\pysigline{\sphinxbfcode{\sphinxupquote{class }}\sphinxcode{\sphinxupquote{src.options.}}\sphinxbfcode{\sphinxupquote{Options}}}
Bases: \sphinxcode{\sphinxupquote{object}}

Class to manage all salomeTools options
\index{add\_option() (src.options.Options method)}

\begin{fulllineitems}
\phantomsection\label{\detokenize{apidoc_src/src:src.options.Options.add_option}}\pysiglinewithargsret{\sphinxbfcode{\sphinxupquote{add\_option}}}{\emph{shortName}, \emph{longName}, \emph{optionType}, \emph{destName}, \emph{helpString=''}, \emph{default=None}}{}
Add an option to a command. It gets all attributes
of an option and append it in the options field
\begin{quote}\begin{description}
\item[{Parameters}] \leavevmode\begin{itemize}
\item {} 
\sphinxstyleliteralstrong{\sphinxupquote{shortName}} \textendash{} (str) 
The short name of the option (as ‘-l’ for level option).

\item {} 
\sphinxstyleliteralstrong{\sphinxupquote{longName}} \textendash{} (str) 
The long name of the option (as ‘\textendash{}level’ for level option).

\item {} 
\sphinxstyleliteralstrong{\sphinxupquote{optionType}} \textendash{} (str) The type of the option (ex “int”).

\item {} 
\sphinxstyleliteralstrong{\sphinxupquote{destName}} \textendash{} (str) The name that will be used in the code.

\item {} 
\sphinxstyleliteralstrong{\sphinxupquote{helpString}} \textendash{} (str) 
The text to display when user ask for help on a command.

\end{itemize}

\item[{Returns}] \leavevmode
None

\end{description}\end{quote}

\end{fulllineitems}

\index{debug\_write() (src.options.Options method)}

\begin{fulllineitems}
\phantomsection\label{\detokenize{apidoc_src/src:src.options.Options.debug_write}}\pysiglinewithargsret{\sphinxbfcode{\sphinxupquote{debug\_write}}}{}{}
\end{fulllineitems}

\index{getDetailOption() (src.options.Options method)}

\begin{fulllineitems}
\phantomsection\label{\detokenize{apidoc_src/src:src.options.Options.getDetailOption}}\pysiglinewithargsret{\sphinxbfcode{\sphinxupquote{getDetailOption}}}{\emph{option}}{}
for convenience
\begin{quote}\begin{description}
\item[{Returns}] \leavevmode
(tuple) 4-elements (shortName, longName, optionType, helpString)

\end{description}\end{quote}

\end{fulllineitems}

\index{get\_help() (src.options.Options method)}

\begin{fulllineitems}
\phantomsection\label{\detokenize{apidoc_src/src:src.options.Options.get_help}}\pysiglinewithargsret{\sphinxbfcode{\sphinxupquote{get\_help}}}{}{}
Returns all options stored in self.options 
as help message colored string
\begin{quote}\begin{description}
\item[{Returns}] \leavevmode
(str) colored string

\end{description}\end{quote}

\end{fulllineitems}

\index{indent() (src.options.Options method)}

\begin{fulllineitems}
\phantomsection\label{\detokenize{apidoc_src/src:src.options.Options.indent}}\pysiglinewithargsret{\sphinxbfcode{\sphinxupquote{indent}}}{\emph{text}, \emph{amount}, \emph{car=' '}}{}
indent multi lines message

\end{fulllineitems}

\index{parse\_args() (src.options.Options method)}

\begin{fulllineitems}
\phantomsection\label{\detokenize{apidoc_src/src:src.options.Options.parse_args}}\pysiglinewithargsret{\sphinxbfcode{\sphinxupquote{parse\_args}}}{\emph{argList=None}}{}
Instantiates the class OptResult 
that gives access to all options in the code
\begin{quote}\begin{description}
\item[{Parameters}] \leavevmode
\sphinxstyleliteralstrong{\sphinxupquote{argList}} \textendash{} (list) the raw list of arguments that were passed

\item[{Returns}] \leavevmode
(OptResult, list) as (optResult, args) 
optResult is the option instance to manipulate in the code. 
args is the full raw list of passed options

\end{description}\end{quote}

\end{fulllineitems}


\end{fulllineitems}



\subsubsection{src.product module}
\label{\detokenize{apidoc_src/src:module-src.product}}\label{\detokenize{apidoc_src/src:src-product-module}}\index{src.product (module)}
Contains the methods 
relative to the product notion of salomeTools
\index{check\_config\_exists() (in module src.product)}

\begin{fulllineitems}
\phantomsection\label{\detokenize{apidoc_src/src:src.product.check_config_exists}}\pysiglinewithargsret{\sphinxcode{\sphinxupquote{src.product.}}\sphinxbfcode{\sphinxupquote{check\_config\_exists}}}{\emph{config}, \emph{prod\_dir}, \emph{prod\_info}}{}
Verify that the installation directory of a product in a base exists
Check all the config-\textless{}i\textgreater{} directory and verify the sat-config.pyconf file
that is in it
\begin{quote}\begin{description}
\item[{Parameters}] \leavevmode\begin{itemize}
\item {} 
\sphinxstyleliteralstrong{\sphinxupquote{config}} \textendash{} (Config) The global configuration

\item {} 
\sphinxstyleliteralstrong{\sphinxupquote{prod\_dir}} \textendash{} (str) 
The product installation directory path 
(without config-\textless{}i\textgreater{})

\item {} 
\sphinxstyleliteralstrong{\sphinxupquote{product\_info}} \textendash{} (Config)
The configuration specific to the product

\end{itemize}

\item[{Returns}] \leavevmode
(tuple) as (boolean, str)
True or false is the installation is found or not 
and if it is found, the path of the found installation

\end{description}\end{quote}

\end{fulllineitems}

\index{check\_installation() (in module src.product)}

\begin{fulllineitems}
\phantomsection\label{\detokenize{apidoc_src/src:src.product.check_installation}}\pysiglinewithargsret{\sphinxcode{\sphinxupquote{src.product.}}\sphinxbfcode{\sphinxupquote{check\_installation}}}{\emph{product\_info}}{}
Verify if a product is well installed. 
Checks install directory presence
and some additional files if it is defined in the config
\begin{quote}\begin{description}
\item[{Parameters}] \leavevmode
\sphinxstyleliteralstrong{\sphinxupquote{product\_info}} \textendash{} (Config) 
The configuration specific to the product

\item[{Returns}] \leavevmode
(bool) True if it is well installed

\end{description}\end{quote}

\end{fulllineitems}

\index{get\_base\_install\_dir() (in module src.product)}

\begin{fulllineitems}
\phantomsection\label{\detokenize{apidoc_src/src:src.product.get_base_install_dir}}\pysiglinewithargsret{\sphinxcode{\sphinxupquote{src.product.}}\sphinxbfcode{\sphinxupquote{get\_base\_install\_dir}}}{\emph{config}, \emph{prod\_info}, \emph{version}}{}
Compute the installation directory of a product in base
\begin{quote}\begin{description}
\item[{Parameters}] \leavevmode\begin{itemize}
\item {} 
\sphinxstyleliteralstrong{\sphinxupquote{config}} \textendash{} (Config) The global configuration

\item {} 
\sphinxstyleliteralstrong{\sphinxupquote{product\_info}} \textendash{} (Config) 
The configuration specific to the product

\item {} 
\sphinxstyleliteralstrong{\sphinxupquote{version}} \textendash{} (str) The version of the product

\end{itemize}

\item[{Returns}] \leavevmode
(str) The path of the product installation

\end{description}\end{quote}

\end{fulllineitems}

\index{get\_install\_dir() (in module src.product)}

\begin{fulllineitems}
\phantomsection\label{\detokenize{apidoc_src/src:src.product.get_install_dir}}\pysiglinewithargsret{\sphinxcode{\sphinxupquote{src.product.}}\sphinxbfcode{\sphinxupquote{get\_install\_dir}}}{\emph{config}, \emph{base}, \emph{version}, \emph{prod\_info}}{}
Compute the installation directory of a given product
\begin{quote}\begin{description}
\item[{Parameters}] \leavevmode\begin{itemize}
\item {} 
\sphinxstyleliteralstrong{\sphinxupquote{config}} \textendash{} (Config) The global configuration

\item {} 
\sphinxstyleliteralstrong{\sphinxupquote{base}} \textendash{} (str) 
This corresponds to the value given by user in its application.pyconf 
for the specific product. 
If “yes”, the user wants the product to be in base. 
If “no”, he wants the product to be in the application workdir

\item {} 
\sphinxstyleliteralstrong{\sphinxupquote{version}} \textendash{} (str) The version of the product

\item {} 
\sphinxstyleliteralstrong{\sphinxupquote{product\_info}} \textendash{} (Config) The configuration specific to the product

\end{itemize}

\item[{Returns}] \leavevmode
(str) The path of the product installation

\end{description}\end{quote}

\end{fulllineitems}

\index{get\_product\_components() (in module src.product)}

\begin{fulllineitems}
\phantomsection\label{\detokenize{apidoc_src/src:src.product.get_product_components}}\pysiglinewithargsret{\sphinxcode{\sphinxupquote{src.product.}}\sphinxbfcode{\sphinxupquote{get\_product\_components}}}{\emph{product\_info}}{}
Get the component list to generate with the product
\begin{quote}\begin{description}
\item[{Parameters}] \leavevmode
\sphinxstyleliteralstrong{\sphinxupquote{product\_info}} \textendash{} (Config)
The configuration specific to the product

\item[{Returns}] \leavevmode
(list) The list of names of the components

\end{description}\end{quote}

\end{fulllineitems}

\index{get\_product\_config() (in module src.product)}

\begin{fulllineitems}
\phantomsection\label{\detokenize{apidoc_src/src:src.product.get_product_config}}\pysiglinewithargsret{\sphinxcode{\sphinxupquote{src.product.}}\sphinxbfcode{\sphinxupquote{get\_product\_config}}}{\emph{config}, \emph{product\_name}, \emph{with\_install\_dir=True}}{}
Get the specific configuration of a product 
from the global configuration
\begin{quote}\begin{description}
\item[{Parameters}] \leavevmode\begin{itemize}
\item {} 
\sphinxstyleliteralstrong{\sphinxupquote{config}} \textendash{} (Config) The global configuration

\item {} 
\sphinxstyleliteralstrong{\sphinxupquote{product\_name}} \textendash{} (str) The name of the product

\item {} 
\sphinxstyleliteralstrong{\sphinxupquote{with\_install\_dir}} \textendash{} (boolean)
If false, do not provide an install directory 
(at false only for internal use of the function check\_config\_exists)

\end{itemize}

\item[{Returns}] \leavevmode
(Config) The specific configuration of the product

\end{description}\end{quote}

\end{fulllineitems}

\index{get\_product\_dependencies() (in module src.product)}

\begin{fulllineitems}
\phantomsection\label{\detokenize{apidoc_src/src:src.product.get_product_dependencies}}\pysiglinewithargsret{\sphinxcode{\sphinxupquote{src.product.}}\sphinxbfcode{\sphinxupquote{get\_product\_dependencies}}}{\emph{config}, \emph{product\_info}}{}
Get recursively the list of products that are 
in the product\_info dependencies
\begin{quote}\begin{description}
\item[{Parameters}] \leavevmode\begin{itemize}
\item {} 
\sphinxstyleliteralstrong{\sphinxupquote{config}} \textendash{} (Config) The global configuration

\item {} 
\sphinxstyleliteralstrong{\sphinxupquote{product\_info}} \textendash{} (Config) 
The configuration specific to the product

\end{itemize}

\item[{Returns}] \leavevmode
(list) the list of products in dependence

\end{description}\end{quote}

\end{fulllineitems}

\index{get\_product\_section() (in module src.product)}

\begin{fulllineitems}
\phantomsection\label{\detokenize{apidoc_src/src:src.product.get_product_section}}\pysiglinewithargsret{\sphinxcode{\sphinxupquote{src.product.}}\sphinxbfcode{\sphinxupquote{get\_product\_section}}}{\emph{config}, \emph{product\_name}, \emph{version}, \emph{section=None}}{}
Get the product description from the configuration
\begin{quote}\begin{description}
\item[{Parameters}] \leavevmode\begin{itemize}
\item {} 
\sphinxstyleliteralstrong{\sphinxupquote{config}} \textendash{} (Config) The global configuration

\item {} 
\sphinxstyleliteralstrong{\sphinxupquote{product\_name}} \textendash{} (str) The product name

\item {} 
\sphinxstyleliteralstrong{\sphinxupquote{version}} \textendash{} (str) The version of the product

\item {} 
\sphinxstyleliteralstrong{\sphinxupquote{section}} \textendash{} (str) 
The searched section
(if not None, the section is explicitly given)

\end{itemize}

\item[{Returns}] \leavevmode
(Config) The product description

\end{description}\end{quote}

\end{fulllineitems}

\index{get\_products\_infos() (in module src.product)}

\begin{fulllineitems}
\phantomsection\label{\detokenize{apidoc_src/src:src.product.get_products_infos}}\pysiglinewithargsret{\sphinxcode{\sphinxupquote{src.product.}}\sphinxbfcode{\sphinxupquote{get\_products\_infos}}}{\emph{products}, \emph{config}}{}
Get the specific configuration of a list of products
\begin{quote}\begin{description}
\item[{Parameters}] \leavevmode\begin{itemize}
\item {} 
\sphinxstyleliteralstrong{\sphinxupquote{products}} \textendash{} (list) The list of product names

\item {} 
\sphinxstyleliteralstrong{\sphinxupquote{config}} \textendash{} (Config) The global configuration

\end{itemize}

\item[{Returns}] \leavevmode
(list) of tuples (str, Config)
as (product name, specific configuration of the product)

\end{description}\end{quote}

\end{fulllineitems}

\index{product\_compiles() (in module src.product)}

\begin{fulllineitems}
\phantomsection\label{\detokenize{apidoc_src/src:src.product.product_compiles}}\pysiglinewithargsret{\sphinxcode{\sphinxupquote{src.product.}}\sphinxbfcode{\sphinxupquote{product\_compiles}}}{\emph{product\_info}}{}~\begin{description}
\item[{Know if a product compiles or not (some products do not have a }] \leavevmode
compilation procedure)

\end{description}
\begin{quote}\begin{description}
\item[{Parameters}] \leavevmode
\sphinxstyleliteralstrong{\sphinxupquote{product\_info}} \textendash{} (Config)
The configuration specific to the product

\item[{Returns}] \leavevmode
(bool) 
True if the product compiles, else False

\end{description}\end{quote}

\end{fulllineitems}

\index{product\_has\_env\_script() (in module src.product)}

\begin{fulllineitems}
\phantomsection\label{\detokenize{apidoc_src/src:src.product.product_has_env_script}}\pysiglinewithargsret{\sphinxcode{\sphinxupquote{src.product.}}\sphinxbfcode{\sphinxupquote{product\_has\_env\_script}}}{\emph{product\_info}}{}
Know if a product has an environment script
\begin{quote}\begin{description}
\item[{Parameters}] \leavevmode
\sphinxstyleliteralstrong{\sphinxupquote{product\_info}} \textendash{} (Config)
The configuration specific to the product

\item[{Returns}] \leavevmode
(bool) 
True if the product it has an environment script, else False

\end{description}\end{quote}

\end{fulllineitems}

\index{product\_has\_logo() (in module src.product)}

\begin{fulllineitems}
\phantomsection\label{\detokenize{apidoc_src/src:src.product.product_has_logo}}\pysiglinewithargsret{\sphinxcode{\sphinxupquote{src.product.}}\sphinxbfcode{\sphinxupquote{product\_has\_logo}}}{\emph{product\_info}}{}
Know if a product has a logo (YACSGEN generate)
\begin{quote}\begin{description}
\item[{Parameters}] \leavevmode
\sphinxstyleliteralstrong{\sphinxupquote{product\_info}} \textendash{} (Config)
The configuration specific to the product

\item[{Returns}] \leavevmode
(str) 
The path of the logo if the product has a logo, else False

\end{description}\end{quote}

\end{fulllineitems}

\index{product\_has\_patches() (in module src.product)}

\begin{fulllineitems}
\phantomsection\label{\detokenize{apidoc_src/src:src.product.product_has_patches}}\pysiglinewithargsret{\sphinxcode{\sphinxupquote{src.product.}}\sphinxbfcode{\sphinxupquote{product\_has\_patches}}}{\emph{product\_info}}{}
Know if a product has one or more patches
\begin{quote}\begin{description}
\item[{Parameters}] \leavevmode
\sphinxstyleliteralstrong{\sphinxupquote{product\_info}} \textendash{} (Config)
The configuration specific to the product

\item[{Returns}] \leavevmode
(bool) 
True if the product has one or more patches

\end{description}\end{quote}

\end{fulllineitems}

\index{product\_has\_salome\_gui() (in module src.product)}

\begin{fulllineitems}
\phantomsection\label{\detokenize{apidoc_src/src:src.product.product_has_salome_gui}}\pysiglinewithargsret{\sphinxcode{\sphinxupquote{src.product.}}\sphinxbfcode{\sphinxupquote{product\_has\_salome\_gui}}}{\emph{product\_info}}{}
Know if a product has a SALOME gui
\begin{quote}\begin{description}
\item[{Parameters}] \leavevmode
\sphinxstyleliteralstrong{\sphinxupquote{product\_info}} \textendash{} (Config)
The configuration specific to the product

\item[{Returns}] \leavevmode
(bool) 
True if the product has a SALOME gui, else False

\end{description}\end{quote}

\end{fulllineitems}

\index{product\_has\_script() (in module src.product)}

\begin{fulllineitems}
\phantomsection\label{\detokenize{apidoc_src/src:src.product.product_has_script}}\pysiglinewithargsret{\sphinxcode{\sphinxupquote{src.product.}}\sphinxbfcode{\sphinxupquote{product\_has\_script}}}{\emph{product\_info}}{}
Know if a product has a compilation script
\begin{quote}\begin{description}
\item[{Parameters}] \leavevmode
\sphinxstyleliteralstrong{\sphinxupquote{product\_info}} \textendash{} (Config)
The configuration specific to the product

\item[{Returns}] \leavevmode
(bool) 
True if the product it has a compilation script, else False

\end{description}\end{quote}

\end{fulllineitems}

\index{product\_is\_SALOME() (in module src.product)}

\begin{fulllineitems}
\phantomsection\label{\detokenize{apidoc_src/src:src.product.product_is_SALOME}}\pysiglinewithargsret{\sphinxcode{\sphinxupquote{src.product.}}\sphinxbfcode{\sphinxupquote{product\_is\_SALOME}}}{\emph{product\_info}}{}
Know if a product is a SALOME module
\begin{quote}\begin{description}
\item[{Parameters}] \leavevmode
\sphinxstyleliteralstrong{\sphinxupquote{product\_info}} \textendash{} (Config)
The configuration specific to the product

\item[{Returns}] \leavevmode
(bool) 
True if the product is a SALOME module, else False

\end{description}\end{quote}

\end{fulllineitems}

\index{product\_is\_autotools() (in module src.product)}

\begin{fulllineitems}
\phantomsection\label{\detokenize{apidoc_src/src:src.product.product_is_autotools}}\pysiglinewithargsret{\sphinxcode{\sphinxupquote{src.product.}}\sphinxbfcode{\sphinxupquote{product\_is\_autotools}}}{\emph{product\_info}}{}
Know if a product is compiled using the autotools
\begin{quote}\begin{description}
\item[{Parameters}] \leavevmode
\sphinxstyleliteralstrong{\sphinxupquote{product\_info}} \textendash{} (Config)
The configuration specific to the product

\item[{Returns}] \leavevmode
(bool) 
True if the product is autotools, else False

\end{description}\end{quote}

\end{fulllineitems}

\index{product\_is\_cmake() (in module src.product)}

\begin{fulllineitems}
\phantomsection\label{\detokenize{apidoc_src/src:src.product.product_is_cmake}}\pysiglinewithargsret{\sphinxcode{\sphinxupquote{src.product.}}\sphinxbfcode{\sphinxupquote{product\_is\_cmake}}}{\emph{product\_info}}{}
Know if a product is compiled using the cmake
\begin{quote}\begin{description}
\item[{Parameters}] \leavevmode
\sphinxstyleliteralstrong{\sphinxupquote{product\_info}} \textendash{} (Config)
The configuration specific to the product

\item[{Returns}] \leavevmode
(bool) 
True if the product is cmake, else False

\end{description}\end{quote}

\end{fulllineitems}

\index{product\_is\_cpp() (in module src.product)}

\begin{fulllineitems}
\phantomsection\label{\detokenize{apidoc_src/src:src.product.product_is_cpp}}\pysiglinewithargsret{\sphinxcode{\sphinxupquote{src.product.}}\sphinxbfcode{\sphinxupquote{product\_is\_cpp}}}{\emph{product\_info}}{}
Know if a product is cpp
\begin{quote}\begin{description}
\item[{Parameters}] \leavevmode
\sphinxstyleliteralstrong{\sphinxupquote{product\_info}} \textendash{} (Config)
The configuration specific to the product

\item[{Returns}] \leavevmode
(bool) 
True if the product is a cpp, else False

\end{description}\end{quote}

\end{fulllineitems}

\index{product\_is\_debug() (in module src.product)}

\begin{fulllineitems}
\phantomsection\label{\detokenize{apidoc_src/src:src.product.product_is_debug}}\pysiglinewithargsret{\sphinxcode{\sphinxupquote{src.product.}}\sphinxbfcode{\sphinxupquote{product\_is\_debug}}}{\emph{product\_info}}{}
Know if a product is in debug mode
\begin{quote}\begin{description}
\item[{Parameters}] \leavevmode
\sphinxstyleliteralstrong{\sphinxupquote{product\_info}} \textendash{} (Config)
The configuration specific to the product

\item[{Returns}] \leavevmode
(bool) 
True if the product is in debug mode, else False

\end{description}\end{quote}

\end{fulllineitems}

\index{product\_is\_dev() (in module src.product)}

\begin{fulllineitems}
\phantomsection\label{\detokenize{apidoc_src/src:src.product.product_is_dev}}\pysiglinewithargsret{\sphinxcode{\sphinxupquote{src.product.}}\sphinxbfcode{\sphinxupquote{product\_is\_dev}}}{\emph{product\_info}}{}
Know if a product is in dev mode
\begin{quote}\begin{description}
\item[{Parameters}] \leavevmode
\sphinxstyleliteralstrong{\sphinxupquote{product\_info}} \textendash{} (Config)
The configuration specific to the product

\item[{Returns}] \leavevmode
(bool) 
True if the product is in dev mode, else False

\end{description}\end{quote}

\end{fulllineitems}

\index{product\_is\_fixed() (in module src.product)}

\begin{fulllineitems}
\phantomsection\label{\detokenize{apidoc_src/src:src.product.product_is_fixed}}\pysiglinewithargsret{\sphinxcode{\sphinxupquote{src.product.}}\sphinxbfcode{\sphinxupquote{product\_is\_fixed}}}{\emph{product\_info}}{}
Know if a product is fixed
\begin{quote}\begin{description}
\item[{Parameters}] \leavevmode
\sphinxstyleliteralstrong{\sphinxupquote{product\_info}} \textendash{} (Config)
The configuration specific to the product

\item[{Returns}] \leavevmode
(bool) 
True if the product is fixed, else False

\end{description}\end{quote}

\end{fulllineitems}

\index{product\_is\_generated() (in module src.product)}

\begin{fulllineitems}
\phantomsection\label{\detokenize{apidoc_src/src:src.product.product_is_generated}}\pysiglinewithargsret{\sphinxcode{\sphinxupquote{src.product.}}\sphinxbfcode{\sphinxupquote{product\_is\_generated}}}{\emph{product\_info}}{}
Know if a product is generated (YACSGEN)
\begin{quote}\begin{description}
\item[{Parameters}] \leavevmode
\sphinxstyleliteralstrong{\sphinxupquote{product\_info}} \textendash{} (Config)
The configuration specific to the product

\item[{Returns}] \leavevmode
(bool) True if the product is generated

\end{description}\end{quote}

\end{fulllineitems}

\index{product\_is\_mpi() (in module src.product)}

\begin{fulllineitems}
\phantomsection\label{\detokenize{apidoc_src/src:src.product.product_is_mpi}}\pysiglinewithargsret{\sphinxcode{\sphinxupquote{src.product.}}\sphinxbfcode{\sphinxupquote{product\_is\_mpi}}}{\emph{product\_info}}{}
Know if a product has openmpi in its dependencies
\begin{quote}\begin{description}
\item[{Parameters}] \leavevmode
\sphinxstyleliteralstrong{\sphinxupquote{product\_info}} \textendash{} (Config)
The configuration specific to the product

\item[{Returns}] \leavevmode
(bool) 
True if the product has openmpi inits dependencies

\end{description}\end{quote}

\end{fulllineitems}

\index{product\_is\_native() (in module src.product)}

\begin{fulllineitems}
\phantomsection\label{\detokenize{apidoc_src/src:src.product.product_is_native}}\pysiglinewithargsret{\sphinxcode{\sphinxupquote{src.product.}}\sphinxbfcode{\sphinxupquote{product\_is\_native}}}{\emph{product\_info}}{}
Know if a product is native
\begin{quote}\begin{description}
\item[{Parameters}] \leavevmode
\sphinxstyleliteralstrong{\sphinxupquote{product\_info}} \textendash{} (Config)
The configuration specific to the product

\item[{Returns}] \leavevmode
(bool) 
True if the product is native, else False

\end{description}\end{quote}

\end{fulllineitems}

\index{product\_is\_salome() (in module src.product)}

\begin{fulllineitems}
\phantomsection\label{\detokenize{apidoc_src/src:src.product.product_is_salome}}\pysiglinewithargsret{\sphinxcode{\sphinxupquote{src.product.}}\sphinxbfcode{\sphinxupquote{product\_is\_salome}}}{\emph{product\_info}}{}
Know if a product is of type salome
\begin{quote}\begin{description}
\item[{Parameters}] \leavevmode
\sphinxstyleliteralstrong{\sphinxupquote{product\_info}} \textendash{} (Config) 
The configuration specific to the product

\item[{Returns}] \leavevmode
(bool) 
True if the product is salome, else False

\end{description}\end{quote}

\end{fulllineitems}

\index{product\_is\_sample() (in module src.product)}

\begin{fulllineitems}
\phantomsection\label{\detokenize{apidoc_src/src:src.product.product_is_sample}}\pysiglinewithargsret{\sphinxcode{\sphinxupquote{src.product.}}\sphinxbfcode{\sphinxupquote{product\_is\_sample}}}{\emph{product\_info}}{}
Know if a product has the sample type
\begin{quote}\begin{description}
\item[{Parameters}] \leavevmode
\sphinxstyleliteralstrong{\sphinxupquote{product\_info}} \textendash{} (Config)
The configuration specific to the product

\item[{Returns}] \leavevmode
(bool) 
True if the product has the sample type, else False

\end{description}\end{quote}

\end{fulllineitems}

\index{product\_is\_smesh\_plugin() (in module src.product)}

\begin{fulllineitems}
\phantomsection\label{\detokenize{apidoc_src/src:src.product.product_is_smesh_plugin}}\pysiglinewithargsret{\sphinxcode{\sphinxupquote{src.product.}}\sphinxbfcode{\sphinxupquote{product\_is\_smesh\_plugin}}}{\emph{product\_info}}{}
Know if a product is a SMESH plugin
\begin{quote}\begin{description}
\item[{Parameters}] \leavevmode
\sphinxstyleliteralstrong{\sphinxupquote{product\_info}} \textendash{} (Config)
The configuration specific to the product

\item[{Returns}] \leavevmode
(bool)
True if the product is a SMESH plugin, else False

\end{description}\end{quote}

\end{fulllineitems}

\index{product\_is\_vcs() (in module src.product)}

\begin{fulllineitems}
\phantomsection\label{\detokenize{apidoc_src/src:src.product.product_is_vcs}}\pysiglinewithargsret{\sphinxcode{\sphinxupquote{src.product.}}\sphinxbfcode{\sphinxupquote{product\_is\_vcs}}}{\emph{product\_info}}{}
Know if a product is download using git, svn or cvs (not archive)
\begin{quote}\begin{description}
\item[{Parameters}] \leavevmode
\sphinxstyleliteralstrong{\sphinxupquote{product\_info}} \textendash{} (Config)
The configuration specific to the product

\item[{Returns}] \leavevmode
(bool)
True if the product is vcs, else False

\end{description}\end{quote}

\end{fulllineitems}



\subsubsection{src.pyconf module}
\label{\detokenize{apidoc_src/src:src-pyconf-module}}\label{\detokenize{apidoc_src/src:module-src.pyconf}}\index{src.pyconf (module)}
This is a configuration module for Python.

This module should work under Python versions \textgreater{}= 2.2, and cannot be used with
earlier versions since it uses new-style classes.

Development and testing has only been carried out (so far) on Python 2.3.4 and
Python 2.4.2. See the test module (test\_config.py) included in the
U\{distribution\textless{}\sphinxurl{http://www.red-dove.com/python\_config}.html\textbar{}\_blank\textgreater{}\} (follow the
download link).

A simple example - with the example configuration file:

\fvset{hllines={, ,}}%
\begin{sphinxVerbatim}[commandchars=\\\{\}]
messages:
[
  \PYGZob{}
    stream : {}`sys.stderr{}`
    message: \PYGZsq{}Welcome\PYGZsq{}
    name: \PYGZsq{}Harry\PYGZsq{}
  \PYGZcb{}
  \PYGZob{}
    stream : {}`sys.stdout{}`
    message: \PYGZsq{}Welkom\PYGZsq{}
    name: \PYGZsq{}Ruud\PYGZsq{}
  \PYGZcb{}
  \PYGZob{}
    stream : \PYGZdl{}messages[0].stream
    message: \PYGZsq{}Bienvenue\PYGZsq{}
    name: Yves
  \PYGZcb{}
]
\end{sphinxVerbatim}

a program to read the configuration would be:

\fvset{hllines={, ,}}%
\begin{sphinxVerbatim}[commandchars=\\\{\}]
\PYG{k+kn}{from} \PYG{n+nn}{config} \PYG{k}{import} \PYG{n}{Config}

\PYG{n}{f} \PYG{o}{=} \PYG{n}{file}\PYG{p}{(}\PYG{l+s+s1}{\PYGZsq{}}\PYG{l+s+s1}{simple.cfg}\PYG{l+s+s1}{\PYGZsq{}}\PYG{p}{)}
\PYG{n}{cfg} \PYG{o}{=} \PYG{n}{Config}\PYG{p}{(}\PYG{n}{f}\PYG{p}{)}
\PYG{k}{for} \PYG{n}{m} \PYG{o+ow}{in} \PYG{n}{cfg}\PYG{o}{.}\PYG{n}{messages}\PYG{p}{:}
    \PYG{n}{s} \PYG{o}{=} \PYG{l+s+s1}{\PYGZsq{}}\PYG{l+s+si}{\PYGZpc{}s}\PYG{l+s+s1}{, }\PYG{l+s+si}{\PYGZpc{}s}\PYG{l+s+s1}{\PYGZsq{}} \PYG{o}{\PYGZpc{}} \PYG{p}{(}\PYG{n}{m}\PYG{o}{.}\PYG{n}{message}\PYG{p}{,} \PYG{n}{m}\PYG{o}{.}\PYG{n}{name}\PYG{p}{)}
    \PYG{k}{try}\PYG{p}{:}
        \PYG{n+nb}{print} \PYG{o}{\PYGZgt{}\PYGZgt{}} \PYG{n}{m}\PYG{o}{.}\PYG{n}{stream}\PYG{p}{,} \PYG{n}{s}
    \PYG{k}{except} \PYG{n+ne}{IOError}\PYG{p}{,} \PYG{n}{e}\PYG{p}{:}
        \PYG{n+nb}{print} \PYG{n}{e}
\end{sphinxVerbatim}

which, when run, would yield the console output:

\fvset{hllines={, ,}}%
\begin{sphinxVerbatim}[commandchars=\\\{\}]
\PYG{n}{Welcome}\PYG{p}{,} \PYG{n}{Harry}
\PYG{n}{Welkom}\PYG{p}{,} \PYG{n}{Ruud}
\PYG{n}{Bienvenue}\PYG{p}{,} \PYG{n}{Yves}
\end{sphinxVerbatim}

See U\{this tutorial\textless{}\sphinxurl{http://www.red-dove.com/python\_config}.html\textbar{}\_blank\textgreater{}\} for more
information.

\#modified for salomeTools
@version: 0.3.7.1

@author: Vinay Sajip

@copyright: Copyright (C) 2004-2007 Vinay Sajip. All Rights Reserved.

@var streamOpener: The default stream opener. This is a factory function which
takes a string (e.g. filename) and returns a stream suitable for reading. If
unable to open the stream, an IOError exception should be thrown.

The default value of this variable is L\{defaultStreamOpener\}. For an example
of how it’s used, see test\_config.py (search for streamOpener).
\index{Config (class in src.pyconf)}

\begin{fulllineitems}
\phantomsection\label{\detokenize{apidoc_src/src:src.pyconf.Config}}\pysiglinewithargsret{\sphinxbfcode{\sphinxupquote{class }}\sphinxcode{\sphinxupquote{src.pyconf.}}\sphinxbfcode{\sphinxupquote{Config}}}{\emph{streamOrFile=None}, \emph{parent=None}, \emph{PWD=None}}{}
Bases: {\hyperref[\detokenize{apidoc_src/src:src.pyconf.Mapping}]{\sphinxcrossref{\sphinxcode{\sphinxupquote{src.pyconf.Mapping}}}}} (\autopageref*{\detokenize{apidoc_src/src:src.pyconf.Mapping}})

This class represents a configuration, and is the only one which clients
need to interface to, under normal circumstances.
\index{Config.Namespace (class in src.pyconf)}

\begin{fulllineitems}
\phantomsection\label{\detokenize{apidoc_src/src:src.pyconf.Config.Namespace}}\pysigline{\sphinxbfcode{\sphinxupquote{class }}\sphinxbfcode{\sphinxupquote{Namespace}}}
Bases: \sphinxcode{\sphinxupquote{object}}

This internal class is used for implementing default namespaces.

An instance acts as a namespace.

\end{fulllineitems}

\index{addNamespace() (src.pyconf.Config method)}

\begin{fulllineitems}
\phantomsection\label{\detokenize{apidoc_src/src:src.pyconf.Config.addNamespace}}\pysiglinewithargsret{\sphinxbfcode{\sphinxupquote{addNamespace}}}{\emph{ns}, \emph{name=None}}{}
Add a namespace to this configuration which can be used to evaluate
(resolve) dotted-identifier expressions.
@param ns: The namespace to be added.
@type ns: A module or other namespace suitable for passing as an
argument to vars().
@param name: A name for the namespace, which, if specified, provides
an additional level of indirection.
@type name: str

\end{fulllineitems}

\index{getByPath() (src.pyconf.Config method)}

\begin{fulllineitems}
\phantomsection\label{\detokenize{apidoc_src/src:src.pyconf.Config.getByPath}}\pysiglinewithargsret{\sphinxbfcode{\sphinxupquote{getByPath}}}{\emph{path}}{}
Obtain a value in the configuration via its path.
@param path: The path of the required value
@type path: str
@return the value at the specified path.
@rtype: any
@raise ConfigError: If the path is invalid

\end{fulllineitems}

\index{load() (src.pyconf.Config method)}

\begin{fulllineitems}
\phantomsection\label{\detokenize{apidoc_src/src:src.pyconf.Config.load}}\pysiglinewithargsret{\sphinxbfcode{\sphinxupquote{load}}}{\emph{stream}}{}
Load the configuration from the specified stream. Multiple streams can
be used to populate the same instance, as long as there are no
clashing keys. The stream is closed.
@param stream: A stream from which the configuration is read.
@type stream: A read-only stream (file-like object).
@raise ConfigError: if keys in the loaded configuration clash with
existing keys.
@raise ConfigFormatError: if there is a syntax error in the stream.

\end{fulllineitems}

\index{removeNamespace() (src.pyconf.Config method)}

\begin{fulllineitems}
\phantomsection\label{\detokenize{apidoc_src/src:src.pyconf.Config.removeNamespace}}\pysiglinewithargsret{\sphinxbfcode{\sphinxupquote{removeNamespace}}}{\emph{ns}, \emph{name=None}}{}
Remove a namespace added with L\{addNamespace\}.
@param ns: The namespace to be removed.
@param name: The name which was specified when L\{addNamespace\} was
called.
@type name: str

\end{fulllineitems}


\end{fulllineitems}

\index{ConfigError}

\begin{fulllineitems}
\phantomsection\label{\detokenize{apidoc_src/src:src.pyconf.ConfigError}}\pysigline{\sphinxbfcode{\sphinxupquote{exception }}\sphinxcode{\sphinxupquote{src.pyconf.}}\sphinxbfcode{\sphinxupquote{ConfigError}}}
Bases: \sphinxcode{\sphinxupquote{exceptions.Exception}}

This is the base class of exceptions raised by this module.

\end{fulllineitems}

\index{ConfigFormatError}

\begin{fulllineitems}
\phantomsection\label{\detokenize{apidoc_src/src:src.pyconf.ConfigFormatError}}\pysigline{\sphinxbfcode{\sphinxupquote{exception }}\sphinxcode{\sphinxupquote{src.pyconf.}}\sphinxbfcode{\sphinxupquote{ConfigFormatError}}}
Bases: {\hyperref[\detokenize{apidoc_src/src:src.pyconf.ConfigError}]{\sphinxcrossref{\sphinxcode{\sphinxupquote{src.pyconf.ConfigError}}}}} (\autopageref*{\detokenize{apidoc_src/src:src.pyconf.ConfigError}})

This is the base class of exceptions raised due to syntax errors in
configurations.

\end{fulllineitems}

\index{ConfigInputStream (class in src.pyconf)}

\begin{fulllineitems}
\phantomsection\label{\detokenize{apidoc_src/src:src.pyconf.ConfigInputStream}}\pysiglinewithargsret{\sphinxbfcode{\sphinxupquote{class }}\sphinxcode{\sphinxupquote{src.pyconf.}}\sphinxbfcode{\sphinxupquote{ConfigInputStream}}}{\emph{stream}}{}
Bases: \sphinxcode{\sphinxupquote{object}}

An input stream which can read either ANSI files with default encoding
or Unicode files with BOMs.

Handles UTF-8, UTF-16LE, UTF-16BE. Could handle UTF-32 if Python had
built-in support.
\index{close() (src.pyconf.ConfigInputStream method)}

\begin{fulllineitems}
\phantomsection\label{\detokenize{apidoc_src/src:src.pyconf.ConfigInputStream.close}}\pysiglinewithargsret{\sphinxbfcode{\sphinxupquote{close}}}{}{}
\end{fulllineitems}

\index{read() (src.pyconf.ConfigInputStream method)}

\begin{fulllineitems}
\phantomsection\label{\detokenize{apidoc_src/src:src.pyconf.ConfigInputStream.read}}\pysiglinewithargsret{\sphinxbfcode{\sphinxupquote{read}}}{\emph{size}}{}
\end{fulllineitems}

\index{readline() (src.pyconf.ConfigInputStream method)}

\begin{fulllineitems}
\phantomsection\label{\detokenize{apidoc_src/src:src.pyconf.ConfigInputStream.readline}}\pysiglinewithargsret{\sphinxbfcode{\sphinxupquote{readline}}}{}{}
\end{fulllineitems}


\end{fulllineitems}

\index{ConfigList (class in src.pyconf)}

\begin{fulllineitems}
\phantomsection\label{\detokenize{apidoc_src/src:src.pyconf.ConfigList}}\pysigline{\sphinxbfcode{\sphinxupquote{class }}\sphinxcode{\sphinxupquote{src.pyconf.}}\sphinxbfcode{\sphinxupquote{ConfigList}}}
Bases: \sphinxcode{\sphinxupquote{list}}

This class implements an ordered list of configurations and allows you
to try getting the configuration from each entry in turn, returning
the first successfully obtained value.
\index{getByPath() (src.pyconf.ConfigList method)}

\begin{fulllineitems}
\phantomsection\label{\detokenize{apidoc_src/src:src.pyconf.ConfigList.getByPath}}\pysiglinewithargsret{\sphinxbfcode{\sphinxupquote{getByPath}}}{\emph{path}}{}
Obtain a value from the first configuration in the list which defines
it.

@param path: The path of the value to retrieve.
@type path: str
@return: The value from the earliest configuration in the list which
defines it.
@rtype: any
@raise ConfigError: If no configuration in the list has an entry with
the specified path.

\end{fulllineitems}


\end{fulllineitems}

\index{ConfigMerger (class in src.pyconf)}

\begin{fulllineitems}
\phantomsection\label{\detokenize{apidoc_src/src:src.pyconf.ConfigMerger}}\pysiglinewithargsret{\sphinxbfcode{\sphinxupquote{class }}\sphinxcode{\sphinxupquote{src.pyconf.}}\sphinxbfcode{\sphinxupquote{ConfigMerger}}}{\emph{resolver=\textless{}function defaultMergeResolve at 0x3561c80\textgreater{}}}{}
Bases: \sphinxcode{\sphinxupquote{object}}

This class is used for merging two configurations. If a key exists in the
merge operand but not the merge target, then the entry is copied from the
merge operand to the merge target. If a key exists in both configurations,
then a resolver (a callable) is called to decide how to handle the
conflict.
\index{handleMismatch() (src.pyconf.ConfigMerger method)}

\begin{fulllineitems}
\phantomsection\label{\detokenize{apidoc_src/src:src.pyconf.ConfigMerger.handleMismatch}}\pysiglinewithargsret{\sphinxbfcode{\sphinxupquote{handleMismatch}}}{\emph{obj1}, \emph{obj2}}{}
Handle a mismatch between two objects.

@param obj1: The object to merge into.
@type obj1: any
@param obj2: The object to merge.
@type obj2: any

\end{fulllineitems}

\index{merge() (src.pyconf.ConfigMerger method)}

\begin{fulllineitems}
\phantomsection\label{\detokenize{apidoc_src/src:src.pyconf.ConfigMerger.merge}}\pysiglinewithargsret{\sphinxbfcode{\sphinxupquote{merge}}}{\emph{merged}, \emph{mergee}}{}
Merge two configurations. The second configuration is unchanged,
and the first is changed to reflect the results of the merge.

@param merged: The configuration to merge into.
@type merged: L\{Config\}.
@param mergee: The configuration to merge.
@type mergee: L\{Config\}.

\end{fulllineitems}

\index{mergeMapping() (src.pyconf.ConfigMerger method)}

\begin{fulllineitems}
\phantomsection\label{\detokenize{apidoc_src/src:src.pyconf.ConfigMerger.mergeMapping}}\pysiglinewithargsret{\sphinxbfcode{\sphinxupquote{mergeMapping}}}{\emph{map1}, \emph{map2}}{}
Merge two mappings recursively. The second mapping is unchanged,
and the first is changed to reflect the results of the merge.

@param map1: The mapping to merge into.
@type map1: L\{Mapping\}.
@param map2: The mapping to merge.
@type map2: L\{Mapping\}.

\end{fulllineitems}

\index{mergeSequence() (src.pyconf.ConfigMerger method)}

\begin{fulllineitems}
\phantomsection\label{\detokenize{apidoc_src/src:src.pyconf.ConfigMerger.mergeSequence}}\pysiglinewithargsret{\sphinxbfcode{\sphinxupquote{mergeSequence}}}{\emph{seq1}, \emph{seq2}}{}
Merge two sequences. The second sequence is unchanged,
and the first is changed to have the elements of the second
appended to it.

@param seq1: The sequence to merge into.
@type seq1: L\{Sequence\}.
@param seq2: The sequence to merge.
@type seq2: L\{Sequence\}.

\end{fulllineitems}

\index{overwriteKeys() (src.pyconf.ConfigMerger method)}

\begin{fulllineitems}
\phantomsection\label{\detokenize{apidoc_src/src:src.pyconf.ConfigMerger.overwriteKeys}}\pysiglinewithargsret{\sphinxbfcode{\sphinxupquote{overwriteKeys}}}{\emph{map1}, \emph{seq2}}{}
Renint variables. The second mapping is unchanged,
and the first is changed depending the keys of the second mapping.
@param map1: The mapping to reinit keys into.
@type map1: L\{Mapping\}.
@param map2: The mapping container reinit information.
@type map2: L\{Mapping\}.

\end{fulllineitems}


\end{fulllineitems}

\index{ConfigOutputStream (class in src.pyconf)}

\begin{fulllineitems}
\phantomsection\label{\detokenize{apidoc_src/src:src.pyconf.ConfigOutputStream}}\pysiglinewithargsret{\sphinxbfcode{\sphinxupquote{class }}\sphinxcode{\sphinxupquote{src.pyconf.}}\sphinxbfcode{\sphinxupquote{ConfigOutputStream}}}{\emph{stream}, \emph{encoding=None}}{}
Bases: \sphinxcode{\sphinxupquote{object}}

An output stream which can write either ANSI files with default encoding
or Unicode files with BOMs.

Handles UTF-8, UTF-16LE, UTF-16BE. Could handle UTF-32 if Python had
built-in support.
\index{close() (src.pyconf.ConfigOutputStream method)}

\begin{fulllineitems}
\phantomsection\label{\detokenize{apidoc_src/src:src.pyconf.ConfigOutputStream.close}}\pysiglinewithargsret{\sphinxbfcode{\sphinxupquote{close}}}{}{}
\end{fulllineitems}

\index{flush() (src.pyconf.ConfigOutputStream method)}

\begin{fulllineitems}
\phantomsection\label{\detokenize{apidoc_src/src:src.pyconf.ConfigOutputStream.flush}}\pysiglinewithargsret{\sphinxbfcode{\sphinxupquote{flush}}}{}{}
\end{fulllineitems}

\index{write() (src.pyconf.ConfigOutputStream method)}

\begin{fulllineitems}
\phantomsection\label{\detokenize{apidoc_src/src:src.pyconf.ConfigOutputStream.write}}\pysiglinewithargsret{\sphinxbfcode{\sphinxupquote{write}}}{\emph{data}}{}
\end{fulllineitems}


\end{fulllineitems}

\index{ConfigReader (class in src.pyconf)}

\begin{fulllineitems}
\phantomsection\label{\detokenize{apidoc_src/src:src.pyconf.ConfigReader}}\pysiglinewithargsret{\sphinxbfcode{\sphinxupquote{class }}\sphinxcode{\sphinxupquote{src.pyconf.}}\sphinxbfcode{\sphinxupquote{ConfigReader}}}{\emph{config}}{}
Bases: \sphinxcode{\sphinxupquote{object}}

This internal class implements a parser for configurations.
\index{getChar() (src.pyconf.ConfigReader method)}

\begin{fulllineitems}
\phantomsection\label{\detokenize{apidoc_src/src:src.pyconf.ConfigReader.getChar}}\pysiglinewithargsret{\sphinxbfcode{\sphinxupquote{getChar}}}{}{}
Get the next char from the stream. Update line and column numbers
appropriately.

@return: The next character from the stream.
@rtype: str

\end{fulllineitems}

\index{getToken() (src.pyconf.ConfigReader method)}

\begin{fulllineitems}
\phantomsection\label{\detokenize{apidoc_src/src:src.pyconf.ConfigReader.getToken}}\pysiglinewithargsret{\sphinxbfcode{\sphinxupquote{getToken}}}{}{}
Get a token from the stream. String values are returned in a form
where you need to eval() the returned value to get the actual
string. The return value is (token\_type, token\_value).

Multiline string tokenizing is thanks to David Janes (BlogMatrix)

@return: The next token.
@rtype: A token tuple.

\end{fulllineitems}

\index{load() (src.pyconf.ConfigReader method)}

\begin{fulllineitems}
\phantomsection\label{\detokenize{apidoc_src/src:src.pyconf.ConfigReader.load}}\pysiglinewithargsret{\sphinxbfcode{\sphinxupquote{load}}}{\emph{stream}, \emph{parent=None}, \emph{suffix=None}}{}
Load the configuration from the specified stream.

@param stream: A stream from which to load the configuration.
@type stream: A stream (file-like object).
@param parent: The parent of the configuration (to which this reader
belongs) in the hierarchy. Specified when the configuration is
included in another one.
@type parent: A L\{Container\} instance.
@param suffix: The suffix of this configuration in the parent
configuration. Should be specified whenever the parent is not None.
@raise ConfigError: If parent is specified but suffix is not.
@raise ConfigFormatError: If there are syntax errors in the stream.

\end{fulllineitems}

\index{location() (src.pyconf.ConfigReader method)}

\begin{fulllineitems}
\phantomsection\label{\detokenize{apidoc_src/src:src.pyconf.ConfigReader.location}}\pysiglinewithargsret{\sphinxbfcode{\sphinxupquote{location}}}{}{}
Return the current location (filename, line, column) in the stream
as a string.

Used when printing error messages,

@return: A string representing a location in the stream being read.
@rtype: str

\end{fulllineitems}

\index{match() (src.pyconf.ConfigReader method)}

\begin{fulllineitems}
\phantomsection\label{\detokenize{apidoc_src/src:src.pyconf.ConfigReader.match}}\pysiglinewithargsret{\sphinxbfcode{\sphinxupquote{match}}}{\emph{t}}{}
Ensure that the current token type matches the specified value, and
advance to the next token.

@param t: The token type to match.
@type t: A valid token type.
@return: The token which was last read from the stream before this
function is called.
@rtype: a token tuple - see L\{getToken\}.
@raise ConfigFormatError: If the token does not match what’s expected.

\end{fulllineitems}

\index{parseFactor() (src.pyconf.ConfigReader method)}

\begin{fulllineitems}
\phantomsection\label{\detokenize{apidoc_src/src:src.pyconf.ConfigReader.parseFactor}}\pysiglinewithargsret{\sphinxbfcode{\sphinxupquote{parseFactor}}}{}{}
Parse a factor in an multiplicative expression (a * b, a / b, a \% b)

@return: the parsed factor
@rtype: any scalar
@raise ConfigFormatError: if a syntax error is found.

\end{fulllineitems}

\index{parseKeyValuePair() (src.pyconf.ConfigReader method)}

\begin{fulllineitems}
\phantomsection\label{\detokenize{apidoc_src/src:src.pyconf.ConfigReader.parseKeyValuePair}}\pysiglinewithargsret{\sphinxbfcode{\sphinxupquote{parseKeyValuePair}}}{\emph{parent}}{}
Parse a key-value pair, and add it to the provided L\{Mapping\}.

@param parent: The mapping to add entries to.
@type parent: A L\{Mapping\} instance.
@raise ConfigFormatError: if a syntax error is found.

\end{fulllineitems}

\index{parseMapping() (src.pyconf.ConfigReader method)}

\begin{fulllineitems}
\phantomsection\label{\detokenize{apidoc_src/src:src.pyconf.ConfigReader.parseMapping}}\pysiglinewithargsret{\sphinxbfcode{\sphinxupquote{parseMapping}}}{\emph{parent}, \emph{suffix}}{}
Parse a mapping.

@param parent: The container to which the mapping will be added.
@type parent: A L\{Container\} instance.
@param suffix: The suffix for the value.
@type suffix: str
@return: a L\{Mapping\} instance representing the mapping.
@rtype: L\{Mapping\}
@raise ConfigFormatError: if a syntax error is found.

\end{fulllineitems}

\index{parseMappingBody() (src.pyconf.ConfigReader method)}

\begin{fulllineitems}
\phantomsection\label{\detokenize{apidoc_src/src:src.pyconf.ConfigReader.parseMappingBody}}\pysiglinewithargsret{\sphinxbfcode{\sphinxupquote{parseMappingBody}}}{\emph{parent}}{}
Parse the internals of a mapping, and add entries to the provided
L\{Mapping\}.

@param parent: The mapping to add entries to.
@type parent: A L\{Mapping\} instance.

\end{fulllineitems}

\index{parseReference() (src.pyconf.ConfigReader method)}

\begin{fulllineitems}
\phantomsection\label{\detokenize{apidoc_src/src:src.pyconf.ConfigReader.parseReference}}\pysiglinewithargsret{\sphinxbfcode{\sphinxupquote{parseReference}}}{\emph{type}}{}
Parse a reference.

@return: the parsed reference
@rtype: L\{Reference\}
@raise ConfigFormatError: if a syntax error is found.

\end{fulllineitems}

\index{parseScalar() (src.pyconf.ConfigReader method)}

\begin{fulllineitems}
\phantomsection\label{\detokenize{apidoc_src/src:src.pyconf.ConfigReader.parseScalar}}\pysiglinewithargsret{\sphinxbfcode{\sphinxupquote{parseScalar}}}{}{}
Parse a scalar - a terminal value such as a string or number, or
an L\{Expression\} or L\{Reference\}.

@return: the parsed scalar
@rtype: any scalar
@raise ConfigFormatError: if a syntax error is found.

\end{fulllineitems}

\index{parseSequence() (src.pyconf.ConfigReader method)}

\begin{fulllineitems}
\phantomsection\label{\detokenize{apidoc_src/src:src.pyconf.ConfigReader.parseSequence}}\pysiglinewithargsret{\sphinxbfcode{\sphinxupquote{parseSequence}}}{\emph{parent}, \emph{suffix}}{}
Parse a sequence.

@param parent: The container to which the sequence will be added.
@type parent: A L\{Container\} instance.
@param suffix: The suffix for the value.
@type suffix: str
@return: a L\{Sequence\} instance representing the sequence.
@rtype: L\{Sequence\}
@raise ConfigFormatError: if a syntax error is found.

\end{fulllineitems}

\index{parseSuffix() (src.pyconf.ConfigReader method)}

\begin{fulllineitems}
\phantomsection\label{\detokenize{apidoc_src/src:src.pyconf.ConfigReader.parseSuffix}}\pysiglinewithargsret{\sphinxbfcode{\sphinxupquote{parseSuffix}}}{\emph{ref}}{}
Parse a reference suffix.

@param ref: The reference of which this suffix is a part.
@type ref: L\{Reference\}.
@raise ConfigFormatError: if a syntax error is found.

\end{fulllineitems}

\index{parseTerm() (src.pyconf.ConfigReader method)}

\begin{fulllineitems}
\phantomsection\label{\detokenize{apidoc_src/src:src.pyconf.ConfigReader.parseTerm}}\pysiglinewithargsret{\sphinxbfcode{\sphinxupquote{parseTerm}}}{}{}
Parse a term in an additive expression (a + b, a - b)

@return: the parsed term
@rtype: any scalar
@raise ConfigFormatError: if a syntax error is found.

\end{fulllineitems}

\index{parseValue() (src.pyconf.ConfigReader method)}

\begin{fulllineitems}
\phantomsection\label{\detokenize{apidoc_src/src:src.pyconf.ConfigReader.parseValue}}\pysiglinewithargsret{\sphinxbfcode{\sphinxupquote{parseValue}}}{\emph{parent}, \emph{suffix}}{}
Parse a value.

@param parent: The container to which the value will be added.
@type parent: A L\{Container\} instance.
@param suffix: The suffix for the value.
@type suffix: str
@return: The value
@rtype: any
@raise ConfigFormatError: if a syntax error is found.

\end{fulllineitems}

\index{setStream() (src.pyconf.ConfigReader method)}

\begin{fulllineitems}
\phantomsection\label{\detokenize{apidoc_src/src:src.pyconf.ConfigReader.setStream}}\pysiglinewithargsret{\sphinxbfcode{\sphinxupquote{setStream}}}{\emph{stream}}{}
Set the stream to the specified value, and prepare to read from it.

@param stream: A stream from which to load the configuration.
@type stream: A stream (file-like object).

\end{fulllineitems}


\end{fulllineitems}

\index{ConfigResolutionError}

\begin{fulllineitems}
\phantomsection\label{\detokenize{apidoc_src/src:src.pyconf.ConfigResolutionError}}\pysigline{\sphinxbfcode{\sphinxupquote{exception }}\sphinxcode{\sphinxupquote{src.pyconf.}}\sphinxbfcode{\sphinxupquote{ConfigResolutionError}}}
Bases: {\hyperref[\detokenize{apidoc_src/src:src.pyconf.ConfigError}]{\sphinxcrossref{\sphinxcode{\sphinxupquote{src.pyconf.ConfigError}}}}} (\autopageref*{\detokenize{apidoc_src/src:src.pyconf.ConfigError}})

This is the base class of exceptions raised due to semantic errors in
configurations.

\end{fulllineitems}

\index{Container (class in src.pyconf)}

\begin{fulllineitems}
\phantomsection\label{\detokenize{apidoc_src/src:src.pyconf.Container}}\pysiglinewithargsret{\sphinxbfcode{\sphinxupquote{class }}\sphinxcode{\sphinxupquote{src.pyconf.}}\sphinxbfcode{\sphinxupquote{Container}}}{\emph{parent}}{}
Bases: \sphinxcode{\sphinxupquote{object}}

This internal class is the base class for mappings and sequences.

@ivar path: A string which describes how to get
to this instance from the root of the hierarchy.

Example:

\fvset{hllines={, ,}}%
\begin{sphinxVerbatim}[commandchars=\\\{\}]
\PYG{n}{a}\PYG{o}{.}\PYG{n}{list}\PYG{o}{.}\PYG{n}{of}\PYG{p}{[}\PYG{l+m+mi}{1}\PYG{p}{]}\PYG{o}{.}\PYG{o+ow}{or}\PYG{p}{[}\PYG{l+s+s1}{\PYGZsq{}}\PYG{l+s+s1}{more}\PYG{l+s+s1}{\PYGZsq{}}\PYG{p}{]}\PYG{o}{.}\PYG{n}{elements}
\end{sphinxVerbatim}
\index{evaluate() (src.pyconf.Container method)}

\begin{fulllineitems}
\phantomsection\label{\detokenize{apidoc_src/src:src.pyconf.Container.evaluate}}\pysiglinewithargsret{\sphinxbfcode{\sphinxupquote{evaluate}}}{\emph{item}}{}
Evaluate items which are instances of L\{Reference\} or L\{Expression\}.

L\{Reference\} instances are evaluated using L\{Reference.resolve\},
and L\{Expression\} instances are evaluated using
L\{Expression.evaluate\}.

@param item: The item to be evaluated.
@type item: any
@return: If the item is an instance of L\{Reference\} or L\{Expression\},
the evaluated value is returned, otherwise the item is returned
unchanged.

\end{fulllineitems}

\index{setPath() (src.pyconf.Container method)}

\begin{fulllineitems}
\phantomsection\label{\detokenize{apidoc_src/src:src.pyconf.Container.setPath}}\pysiglinewithargsret{\sphinxbfcode{\sphinxupquote{setPath}}}{\emph{path}}{}
Set the path for this instance.
@param path: The path - a string which describes how to get
to this instance from the root of the hierarchy.
@type path: str

\end{fulllineitems}

\index{writeToStream() (src.pyconf.Container method)}

\begin{fulllineitems}
\phantomsection\label{\detokenize{apidoc_src/src:src.pyconf.Container.writeToStream}}\pysiglinewithargsret{\sphinxbfcode{\sphinxupquote{writeToStream}}}{\emph{stream}, \emph{indent}, \emph{container}}{}
Write this instance to a stream at the specified indentation level.

Should be redefined in subclasses.

@param stream: The stream to write to
@type stream: A writable stream (file-like object)
@param indent: The indentation level
@type indent: int
@param container: The container of this instance
@type container: L\{Container\}
@raise NotImplementedError: If a subclass does not override this

\end{fulllineitems}

\index{writeValue() (src.pyconf.Container method)}

\begin{fulllineitems}
\phantomsection\label{\detokenize{apidoc_src/src:src.pyconf.Container.writeValue}}\pysiglinewithargsret{\sphinxbfcode{\sphinxupquote{writeValue}}}{\emph{value}, \emph{stream}, \emph{indent}}{}
\end{fulllineitems}


\end{fulllineitems}

\index{Expression (class in src.pyconf)}

\begin{fulllineitems}
\phantomsection\label{\detokenize{apidoc_src/src:src.pyconf.Expression}}\pysiglinewithargsret{\sphinxbfcode{\sphinxupquote{class }}\sphinxcode{\sphinxupquote{src.pyconf.}}\sphinxbfcode{\sphinxupquote{Expression}}}{\emph{op}, \emph{lhs}, \emph{rhs}}{}
Bases: \sphinxcode{\sphinxupquote{object}}

This internal class implements a value which is obtained by evaluating an expression.
\index{evaluate() (src.pyconf.Expression method)}

\begin{fulllineitems}
\phantomsection\label{\detokenize{apidoc_src/src:src.pyconf.Expression.evaluate}}\pysiglinewithargsret{\sphinxbfcode{\sphinxupquote{evaluate}}}{\emph{container}}{}
Evaluate this instance in the context of a container.

@param container: The container to evaluate in from.
@type container: L\{Container\}
@return: The evaluated value.
@rtype: any
@raise ConfigResolutionError: If evaluation fails.
@raise ZeroDivideError: If division by zero occurs.
@raise TypeError: If the operation is invalid, e.g.
subtracting one string from another.

\end{fulllineitems}


\end{fulllineitems}

\index{Mapping (class in src.pyconf)}

\begin{fulllineitems}
\phantomsection\label{\detokenize{apidoc_src/src:src.pyconf.Mapping}}\pysiglinewithargsret{\sphinxbfcode{\sphinxupquote{class }}\sphinxcode{\sphinxupquote{src.pyconf.}}\sphinxbfcode{\sphinxupquote{Mapping}}}{\emph{parent=None}}{}
Bases: {\hyperref[\detokenize{apidoc_src/src:src.pyconf.Container}]{\sphinxcrossref{\sphinxcode{\sphinxupquote{src.pyconf.Container}}}}} (\autopageref*{\detokenize{apidoc_src/src:src.pyconf.Container}})

This internal class implements key-value mappings in configurations.
\index{addMapping() (src.pyconf.Mapping method)}

\begin{fulllineitems}
\phantomsection\label{\detokenize{apidoc_src/src:src.pyconf.Mapping.addMapping}}\pysiglinewithargsret{\sphinxbfcode{\sphinxupquote{addMapping}}}{\emph{key}, \emph{value}, \emph{comment}, \emph{setting=False}}{}
Add a key-value mapping with a comment.

@param key: The key for the mapping.
@type key: str
@param value: The value for the mapping.
@type value: any
@param comment: The comment for the key (can be None).
@type comment: str
@param setting: If True, ignore clashes. This is set
to true when called from L\{\_\_setattr\_\_\}.
@raise ConfigFormatError: If an existing key is seen
again and setting is False.

\end{fulllineitems}

\index{get() (src.pyconf.Mapping method)}

\begin{fulllineitems}
\phantomsection\label{\detokenize{apidoc_src/src:src.pyconf.Mapping.get}}\pysiglinewithargsret{\sphinxbfcode{\sphinxupquote{get}}}{\emph{key}, \emph{default=None}}{}
Allows a dictionary-style get operation.

\end{fulllineitems}

\index{iteritems() (src.pyconf.Mapping method)}

\begin{fulllineitems}
\phantomsection\label{\detokenize{apidoc_src/src:src.pyconf.Mapping.iteritems}}\pysiglinewithargsret{\sphinxbfcode{\sphinxupquote{iteritems}}}{}{}
\end{fulllineitems}

\index{iterkeys() (src.pyconf.Mapping method)}

\begin{fulllineitems}
\phantomsection\label{\detokenize{apidoc_src/src:src.pyconf.Mapping.iterkeys}}\pysiglinewithargsret{\sphinxbfcode{\sphinxupquote{iterkeys}}}{}{}
\end{fulllineitems}

\index{keys() (src.pyconf.Mapping method)}

\begin{fulllineitems}
\phantomsection\label{\detokenize{apidoc_src/src:src.pyconf.Mapping.keys}}\pysiglinewithargsret{\sphinxbfcode{\sphinxupquote{keys}}}{}{}
Return the keys in a similar way to a dictionary.

\end{fulllineitems}

\index{writeToStream() (src.pyconf.Mapping method)}

\begin{fulllineitems}
\phantomsection\label{\detokenize{apidoc_src/src:src.pyconf.Mapping.writeToStream}}\pysiglinewithargsret{\sphinxbfcode{\sphinxupquote{writeToStream}}}{\emph{stream}, \emph{indent}, \emph{container}}{}
Write this instance to a stream at the specified indentation level.

Should be redefined in subclasses.

@param stream: The stream to write to
@type stream: A writable stream (file-like object)
@param indent: The indentation level
@type indent: int
@param container: The container of this instance
@type container: L\{Container\}

\end{fulllineitems}


\end{fulllineitems}

\index{Reference (class in src.pyconf)}

\begin{fulllineitems}
\phantomsection\label{\detokenize{apidoc_src/src:src.pyconf.Reference}}\pysiglinewithargsret{\sphinxbfcode{\sphinxupquote{class }}\sphinxcode{\sphinxupquote{src.pyconf.}}\sphinxbfcode{\sphinxupquote{Reference}}}{\emph{config}, \emph{type}, \emph{ident}}{}
Bases: \sphinxcode{\sphinxupquote{object}}

This internal class implements a value which is a reference to another value.
\index{addElement() (src.pyconf.Reference method)}

\begin{fulllineitems}
\phantomsection\label{\detokenize{apidoc_src/src:src.pyconf.Reference.addElement}}\pysiglinewithargsret{\sphinxbfcode{\sphinxupquote{addElement}}}{\emph{type}, \emph{ident}}{}
Add an element to the reference.

@param type: The type of reference.
@type type: BACKTICK or DOLLAR
@param ident: The identifier which continues the reference.
@type ident: str

\end{fulllineitems}

\index{findConfig() (src.pyconf.Reference method)}

\begin{fulllineitems}
\phantomsection\label{\detokenize{apidoc_src/src:src.pyconf.Reference.findConfig}}\pysiglinewithargsret{\sphinxbfcode{\sphinxupquote{findConfig}}}{\emph{container}}{}
Find the closest enclosing configuration to the specified container.

@param container: The container to start from.
@type container: L\{Container\}
@return: The closest enclosing configuration, or None.
@rtype: L\{Config\}

\end{fulllineitems}

\index{resolve() (src.pyconf.Reference method)}

\begin{fulllineitems}
\phantomsection\label{\detokenize{apidoc_src/src:src.pyconf.Reference.resolve}}\pysiglinewithargsret{\sphinxbfcode{\sphinxupquote{resolve}}}{\emph{container}}{}
Resolve this instance in the context of a container.

@param container: The container to resolve from.
@type container: L\{Container\}
@return: The resolved value.
@rtype: any
@raise ConfigResolutionError: If resolution fails.

\end{fulllineitems}


\end{fulllineitems}

\index{Sequence (class in src.pyconf)}

\begin{fulllineitems}
\phantomsection\label{\detokenize{apidoc_src/src:src.pyconf.Sequence}}\pysiglinewithargsret{\sphinxbfcode{\sphinxupquote{class }}\sphinxcode{\sphinxupquote{src.pyconf.}}\sphinxbfcode{\sphinxupquote{Sequence}}}{\emph{parent=None}}{}
Bases: {\hyperref[\detokenize{apidoc_src/src:src.pyconf.Container}]{\sphinxcrossref{\sphinxcode{\sphinxupquote{src.pyconf.Container}}}}} (\autopageref*{\detokenize{apidoc_src/src:src.pyconf.Container}})

This internal class implements a value which is a sequence of other values.
\index{Sequence.SeqIter (class in src.pyconf)}

\begin{fulllineitems}
\phantomsection\label{\detokenize{apidoc_src/src:src.pyconf.Sequence.SeqIter}}\pysiglinewithargsret{\sphinxbfcode{\sphinxupquote{class }}\sphinxbfcode{\sphinxupquote{SeqIter}}}{\emph{seq}}{}
Bases: \sphinxcode{\sphinxupquote{object}}

This internal class implements an iterator for a L\{Sequence\} instance.
\index{next() (src.pyconf.Sequence.SeqIter method)}

\begin{fulllineitems}
\phantomsection\label{\detokenize{apidoc_src/src:src.pyconf.Sequence.SeqIter.next}}\pysiglinewithargsret{\sphinxbfcode{\sphinxupquote{next}}}{}{}
\end{fulllineitems}


\end{fulllineitems}

\index{append() (src.pyconf.Sequence method)}

\begin{fulllineitems}
\phantomsection\label{\detokenize{apidoc_src/src:src.pyconf.Sequence.append}}\pysiglinewithargsret{\sphinxbfcode{\sphinxupquote{append}}}{\emph{item}, \emph{comment}}{}
Add an item to the sequence.

@param item: The item to add.
@type item: any
@param comment: A comment for the item.
@type comment: str

\end{fulllineitems}

\index{writeToStream() (src.pyconf.Sequence method)}

\begin{fulllineitems}
\phantomsection\label{\detokenize{apidoc_src/src:src.pyconf.Sequence.writeToStream}}\pysiglinewithargsret{\sphinxbfcode{\sphinxupquote{writeToStream}}}{\emph{stream}, \emph{indent}, \emph{container}}{}
Write this instance to a stream at the specified indentation level.

Should be redefined in subclasses.

@param stream: The stream to write to
@type stream: A writable stream (file-like object)
@param indent: The indentation level
@type indent: int
@param container: The container of this instance
@type container: L\{Container\}

\end{fulllineitems}


\end{fulllineitems}

\index{deepCopyMapping() (in module src.pyconf)}

\begin{fulllineitems}
\phantomsection\label{\detokenize{apidoc_src/src:src.pyconf.deepCopyMapping}}\pysiglinewithargsret{\sphinxcode{\sphinxupquote{src.pyconf.}}\sphinxbfcode{\sphinxupquote{deepCopyMapping}}}{\emph{inMapping}}{}
\end{fulllineitems}

\index{defaultMergeResolve() (in module src.pyconf)}

\begin{fulllineitems}
\phantomsection\label{\detokenize{apidoc_src/src:src.pyconf.defaultMergeResolve}}\pysiglinewithargsret{\sphinxcode{\sphinxupquote{src.pyconf.}}\sphinxbfcode{\sphinxupquote{defaultMergeResolve}}}{\emph{map1}, \emph{map2}, \emph{key}}{}
A default resolver for merge conflicts. 
Returns a string indicating what action to take to resolve the conflict.

@param map1: The map being merged into.
@type map1: L\{Mapping\}.
@param map2: The map being used as the merge operand.
@type map2: L\{Mapping\}.
@param key: The key in map2 (which also exists in map1).
@type key: str
@return: One of “merge”, “append”, “mismatch” or “overwrite” 
indicating what action should be taken. This should
be appropriate to the objects being merged - e.g.
there is no point returning “merge” if the two objects
are instances of L\{Sequence\}.
@rtype: str

\end{fulllineitems}

\index{defaultStreamOpener() (in module src.pyconf)}

\begin{fulllineitems}
\phantomsection\label{\detokenize{apidoc_src/src:src.pyconf.defaultStreamOpener}}\pysiglinewithargsret{\sphinxcode{\sphinxupquote{src.pyconf.}}\sphinxbfcode{\sphinxupquote{defaultStreamOpener}}}{\emph{name}}{}
This function returns a read-only stream, given its name. The name passed
in should correspond to an existing stream, otherwise an exception will be
raised.

This is the default value of L\{streamOpener\}; assign your own callable to
streamOpener to return streams based on names. For example, you could use
urllib2.urlopen().

@param name: The name of a stream, most commonly a file name.
@type name: str
@return: A stream with the specified name.
@rtype: A read-only stream (file-like object)

\end{fulllineitems}

\index{isWord() (in module src.pyconf)}

\begin{fulllineitems}
\phantomsection\label{\detokenize{apidoc_src/src:src.pyconf.isWord}}\pysiglinewithargsret{\sphinxcode{\sphinxupquote{src.pyconf.}}\sphinxbfcode{\sphinxupquote{isWord}}}{\emph{s}}{}
See if a passed-in value is an identifier. If the value passed in is not a
string, False is returned. An identifier consists of alphanumerics or
underscore characters.

Examples:

\fvset{hllines={, ,}}%
\begin{sphinxVerbatim}[commandchars=\\\{\}]
\PYG{n}{isWord}\PYG{p}{(}\PYG{l+s+s1}{\PYGZsq{}}\PYG{l+s+s1}{a word}\PYG{l+s+s1}{\PYGZsq{}}\PYG{p}{)} \PYG{o}{\PYGZhy{}}\PYG{o}{\PYGZgt{}}\PYG{k+kc}{False}
\PYG{n}{isWord}\PYG{p}{(}\PYG{l+s+s1}{\PYGZsq{}}\PYG{l+s+s1}{award}\PYG{l+s+s1}{\PYGZsq{}}\PYG{p}{)} \PYG{o}{\PYGZhy{}}\PYG{o}{\PYGZgt{}} \PYG{k+kc}{True}
\PYG{n}{isWord}\PYG{p}{(}\PYG{l+m+mi}{9}\PYG{p}{)} \PYG{o}{\PYGZhy{}}\PYG{o}{\PYGZgt{}} \PYG{k+kc}{False}
\PYG{n}{isWord}\PYG{p}{(}\PYG{l+s+s1}{\PYGZsq{}}\PYG{l+s+s1}{a\PYGZus{}b\PYGZus{}c\PYGZus{}}\PYG{l+s+s1}{\PYGZsq{}}\PYG{p}{)} \PYG{o}{\PYGZhy{}}\PYG{o}{\PYGZgt{}}\PYG{k+kc}{True}
\end{sphinxVerbatim}

@note: isWord(‘9abc’) will return True - not exactly correct, but adequate
for the way it’s used here.

@param s: The name to be tested
@type s: any
@return: True if a word, else False
@rtype: bool

\end{fulllineitems}

\index{makePath() (in module src.pyconf)}

\begin{fulllineitems}
\phantomsection\label{\detokenize{apidoc_src/src:src.pyconf.makePath}}\pysiglinewithargsret{\sphinxcode{\sphinxupquote{src.pyconf.}}\sphinxbfcode{\sphinxupquote{makePath}}}{\emph{prefix}, \emph{suffix}}{}
Make a path from a prefix and suffix.

Examples::
makePath(‘’, ‘suffix’) -\textgreater{} ‘suffix’
makePath(‘prefix’, ‘suffix’) -\textgreater{} ‘prefix.suffix’
makePath(‘prefix’, ‘{[}1{]}’) -\textgreater{} ‘prefix{[}1{]}’

@param prefix:  The prefix to use. If it evaluates as false, the suffix is returned.
@type prefix:   str
@param suffix:  The suffix to use. It is either an identifier or an index in brackets.
@type suffix:   str
@return:        The path concatenation of prefix and suffix, with a dot if the suffix is not a bracketed index.
@rtype:         str

\end{fulllineitems}

\index{overwriteMergeResolve() (in module src.pyconf)}

\begin{fulllineitems}
\phantomsection\label{\detokenize{apidoc_src/src:src.pyconf.overwriteMergeResolve}}\pysiglinewithargsret{\sphinxcode{\sphinxupquote{src.pyconf.}}\sphinxbfcode{\sphinxupquote{overwriteMergeResolve}}}{\emph{map1}, \emph{map2}, \emph{key}}{}
An overwriting resolver for merge conflicts. Calls L\{defaultMergeResolve\},
but where a “mismatch” is detected, returns “overwrite” instead.

@param map1: The map being merged into.
@type map1: L\{Mapping\}.
@param map2: The map being used as the merge operand.
@type map2: L\{Mapping\}.
@param key: The key in map2 (which also exists in map1).
@type key: str

\end{fulllineitems}



\subsubsection{src.returnCode module}
\label{\detokenize{apidoc_src/src:module-src.returnCode}}\label{\detokenize{apidoc_src/src:src-returncode-module}}\index{src.returnCode (module)}
This file contains ReturnCode class

usage:
\textgreater{}\textgreater{} import returnCode as RCO
\index{ReturnCode (class in src.returnCode)}

\begin{fulllineitems}
\phantomsection\label{\detokenize{apidoc_src/src:src.returnCode.ReturnCode}}\pysiglinewithargsret{\sphinxbfcode{\sphinxupquote{class }}\sphinxcode{\sphinxupquote{src.returnCode.}}\sphinxbfcode{\sphinxupquote{ReturnCode}}}{\emph{status=None}, \emph{why=None}, \emph{value=None}}{}
Bases: \sphinxcode{\sphinxupquote{object}}

assume simple return code for methods, with explanation as ‘why’
obviously why is why it is not OK, 
but also why is why it is OK (if you want). 
and optionnaly contains a return value as self.getValue()

usage:
\textgreater{}\textgreater{} import returnCode as RCO

\textgreater{}\textgreater{} aValue = doSomethingToReturn()
\textgreater{}\textgreater{} return RCO.ReturnCode(“KO”, “there is no problem here”, aValue)
\textgreater{}\textgreater{} return RCO.ReturnCode(“KO”, “there is a problem here because etc”, None)
\textgreater{}\textgreater{} return RCO.ReturnCode(“TIMEOUT\_STATUS”, “too long here because etc”)
\textgreater{}\textgreater{} return RCO.ReturnCode(“NA”, “not applicable here because etc”)

\textgreater{}\textgreater{} rc = doSomething()
\textgreater{}\textgreater{} print(“short returnCode string”, str(rc))
\textgreater{}\textgreater{} print(“long returnCode string with value”, repr(rc))

\textgreater{}\textgreater{} rc1 = RCO.ReturnCode(“OK”, …)
\textgreater{}\textgreater{} rc2 = RCO.ReturnCode(“KO”, …)
\textgreater{}\textgreater{} rcFinal = rc1 + rc2
\textgreater{}\textgreater{} print(“long returnCode string with value”, repr(rcFinal)) \# KO!
\index{KFSYS (src.returnCode.ReturnCode attribute)}

\begin{fulllineitems}
\phantomsection\label{\detokenize{apidoc_src/src:src.returnCode.ReturnCode.KFSYS}}\pysigline{\sphinxbfcode{\sphinxupquote{KFSYS}}\sphinxbfcode{\sphinxupquote{ = 4}}}
\end{fulllineitems}

\index{KNOWNFAILURE\_STATUS (src.returnCode.ReturnCode attribute)}

\begin{fulllineitems}
\phantomsection\label{\detokenize{apidoc_src/src:src.returnCode.ReturnCode.KNOWNFAILURE_STATUS}}\pysigline{\sphinxbfcode{\sphinxupquote{KNOWNFAILURE\_STATUS}}\sphinxbfcode{\sphinxupquote{ = 'KF'}}}
\end{fulllineitems}

\index{KOSYS (src.returnCode.ReturnCode attribute)}

\begin{fulllineitems}
\phantomsection\label{\detokenize{apidoc_src/src:src.returnCode.ReturnCode.KOSYS}}\pysigline{\sphinxbfcode{\sphinxupquote{KOSYS}}\sphinxbfcode{\sphinxupquote{ = 1}}}
\end{fulllineitems}

\index{KO\_STATUS (src.returnCode.ReturnCode attribute)}

\begin{fulllineitems}
\phantomsection\label{\detokenize{apidoc_src/src:src.returnCode.ReturnCode.KO_STATUS}}\pysigline{\sphinxbfcode{\sphinxupquote{KO\_STATUS}}\sphinxbfcode{\sphinxupquote{ = 'KO'}}}
\end{fulllineitems}

\index{NASYS (src.returnCode.ReturnCode attribute)}

\begin{fulllineitems}
\phantomsection\label{\detokenize{apidoc_src/src:src.returnCode.ReturnCode.NASYS}}\pysigline{\sphinxbfcode{\sphinxupquote{NASYS}}\sphinxbfcode{\sphinxupquote{ = 2}}}
\end{fulllineitems}

\index{NA\_STATUS (src.returnCode.ReturnCode attribute)}

\begin{fulllineitems}
\phantomsection\label{\detokenize{apidoc_src/src:src.returnCode.ReturnCode.NA_STATUS}}\pysigline{\sphinxbfcode{\sphinxupquote{NA\_STATUS}}\sphinxbfcode{\sphinxupquote{ = 'NA'}}}
\end{fulllineitems}

\index{NDSYS (src.returnCode.ReturnCode attribute)}

\begin{fulllineitems}
\phantomsection\label{\detokenize{apidoc_src/src:src.returnCode.ReturnCode.NDSYS}}\pysigline{\sphinxbfcode{\sphinxupquote{NDSYS}}\sphinxbfcode{\sphinxupquote{ = 3}}}
\end{fulllineitems}

\index{OKSYS (src.returnCode.ReturnCode attribute)}

\begin{fulllineitems}
\phantomsection\label{\detokenize{apidoc_src/src:src.returnCode.ReturnCode.OKSYS}}\pysigline{\sphinxbfcode{\sphinxupquote{OKSYS}}\sphinxbfcode{\sphinxupquote{ = 0}}}
\end{fulllineitems}

\index{OK\_STATUS (src.returnCode.ReturnCode attribute)}

\begin{fulllineitems}
\phantomsection\label{\detokenize{apidoc_src/src:src.returnCode.ReturnCode.OK_STATUS}}\pysigline{\sphinxbfcode{\sphinxupquote{OK\_STATUS}}\sphinxbfcode{\sphinxupquote{ = 'OK'}}}
\end{fulllineitems}

\index{TIMEOUT\_STATUS (src.returnCode.ReturnCode attribute)}

\begin{fulllineitems}
\phantomsection\label{\detokenize{apidoc_src/src:src.returnCode.ReturnCode.TIMEOUT_STATUS}}\pysigline{\sphinxbfcode{\sphinxupquote{TIMEOUT\_STATUS}}\sphinxbfcode{\sphinxupquote{ = 'TIMEOUT'}}}
\end{fulllineitems}

\index{TOSYS (src.returnCode.ReturnCode attribute)}

\begin{fulllineitems}
\phantomsection\label{\detokenize{apidoc_src/src:src.returnCode.ReturnCode.TOSYS}}\pysigline{\sphinxbfcode{\sphinxupquote{TOSYS}}\sphinxbfcode{\sphinxupquote{ = 5}}}
\end{fulllineitems}

\index{UNKNOWN\_STATUS (src.returnCode.ReturnCode attribute)}

\begin{fulllineitems}
\phantomsection\label{\detokenize{apidoc_src/src:src.returnCode.ReturnCode.UNKNOWN_STATUS}}\pysigline{\sphinxbfcode{\sphinxupquote{UNKNOWN\_STATUS}}\sphinxbfcode{\sphinxupquote{ = 'ND'}}}
\end{fulllineitems}

\index{getValue() (src.returnCode.ReturnCode method)}

\begin{fulllineitems}
\phantomsection\label{\detokenize{apidoc_src/src:src.returnCode.ReturnCode.getValue}}\pysiglinewithargsret{\sphinxbfcode{\sphinxupquote{getValue}}}{}{}
\end{fulllineitems}

\index{getWhy() (src.returnCode.ReturnCode method)}

\begin{fulllineitems}
\phantomsection\label{\detokenize{apidoc_src/src:src.returnCode.ReturnCode.getWhy}}\pysiglinewithargsret{\sphinxbfcode{\sphinxupquote{getWhy}}}{}{}
return why as str or list if sum or some ReturnCode

\end{fulllineitems}

\index{indent() (src.returnCode.ReturnCode method)}

\begin{fulllineitems}
\phantomsection\label{\detokenize{apidoc_src/src:src.returnCode.ReturnCode.indent}}\pysiglinewithargsret{\sphinxbfcode{\sphinxupquote{indent}}}{\emph{text}, \emph{amount=5}, \emph{ch=' '}}{}
indent multi lines message

\end{fulllineitems}

\index{isOk() (src.returnCode.ReturnCode method)}

\begin{fulllineitems}
\phantomsection\label{\detokenize{apidoc_src/src:src.returnCode.ReturnCode.isOk}}\pysiglinewithargsret{\sphinxbfcode{\sphinxupquote{isOk}}}{}{}
return True if ok

\end{fulllineitems}

\index{raiseIfKo() (src.returnCode.ReturnCode method)}

\begin{fulllineitems}
\phantomsection\label{\detokenize{apidoc_src/src:src.returnCode.ReturnCode.raiseIfKo}}\pysiglinewithargsret{\sphinxbfcode{\sphinxupquote{raiseIfKo}}}{}{}
raise an exception with message why if not ok

\end{fulllineitems}

\index{setStatus() (src.returnCode.ReturnCode method)}

\begin{fulllineitems}
\phantomsection\label{\detokenize{apidoc_src/src:src.returnCode.ReturnCode.setStatus}}\pysiglinewithargsret{\sphinxbfcode{\sphinxupquote{setStatus}}}{\emph{status}, \emph{why=None}, \emph{value=None}}{}
\end{fulllineitems}

\index{setValue() (src.returnCode.ReturnCode method)}

\begin{fulllineitems}
\phantomsection\label{\detokenize{apidoc_src/src:src.returnCode.ReturnCode.setValue}}\pysiglinewithargsret{\sphinxbfcode{\sphinxupquote{setValue}}}{\emph{value}}{}
\end{fulllineitems}

\index{setWhy() (src.returnCode.ReturnCode method)}

\begin{fulllineitems}
\phantomsection\label{\detokenize{apidoc_src/src:src.returnCode.ReturnCode.setWhy}}\pysiglinewithargsret{\sphinxbfcode{\sphinxupquote{setWhy}}}{\emph{why}}{}
\end{fulllineitems}

\index{toSys() (src.returnCode.ReturnCode method)}

\begin{fulllineitems}
\phantomsection\label{\detokenize{apidoc_src/src:src.returnCode.ReturnCode.toSys}}\pysiglinewithargsret{\sphinxbfcode{\sphinxupquote{toSys}}}{}{}
return system return code as bash or bat

\end{fulllineitems}


\end{fulllineitems}



\subsubsection{src.salomeTools module}
\label{\detokenize{apidoc_src/src:src-salometools-module}}\label{\detokenize{apidoc_src/src:module-src.salomeTools}}\index{src.salomeTools (module)}
This file is the main entry file to salomeTools
NO \_\_main\_\_ entry allowed, use ‘sat’ (in parent directory)
\index{Sat (class in src.salomeTools)}

\begin{fulllineitems}
\phantomsection\label{\detokenize{apidoc_src/src:src.salomeTools.Sat}}\pysiglinewithargsret{\sphinxbfcode{\sphinxupquote{class }}\sphinxcode{\sphinxupquote{src.salomeTools.}}\sphinxbfcode{\sphinxupquote{Sat}}}{\emph{logger}}{}
Bases: \sphinxcode{\sphinxupquote{object}}

The main class that stores all the commands of salomeTools
(usually known as ‘runner’ argument in Command classes)
\index{assumeAsList() (src.salomeTools.Sat method)}

\begin{fulllineitems}
\phantomsection\label{\detokenize{apidoc_src/src:src.salomeTools.Sat.assumeAsList}}\pysiglinewithargsret{\sphinxbfcode{\sphinxupquote{assumeAsList}}}{\emph{strOrList}}{}
\end{fulllineitems}

\index{execute\_cli() (src.salomeTools.Sat method)}

\begin{fulllineitems}
\phantomsection\label{\detokenize{apidoc_src/src:src.salomeTools.Sat.execute_cli}}\pysiglinewithargsret{\sphinxbfcode{\sphinxupquote{execute\_cli}}}{\emph{cli\_arguments}}{}
select first argument as a command in directory ‘commands’, and launch on arguments
\begin{quote}\begin{description}
\item[{Parameters}] \leavevmode
\sphinxstyleliteralstrong{\sphinxupquote{cli\_arguments}} \textendash{} (str or list) The sat cli arguments (as sys.argv)

\end{description}\end{quote}

\end{fulllineitems}

\index{getColoredVersion() (src.salomeTools.Sat method)}

\begin{fulllineitems}
\phantomsection\label{\detokenize{apidoc_src/src:src.salomeTools.Sat.getColoredVersion}}\pysiglinewithargsret{\sphinxbfcode{\sphinxupquote{getColoredVersion}}}{}{}
get colored salomeTools version message

\end{fulllineitems}

\index{getCommandAndAppli() (src.salomeTools.Sat method)}

\begin{fulllineitems}
\phantomsection\label{\detokenize{apidoc_src/src:src.salomeTools.Sat.getCommandAndAppli}}\pysiglinewithargsret{\sphinxbfcode{\sphinxupquote{getCommandAndAppli}}}{\emph{arguments}}{}
\end{fulllineitems}

\index{getCommandInstance() (src.salomeTools.Sat method)}

\begin{fulllineitems}
\phantomsection\label{\detokenize{apidoc_src/src:src.salomeTools.Sat.getCommandInstance}}\pysiglinewithargsret{\sphinxbfcode{\sphinxupquote{getCommandInstance}}}{\emph{name}}{}
returns inherited instance of Command(\_BaseCmd) for command ‘name’
if module not loaded yet, load it.

\end{fulllineitems}

\index{getConfig() (src.salomeTools.Sat method)}

\begin{fulllineitems}
\phantomsection\label{\detokenize{apidoc_src/src:src.salomeTools.Sat.getConfig}}\pysiglinewithargsret{\sphinxbfcode{\sphinxupquote{getConfig}}}{}{}
\end{fulllineitems}

\index{getConfigManager() (src.salomeTools.Sat method)}

\begin{fulllineitems}
\phantomsection\label{\detokenize{apidoc_src/src:src.salomeTools.Sat.getConfigManager}}\pysiglinewithargsret{\sphinxbfcode{\sphinxupquote{getConfigManager}}}{}{}
\end{fulllineitems}

\index{getLogger() (src.salomeTools.Sat method)}

\begin{fulllineitems}
\phantomsection\label{\detokenize{apidoc_src/src:src.salomeTools.Sat.getLogger}}\pysiglinewithargsret{\sphinxbfcode{\sphinxupquote{getLogger}}}{}{}
\end{fulllineitems}

\index{getModule() (src.salomeTools.Sat method)}

\begin{fulllineitems}
\phantomsection\label{\detokenize{apidoc_src/src:src.salomeTools.Sat.getModule}}\pysiglinewithargsret{\sphinxbfcode{\sphinxupquote{getModule}}}{\emph{name}}{}
returns only-one-time loaded module Command ‘name’
assume load if not done yet

\end{fulllineitems}

\index{get\_help() (src.salomeTools.Sat method)}

\begin{fulllineitems}
\phantomsection\label{\detokenize{apidoc_src/src:src.salomeTools.Sat.get_help}}\pysiglinewithargsret{\sphinxbfcode{\sphinxupquote{get\_help}}}{}{}
get general help colored string

\end{fulllineitems}

\index{parseArguments() (src.salomeTools.Sat method)}

\begin{fulllineitems}
\phantomsection\label{\detokenize{apidoc_src/src:src.salomeTools.Sat.parseArguments}}\pysiglinewithargsret{\sphinxbfcode{\sphinxupquote{parseArguments}}}{\emph{arguments}}{}
\end{fulllineitems}

\index{print\_help() (src.salomeTools.Sat method)}

\begin{fulllineitems}
\phantomsection\label{\detokenize{apidoc_src/src:src.salomeTools.Sat.print_help}}\pysiglinewithargsret{\sphinxbfcode{\sphinxupquote{print\_help}}}{}{}
prints salomeTools general help

\end{fulllineitems}


\end{fulllineitems}

\index{assumeAsList() (in module src.salomeTools)}

\begin{fulllineitems}
\phantomsection\label{\detokenize{apidoc_src/src:src.salomeTools.assumeAsList}}\pysiglinewithargsret{\sphinxcode{\sphinxupquote{src.salomeTools.}}\sphinxbfcode{\sphinxupquote{assumeAsList}}}{\emph{strOrList}}{}
return a list as sys.argv if string

\end{fulllineitems}

\index{find\_command\_list() (in module src.salomeTools)}

\begin{fulllineitems}
\phantomsection\label{\detokenize{apidoc_src/src:src.salomeTools.find_command_list}}\pysiglinewithargsret{\sphinxcode{\sphinxupquote{src.salomeTools.}}\sphinxbfcode{\sphinxupquote{find\_command\_list}}}{\emph{dirPath}}{}
Parse files in dirPath that end with ‘.py’ : it gives commands list
\begin{quote}\begin{description}
\item[{Parameters}] \leavevmode
\sphinxstyleliteralstrong{\sphinxupquote{dirPath}} \textendash{} (str) The directory path where to search the commands

\item[{Returns}] \leavevmode
(list) the list containing the commands name

\end{description}\end{quote}

\end{fulllineitems}

\index{getCommandsList() (in module src.salomeTools)}

\begin{fulllineitems}
\phantomsection\label{\detokenize{apidoc_src/src:src.salomeTools.getCommandsList}}\pysiglinewithargsret{\sphinxcode{\sphinxupquote{src.salomeTools.}}\sphinxbfcode{\sphinxupquote{getCommandsList}}}{}{}
Gives commands list (as basename of files .py in directory commands

\end{fulllineitems}

\index{getVersion() (in module src.salomeTools)}

\begin{fulllineitems}
\phantomsection\label{\detokenize{apidoc_src/src:src.salomeTools.getVersion}}\pysiglinewithargsret{\sphinxcode{\sphinxupquote{src.salomeTools.}}\sphinxbfcode{\sphinxupquote{getVersion}}}{}{}
get version number as string

\end{fulllineitems}

\index{launchSat() (in module src.salomeTools)}

\begin{fulllineitems}
\phantomsection\label{\detokenize{apidoc_src/src:src.salomeTools.launchSat}}\pysiglinewithargsret{\sphinxcode{\sphinxupquote{src.salomeTools.}}\sphinxbfcode{\sphinxupquote{launchSat}}}{\emph{command}}{}
launch sat as subprocess.Popen
command as string (‘sat \textendash{}help’ for example)
used for unittest, or else…
\begin{quote}\begin{description}
\item[{Returns}] \leavevmode
(stdout, stderr) tuple of subprocess.Popen output

\end{description}\end{quote}

\end{fulllineitems}

\index{setLocale() (in module src.salomeTools)}

\begin{fulllineitems}
\phantomsection\label{\detokenize{apidoc_src/src:src.salomeTools.setLocale}}\pysiglinewithargsret{\sphinxcode{\sphinxupquote{src.salomeTools.}}\sphinxbfcode{\sphinxupquote{setLocale}}}{}{}
reset initial locale at any moment 
‘fr’ or else (TODO) from initial environment var ‘\$LANG’
‘i18n’ as ‘internationalization’

\end{fulllineitems}

\index{setNotLocale() (in module src.salomeTools)}

\begin{fulllineitems}
\phantomsection\label{\detokenize{apidoc_src/src:src.salomeTools.setNotLocale}}\pysiglinewithargsret{\sphinxcode{\sphinxupquote{src.salomeTools.}}\sphinxbfcode{\sphinxupquote{setNotLocale}}}{}{}
force english at any moment

\end{fulllineitems}



\subsubsection{src.system module}
\label{\detokenize{apidoc_src/src:src-system-module}}\label{\detokenize{apidoc_src/src:module-src.system}}\index{src.system (module)}
All utilities method doing a system call, 
like open a browser or an editor, or call a git command
\begin{description}
\item[{usage:}] \leavevmode
\textgreater{}\textgreater{} import src.system as SYSS

\end{description}
\index{archive\_extract() (in module src.system)}

\begin{fulllineitems}
\phantomsection\label{\detokenize{apidoc_src/src:src.system.archive_extract}}\pysiglinewithargsret{\sphinxcode{\sphinxupquote{src.system.}}\sphinxbfcode{\sphinxupquote{archive\_extract}}}{\emph{from\_what}, \emph{where}, \emph{logger}}{}
Extracts sources from an archive.
\begin{quote}\begin{description}
\item[{Parameters}] \leavevmode\begin{itemize}
\item {} 
\sphinxstyleliteralstrong{\sphinxupquote{from\_what}} \textendash{} (str) The path to the archive.

\item {} 
\sphinxstyleliteralstrong{\sphinxupquote{where}} \textendash{} (str) The path where to extract.

\item {} 
\sphinxstyleliteralstrong{\sphinxupquote{logger}} \textendash{} (Logger) The logger instance to use.

\end{itemize}

\item[{Returns}] \leavevmode
(bool) True if the extraction is successful

\end{description}\end{quote}

\end{fulllineitems}

\index{cvs\_extract() (in module src.system)}

\begin{fulllineitems}
\phantomsection\label{\detokenize{apidoc_src/src:src.system.cvs_extract}}\pysiglinewithargsret{\sphinxcode{\sphinxupquote{src.system.}}\sphinxbfcode{\sphinxupquote{cvs\_extract}}}{\emph{protocol}, \emph{user}, \emph{server}, \emph{base}, \emph{tag}, \emph{product}, \emph{where}, \emph{logger}, \emph{checkout=False}, \emph{environment=None}}{}
Extracts sources from a cvs repository.
\begin{quote}\begin{description}
\item[{Parameters}] \leavevmode\begin{itemize}
\item {} 
\sphinxstyleliteralstrong{\sphinxupquote{protocol}} \textendash{} (str) The cvs protocol.

\item {} 
\sphinxstyleliteralstrong{\sphinxupquote{user}} \textendash{} (str) The user to be used.

\item {} 
\sphinxstyleliteralstrong{\sphinxupquote{server}} \textendash{} (str) The remote cvs server.

\item {} 
\sphinxstyleliteralstrong{\sphinxupquote{base}} \textendash{} (str) .

\item {} 
\sphinxstyleliteralstrong{\sphinxupquote{tag}} \textendash{} (str) The tag.

\item {} 
\sphinxstyleliteralstrong{\sphinxupquote{product}} \textendash{} (str) The product.

\item {} 
\sphinxstyleliteralstrong{\sphinxupquote{where}} \textendash{} (str) The path where to extract.

\item {} 
\sphinxstyleliteralstrong{\sphinxupquote{logger}} \textendash{} (Logger) The logger instance to use.

\item {} 
\sphinxstyleliteralstrong{\sphinxupquote{checkout}} \textendash{} (bool) If true use checkout cvs.

\item {} 
\sphinxstyleliteralstrong{\sphinxupquote{environment}} \textendash{} (Environ) 
The environment to source when extracting.

\end{itemize}

\item[{Returns}] \leavevmode
(bool) True if the extraction is successful

\end{description}\end{quote}

\end{fulllineitems}

\index{git\_extract() (in module src.system)}

\begin{fulllineitems}
\phantomsection\label{\detokenize{apidoc_src/src:src.system.git_extract}}\pysiglinewithargsret{\sphinxcode{\sphinxupquote{src.system.}}\sphinxbfcode{\sphinxupquote{git\_extract}}}{\emph{from\_what}, \emph{tag}, \emph{where}, \emph{logger}, \emph{environment=None}}{}
Extracts sources from a git repository.
\begin{quote}\begin{description}
\item[{Parameters}] \leavevmode\begin{itemize}
\item {} 
\sphinxstyleliteralstrong{\sphinxupquote{from\_what}} \textendash{} (str) The remote git repository.

\item {} 
\sphinxstyleliteralstrong{\sphinxupquote{tag}} \textendash{} (str) The tag.

\item {} 
\sphinxstyleliteralstrong{\sphinxupquote{where}} \textendash{} (str) The path where to extract.

\item {} 
\sphinxstyleliteralstrong{\sphinxupquote{logger}} \textendash{} (Logger) The logger instance to use.

\item {} 
\sphinxstyleliteralstrong{\sphinxupquote{environment}} \textendash{} (Environ) 
The environment to source when extracting.

\end{itemize}

\item[{Returns}] \leavevmode
(bool) True if the extraction is successful

\end{description}\end{quote}

\end{fulllineitems}

\index{show\_in\_editor() (in module src.system)}

\begin{fulllineitems}
\phantomsection\label{\detokenize{apidoc_src/src:src.system.show_in_editor}}\pysiglinewithargsret{\sphinxcode{\sphinxupquote{src.system.}}\sphinxbfcode{\sphinxupquote{show\_in\_editor}}}{\emph{editor}, \emph{filePath}, \emph{logger}}{}
open filePath using editor.
\begin{quote}\begin{description}
\item[{Parameters}] \leavevmode\begin{itemize}
\item {} 
\sphinxstyleliteralstrong{\sphinxupquote{editor}} \textendash{} (str) The editor to use.

\item {} 
\sphinxstyleliteralstrong{\sphinxupquote{filePath}} \textendash{} (str) The path to the file to open.

\end{itemize}

\end{description}\end{quote}

\end{fulllineitems}

\index{svn\_extract() (in module src.system)}

\begin{fulllineitems}
\phantomsection\label{\detokenize{apidoc_src/src:src.system.svn_extract}}\pysiglinewithargsret{\sphinxcode{\sphinxupquote{src.system.}}\sphinxbfcode{\sphinxupquote{svn\_extract}}}{\emph{user}, \emph{from\_what}, \emph{tag}, \emph{where}, \emph{logger}, \emph{checkout=False}, \emph{environment=None}}{}
Extracts sources from a svn repository.
\begin{quote}\begin{description}
\item[{Parameters}] \leavevmode\begin{itemize}
\item {} 
\sphinxstyleliteralstrong{\sphinxupquote{user}} \textendash{} (str) The user to be used.

\item {} 
\sphinxstyleliteralstrong{\sphinxupquote{from\_what}} \textendash{} (str) The remote git repository.

\item {} 
\sphinxstyleliteralstrong{\sphinxupquote{tag}} \textendash{} (str) The tag.

\item {} 
\sphinxstyleliteralstrong{\sphinxupquote{where}} \textendash{} (str) The path where to extract.

\item {} 
\sphinxstyleliteralstrong{\sphinxupquote{logger}} \textendash{} (Logger) The logger instance to use.

\item {} 
\sphinxstyleliteralstrong{\sphinxupquote{checkout}} \textendash{} (bool) If true use checkout svn.

\item {} 
\sphinxstyleliteralstrong{\sphinxupquote{environment}} \textendash{} (Environ)
The environment to source when extracting.

\end{itemize}

\item[{Returns}] \leavevmode
(bool) True if the extraction is successful

\end{description}\end{quote}

\end{fulllineitems}



\subsubsection{src.template module}
\label{\detokenize{apidoc_src/src:module-src.template}}\label{\detokenize{apidoc_src/src:src-template-module}}\index{src.template (module)}\index{MyTemplate (class in src.template)}

\begin{fulllineitems}
\phantomsection\label{\detokenize{apidoc_src/src:src.template.MyTemplate}}\pysiglinewithargsret{\sphinxbfcode{\sphinxupquote{class }}\sphinxcode{\sphinxupquote{src.template.}}\sphinxbfcode{\sphinxupquote{MyTemplate}}}{\emph{template}}{}
Bases: \sphinxcode{\sphinxupquote{string.Template}}
\index{delimiter (src.template.MyTemplate attribute)}

\begin{fulllineitems}
\phantomsection\label{\detokenize{apidoc_src/src:src.template.MyTemplate.delimiter}}\pysigline{\sphinxbfcode{\sphinxupquote{delimiter}}\sphinxbfcode{\sphinxupquote{ = '\textbackslash{}xc2\textbackslash{}xa4'}}}
\end{fulllineitems}

\index{pattern (src.template.MyTemplate attribute)}

\begin{fulllineitems}
\phantomsection\label{\detokenize{apidoc_src/src:src.template.MyTemplate.pattern}}\pysigline{\sphinxbfcode{\sphinxupquote{pattern}}\sphinxbfcode{\sphinxupquote{ = \textless{}\_sre.SRE\_Pattern object at 0x412cde0\textgreater{}}}}
\end{fulllineitems}


\end{fulllineitems}

\index{substitute() (in module src.template)}

\begin{fulllineitems}
\phantomsection\label{\detokenize{apidoc_src/src:src.template.substitute}}\pysiglinewithargsret{\sphinxcode{\sphinxupquote{src.template.}}\sphinxbfcode{\sphinxupquote{substitute}}}{\emph{template\_file}, \emph{subst\_dic}}{}
\end{fulllineitems}



\subsubsection{src.test\_module module}
\label{\detokenize{apidoc_src/src:module-src.test_module}}\label{\detokenize{apidoc_src/src:src-test-module-module}}\index{src.test\_module (module)}\index{Test (class in src.test\_module)}

\begin{fulllineitems}
\phantomsection\label{\detokenize{apidoc_src/src:src.test_module.Test}}\pysiglinewithargsret{\sphinxbfcode{\sphinxupquote{class }}\sphinxcode{\sphinxupquote{src.test\_module.}}\sphinxbfcode{\sphinxupquote{Test}}}{\emph{config}, \emph{logger}, \emph{tmp\_working\_dir}, \emph{testbase=''}, \emph{grids=None}, \emph{sessions=None}, \emph{launcher=''}, \emph{show\_desktop=True}}{}~\index{generate\_launching\_commands() (src.test\_module.Test method)}

\begin{fulllineitems}
\phantomsection\label{\detokenize{apidoc_src/src:src.test_module.Test.generate_launching_commands}}\pysiglinewithargsret{\sphinxbfcode{\sphinxupquote{generate\_launching\_commands}}}{}{}
\end{fulllineitems}

\index{generate\_script() (src.test\_module.Test method)}

\begin{fulllineitems}
\phantomsection\label{\detokenize{apidoc_src/src:src.test_module.Test.generate_script}}\pysiglinewithargsret{\sphinxbfcode{\sphinxupquote{generate\_script}}}{\emph{listTest}, \emph{script\_path}, \emph{ignoreList}}{}
\end{fulllineitems}

\index{get\_test\_timeout() (src.test\_module.Test method)}

\begin{fulllineitems}
\phantomsection\label{\detokenize{apidoc_src/src:src.test_module.Test.get_test_timeout}}\pysiglinewithargsret{\sphinxbfcode{\sphinxupquote{get\_test\_timeout}}}{\emph{test\_name}, \emph{default\_value}}{}
\end{fulllineitems}

\index{get\_tmp\_dir() (src.test\_module.Test method)}

\begin{fulllineitems}
\phantomsection\label{\detokenize{apidoc_src/src:src.test_module.Test.get_tmp_dir}}\pysiglinewithargsret{\sphinxbfcode{\sphinxupquote{get\_tmp\_dir}}}{}{}
\end{fulllineitems}

\index{prepare\_testbase() (src.test\_module.Test method)}

\begin{fulllineitems}
\phantomsection\label{\detokenize{apidoc_src/src:src.test_module.Test.prepare_testbase}}\pysiglinewithargsret{\sphinxbfcode{\sphinxupquote{prepare\_testbase}}}{\emph{test\_base\_name}}{}
\end{fulllineitems}

\index{prepare\_testbase\_from\_dir() (src.test\_module.Test method)}

\begin{fulllineitems}
\phantomsection\label{\detokenize{apidoc_src/src:src.test_module.Test.prepare_testbase_from_dir}}\pysiglinewithargsret{\sphinxbfcode{\sphinxupquote{prepare\_testbase\_from\_dir}}}{\emph{testbase\_name}, \emph{testbase\_dir}}{}
\end{fulllineitems}

\index{prepare\_testbase\_from\_git() (src.test\_module.Test method)}

\begin{fulllineitems}
\phantomsection\label{\detokenize{apidoc_src/src:src.test_module.Test.prepare_testbase_from_git}}\pysiglinewithargsret{\sphinxbfcode{\sphinxupquote{prepare\_testbase\_from\_git}}}{\emph{testbase\_name}, \emph{testbase\_base}, \emph{testbase\_tag}}{}
\end{fulllineitems}

\index{prepare\_testbase\_from\_svn() (src.test\_module.Test method)}

\begin{fulllineitems}
\phantomsection\label{\detokenize{apidoc_src/src:src.test_module.Test.prepare_testbase_from_svn}}\pysiglinewithargsret{\sphinxbfcode{\sphinxupquote{prepare\_testbase\_from\_svn}}}{\emph{user}, \emph{testbase\_name}, \emph{testbase\_base}}{}
\end{fulllineitems}

\index{read\_results() (src.test\_module.Test method)}

\begin{fulllineitems}
\phantomsection\label{\detokenize{apidoc_src/src:src.test_module.Test.read_results}}\pysiglinewithargsret{\sphinxbfcode{\sphinxupquote{read\_results}}}{\emph{listTest}, \emph{has\_timed\_out}}{}
\end{fulllineitems}

\index{run\_all\_tests() (src.test\_module.Test method)}

\begin{fulllineitems}
\phantomsection\label{\detokenize{apidoc_src/src:src.test_module.Test.run_all_tests}}\pysiglinewithargsret{\sphinxbfcode{\sphinxupquote{run\_all\_tests}}}{}{}
\end{fulllineitems}

\index{run\_grid\_tests() (src.test\_module.Test method)}

\begin{fulllineitems}
\phantomsection\label{\detokenize{apidoc_src/src:src.test_module.Test.run_grid_tests}}\pysiglinewithargsret{\sphinxbfcode{\sphinxupquote{run\_grid\_tests}}}{}{}
\end{fulllineitems}

\index{run\_script() (src.test\_module.Test method)}

\begin{fulllineitems}
\phantomsection\label{\detokenize{apidoc_src/src:src.test_module.Test.run_script}}\pysiglinewithargsret{\sphinxbfcode{\sphinxupquote{run\_script}}}{\emph{script\_name}}{}
\end{fulllineitems}

\index{run\_session\_tests() (src.test\_module.Test method)}

\begin{fulllineitems}
\phantomsection\label{\detokenize{apidoc_src/src:src.test_module.Test.run_session_tests}}\pysiglinewithargsret{\sphinxbfcode{\sphinxupquote{run\_session\_tests}}}{}{}
\end{fulllineitems}

\index{run\_testbase\_tests() (src.test\_module.Test method)}

\begin{fulllineitems}
\phantomsection\label{\detokenize{apidoc_src/src:src.test_module.Test.run_testbase_tests}}\pysiglinewithargsret{\sphinxbfcode{\sphinxupquote{run\_testbase\_tests}}}{}{}
Runs test testbase

\end{fulllineitems}

\index{run\_tests() (src.test\_module.Test method)}

\begin{fulllineitems}
\phantomsection\label{\detokenize{apidoc_src/src:src.test_module.Test.run_tests}}\pysiglinewithargsret{\sphinxbfcode{\sphinxupquote{run\_tests}}}{\emph{listTest}, \emph{ignoreList}}{}
\end{fulllineitems}

\index{search\_known\_errors() (src.test\_module.Test method)}

\begin{fulllineitems}
\phantomsection\label{\detokenize{apidoc_src/src:src.test_module.Test.search_known_errors}}\pysiglinewithargsret{\sphinxbfcode{\sphinxupquote{search\_known\_errors}}}{\emph{status}, \emph{test\_grid}, \emph{test\_session}, \emph{test}}{}
\end{fulllineitems}

\index{write\_test\_margin() (src.test\_module.Test method)}

\begin{fulllineitems}
\phantomsection\label{\detokenize{apidoc_src/src:src.test_module.Test.write_test_margin}}\pysiglinewithargsret{\sphinxbfcode{\sphinxupquote{write\_test\_margin}}}{\emph{tab}}{}
indent with ‘\textbar{} … +’ to show test results.

\end{fulllineitems}


\end{fulllineitems}

\index{getTmpDirDEFAULT() (in module src.test\_module)}

\begin{fulllineitems}
\phantomsection\label{\detokenize{apidoc_src/src:src.test_module.getTmpDirDEFAULT}}\pysiglinewithargsret{\sphinxcode{\sphinxupquote{src.test\_module.}}\sphinxbfcode{\sphinxupquote{getTmpDirDEFAULT}}}{}{}
\end{fulllineitems}



\subsubsection{src.utilsSat module}
\label{\detokenize{apidoc_src/src:src-utilssat-module}}\label{\detokenize{apidoc_src/src:module-src.utilsSat}}\index{src.utilsSat (module)}
utilities for sat
general useful simple methods
all-in-one import srs.utilsSat as UTS

usage:
\textgreater{}\textgreater{} import srsc.utilsSat as UTS
\textgreater{}\textgreater{} UTS.ensure\_path\_exists(path)
\index{Path (class in src.utilsSat)}

\begin{fulllineitems}
\phantomsection\label{\detokenize{apidoc_src/src:src.utilsSat.Path}}\pysiglinewithargsret{\sphinxbfcode{\sphinxupquote{class }}\sphinxcode{\sphinxupquote{src.utilsSat.}}\sphinxbfcode{\sphinxupquote{Path}}}{\emph{path}}{}~\index{base() (src.utilsSat.Path method)}

\begin{fulllineitems}
\phantomsection\label{\detokenize{apidoc_src/src:src.utilsSat.Path.base}}\pysiglinewithargsret{\sphinxbfcode{\sphinxupquote{base}}}{}{}
\end{fulllineitems}

\index{chmod() (src.utilsSat.Path method)}

\begin{fulllineitems}
\phantomsection\label{\detokenize{apidoc_src/src:src.utilsSat.Path.chmod}}\pysiglinewithargsret{\sphinxbfcode{\sphinxupquote{chmod}}}{\emph{mode}}{}
\end{fulllineitems}

\index{copy() (src.utilsSat.Path method)}

\begin{fulllineitems}
\phantomsection\label{\detokenize{apidoc_src/src:src.utilsSat.Path.copy}}\pysiglinewithargsret{\sphinxbfcode{\sphinxupquote{copy}}}{\emph{path}, \emph{smart=False}}{}
\end{fulllineitems}

\index{copydir() (src.utilsSat.Path method)}

\begin{fulllineitems}
\phantomsection\label{\detokenize{apidoc_src/src:src.utilsSat.Path.copydir}}\pysiglinewithargsret{\sphinxbfcode{\sphinxupquote{copydir}}}{\emph{dst}, \emph{smart=False}}{}
\end{fulllineitems}

\index{copyfile() (src.utilsSat.Path method)}

\begin{fulllineitems}
\phantomsection\label{\detokenize{apidoc_src/src:src.utilsSat.Path.copyfile}}\pysiglinewithargsret{\sphinxbfcode{\sphinxupquote{copyfile}}}{\emph{path}}{}
\end{fulllineitems}

\index{copylink() (src.utilsSat.Path method)}

\begin{fulllineitems}
\phantomsection\label{\detokenize{apidoc_src/src:src.utilsSat.Path.copylink}}\pysiglinewithargsret{\sphinxbfcode{\sphinxupquote{copylink}}}{\emph{path}}{}
\end{fulllineitems}

\index{dir() (src.utilsSat.Path method)}

\begin{fulllineitems}
\phantomsection\label{\detokenize{apidoc_src/src:src.utilsSat.Path.dir}}\pysiglinewithargsret{\sphinxbfcode{\sphinxupquote{dir}}}{}{}
\end{fulllineitems}

\index{exists() (src.utilsSat.Path method)}

\begin{fulllineitems}
\phantomsection\label{\detokenize{apidoc_src/src:src.utilsSat.Path.exists}}\pysiglinewithargsret{\sphinxbfcode{\sphinxupquote{exists}}}{}{}
\end{fulllineitems}

\index{isdir() (src.utilsSat.Path method)}

\begin{fulllineitems}
\phantomsection\label{\detokenize{apidoc_src/src:src.utilsSat.Path.isdir}}\pysiglinewithargsret{\sphinxbfcode{\sphinxupquote{isdir}}}{}{}
\end{fulllineitems}

\index{isfile() (src.utilsSat.Path method)}

\begin{fulllineitems}
\phantomsection\label{\detokenize{apidoc_src/src:src.utilsSat.Path.isfile}}\pysiglinewithargsret{\sphinxbfcode{\sphinxupquote{isfile}}}{}{}
\end{fulllineitems}

\index{islink() (src.utilsSat.Path method)}

\begin{fulllineitems}
\phantomsection\label{\detokenize{apidoc_src/src:src.utilsSat.Path.islink}}\pysiglinewithargsret{\sphinxbfcode{\sphinxupquote{islink}}}{}{}
\end{fulllineitems}

\index{list() (src.utilsSat.Path method)}

\begin{fulllineitems}
\phantomsection\label{\detokenize{apidoc_src/src:src.utilsSat.Path.list}}\pysiglinewithargsret{\sphinxbfcode{\sphinxupquote{list}}}{}{}
\end{fulllineitems}

\index{make() (src.utilsSat.Path method)}

\begin{fulllineitems}
\phantomsection\label{\detokenize{apidoc_src/src:src.utilsSat.Path.make}}\pysiglinewithargsret{\sphinxbfcode{\sphinxupquote{make}}}{\emph{mode=None}}{}
\end{fulllineitems}

\index{readlink() (src.utilsSat.Path method)}

\begin{fulllineitems}
\phantomsection\label{\detokenize{apidoc_src/src:src.utilsSat.Path.readlink}}\pysiglinewithargsret{\sphinxbfcode{\sphinxupquote{readlink}}}{}{}
\end{fulllineitems}

\index{rm() (src.utilsSat.Path method)}

\begin{fulllineitems}
\phantomsection\label{\detokenize{apidoc_src/src:src.utilsSat.Path.rm}}\pysiglinewithargsret{\sphinxbfcode{\sphinxupquote{rm}}}{}{}
\end{fulllineitems}

\index{smartcopy() (src.utilsSat.Path method)}

\begin{fulllineitems}
\phantomsection\label{\detokenize{apidoc_src/src:src.utilsSat.Path.smartcopy}}\pysiglinewithargsret{\sphinxbfcode{\sphinxupquote{smartcopy}}}{\emph{path}}{}
\end{fulllineitems}

\index{symlink() (src.utilsSat.Path method)}

\begin{fulllineitems}
\phantomsection\label{\detokenize{apidoc_src/src:src.utilsSat.Path.symlink}}\pysiglinewithargsret{\sphinxbfcode{\sphinxupquote{symlink}}}{\emph{path}}{}
\end{fulllineitems}


\end{fulllineitems}

\index{black() (in module src.utilsSat)}

\begin{fulllineitems}
\phantomsection\label{\detokenize{apidoc_src/src:src.utilsSat.black}}\pysiglinewithargsret{\sphinxcode{\sphinxupquote{src.utilsSat.}}\sphinxbfcode{\sphinxupquote{black}}}{\emph{msg}}{}
\end{fulllineitems}

\index{blue() (in module src.utilsSat)}

\begin{fulllineitems}
\phantomsection\label{\detokenize{apidoc_src/src:src.utilsSat.blue}}\pysiglinewithargsret{\sphinxcode{\sphinxupquote{src.utilsSat.}}\sphinxbfcode{\sphinxupquote{blue}}}{\emph{msg}}{}
\end{fulllineitems}

\index{check\_config\_has\_application() (in module src.utilsSat)}

\begin{fulllineitems}
\phantomsection\label{\detokenize{apidoc_src/src:src.utilsSat.check_config_has_application}}\pysiglinewithargsret{\sphinxcode{\sphinxupquote{src.utilsSat.}}\sphinxbfcode{\sphinxupquote{check\_config\_has\_application}}}{\emph{config}}{}
Check that the config has the key APPLICATION. 
Else raise an exception.
\begin{quote}\begin{description}
\item[{Parameters}] \leavevmode
\sphinxstyleliteralstrong{\sphinxupquote{config}} \textendash{} (Config) The config.

\end{description}\end{quote}

\end{fulllineitems}

\index{check\_config\_has\_profile() (in module src.utilsSat)}

\begin{fulllineitems}
\phantomsection\label{\detokenize{apidoc_src/src:src.utilsSat.check_config_has_profile}}\pysiglinewithargsret{\sphinxcode{\sphinxupquote{src.utilsSat.}}\sphinxbfcode{\sphinxupquote{check\_config\_has\_profile}}}{\emph{config}}{}
Check that the config has the key APPLICATION.profile.
Else, raise an exception.
\begin{quote}\begin{description}
\item[{Parameters}] \leavevmode
\sphinxstyleliteralstrong{\sphinxupquote{config}} \textendash{} (Config) The config.

\end{description}\end{quote}

\end{fulllineitems}

\index{check\_has\_key() (in module src.utilsSat)}

\begin{fulllineitems}
\phantomsection\label{\detokenize{apidoc_src/src:src.utilsSat.check_has_key}}\pysiglinewithargsret{\sphinxcode{\sphinxupquote{src.utilsSat.}}\sphinxbfcode{\sphinxupquote{check\_has\_key}}}{\emph{inConfig}, \emph{key}}{}
Check that the in-Config node has the named key (as an attribute)
\begin{quote}\begin{description}
\item[{Parameters}] \leavevmode\begin{itemize}
\item {} 
\sphinxstyleliteralstrong{\sphinxupquote{inConfig}} \textendash{} (Config or Mapping etc) The in-Config node

\item {} 
\sphinxstyleliteralstrong{\sphinxupquote{key}} \textendash{} (str) The key to check presence in in-Config node

\end{itemize}

\item[{Returns}] \leavevmode
(RCO.ReturnCode) ‘OK’ if presence

\end{description}\end{quote}

\end{fulllineitems}

\index{config\_has\_application() (in module src.utilsSat)}

\begin{fulllineitems}
\phantomsection\label{\detokenize{apidoc_src/src:src.utilsSat.config_has_application}}\pysiglinewithargsret{\sphinxcode{\sphinxupquote{src.utilsSat.}}\sphinxbfcode{\sphinxupquote{config\_has\_application}}}{\emph{config}}{}
\end{fulllineitems}

\index{critical() (in module src.utilsSat)}

\begin{fulllineitems}
\phantomsection\label{\detokenize{apidoc_src/src:src.utilsSat.critical}}\pysiglinewithargsret{\sphinxcode{\sphinxupquote{src.utilsSat.}}\sphinxbfcode{\sphinxupquote{critical}}}{\emph{msg}}{}
\end{fulllineitems}

\index{cyan() (in module src.utilsSat)}

\begin{fulllineitems}
\phantomsection\label{\detokenize{apidoc_src/src:src.utilsSat.cyan}}\pysiglinewithargsret{\sphinxcode{\sphinxupquote{src.utilsSat.}}\sphinxbfcode{\sphinxupquote{cyan}}}{\emph{msg}}{}
\end{fulllineitems}

\index{date\_to\_datetime() (in module src.utilsSat)}

\begin{fulllineitems}
\phantomsection\label{\detokenize{apidoc_src/src:src.utilsSat.date_to_datetime}}\pysiglinewithargsret{\sphinxcode{\sphinxupquote{src.utilsSat.}}\sphinxbfcode{\sphinxupquote{date\_to\_datetime}}}{\emph{date}}{}
From a string date in format YYYYMMDD\_HHMMSS
returns list year, mon, day, hour, minutes, seconds
\begin{quote}\begin{description}
\item[{Parameters}] \leavevmode
\sphinxstyleliteralstrong{\sphinxupquote{date}} \textendash{} (str) The date in format YYYYMMDD\_HHMMSS

\item[{Returns}] \leavevmode
(tuple) as (str,str,str,str,str,str)
The same date and time in separate variables.

\end{description}\end{quote}

\end{fulllineitems}

\index{deepcopy\_list() (in module src.utilsSat)}

\begin{fulllineitems}
\phantomsection\label{\detokenize{apidoc_src/src:src.utilsSat.deepcopy_list}}\pysiglinewithargsret{\sphinxcode{\sphinxupquote{src.utilsSat.}}\sphinxbfcode{\sphinxupquote{deepcopy\_list}}}{\emph{input\_list}}{}
Do a deep copy of a list
\begin{quote}\begin{description}
\item[{Parameters}] \leavevmode
\sphinxstyleliteralstrong{\sphinxupquote{input\_list}} \textendash{} (list) The list to copy

\item[{Returns}] \leavevmode
(list) The copy of the list

\end{description}\end{quote}

\end{fulllineitems}

\index{ensure\_path\_exists() (in module src.utilsSat)}

\begin{fulllineitems}
\phantomsection\label{\detokenize{apidoc_src/src:src.utilsSat.ensure_path_exists}}\pysiglinewithargsret{\sphinxcode{\sphinxupquote{src.utilsSat.}}\sphinxbfcode{\sphinxupquote{ensure\_path\_exists}}}{\emph{path}}{}
Create a path if not existing
\begin{quote}\begin{description}
\item[{Parameters}] \leavevmode
\sphinxstyleliteralstrong{\sphinxupquote{path}} \textendash{} (str) The path.

\end{description}\end{quote}

\end{fulllineitems}

\index{error() (in module src.utilsSat)}

\begin{fulllineitems}
\phantomsection\label{\detokenize{apidoc_src/src:src.utilsSat.error}}\pysiglinewithargsret{\sphinxcode{\sphinxupquote{src.utilsSat.}}\sphinxbfcode{\sphinxupquote{error}}}{\emph{msg}}{}
\end{fulllineitems}

\index{find\_file\_in\_lpath() (in module src.utilsSat)}

\begin{fulllineitems}
\phantomsection\label{\detokenize{apidoc_src/src:src.utilsSat.find_file_in_lpath}}\pysiglinewithargsret{\sphinxcode{\sphinxupquote{src.utilsSat.}}\sphinxbfcode{\sphinxupquote{find\_file\_in\_lpath}}}{\emph{file\_name}, \emph{lpath}, \emph{additional\_dir=''}}{}
Find in all the directories in lpath list the file 
that has the same name as file\_name. 
If it is found, return the full path of the file, else, return False. 
The additional\_dir (optional) is the name of the directory 
to add to all paths in lpath.
\begin{quote}\begin{description}
\item[{Parameters}] \leavevmode\begin{itemize}
\item {} 
\sphinxstyleliteralstrong{\sphinxupquote{file\_name}} \textendash{} (str) The file name to search

\item {} 
\sphinxstyleliteralstrong{\sphinxupquote{lpath}} \textendash{} (list) The list of directories where to search

\item {} 
\sphinxstyleliteralstrong{\sphinxupquote{additional\_dir}} \textendash{} (str) The name of the additional directory

\end{itemize}

\item[{Returns}] \leavevmode
(str) The full path of the file or False if not found

\end{description}\end{quote}

\end{fulllineitems}

\index{formatTuples() (in module src.utilsSat)}

\begin{fulllineitems}
\phantomsection\label{\detokenize{apidoc_src/src:src.utilsSat.formatTuples}}\pysiglinewithargsret{\sphinxcode{\sphinxupquote{src.utilsSat.}}\sphinxbfcode{\sphinxupquote{formatTuples}}}{\emph{tuples}}{}
Format ‘label = value’ the tuples in a tabulated way.
\begin{quote}\begin{description}
\item[{Parameters}] \leavevmode
\sphinxstyleliteralstrong{\sphinxupquote{tuples}} \textendash{} (list) The list of tuples to format

\item[{Returns}] \leavevmode
(str) The tabulated text. (as mutiples lines)

\end{description}\end{quote}

\end{fulllineitems}

\index{formatValue() (in module src.utilsSat)}

\begin{fulllineitems}
\phantomsection\label{\detokenize{apidoc_src/src:src.utilsSat.formatValue}}\pysiglinewithargsret{\sphinxcode{\sphinxupquote{src.utilsSat.}}\sphinxbfcode{\sphinxupquote{formatValue}}}{\emph{label}, \emph{value}, \emph{suffix=''}}{}
format ‘label = value’ with the info color
\begin{quote}\begin{description}
\item[{Parameters}] \leavevmode\begin{itemize}
\item {} 
\sphinxstyleliteralstrong{\sphinxupquote{label}} \textendash{} (int) the label to print.

\item {} 
\sphinxstyleliteralstrong{\sphinxupquote{value}} \textendash{} (str) the value to print.

\item {} 
\sphinxstyleliteralstrong{\sphinxupquote{suffix}} \textendash{} (str) the optionnal suffix to add at the end.

\end{itemize}

\end{description}\end{quote}

\end{fulllineitems}

\index{get\_base\_path() (in module src.utilsSat)}

\begin{fulllineitems}
\phantomsection\label{\detokenize{apidoc_src/src:src.utilsSat.get_base_path}}\pysiglinewithargsret{\sphinxcode{\sphinxupquote{src.utilsSat.}}\sphinxbfcode{\sphinxupquote{get\_base\_path}}}{\emph{config}}{}
Returns the path of the products base.
\begin{quote}\begin{description}
\item[{Parameters}] \leavevmode
\sphinxstyleliteralstrong{\sphinxupquote{config}} \textendash{} (Config) The global Config instance.

\item[{Returns}] \leavevmode
(str) The path of the products base.

\end{description}\end{quote}

\end{fulllineitems}

\index{get\_cfg\_param() (in module src.utilsSat)}

\begin{fulllineitems}
\phantomsection\label{\detokenize{apidoc_src/src:src.utilsSat.get_cfg_param}}\pysiglinewithargsret{\sphinxcode{\sphinxupquote{src.utilsSat.}}\sphinxbfcode{\sphinxupquote{get\_cfg\_param}}}{\emph{config}, \emph{param\_name}, \emph{default}}{}
Search for param\_name value in config.
If param\_name is not in config, then return default,
else, return the found value
\begin{quote}\begin{description}
\item[{Parameters}] \leavevmode\begin{itemize}
\item {} 
\sphinxstyleliteralstrong{\sphinxupquote{config}} \textendash{} (Config) The config.

\item {} 
\sphinxstyleliteralstrong{\sphinxupquote{param\_name}} \textendash{} (str) the name of the parameter to get the value

\item {} 
\sphinxstyleliteralstrong{\sphinxupquote{default}} \textendash{} (str) The value to return if param\_name is not in config

\end{itemize}

\item[{Returns}] \leavevmode
(str) see initial description of the function

\end{description}\end{quote}

\end{fulllineitems}

\index{get\_launcher\_name() (in module src.utilsSat)}

\begin{fulllineitems}
\phantomsection\label{\detokenize{apidoc_src/src:src.utilsSat.get_launcher_name}}\pysiglinewithargsret{\sphinxcode{\sphinxupquote{src.utilsSat.}}\sphinxbfcode{\sphinxupquote{get\_launcher\_name}}}{\emph{config}}{}
Returns the name of application file launcher, ‘salome’ by default.
\begin{quote}\begin{description}
\item[{Parameters}] \leavevmode
\sphinxstyleliteralstrong{\sphinxupquote{config}} \textendash{} (Config) The global Config instance.

\item[{Returns}] \leavevmode
(str) The name of salome launcher.

\end{description}\end{quote}

\end{fulllineitems}

\index{get\_log\_path() (in module src.utilsSat)}

\begin{fulllineitems}
\phantomsection\label{\detokenize{apidoc_src/src:src.utilsSat.get_log_path}}\pysiglinewithargsret{\sphinxcode{\sphinxupquote{src.utilsSat.}}\sphinxbfcode{\sphinxupquote{get\_log\_path}}}{\emph{config}}{}
Returns the path of the logs.
\begin{quote}\begin{description}
\item[{Parameters}] \leavevmode
\sphinxstyleliteralstrong{\sphinxupquote{config}} \textendash{} (Config) The global Config instance.

\item[{Returns}] \leavevmode
(str) The path of the logs.

\end{description}\end{quote}

\end{fulllineitems}

\index{get\_property\_in\_product\_cfg() (in module src.utilsSat)}

\begin{fulllineitems}
\phantomsection\label{\detokenize{apidoc_src/src:src.utilsSat.get_property_in_product_cfg}}\pysiglinewithargsret{\sphinxcode{\sphinxupquote{src.utilsSat.}}\sphinxbfcode{\sphinxupquote{get\_property\_in\_product\_cfg}}}{\emph{product\_cfg}, \emph{pprty}}{}
\end{fulllineitems}

\index{get\_salome\_version() (in module src.utilsSat)}

\begin{fulllineitems}
\phantomsection\label{\detokenize{apidoc_src/src:src.utilsSat.get_salome_version}}\pysiglinewithargsret{\sphinxcode{\sphinxupquote{src.utilsSat.}}\sphinxbfcode{\sphinxupquote{get\_salome\_version}}}{\emph{config}}{}
\end{fulllineitems}

\index{get\_tmp\_filename() (in module src.utilsSat)}

\begin{fulllineitems}
\phantomsection\label{\detokenize{apidoc_src/src:src.utilsSat.get_tmp_filename}}\pysiglinewithargsret{\sphinxcode{\sphinxupquote{src.utilsSat.}}\sphinxbfcode{\sphinxupquote{get\_tmp\_filename}}}{\emph{config}, \emph{name}}{}
\end{fulllineitems}

\index{green() (in module src.utilsSat)}

\begin{fulllineitems}
\phantomsection\label{\detokenize{apidoc_src/src:src.utilsSat.green}}\pysiglinewithargsret{\sphinxcode{\sphinxupquote{src.utilsSat.}}\sphinxbfcode{\sphinxupquote{green}}}{\emph{msg}}{}
\end{fulllineitems}

\index{handleRemoveReadonly() (in module src.utilsSat)}

\begin{fulllineitems}
\phantomsection\label{\detokenize{apidoc_src/src:src.utilsSat.handleRemoveReadonly}}\pysiglinewithargsret{\sphinxcode{\sphinxupquote{src.utilsSat.}}\sphinxbfcode{\sphinxupquote{handleRemoveReadonly}}}{\emph{func}, \emph{path}, \emph{exc}}{}
\end{fulllineitems}

\index{header() (in module src.utilsSat)}

\begin{fulllineitems}
\phantomsection\label{\detokenize{apidoc_src/src:src.utilsSat.header}}\pysiglinewithargsret{\sphinxcode{\sphinxupquote{src.utilsSat.}}\sphinxbfcode{\sphinxupquote{header}}}{\emph{msg}}{}
\end{fulllineitems}

\index{info() (in module src.utilsSat)}

\begin{fulllineitems}
\phantomsection\label{\detokenize{apidoc_src/src:src.utilsSat.info}}\pysiglinewithargsret{\sphinxcode{\sphinxupquote{src.utilsSat.}}\sphinxbfcode{\sphinxupquote{info}}}{\emph{msg}}{}
\end{fulllineitems}

\index{label() (in module src.utilsSat)}

\begin{fulllineitems}
\phantomsection\label{\detokenize{apidoc_src/src:src.utilsSat.label}}\pysiglinewithargsret{\sphinxcode{\sphinxupquote{src.utilsSat.}}\sphinxbfcode{\sphinxupquote{label}}}{\emph{msg}}{}
\end{fulllineitems}

\index{list\_log\_file() (in module src.utilsSat)}

\begin{fulllineitems}
\phantomsection\label{\detokenize{apidoc_src/src:src.utilsSat.list_log_file}}\pysiglinewithargsret{\sphinxcode{\sphinxupquote{src.utilsSat.}}\sphinxbfcode{\sphinxupquote{list\_log\_file}}}{\emph{dirPath}, \emph{expression}}{}
Find all files corresponding to expression in dirPath
\begin{quote}\begin{description}
\item[{Parameters}] \leavevmode\begin{itemize}
\item {} 
\sphinxstyleliteralstrong{\sphinxupquote{dirPath}} \textendash{} (str) the directory where to search the files

\item {} 
\sphinxstyleliteralstrong{\sphinxupquote{expression}} \textendash{} (str) the regular expression of files to find

\end{itemize}

\item[{Returns}] \leavevmode
(list) the list of files path and informations about it

\end{description}\end{quote}

\end{fulllineitems}

\index{log\_res\_step() (in module src.utilsSat)}

\begin{fulllineitems}
\phantomsection\label{\detokenize{apidoc_src/src:src.utilsSat.log_res_step}}\pysiglinewithargsret{\sphinxcode{\sphinxupquote{src.utilsSat.}}\sphinxbfcode{\sphinxupquote{log\_res\_step}}}{\emph{logger}, \emph{res}}{}
\end{fulllineitems}

\index{log\_step() (in module src.utilsSat)}

\begin{fulllineitems}
\phantomsection\label{\detokenize{apidoc_src/src:src.utilsSat.log_step}}\pysiglinewithargsret{\sphinxcode{\sphinxupquote{src.utilsSat.}}\sphinxbfcode{\sphinxupquote{log\_step}}}{\emph{logger}, \emph{header}, \emph{step}}{}
\end{fulllineitems}

\index{logger\_info\_tuples() (in module src.utilsSat)}

\begin{fulllineitems}
\phantomsection\label{\detokenize{apidoc_src/src:src.utilsSat.logger_info_tuples}}\pysiglinewithargsret{\sphinxcode{\sphinxupquote{src.utilsSat.}}\sphinxbfcode{\sphinxupquote{logger\_info\_tuples}}}{\emph{logger}, \emph{tuples}}{}
For convenience
format as formatTuples() and call logger.info()

\end{fulllineitems}

\index{magenta() (in module src.utilsSat)}

\begin{fulllineitems}
\phantomsection\label{\detokenize{apidoc_src/src:src.utilsSat.magenta}}\pysiglinewithargsret{\sphinxcode{\sphinxupquote{src.utilsSat.}}\sphinxbfcode{\sphinxupquote{magenta}}}{\emph{msg}}{}
\end{fulllineitems}

\index{merge\_dicts() (in module src.utilsSat)}

\begin{fulllineitems}
\phantomsection\label{\detokenize{apidoc_src/src:src.utilsSat.merge_dicts}}\pysiglinewithargsret{\sphinxcode{\sphinxupquote{src.utilsSat.}}\sphinxbfcode{\sphinxupquote{merge\_dicts}}}{\emph{*dict\_args}}{}
Given any number of dicts, shallow copy and merge into a new dict,
precedence goes to key value pairs in latter dicts.

\end{fulllineitems}

\index{normal() (in module src.utilsSat)}

\begin{fulllineitems}
\phantomsection\label{\detokenize{apidoc_src/src:src.utilsSat.normal}}\pysiglinewithargsret{\sphinxcode{\sphinxupquote{src.utilsSat.}}\sphinxbfcode{\sphinxupquote{normal}}}{\emph{msg}}{}
\end{fulllineitems}

\index{only\_numbers() (in module src.utilsSat)}

\begin{fulllineitems}
\phantomsection\label{\detokenize{apidoc_src/src:src.utilsSat.only_numbers}}\pysiglinewithargsret{\sphinxcode{\sphinxupquote{src.utilsSat.}}\sphinxbfcode{\sphinxupquote{only\_numbers}}}{\emph{str\_num}}{}
\end{fulllineitems}

\index{parse\_date() (in module src.utilsSat)}

\begin{fulllineitems}
\phantomsection\label{\detokenize{apidoc_src/src:src.utilsSat.parse_date}}\pysiglinewithargsret{\sphinxcode{\sphinxupquote{src.utilsSat.}}\sphinxbfcode{\sphinxupquote{parse\_date}}}{\emph{date}}{}
Transform YYYYMMDD\_hhmmss into YYYY-MM-DD hh:mm:ss.
\begin{quote}\begin{description}
\item[{Parameters}] \leavevmode
\sphinxstyleliteralstrong{\sphinxupquote{date}} \textendash{} (str) The date to transform

\item[{Returns}] \leavevmode
(str) The date in the new format

\end{description}\end{quote}

\end{fulllineitems}

\index{read\_config\_from\_a\_file() (in module src.utilsSat)}

\begin{fulllineitems}
\phantomsection\label{\detokenize{apidoc_src/src:src.utilsSat.read_config_from_a_file}}\pysiglinewithargsret{\sphinxcode{\sphinxupquote{src.utilsSat.}}\sphinxbfcode{\sphinxupquote{read\_config\_from\_a\_file}}}{\emph{filePath}}{}
\end{fulllineitems}

\index{red() (in module src.utilsSat)}

\begin{fulllineitems}
\phantomsection\label{\detokenize{apidoc_src/src:src.utilsSat.red}}\pysiglinewithargsret{\sphinxcode{\sphinxupquote{src.utilsSat.}}\sphinxbfcode{\sphinxupquote{red}}}{\emph{msg}}{}
\end{fulllineitems}

\index{remove\_item\_from\_list() (in module src.utilsSat)}

\begin{fulllineitems}
\phantomsection\label{\detokenize{apidoc_src/src:src.utilsSat.remove_item_from_list}}\pysiglinewithargsret{\sphinxcode{\sphinxupquote{src.utilsSat.}}\sphinxbfcode{\sphinxupquote{remove\_item\_from\_list}}}{\emph{input\_list}, \emph{item}}{}
Remove all occurences of item from input\_list
\begin{quote}\begin{description}
\item[{Parameters}] \leavevmode
\sphinxstyleliteralstrong{\sphinxupquote{input\_list}} \textendash{} (list) The list to modify

\item[{Returns}] \leavevmode
(list) The without any item

\end{description}\end{quote}

\end{fulllineitems}

\index{replace\_in\_file() (in module src.utilsSat)}

\begin{fulllineitems}
\phantomsection\label{\detokenize{apidoc_src/src:src.utilsSat.replace_in_file}}\pysiglinewithargsret{\sphinxcode{\sphinxupquote{src.utilsSat.}}\sphinxbfcode{\sphinxupquote{replace\_in\_file}}}{\emph{file\_in}, \emph{str\_in}, \emph{str\_out}}{}
Replace \textless{}str\_in\textgreater{} by \textless{}str\_out\textgreater{} in file \textless{}file\_in\textgreater{}
\begin{quote}\begin{description}
\item[{Parameters}] \leavevmode\begin{itemize}
\item {} 
\sphinxstyleliteralstrong{\sphinxupquote{file\_in}} \textendash{} (str) The file name

\item {} 
\sphinxstyleliteralstrong{\sphinxupquote{str\_in}} \textendash{} (str) The string to search

\item {} 
\sphinxstyleliteralstrong{\sphinxupquote{str\_out}} \textendash{} (str) The string to replace.

\end{itemize}

\end{description}\end{quote}

\end{fulllineitems}

\index{reset() (in module src.utilsSat)}

\begin{fulllineitems}
\phantomsection\label{\detokenize{apidoc_src/src:src.utilsSat.reset}}\pysiglinewithargsret{\sphinxcode{\sphinxupquote{src.utilsSat.}}\sphinxbfcode{\sphinxupquote{reset}}}{\emph{msg}}{}
\end{fulllineitems}

\index{show\_command\_log() (in module src.utilsSat)}

\begin{fulllineitems}
\phantomsection\label{\detokenize{apidoc_src/src:src.utilsSat.show_command_log}}\pysiglinewithargsret{\sphinxcode{\sphinxupquote{src.utilsSat.}}\sphinxbfcode{\sphinxupquote{show\_command\_log}}}{\emph{logFilePath}, \emph{cmd}, \emph{application}, \emph{notShownCommands}}{}
Used in updateHatXml. 
Determine if the log xml file logFilePath 
has to be shown or not in the hat log.
\begin{quote}\begin{description}
\item[{Parameters}] \leavevmode\begin{itemize}
\item {} 
\sphinxstyleliteralstrong{\sphinxupquote{logFilePath}} \textendash{} (str) the path to the command xml log file

\item {} 
\sphinxstyleliteralstrong{\sphinxupquote{cmd}} \textendash{} (str) the command of the log file

\item {} 
\sphinxstyleliteralstrong{\sphinxupquote{application}} \textendash{} (str) 
The application passed as parameter to the salomeTools command

\item {} 
\sphinxstyleliteralstrong{\sphinxupquote{notShownCommands}} \textendash{} (list) 
The list of commands that are not shown by default

\end{itemize}

\item[{Returns}] \leavevmode
(RCO.ReturnCode)
OK if cmd is not in notShownCommands and the application 
in the log file corresponds to application
ReturnCode value is tuple (appliLog, launched\_cmd)

\end{description}\end{quote}

\end{fulllineitems}

\index{success() (in module src.utilsSat)}

\begin{fulllineitems}
\phantomsection\label{\detokenize{apidoc_src/src:src.utilsSat.success}}\pysiglinewithargsret{\sphinxcode{\sphinxupquote{src.utilsSat.}}\sphinxbfcode{\sphinxupquote{success}}}{\emph{msg}}{}
\end{fulllineitems}

\index{timedelta\_total\_seconds() (in module src.utilsSat)}

\begin{fulllineitems}
\phantomsection\label{\detokenize{apidoc_src/src:src.utilsSat.timedelta_total_seconds}}\pysiglinewithargsret{\sphinxcode{\sphinxupquote{src.utilsSat.}}\sphinxbfcode{\sphinxupquote{timedelta\_total\_seconds}}}{\emph{timedelta}}{}
Replace total\_seconds from datetime module 
in order to be compatible with old python versions
\begin{quote}\begin{description}
\item[{Parameters}] \leavevmode
\sphinxstyleliteralstrong{\sphinxupquote{timedelta}} \textendash{} (datetime.timedelta) 
The delta between two dates

\item[{Returns}] \leavevmode
(float) 
The number of seconds corresponding to timedelta.

\end{description}\end{quote}

\end{fulllineitems}

\index{update\_hat\_xml() (in module src.utilsSat)}

\begin{fulllineitems}
\phantomsection\label{\detokenize{apidoc_src/src:src.utilsSat.update_hat_xml}}\pysiglinewithargsret{\sphinxcode{\sphinxupquote{src.utilsSat.}}\sphinxbfcode{\sphinxupquote{update\_hat\_xml}}}{\emph{logDir}, \emph{application=None}, \emph{notShownCommands={[}{]}}}{}
Create the xml file in logDir that contain all the xml file 
and have a name like YYYYMMDD\_HHMMSS\_namecmd.xml
\begin{quote}\begin{description}
\item[{Parameters}] \leavevmode\begin{itemize}
\item {} 
\sphinxstyleliteralstrong{\sphinxupquote{logDir}} \textendash{} (str) the directory to parse

\item {} 
\sphinxstyleliteralstrong{\sphinxupquote{application}} \textendash{} (str) the name of the application if there is any

\end{itemize}

\end{description}\end{quote}

\end{fulllineitems}

\index{warning() (in module src.utilsSat)}

\begin{fulllineitems}
\phantomsection\label{\detokenize{apidoc_src/src:src.utilsSat.warning}}\pysiglinewithargsret{\sphinxcode{\sphinxupquote{src.utilsSat.}}\sphinxbfcode{\sphinxupquote{warning}}}{\emph{msg}}{}
\end{fulllineitems}

\index{white() (in module src.utilsSat)}

\begin{fulllineitems}
\phantomsection\label{\detokenize{apidoc_src/src:src.utilsSat.white}}\pysiglinewithargsret{\sphinxcode{\sphinxupquote{src.utilsSat.}}\sphinxbfcode{\sphinxupquote{white}}}{\emph{msg}}{}
\end{fulllineitems}

\index{yellow() (in module src.utilsSat)}

\begin{fulllineitems}
\phantomsection\label{\detokenize{apidoc_src/src:src.utilsSat.yellow}}\pysiglinewithargsret{\sphinxcode{\sphinxupquote{src.utilsSat.}}\sphinxbfcode{\sphinxupquote{yellow}}}{\emph{msg}}{}
\end{fulllineitems}



\subsubsection{src.xmlManager module}
\label{\detokenize{apidoc_src/src:src-xmlmanager-module}}\label{\detokenize{apidoc_src/src:module-src.xmlManager}}\index{src.xmlManager (module)}
Utilities to read xml logging files
\begin{description}
\item[{usage:}] \leavevmode
\textgreater{}\textgreater{} import src.xmlManager as XMLMGR

\end{description}
\index{ReadXmlFile (class in src.xmlManager)}

\begin{fulllineitems}
\phantomsection\label{\detokenize{apidoc_src/src:src.xmlManager.ReadXmlFile}}\pysiglinewithargsret{\sphinxbfcode{\sphinxupquote{class }}\sphinxcode{\sphinxupquote{src.xmlManager.}}\sphinxbfcode{\sphinxupquote{ReadXmlFile}}}{\emph{filePath}}{}
Bases: \sphinxcode{\sphinxupquote{object}}

Class to manage reading of an xml log file
\index{getRootAttrib() (src.xmlManager.ReadXmlFile method)}

\begin{fulllineitems}
\phantomsection\label{\detokenize{apidoc_src/src:src.xmlManager.ReadXmlFile.getRootAttrib}}\pysiglinewithargsret{\sphinxbfcode{\sphinxupquote{getRootAttrib}}}{}{}
Get the attibutes of the self.xmlroot
\begin{quote}\begin{description}
\item[{Returns}] \leavevmode
(dict) The attributes of the root node

\end{description}\end{quote}

\end{fulllineitems}

\index{get\_attrib() (src.xmlManager.ReadXmlFile method)}

\begin{fulllineitems}
\phantomsection\label{\detokenize{apidoc_src/src:src.xmlManager.ReadXmlFile.get_attrib}}\pysiglinewithargsret{\sphinxbfcode{\sphinxupquote{get\_attrib}}}{\emph{node\_name}}{}
Get the attibutes of the node node\_name in self.xmlroot
\begin{quote}\begin{description}
\item[{Parameters}] \leavevmode
\sphinxstyleliteralstrong{\sphinxupquote{node\_name}} \textendash{} (str) the name of the node

\item[{Returns}] \leavevmode
(dict) the attibutes of the node node\_name in self.xmlroot

\end{description}\end{quote}

\end{fulllineitems}

\index{get\_node\_text() (src.xmlManager.ReadXmlFile method)}

\begin{fulllineitems}
\phantomsection\label{\detokenize{apidoc_src/src:src.xmlManager.ReadXmlFile.get_node_text}}\pysiglinewithargsret{\sphinxbfcode{\sphinxupquote{get\_node\_text}}}{\emph{node}}{}
Get the text of the first node that has name 
that corresponds to the parameter node
\begin{quote}\begin{description}
\item[{Parameters}] \leavevmode
\sphinxstyleliteralstrong{\sphinxupquote{node}} \textendash{} (str) the name of the node from which get the text

\item[{Returns}] \leavevmode
(str) 
The text of the first node that has name 
that corresponds to the parameter node

\end{description}\end{quote}

\end{fulllineitems}


\end{fulllineitems}

\index{XmlLogFile (class in src.xmlManager)}

\begin{fulllineitems}
\phantomsection\label{\detokenize{apidoc_src/src:src.xmlManager.XmlLogFile}}\pysiglinewithargsret{\sphinxbfcode{\sphinxupquote{class }}\sphinxcode{\sphinxupquote{src.xmlManager.}}\sphinxbfcode{\sphinxupquote{XmlLogFile}}}{\emph{filePath}, \emph{rootname}, \emph{attrib=\{\}}}{}
Bases: \sphinxcode{\sphinxupquote{object}}

Class to manage writing in salomeTools xml log file
\index{add\_simple\_node() (src.xmlManager.XmlLogFile method)}

\begin{fulllineitems}
\phantomsection\label{\detokenize{apidoc_src/src:src.xmlManager.XmlLogFile.add_simple_node}}\pysiglinewithargsret{\sphinxbfcode{\sphinxupquote{add\_simple\_node}}}{\emph{node\_name}, \emph{text=None}, \emph{attrib=\{\}}}{}
Add a node with some attibutes and text to the root node.
\begin{quote}\begin{description}
\item[{Parameters}] \leavevmode\begin{itemize}
\item {} 
\sphinxstyleliteralstrong{\sphinxupquote{node\_name}} \textendash{} (str) the name of the node to add

\item {} 
\sphinxstyleliteralstrong{\sphinxupquote{text}} \textendash{} (str) the text of the node

\item {} 
\sphinxstyleliteralstrong{\sphinxupquote{attrib}} \textendash{} (dict)
The dictionary containing the attribute of the new node

\end{itemize}

\end{description}\end{quote}

\end{fulllineitems}

\index{append\_node\_attrib() (src.xmlManager.XmlLogFile method)}

\begin{fulllineitems}
\phantomsection\label{\detokenize{apidoc_src/src:src.xmlManager.XmlLogFile.append_node_attrib}}\pysiglinewithargsret{\sphinxbfcode{\sphinxupquote{append\_node\_attrib}}}{\emph{node\_name}, \emph{attrib}}{}
Append a new attributes to the node that has node\_name as name
\begin{quote}\begin{description}
\item[{Parameters}] \leavevmode\begin{itemize}
\item {} 
\sphinxstyleliteralstrong{\sphinxupquote{node\_name}} \textendash{} (str) The name of the node on which append text

\item {} 
\sphinxstyleliteralstrong{\sphinxupquote{attrib}} \textendash{} (dict) The attrib to append

\end{itemize}

\end{description}\end{quote}

\end{fulllineitems}

\index{append\_node\_text() (src.xmlManager.XmlLogFile method)}

\begin{fulllineitems}
\phantomsection\label{\detokenize{apidoc_src/src:src.xmlManager.XmlLogFile.append_node_text}}\pysiglinewithargsret{\sphinxbfcode{\sphinxupquote{append\_node\_text}}}{\emph{node\_name}, \emph{text}}{}
Append a new text to the node that has node\_name as name
\begin{quote}\begin{description}
\item[{Parameters}] \leavevmode\begin{itemize}
\item {} 
\sphinxstyleliteralstrong{\sphinxupquote{node\_name}} \textendash{} (str) The name of the node on which append text

\item {} 
\sphinxstyleliteralstrong{\sphinxupquote{text}} \textendash{} (str) The text to append

\end{itemize}

\end{description}\end{quote}

\end{fulllineitems}

\index{write\_tree() (src.xmlManager.XmlLogFile method)}

\begin{fulllineitems}
\phantomsection\label{\detokenize{apidoc_src/src:src.xmlManager.XmlLogFile.write_tree}}\pysiglinewithargsret{\sphinxbfcode{\sphinxupquote{write\_tree}}}{\emph{stylesheet=None}, \emph{file\_path=None}}{}
Write the xml tree in the log file path. Add the stylesheet if asked.
\begin{quote}\begin{description}
\item[{Parameters}] \leavevmode
\sphinxstyleliteralstrong{\sphinxupquote{stylesheet}} \textendash{} (str) The stylesheet to apply to the xml file

\end{description}\end{quote}

\end{fulllineitems}


\end{fulllineitems}

\index{add\_simple\_node() (in module src.xmlManager)}

\begin{fulllineitems}
\phantomsection\label{\detokenize{apidoc_src/src:src.xmlManager.add_simple_node}}\pysiglinewithargsret{\sphinxcode{\sphinxupquote{src.xmlManager.}}\sphinxbfcode{\sphinxupquote{add\_simple\_node}}}{\emph{root\_node}, \emph{node\_name}, \emph{text=None}, \emph{attrib=\{\}}}{}
Add a node with some attibutes and text to the root node.
\begin{quote}\begin{description}
\item[{Parameters}] \leavevmode\begin{itemize}
\item {} 
\sphinxstyleliteralstrong{\sphinxupquote{root\_node}} \textendash{} (etree.Element) 
the Etree element where to add the new node

\item {} 
\sphinxstyleliteralstrong{\sphinxupquote{node\_name}} \textendash{} (str) the name of the node to add

\item {} 
\sphinxstyleliteralstrong{\sphinxupquote{text}} \textendash{} (str) the text of the node

\item {} 
\sphinxstyleliteralstrong{\sphinxupquote{attrib}} \textendash{} (dict) 
the dictionary containing the attribute(s) of the new node

\end{itemize}

\end{description}\end{quote}

\end{fulllineitems}

\index{append\_node\_attrib() (in module src.xmlManager)}

\begin{fulllineitems}
\phantomsection\label{\detokenize{apidoc_src/src:src.xmlManager.append_node_attrib}}\pysiglinewithargsret{\sphinxcode{\sphinxupquote{src.xmlManager.}}\sphinxbfcode{\sphinxupquote{append\_node\_attrib}}}{\emph{root\_node}, \emph{attrib}}{}
Append a new attributes to the node that has node\_name as name
\begin{quote}\begin{description}
\item[{Parameters}] \leavevmode\begin{itemize}
\item {} 
\sphinxstyleliteralstrong{\sphinxupquote{root\_node}} \textendash{} (etree.Element)
the Etree element where to append the new attibutes

\item {} 
\sphinxstyleliteralstrong{\sphinxupquote{attrib}} \textendash{} (dict) The attrib to append

\end{itemize}

\end{description}\end{quote}

\end{fulllineitems}

\index{find\_node\_by\_attrib() (in module src.xmlManager)}

\begin{fulllineitems}
\phantomsection\label{\detokenize{apidoc_src/src:src.xmlManager.find_node_by_attrib}}\pysiglinewithargsret{\sphinxcode{\sphinxupquote{src.xmlManager.}}\sphinxbfcode{\sphinxupquote{find\_node\_by\_attrib}}}{\emph{xmlroot}, \emph{name\_node}, \emph{key}, \emph{value}}{}
Find the first node from xmlroot that has name name\_node 
and that has in its attributes \{key : value\}. 
Return the node
\begin{quote}\begin{description}
\item[{Parameters}] \leavevmode\begin{itemize}
\item {} 
\sphinxstyleliteralstrong{\sphinxupquote{xmlroot}} \textendash{} (etree.Element) 
the Etree element where to search

\item {} 
\sphinxstyleliteralstrong{\sphinxupquote{name\_node}} \textendash{} (str) the name of node to search

\item {} 
\sphinxstyleliteralstrong{\sphinxupquote{key}} \textendash{} (str) the key to search

\item {} 
\sphinxstyleliteralstrong{\sphinxupquote{value}} \textendash{} (str) the value to search

\end{itemize}

\item[{Returns}] \leavevmode
(etree.Element) the found node

\end{description}\end{quote}

\end{fulllineitems}

\index{write\_report() (in module src.xmlManager)}

\begin{fulllineitems}
\phantomsection\label{\detokenize{apidoc_src/src:src.xmlManager.write_report}}\pysiglinewithargsret{\sphinxcode{\sphinxupquote{src.xmlManager.}}\sphinxbfcode{\sphinxupquote{write\_report}}}{\emph{filename}, \emph{xmlroot}, \emph{stylesheet}}{}
Writes a report file from a XML tree.
\begin{quote}\begin{description}
\item[{Parameters}] \leavevmode\begin{itemize}
\item {} 
\sphinxstyleliteralstrong{\sphinxupquote{filename}} \textendash{} (str) The path to the file to create

\item {} 
\sphinxstyleliteralstrong{\sphinxupquote{xmlroot}} \textendash{} (etree.Element) the Etree element to write to the file

\item {} 
\sphinxstyleliteralstrong{\sphinxupquote{stylesheet}} \textendash{} (str) The stylesheet to add to the begin of the file

\end{itemize}

\end{description}\end{quote}

\end{fulllineitems}



\subsubsection{Module contents}
\label{\detokenize{apidoc_src/src:module-src}}\label{\detokenize{apidoc_src/src:module-contents}}\index{src (module)}

\section{commands}
\label{\detokenize{apidoc_commands/modules:commands}}\label{\detokenize{apidoc_commands/modules::doc}}

\subsection{commands package}
\label{\detokenize{apidoc_commands/commands::doc}}\label{\detokenize{apidoc_commands/commands:commands-package}}

\subsubsection{Submodules}
\label{\detokenize{apidoc_commands/commands:submodules}}

\subsubsection{commands.application module}
\label{\detokenize{apidoc_commands/commands:module-commands.application}}\label{\detokenize{apidoc_commands/commands:commands-application-module}}\index{commands.application (module)}
Is a salomeTools command module
see Command class docstring, also used for help
\index{Command (class in commands.application)}

\begin{fulllineitems}
\phantomsection\label{\detokenize{apidoc_commands/commands:commands.application.Command}}\pysiglinewithargsret{\sphinxbfcode{\sphinxupquote{class }}\sphinxcode{\sphinxupquote{commands.application.}}\sphinxbfcode{\sphinxupquote{Command}}}{\emph{runner}}{}
Bases: \sphinxcode{\sphinxupquote{src.salomeTools.\_BaseCommand}}

The application command creates a SALOME application.

\begin{DUlineblock}{0em}
\item[] Warning:
\item[]
\begin{DUlineblock}{\DUlineblockindent}
\item[] It works only for SALOME 6.
\item[] Use the ‘launcher’ command for newer versions of SALOME
\item[] 
\end{DUlineblock}
\item[] Examples:
\item[] \textgreater{}\textgreater{} sat application SALOME-6.6.0
\end{DUlineblock}
\index{getParser() (commands.application.Command method)}

\begin{fulllineitems}
\phantomsection\label{\detokenize{apidoc_commands/commands:commands.application.Command.getParser}}\pysiglinewithargsret{\sphinxbfcode{\sphinxupquote{getParser}}}{}{}
Define all options for command ‘sat application \textless{}options\textgreater{}’

\end{fulllineitems}

\index{name (commands.application.Command attribute)}

\begin{fulllineitems}
\phantomsection\label{\detokenize{apidoc_commands/commands:commands.application.Command.name}}\pysigline{\sphinxbfcode{\sphinxupquote{name}}\sphinxbfcode{\sphinxupquote{ = 'application'}}}
\end{fulllineitems}

\index{run() (commands.application.Command method)}

\begin{fulllineitems}
\phantomsection\label{\detokenize{apidoc_commands/commands:commands.application.Command.run}}\pysiglinewithargsret{\sphinxbfcode{\sphinxupquote{run}}}{\emph{cmd\_arguments}}{}
method called for command ‘sat application \textless{}options\textgreater{}’

\end{fulllineitems}


\end{fulllineitems}

\index{add\_module\_to\_appli() (in module commands.application)}

\begin{fulllineitems}
\phantomsection\label{\detokenize{apidoc_commands/commands:commands.application.add_module_to_appli}}\pysiglinewithargsret{\sphinxcode{\sphinxupquote{commands.application.}}\sphinxbfcode{\sphinxupquote{add\_module\_to\_appli}}}{\emph{out}, \emph{module}, \emph{has\_gui}, \emph{module\_path}, \emph{logger}, \emph{flagline}}{}
add the definition of a module to out stream.

\end{fulllineitems}

\index{create\_application() (in module commands.application)}

\begin{fulllineitems}
\phantomsection\label{\detokenize{apidoc_commands/commands:commands.application.create_application}}\pysiglinewithargsret{\sphinxcode{\sphinxupquote{commands.application.}}\sphinxbfcode{\sphinxupquote{create\_application}}}{\emph{config}, \emph{appli\_dir}, \emph{catalog}, \emph{logger}, \emph{display=True}}{}
\end{fulllineitems}

\index{create\_config\_file() (in module commands.application)}

\begin{fulllineitems}
\phantomsection\label{\detokenize{apidoc_commands/commands:commands.application.create_config_file}}\pysiglinewithargsret{\sphinxcode{\sphinxupquote{commands.application.}}\sphinxbfcode{\sphinxupquote{create\_config\_file}}}{\emph{config}, \emph{modules}, \emph{env\_file}, \emph{logger}}{}
\end{fulllineitems}

\index{customize\_app() (in module commands.application)}

\begin{fulllineitems}
\phantomsection\label{\detokenize{apidoc_commands/commands:commands.application.customize_app}}\pysiglinewithargsret{\sphinxcode{\sphinxupquote{commands.application.}}\sphinxbfcode{\sphinxupquote{customize\_app}}}{\emph{config}, \emph{appli\_dir}, \emph{logger}}{}
Customizes the application by editing SalomeApp.xml.

\end{fulllineitems}

\index{generate\_application() (in module commands.application)}

\begin{fulllineitems}
\phantomsection\label{\detokenize{apidoc_commands/commands:commands.application.generate_application}}\pysiglinewithargsret{\sphinxcode{\sphinxupquote{commands.application.}}\sphinxbfcode{\sphinxupquote{generate\_application}}}{\emph{config}, \emph{appli\_dir}, \emph{config\_file}, \emph{logger}}{}
\end{fulllineitems}

\index{generate\_catalog() (in module commands.application)}

\begin{fulllineitems}
\phantomsection\label{\detokenize{apidoc_commands/commands:commands.application.generate_catalog}}\pysiglinewithargsret{\sphinxcode{\sphinxupquote{commands.application.}}\sphinxbfcode{\sphinxupquote{generate\_catalog}}}{\emph{machines}, \emph{config}, \emph{logger}}{}
Generates the catalog from a list of machines.

\end{fulllineitems}

\index{generate\_launch\_file() (in module commands.application)}

\begin{fulllineitems}
\phantomsection\label{\detokenize{apidoc_commands/commands:commands.application.generate_launch_file}}\pysiglinewithargsret{\sphinxcode{\sphinxupquote{commands.application.}}\sphinxbfcode{\sphinxupquote{generate\_launch\_file}}}{\emph{config}, \emph{appli\_dir}, \emph{catalog}, \emph{logger}, \emph{l\_SALOME\_modules}}{}
Obsolescent way of creating the application.
This method will use appli\_gen to create the application directory.

\end{fulllineitems}

\index{get\_SALOME\_modules() (in module commands.application)}

\begin{fulllineitems}
\phantomsection\label{\detokenize{apidoc_commands/commands:commands.application.get_SALOME_modules}}\pysiglinewithargsret{\sphinxcode{\sphinxupquote{commands.application.}}\sphinxbfcode{\sphinxupquote{get\_SALOME\_modules}}}{\emph{config}}{}
\end{fulllineitems}

\index{get\_step() (in module commands.application)}

\begin{fulllineitems}
\phantomsection\label{\detokenize{apidoc_commands/commands:commands.application.get_step}}\pysiglinewithargsret{\sphinxcode{\sphinxupquote{commands.application.}}\sphinxbfcode{\sphinxupquote{get\_step}}}{\emph{logger}, \emph{message}, \emph{pad=50}}{}
returns ‘message …….. ‘ with pad 50 by default
avoid colors ‘\textless{}color\textgreater{}’ for now in message

\end{fulllineitems}

\index{make\_alias() (in module commands.application)}

\begin{fulllineitems}
\phantomsection\label{\detokenize{apidoc_commands/commands:commands.application.make_alias}}\pysiglinewithargsret{\sphinxcode{\sphinxupquote{commands.application.}}\sphinxbfcode{\sphinxupquote{make\_alias}}}{\emph{appli\_path}, \emph{alias\_path}, \emph{force=False}}{}
\end{fulllineitems}



\subsubsection{commands.check module}
\label{\detokenize{apidoc_commands/commands:module-commands.check}}\label{\detokenize{apidoc_commands/commands:commands-check-module}}\index{commands.check (module)}\index{Command (class in commands.check)}

\begin{fulllineitems}
\phantomsection\label{\detokenize{apidoc_commands/commands:commands.check.Command}}\pysiglinewithargsret{\sphinxbfcode{\sphinxupquote{class }}\sphinxcode{\sphinxupquote{commands.check.}}\sphinxbfcode{\sphinxupquote{Command}}}{\emph{runner}}{}
Bases: \sphinxcode{\sphinxupquote{src.salomeTools.\_BaseCommand}}

The check command executes the ‘check’ command in the build directory of 
all the products of the application.
It is possible to reduce the list of products to check
by using the \textendash{}products option

\begin{DUlineblock}{0em}
\item[] examples:
\item[] \textgreater{}\textgreater{} sat check SALOME \textendash{}products KERNEL,GUI,GEOM
\end{DUlineblock}
\index{getParser() (commands.check.Command method)}

\begin{fulllineitems}
\phantomsection\label{\detokenize{apidoc_commands/commands:commands.check.Command.getParser}}\pysiglinewithargsret{\sphinxbfcode{\sphinxupquote{getParser}}}{}{}
Define all options for the check command ‘sat check \textless{}options\textgreater{}’

\end{fulllineitems}

\index{name (commands.check.Command attribute)}

\begin{fulllineitems}
\phantomsection\label{\detokenize{apidoc_commands/commands:commands.check.Command.name}}\pysigline{\sphinxbfcode{\sphinxupquote{name}}\sphinxbfcode{\sphinxupquote{ = 'check'}}}
\end{fulllineitems}

\index{run() (commands.check.Command method)}

\begin{fulllineitems}
\phantomsection\label{\detokenize{apidoc_commands/commands:commands.check.Command.run}}\pysiglinewithargsret{\sphinxbfcode{\sphinxupquote{run}}}{\emph{cmd\_arguments}}{}
method called for command ‘sat check \textless{}options\textgreater{}’

\end{fulllineitems}


\end{fulllineitems}

\index{check\_all\_products() (in module commands.check)}

\begin{fulllineitems}
\phantomsection\label{\detokenize{apidoc_commands/commands:commands.check.check_all_products}}\pysiglinewithargsret{\sphinxcode{\sphinxupquote{commands.check.}}\sphinxbfcode{\sphinxupquote{check\_all\_products}}}{\emph{config}, \emph{products\_infos}, \emph{logger}}{}
Execute the proper configuration commands 
in each product build directory.
\begin{quote}\begin{description}
\item[{Parameters}] \leavevmode\begin{itemize}
\item {} 
\sphinxstyleliteralstrong{\sphinxupquote{config}} \textendash{} (Config) The global configuration

\item {} 
\sphinxstyleliteralstrong{\sphinxupquote{products\_info}} \textendash{} (list) 
List of (str, Config) =\textgreater{} (product\_name, product\_info)

\item {} 
\sphinxstyleliteralstrong{\sphinxupquote{logger}} \textendash{} (Logger) 
The logger instance to use for the display and logging

\end{itemize}

\item[{Returns}] \leavevmode
(int) the number of failing commands.

\end{description}\end{quote}

\end{fulllineitems}

\index{check\_product() (in module commands.check)}

\begin{fulllineitems}
\phantomsection\label{\detokenize{apidoc_commands/commands:commands.check.check_product}}\pysiglinewithargsret{\sphinxcode{\sphinxupquote{commands.check.}}\sphinxbfcode{\sphinxupquote{check\_product}}}{\emph{p\_name\_info}, \emph{config}, \emph{logger}}{}
Execute the proper configuration command(s) 
in the product build directory.
\begin{quote}\begin{description}
\item[{Parameters}] \leavevmode\begin{itemize}
\item {} 
\sphinxstyleliteralstrong{\sphinxupquote{p\_name\_info}} \textendash{} (tuple) 
(str, Config) =\textgreater{} (product\_name, product\_info)

\item {} 
\sphinxstyleliteralstrong{\sphinxupquote{config}} \textendash{} (Config) The global configuration

\item {} 
\sphinxstyleliteralstrong{\sphinxupquote{logger}} \textendash{} (Logger) 
The logger instance to use for the display and logging

\end{itemize}

\item[{Returns}] \leavevmode
(int) 1 if it fails, else 0.

\end{description}\end{quote}

\end{fulllineitems}

\index{get\_products\_list() (in module commands.check)}

\begin{fulllineitems}
\phantomsection\label{\detokenize{apidoc_commands/commands:commands.check.get_products_list}}\pysiglinewithargsret{\sphinxcode{\sphinxupquote{commands.check.}}\sphinxbfcode{\sphinxupquote{get\_products\_list}}}{\emph{options}, \emph{cfg}, \emph{logger}}{}
method that gives the product list with their informations from 
configuration regarding the passed options.
\begin{quote}\begin{description}
\item[{Parameters}] \leavevmode\begin{itemize}
\item {} 
\sphinxstyleliteralstrong{\sphinxupquote{options}} \textendash{} (Options) The Options instance that stores 
the commands arguments

\item {} 
\sphinxstyleliteralstrong{\sphinxupquote{cfg}} \textendash{} (Config) The global configuration

\item {} 
\sphinxstyleliteralstrong{\sphinxupquote{logger}} \textendash{} (Logger) The logger instance to use 
for the display and logging

\end{itemize}

\item[{Returns}] \leavevmode
(list) The list of (product name, product\_informations).

\end{description}\end{quote}

\end{fulllineitems}



\subsubsection{commands.clean module}
\label{\detokenize{apidoc_commands/commands:commands-clean-module}}\label{\detokenize{apidoc_commands/commands:module-commands.clean}}\index{commands.clean (module)}\index{Command (class in commands.clean)}

\begin{fulllineitems}
\phantomsection\label{\detokenize{apidoc_commands/commands:commands.clean.Command}}\pysiglinewithargsret{\sphinxbfcode{\sphinxupquote{class }}\sphinxcode{\sphinxupquote{commands.clean.}}\sphinxbfcode{\sphinxupquote{Command}}}{\emph{runner}}{}
Bases: \sphinxcode{\sphinxupquote{src.salomeTools.\_BaseCommand}}

The clean command suppresses the source, build, or install directories 
of the application products.
Use the options to define what directories you want to suppress and 
to reduce the list of products

\begin{DUlineblock}{0em}
\item[] examples:
\item[] \textgreater{}\textgreater{} sat clean SALOME \textendash{}build \textendash{}install \textendash{}properties is\_salome\_module:yes
\end{DUlineblock}
\index{getParser() (commands.clean.Command method)}

\begin{fulllineitems}
\phantomsection\label{\detokenize{apidoc_commands/commands:commands.clean.Command.getParser}}\pysiglinewithargsret{\sphinxbfcode{\sphinxupquote{getParser}}}{}{}
Define all options for the command ‘sat clean \textless{}options\textgreater{}’

\end{fulllineitems}

\index{name (commands.clean.Command attribute)}

\begin{fulllineitems}
\phantomsection\label{\detokenize{apidoc_commands/commands:commands.clean.Command.name}}\pysigline{\sphinxbfcode{\sphinxupquote{name}}\sphinxbfcode{\sphinxupquote{ = 'clean'}}}
\end{fulllineitems}

\index{run() (commands.clean.Command method)}

\begin{fulllineitems}
\phantomsection\label{\detokenize{apidoc_commands/commands:commands.clean.Command.run}}\pysiglinewithargsret{\sphinxbfcode{\sphinxupquote{run}}}{\emph{cmd\_arguments}}{}
method called for command ‘sat clean \textless{}options\textgreater{}’

\end{fulllineitems}


\end{fulllineitems}

\index{get\_build\_directories() (in module commands.clean)}

\begin{fulllineitems}
\phantomsection\label{\detokenize{apidoc_commands/commands:commands.clean.get_build_directories}}\pysiglinewithargsret{\sphinxcode{\sphinxupquote{commands.clean.}}\sphinxbfcode{\sphinxupquote{get\_build\_directories}}}{\emph{products\_infos}}{}
Returns the list of directory build paths corresponding to the list of 
product information given as input.
\begin{quote}\begin{description}
\item[{Parameters}] \leavevmode
\sphinxstyleliteralstrong{\sphinxupquote{products\_infos}} \textendash{} (list)
The list of (name, config) corresponding to one product.

\item[{Returns}] \leavevmode
(list) the list of build paths.

\end{description}\end{quote}

\end{fulllineitems}

\index{get\_install\_directories() (in module commands.clean)}

\begin{fulllineitems}
\phantomsection\label{\detokenize{apidoc_commands/commands:commands.clean.get_install_directories}}\pysiglinewithargsret{\sphinxcode{\sphinxupquote{commands.clean.}}\sphinxbfcode{\sphinxupquote{get\_install\_directories}}}{\emph{products\_infos}}{}
Returns the list of directory install paths corresponding to the list of 
product information given as input.
\begin{quote}\begin{description}
\item[{Parameters}] \leavevmode
\sphinxstyleliteralstrong{\sphinxupquote{products\_infos}} \textendash{} (list) 
The list of (name, config) corresponding to one product.

\item[{Returns}] \leavevmode
(list) the list of install paths.

\end{description}\end{quote}

\end{fulllineitems}

\index{get\_source\_directories() (in module commands.clean)}

\begin{fulllineitems}
\phantomsection\label{\detokenize{apidoc_commands/commands:commands.clean.get_source_directories}}\pysiglinewithargsret{\sphinxcode{\sphinxupquote{commands.clean.}}\sphinxbfcode{\sphinxupquote{get\_source\_directories}}}{\emph{products\_infos}, \emph{without\_dev}}{}
Returns the list of directory source paths corresponding to the list of 
product information given as input. If without\_dev (bool), then
the dev products are ignored.
\begin{quote}\begin{description}
\item[{Parameters}] \leavevmode\begin{itemize}
\item {} 
\sphinxstyleliteralstrong{\sphinxupquote{products\_infos}} \textendash{} (list) 
The list of (name, config) corresponding to one product.

\item {} 
\sphinxstyleliteralstrong{\sphinxupquote{without\_dev}} \textendash{} (boolean) If True, then ignore the dev products.

\end{itemize}

\item[{Returns}] \leavevmode
(list) the list of source paths.

\end{description}\end{quote}

\end{fulllineitems}

\index{product\_has\_dir() (in module commands.clean)}

\begin{fulllineitems}
\phantomsection\label{\detokenize{apidoc_commands/commands:commands.clean.product_has_dir}}\pysiglinewithargsret{\sphinxcode{\sphinxupquote{commands.clean.}}\sphinxbfcode{\sphinxupquote{product\_has\_dir}}}{\emph{product\_info}, \emph{without\_dev=False}}{}
Returns a boolean at True if there is a source, build and install
directory corresponding to the product described by product\_info.
\begin{quote}\begin{description}
\item[{Parameters}] \leavevmode
\sphinxstyleliteralstrong{\sphinxupquote{products\_info}} \textendash{} (Config) 
The config corresponding to the product.

\item[{Returns}] \leavevmode
(bool)
True if there is a source, build and install
directory corresponding to the product described by product\_info.

\end{description}\end{quote}

\end{fulllineitems}

\index{suppress\_directories() (in module commands.clean)}

\begin{fulllineitems}
\phantomsection\label{\detokenize{apidoc_commands/commands:commands.clean.suppress_directories}}\pysiglinewithargsret{\sphinxcode{\sphinxupquote{commands.clean.}}\sphinxbfcode{\sphinxupquote{suppress\_directories}}}{\emph{l\_paths}, \emph{logger}}{}
Suppress the paths given in the list in l\_paths.
\begin{quote}\begin{description}
\item[{Parameters}] \leavevmode\begin{itemize}
\item {} 
\sphinxstyleliteralstrong{\sphinxupquote{l\_paths}} \textendash{} (list) The list of Path to be suppressed

\item {} 
\sphinxstyleliteralstrong{\sphinxupquote{logger}} \textendash{} (Logger) 
The logger instance to use for the display and logging

\end{itemize}

\end{description}\end{quote}

\end{fulllineitems}



\subsubsection{commands.compile module}
\label{\detokenize{apidoc_commands/commands:commands-compile-module}}\label{\detokenize{apidoc_commands/commands:module-commands.compile}}\index{commands.compile (module)}\index{Command (class in commands.compile)}

\begin{fulllineitems}
\phantomsection\label{\detokenize{apidoc_commands/commands:commands.compile.Command}}\pysiglinewithargsret{\sphinxbfcode{\sphinxupquote{class }}\sphinxcode{\sphinxupquote{commands.compile.}}\sphinxbfcode{\sphinxupquote{Command}}}{\emph{runner}}{}
Bases: \sphinxcode{\sphinxupquote{src.salomeTools.\_BaseCommand}}

The compile command constructs the products of the application

\begin{DUlineblock}{0em}
\item[] examples:
\item[] \textgreater{}\textgreater{} sat compile SALOME \textendash{}products KERNEL,GUI,MEDCOUPLING \textendash{}clean\_all
\end{DUlineblock}
\index{getParser() (commands.compile.Command method)}

\begin{fulllineitems}
\phantomsection\label{\detokenize{apidoc_commands/commands:commands.compile.Command.getParser}}\pysiglinewithargsret{\sphinxbfcode{\sphinxupquote{getParser}}}{}{}
Define all options for the command ‘sat compile \textless{}options\textgreater{}’

\end{fulllineitems}

\index{name (commands.compile.Command attribute)}

\begin{fulllineitems}
\phantomsection\label{\detokenize{apidoc_commands/commands:commands.compile.Command.name}}\pysigline{\sphinxbfcode{\sphinxupquote{name}}\sphinxbfcode{\sphinxupquote{ = 'compile'}}}
\end{fulllineitems}

\index{run() (commands.compile.Command method)}

\begin{fulllineitems}
\phantomsection\label{\detokenize{apidoc_commands/commands:commands.compile.Command.run}}\pysiglinewithargsret{\sphinxbfcode{\sphinxupquote{run}}}{\emph{cmd\_arguments}}{}
method called for command ‘sat compile \textless{}options\textgreater{}’

\end{fulllineitems}


\end{fulllineitems}

\index{add\_compile\_config\_file() (in module commands.compile)}

\begin{fulllineitems}
\phantomsection\label{\detokenize{apidoc_commands/commands:commands.compile.add_compile_config_file}}\pysiglinewithargsret{\sphinxcode{\sphinxupquote{commands.compile.}}\sphinxbfcode{\sphinxupquote{add\_compile\_config\_file}}}{\emph{p\_info}, \emph{config}}{}
Execute the proper configuration command(s) 
in the product build directory.
\begin{quote}\begin{description}
\item[{Parameters}] \leavevmode\begin{itemize}
\item {} 
\sphinxstyleliteralstrong{\sphinxupquote{p\_info}} \textendash{} (Config) The specific config of the product

\item {} 
\sphinxstyleliteralstrong{\sphinxupquote{config}} \textendash{} (Config) The global configuration

\end{itemize}

\end{description}\end{quote}

\end{fulllineitems}

\index{check\_dependencies() (in module commands.compile)}

\begin{fulllineitems}
\phantomsection\label{\detokenize{apidoc_commands/commands:commands.compile.check_dependencies}}\pysiglinewithargsret{\sphinxcode{\sphinxupquote{commands.compile.}}\sphinxbfcode{\sphinxupquote{check\_dependencies}}}{\emph{config}, \emph{p\_name\_p\_info}}{}
\end{fulllineitems}

\index{compile\_all\_products() (in module commands.compile)}

\begin{fulllineitems}
\phantomsection\label{\detokenize{apidoc_commands/commands:commands.compile.compile_all_products}}\pysiglinewithargsret{\sphinxcode{\sphinxupquote{commands.compile.}}\sphinxbfcode{\sphinxupquote{compile\_all\_products}}}{\emph{sat}, \emph{config}, \emph{options}, \emph{products\_infos}, \emph{logger}}{}
Execute the proper configuration commands 
in each product build directory.
\begin{quote}\begin{description}
\item[{Parameters}] \leavevmode\begin{itemize}
\item {} 
\sphinxstyleliteralstrong{\sphinxupquote{config}} \textendash{} (Config) The global configuration

\item {} 
\sphinxstyleliteralstrong{\sphinxupquote{products\_info}} \textendash{} (list)
List of (str, Config) =\textgreater{} (product\_name, product\_info)

\item {} 
\sphinxstyleliteralstrong{\sphinxupquote{logger}} \textendash{} (Logger) 
The logger instance to use for the display and logging

\end{itemize}

\item[{Returns}] \leavevmode
(int) the number of failing commands.

\end{description}\end{quote}

\end{fulllineitems}

\index{compile\_product() (in module commands.compile)}

\begin{fulllineitems}
\phantomsection\label{\detokenize{apidoc_commands/commands:commands.compile.compile_product}}\pysiglinewithargsret{\sphinxcode{\sphinxupquote{commands.compile.}}\sphinxbfcode{\sphinxupquote{compile\_product}}}{\emph{sat}, \emph{p\_name\_info}, \emph{config}, \emph{options}, \emph{logger}, \emph{header}, \emph{len\_end}}{}
Execute the proper configuration command(s) 
in the product build directory.
\begin{quote}\begin{description}
\item[{Parameters}] \leavevmode\begin{itemize}
\item {} 
\sphinxstyleliteralstrong{\sphinxupquote{p\_name\_info}} \textendash{} (tuple) (str, Config) =\textgreater{} (product\_name, product\_info)

\item {} 
\sphinxstyleliteralstrong{\sphinxupquote{config}} \textendash{} (Config) The global configuration

\item {} 
\sphinxstyleliteralstrong{\sphinxupquote{logger}} \textendash{} (Logger) 
The logger instance to use for the display and logging

\item {} 
\sphinxstyleliteralstrong{\sphinxupquote{header}} \textendash{} (str) the header to display when logging

\item {} 
\sphinxstyleliteralstrong{\sphinxupquote{len\_end}} \textendash{} (int) the lenght of the the end of line (used in display)

\end{itemize}

\item[{Returns}] \leavevmode
(int) 1 if it fails, else 0.

\end{description}\end{quote}

\end{fulllineitems}

\index{compile\_product\_cmake\_autotools() (in module commands.compile)}

\begin{fulllineitems}
\phantomsection\label{\detokenize{apidoc_commands/commands:commands.compile.compile_product_cmake_autotools}}\pysiglinewithargsret{\sphinxcode{\sphinxupquote{commands.compile.}}\sphinxbfcode{\sphinxupquote{compile\_product\_cmake\_autotools}}}{\emph{sat}, \emph{p\_name\_info}, \emph{config}, \emph{options}, \emph{logger}, \emph{header}, \emph{len\_end}}{}
Execute the proper build procedure for autotools or cmake
in the product build directory.
\begin{quote}\begin{description}
\item[{Parameters}] \leavevmode\begin{itemize}
\item {} 
\sphinxstyleliteralstrong{\sphinxupquote{p\_name\_info}} \textendash{} (tuple) 
(str, Config) =\textgreater{} (product\_name, product\_info)

\item {} 
\sphinxstyleliteralstrong{\sphinxupquote{config}} \textendash{} (Config) The global configuration

\item {} 
\sphinxstyleliteralstrong{\sphinxupquote{logger}} \textendash{} (Logger) 
The logger instance to use for the display and logging

\item {} 
\sphinxstyleliteralstrong{\sphinxupquote{header}} \textendash{} (str) the header to display when logging

\item {} 
\sphinxstyleliteralstrong{\sphinxupquote{len\_end}} \textendash{} (int) the length of the the end of line (used in display)

\end{itemize}

\item[{Returns}] \leavevmode
(int) 1 if it fails, else 0.

\end{description}\end{quote}

\end{fulllineitems}

\index{compile\_product\_script() (in module commands.compile)}

\begin{fulllineitems}
\phantomsection\label{\detokenize{apidoc_commands/commands:commands.compile.compile_product_script}}\pysiglinewithargsret{\sphinxcode{\sphinxupquote{commands.compile.}}\sphinxbfcode{\sphinxupquote{compile\_product\_script}}}{\emph{sat}, \emph{p\_name\_info}, \emph{config}, \emph{options}, \emph{logger}, \emph{header}, \emph{len\_end}}{}
Execute the script build procedure in the product build directory.
\begin{quote}\begin{description}
\item[{Parameters}] \leavevmode\begin{itemize}
\item {} 
\sphinxstyleliteralstrong{\sphinxupquote{p\_name\_info}} \textendash{} (tuple) 
(str, Config) =\textgreater{} (product\_name, product\_info)

\item {} 
\sphinxstyleliteralstrong{\sphinxupquote{config}} \textendash{} (Config) The global configuration

\item {} 
\sphinxstyleliteralstrong{\sphinxupquote{logger}} \textendash{} (Logger) 
The logger instance to use for the display and logging

\item {} 
\sphinxstyleliteralstrong{\sphinxupquote{header}} \textendash{} (str) the header to display when logging

\item {} 
\sphinxstyleliteralstrong{\sphinxupquote{len\_end}} \textendash{} (int) the lenght of the the end of line (used in display)

\end{itemize}

\item[{Returns}] \leavevmode
(int) 1 if it fails, else 0.

\end{description}\end{quote}

\end{fulllineitems}

\index{extend\_with\_children() (in module commands.compile)}

\begin{fulllineitems}
\phantomsection\label{\detokenize{apidoc_commands/commands:commands.compile.extend_with_children}}\pysiglinewithargsret{\sphinxcode{\sphinxupquote{commands.compile.}}\sphinxbfcode{\sphinxupquote{extend\_with\_children}}}{\emph{config}, \emph{p\_infos}}{}
\end{fulllineitems}

\index{extend\_with\_fathers() (in module commands.compile)}

\begin{fulllineitems}
\phantomsection\label{\detokenize{apidoc_commands/commands:commands.compile.extend_with_fathers}}\pysiglinewithargsret{\sphinxcode{\sphinxupquote{commands.compile.}}\sphinxbfcode{\sphinxupquote{extend\_with\_fathers}}}{\emph{config}, \emph{p\_infos}}{}
\end{fulllineitems}

\index{get\_children() (in module commands.compile)}

\begin{fulllineitems}
\phantomsection\label{\detokenize{apidoc_commands/commands:commands.compile.get_children}}\pysiglinewithargsret{\sphinxcode{\sphinxupquote{commands.compile.}}\sphinxbfcode{\sphinxupquote{get\_children}}}{\emph{config}, \emph{p\_name\_p\_info}}{}
\end{fulllineitems}

\index{get\_products\_list() (in module commands.compile)}

\begin{fulllineitems}
\phantomsection\label{\detokenize{apidoc_commands/commands:commands.compile.get_products_list}}\pysiglinewithargsret{\sphinxcode{\sphinxupquote{commands.compile.}}\sphinxbfcode{\sphinxupquote{get\_products\_list}}}{\emph{options}, \emph{cfg}, \emph{logger}}{}
method that gives the product list with their informations from 
configuration regarding the passed options.
\begin{quote}\begin{description}
\item[{Parameters}] \leavevmode\begin{itemize}
\item {} 
\sphinxstyleliteralstrong{\sphinxupquote{options}} \textendash{} (Options) 
The Options instance that stores the commands arguments

\item {} 
\sphinxstyleliteralstrong{\sphinxupquote{cfg}} \textendash{} (Config) The global configuration

\item {} 
\sphinxstyleliteralstrong{\sphinxupquote{logger}} \textendash{} (Logger) 
The logger instance to use for the display and logging

\end{itemize}

\item[{Returns}] \leavevmode
(list) The list of (product name, product\_informations).

\end{description}\end{quote}

\end{fulllineitems}

\index{get\_recursive\_children() (in module commands.compile)}

\begin{fulllineitems}
\phantomsection\label{\detokenize{apidoc_commands/commands:commands.compile.get_recursive_children}}\pysiglinewithargsret{\sphinxcode{\sphinxupquote{commands.compile.}}\sphinxbfcode{\sphinxupquote{get\_recursive\_children}}}{\emph{config}, \emph{p\_name\_p\_info}, \emph{without\_native\_fixed=False}}{}
Get the recursive list of the product that depend on 
the product defined by prod\_info
\begin{quote}\begin{description}
\item[{Parameters}] \leavevmode\begin{itemize}
\item {} 
\sphinxstyleliteralstrong{\sphinxupquote{config}} \textendash{} (Config) The global configuration

\item {} 
\sphinxstyleliteralstrong{\sphinxupquote{prod\_info}} \textendash{} (Config) The specific config of the product

\item {} 
\sphinxstyleliteralstrong{\sphinxupquote{without\_native\_fixed}} \textendash{} (bool) 
If true, do not include the fixed or native products in the result

\end{itemize}

\item[{Returns}] \leavevmode
(list) The list of product\_informations.

\end{description}\end{quote}

\end{fulllineitems}

\index{get\_recursive\_fathers() (in module commands.compile)}

\begin{fulllineitems}
\phantomsection\label{\detokenize{apidoc_commands/commands:commands.compile.get_recursive_fathers}}\pysiglinewithargsret{\sphinxcode{\sphinxupquote{commands.compile.}}\sphinxbfcode{\sphinxupquote{get\_recursive\_fathers}}}{\emph{config}, \emph{p\_name\_p\_info}, \emph{without\_native\_fixed=False}}{}
Get the recursive list of the dependencies of the product defined 
by prod\_info
\begin{quote}\begin{description}
\item[{Parameters}] \leavevmode\begin{itemize}
\item {} 
\sphinxstyleliteralstrong{\sphinxupquote{config}} \textendash{} (Config) The global configuration

\item {} 
\sphinxstyleliteralstrong{\sphinxupquote{prod\_info}} \textendash{} (Config) The specific config of the product

\item {} 
\sphinxstyleliteralstrong{\sphinxupquote{without\_native\_fixed}} \textendash{} (bool) 
If true, do not include the fixed or native products in the result

\end{itemize}

\item[{Returns}] \leavevmode
(list) The list of product\_informations.

\end{description}\end{quote}

\end{fulllineitems}

\index{sort\_products() (in module commands.compile)}

\begin{fulllineitems}
\phantomsection\label{\detokenize{apidoc_commands/commands:commands.compile.sort_products}}\pysiglinewithargsret{\sphinxcode{\sphinxupquote{commands.compile.}}\sphinxbfcode{\sphinxupquote{sort\_products}}}{\emph{config}, \emph{p\_infos}}{}
Sort the p\_infos regarding the dependencies between the products
\begin{quote}\begin{description}
\item[{Parameters}] \leavevmode\begin{itemize}
\item {} 
\sphinxstyleliteralstrong{\sphinxupquote{config}} \textendash{} (Config) The global configuration

\item {} 
\sphinxstyleliteralstrong{\sphinxupquote{p\_infos}} \textendash{} (list) 
List of (str, Config) =\textgreater{} (product\_name, product\_info)

\end{itemize}

\end{description}\end{quote}

\end{fulllineitems}



\subsubsection{commands.config module}
\label{\detokenize{apidoc_commands/commands:commands-config-module}}\label{\detokenize{apidoc_commands/commands:module-commands.config}}\index{commands.config (module)}\index{Command (class in commands.config)}

\begin{fulllineitems}
\phantomsection\label{\detokenize{apidoc_commands/commands:commands.config.Command}}\pysiglinewithargsret{\sphinxbfcode{\sphinxupquote{class }}\sphinxcode{\sphinxupquote{commands.config.}}\sphinxbfcode{\sphinxupquote{Command}}}{\emph{runner}}{}
Bases: \sphinxcode{\sphinxupquote{src.salomeTools.\_BaseCommand}}

The config command allows manipulation and operation on config ‘.pyconf’ files.

\begin{DUlineblock}{0em}
\item[] examples:
\item[] \textgreater{}\textgreater{} sat config \textendash{}list
\item[] \textgreater{}\textgreater{} sat config SALOME \textendash{}edit
\item[] \textgreater{}\textgreater{} sat config SALOME \textendash{}copy SALOME-new
\item[] \textgreater{}\textgreater{} sat config SALOME \textendash{}value VARS
\item[] \textgreater{}\textgreater{} sat config SALOME \textendash{}debug VARS
\item[] \textgreater{}\textgreater{} sat config SALOME \textendash{}info ParaView
\item[] \textgreater{}\textgreater{} sat config SALOME \textendash{}show\_patchs
\end{DUlineblock}
\index{getParser() (commands.config.Command method)}

\begin{fulllineitems}
\phantomsection\label{\detokenize{apidoc_commands/commands:commands.config.Command.getParser}}\pysiglinewithargsret{\sphinxbfcode{\sphinxupquote{getParser}}}{}{}
Define all options for command ‘sat config \textless{}options\textgreater{}’

\end{fulllineitems}

\index{name (commands.config.Command attribute)}

\begin{fulllineitems}
\phantomsection\label{\detokenize{apidoc_commands/commands:commands.config.Command.name}}\pysigline{\sphinxbfcode{\sphinxupquote{name}}\sphinxbfcode{\sphinxupquote{ = 'config'}}}
\end{fulllineitems}

\index{run() (commands.config.Command method)}

\begin{fulllineitems}
\phantomsection\label{\detokenize{apidoc_commands/commands:commands.config.Command.run}}\pysiglinewithargsret{\sphinxbfcode{\sphinxupquote{run}}}{\emph{cmd\_arguments}}{}
method called for command ‘sat config \textless{}options\textgreater{}’

\end{fulllineitems}


\end{fulllineitems}



\subsubsection{commands.configure module}
\label{\detokenize{apidoc_commands/commands:module-commands.configure}}\label{\detokenize{apidoc_commands/commands:commands-configure-module}}\index{commands.configure (module)}\index{Command (class in commands.configure)}

\begin{fulllineitems}
\phantomsection\label{\detokenize{apidoc_commands/commands:commands.configure.Command}}\pysiglinewithargsret{\sphinxbfcode{\sphinxupquote{class }}\sphinxcode{\sphinxupquote{commands.configure.}}\sphinxbfcode{\sphinxupquote{Command}}}{\emph{runner}}{}
Bases: \sphinxcode{\sphinxupquote{src.salomeTools.\_BaseCommand}}

The configure command executes in the build directory commands
corresponding to the compilation mode of the application products.
The possible compilation modes are ‘cmake’, ‘autotools’, or ‘script’.

\begin{DUlineblock}{0em}
\item[] Here are the commands to be run:
\item[]
\begin{DUlineblock}{\DUlineblockindent}
\item[] autotools: \textgreater{}\textgreater{} build\_configure and configure
\item[] cmake:     \textgreater{}\textgreater{} cmake
\item[] script:    (do nothing)
\item[] 
\end{DUlineblock}
\item[] examples:
\item[] \textgreater{}\textgreater{} sat configure SALOME \textendash{}products KERNEL,GUI,PARAVIS
\end{DUlineblock}
\index{getParser() (commands.configure.Command method)}

\begin{fulllineitems}
\phantomsection\label{\detokenize{apidoc_commands/commands:commands.configure.Command.getParser}}\pysiglinewithargsret{\sphinxbfcode{\sphinxupquote{getParser}}}{}{}
Define all options for command ‘sat configure \textless{}options\textgreater{}’

\end{fulllineitems}

\index{name (commands.configure.Command attribute)}

\begin{fulllineitems}
\phantomsection\label{\detokenize{apidoc_commands/commands:commands.configure.Command.name}}\pysigline{\sphinxbfcode{\sphinxupquote{name}}\sphinxbfcode{\sphinxupquote{ = 'configure'}}}
\end{fulllineitems}

\index{run() (commands.configure.Command method)}

\begin{fulllineitems}
\phantomsection\label{\detokenize{apidoc_commands/commands:commands.configure.Command.run}}\pysiglinewithargsret{\sphinxbfcode{\sphinxupquote{run}}}{\emph{cmd\_arguments}}{}
method called for command ‘sat configure \textless{}options\textgreater{}’

\end{fulllineitems}


\end{fulllineitems}

\index{configure\_all\_products() (in module commands.configure)}

\begin{fulllineitems}
\phantomsection\label{\detokenize{apidoc_commands/commands:commands.configure.configure_all_products}}\pysiglinewithargsret{\sphinxcode{\sphinxupquote{commands.configure.}}\sphinxbfcode{\sphinxupquote{configure\_all\_products}}}{\emph{config}, \emph{products\_infos}, \emph{conf\_option}, \emph{logger}}{}
Execute the proper configuration commands 
in each product build directory.
\begin{quote}\begin{description}
\item[{Parameters}] \leavevmode\begin{itemize}
\item {} 
\sphinxstyleliteralstrong{\sphinxupquote{config}} \textendash{} (Config) The global configuration

\item {} 
\sphinxstyleliteralstrong{\sphinxupquote{products\_info}} \textendash{} (list) 
List of (str, Config) =\textgreater{} (product\_name, product\_info)

\item {} 
\sphinxstyleliteralstrong{\sphinxupquote{conf\_option}} \textendash{} (str) The options to add to the command

\item {} 
\sphinxstyleliteralstrong{\sphinxupquote{logger}} \textendash{} (Logger) The logger instance to use for the display and logging

\end{itemize}

\item[{Returns}] \leavevmode
(int) the number of failing commands.

\end{description}\end{quote}

\end{fulllineitems}

\index{configure\_product() (in module commands.configure)}

\begin{fulllineitems}
\phantomsection\label{\detokenize{apidoc_commands/commands:commands.configure.configure_product}}\pysiglinewithargsret{\sphinxcode{\sphinxupquote{commands.configure.}}\sphinxbfcode{\sphinxupquote{configure\_product}}}{\emph{p\_name\_info}, \emph{conf\_option}, \emph{config}, \emph{logger}}{}
Execute the proper configuration command(s) 
in the product build directory.
\begin{quote}\begin{description}
\item[{Parameters}] \leavevmode\begin{itemize}
\item {} 
\sphinxstyleliteralstrong{\sphinxupquote{p\_name\_info}} \textendash{} (tuple) 
(str, Config) =\textgreater{} (product\_name, product\_info)

\item {} 
\sphinxstyleliteralstrong{\sphinxupquote{conf\_option}} \textendash{} (str) The options to add to the command

\item {} 
\sphinxstyleliteralstrong{\sphinxupquote{config}} \textendash{} (Config) The global configuration

\item {} 
\sphinxstyleliteralstrong{\sphinxupquote{logger}} \textendash{} (Logger) 
The logger instance to use for the display and logging

\end{itemize}

\item[{Returns}] \leavevmode
(int) 1 if it fails, else 0.

\end{description}\end{quote}

\end{fulllineitems}

\index{get\_products\_list() (in module commands.configure)}

\begin{fulllineitems}
\phantomsection\label{\detokenize{apidoc_commands/commands:commands.configure.get_products_list}}\pysiglinewithargsret{\sphinxcode{\sphinxupquote{commands.configure.}}\sphinxbfcode{\sphinxupquote{get\_products\_list}}}{\emph{options}, \emph{cfg}, \emph{logger}}{}
method that gives the product list with their informations from 
configuration regarding the passed options.
\begin{quote}\begin{description}
\item[{Parameters}] \leavevmode\begin{itemize}
\item {} 
\sphinxstyleliteralstrong{\sphinxupquote{options}} \textendash{} (Options) 
The Options instance that stores the commands arguments

\item {} 
\sphinxstyleliteralstrong{\sphinxupquote{cfg}} \textendash{} (Config) The global configuration

\item {} 
\sphinxstyleliteralstrong{\sphinxupquote{logger}} \textendash{} (Logger) 
The logger instance to use for the display and logging

\end{itemize}

\item[{Returns}] \leavevmode
(list) The list of (product name, product\_informations).

\end{description}\end{quote}

\end{fulllineitems}



\subsubsection{commands.environ module}
\label{\detokenize{apidoc_commands/commands:commands-environ-module}}\label{\detokenize{apidoc_commands/commands:module-commands.environ}}\index{commands.environ (module)}\index{Command (class in commands.environ)}

\begin{fulllineitems}
\phantomsection\label{\detokenize{apidoc_commands/commands:commands.environ.Command}}\pysiglinewithargsret{\sphinxbfcode{\sphinxupquote{class }}\sphinxcode{\sphinxupquote{commands.environ.}}\sphinxbfcode{\sphinxupquote{Command}}}{\emph{runner}}{}
Bases: \sphinxcode{\sphinxupquote{src.salomeTools.\_BaseCommand}}

The environ command generates the environment files of your application.

\begin{DUlineblock}{0em}
\item[] examples:
\item[] \textgreater{}\textgreater{} sat environ SALOME
\end{DUlineblock}
\index{getParser() (commands.environ.Command method)}

\begin{fulllineitems}
\phantomsection\label{\detokenize{apidoc_commands/commands:commands.environ.Command.getParser}}\pysiglinewithargsret{\sphinxbfcode{\sphinxupquote{getParser}}}{}{}
Define all options for command ‘sat environ \textless{}options\textgreater{}’

\end{fulllineitems}

\index{name (commands.environ.Command attribute)}

\begin{fulllineitems}
\phantomsection\label{\detokenize{apidoc_commands/commands:commands.environ.Command.name}}\pysigline{\sphinxbfcode{\sphinxupquote{name}}\sphinxbfcode{\sphinxupquote{ = 'environ'}}}
\end{fulllineitems}

\index{run() (commands.environ.Command method)}

\begin{fulllineitems}
\phantomsection\label{\detokenize{apidoc_commands/commands:commands.environ.Command.run}}\pysiglinewithargsret{\sphinxbfcode{\sphinxupquote{run}}}{\emph{cmd\_arguments}}{}
method called for command ‘sat environ \textless{}options\textgreater{}’

\end{fulllineitems}


\end{fulllineitems}

\index{write\_all\_source\_files() (in module commands.environ)}

\begin{fulllineitems}
\phantomsection\label{\detokenize{apidoc_commands/commands:commands.environ.write_all_source_files}}\pysiglinewithargsret{\sphinxcode{\sphinxupquote{commands.environ.}}\sphinxbfcode{\sphinxupquote{write\_all\_source\_files}}}{\emph{config, logger, out\_dir=None, src\_root=None, silent=False, shells={[}'bash'{]}, prefix='env', env\_info=None}}{}
Generates the environment files.
\begin{quote}\begin{description}
\item[{Parameters}] \leavevmode\begin{itemize}
\item {} 
\sphinxstyleliteralstrong{\sphinxupquote{config}} \textendash{} (Config) The global configuration

\item {} 
\sphinxstyleliteralstrong{\sphinxupquote{logger}} \textendash{} (Logger)
The logger instance to use for the display and logging

\item {} 
\sphinxstyleliteralstrong{\sphinxupquote{out\_dir}} \textendash{} (str) 
The path to the directory where the files will be put

\item {} 
\sphinxstyleliteralstrong{\sphinxupquote{src\_root}} \textendash{} (str) 
The path to the directory where the sources are

\item {} 
\sphinxstyleliteralstrong{\sphinxupquote{silent}} \textendash{} (bool) 
If True, do not print anything in the terminal

\item {} 
\sphinxstyleliteralstrong{\sphinxupquote{shells}} \textendash{} (list) The list of shells to generate

\item {} 
\sphinxstyleliteralstrong{\sphinxupquote{prefix}} \textendash{} (str) The prefix to add to the file names.

\item {} 
\sphinxstyleliteralstrong{\sphinxupquote{env\_info}} \textendash{} (str) The list of products to add in the files.

\end{itemize}

\item[{Returns}] \leavevmode
(list) The list of the generated files.

\end{description}\end{quote}

\end{fulllineitems}



\subsubsection{commands.find\_duplicates module}
\label{\detokenize{apidoc_commands/commands:module-commands.find_duplicates}}\label{\detokenize{apidoc_commands/commands:commands-find-duplicates-module}}\index{commands.find\_duplicates (module)}\index{Command (class in commands.find\_duplicates)}

\begin{fulllineitems}
\phantomsection\label{\detokenize{apidoc_commands/commands:commands.find_duplicates.Command}}\pysiglinewithargsret{\sphinxbfcode{\sphinxupquote{class }}\sphinxcode{\sphinxupquote{commands.find\_duplicates.}}\sphinxbfcode{\sphinxupquote{Command}}}{\emph{runner}}{}
Bases: \sphinxcode{\sphinxupquote{src.salomeTools.\_BaseCommand}}

The find\_duplicates command search recursively for all duplicates files
in INSTALL directory (or the optionally given directory) and 
prints the found files to the terminal.

\begin{DUlineblock}{0em}
\item[] examples:
\item[] \textgreater{}\textgreater{} sat find\_duplicates \textendash{}path /tmp
\end{DUlineblock}
\index{getParser() (commands.find\_duplicates.Command method)}

\begin{fulllineitems}
\phantomsection\label{\detokenize{apidoc_commands/commands:commands.find_duplicates.Command.getParser}}\pysiglinewithargsret{\sphinxbfcode{\sphinxupquote{getParser}}}{}{}
Define all options for command ‘sat find\_duplicates \textless{}options\textgreater{}’

\end{fulllineitems}

\index{name (commands.find\_duplicates.Command attribute)}

\begin{fulllineitems}
\phantomsection\label{\detokenize{apidoc_commands/commands:commands.find_duplicates.Command.name}}\pysigline{\sphinxbfcode{\sphinxupquote{name}}\sphinxbfcode{\sphinxupquote{ = 'find\_duplicates'}}}
\end{fulllineitems}

\index{run() (commands.find\_duplicates.Command method)}

\begin{fulllineitems}
\phantomsection\label{\detokenize{apidoc_commands/commands:commands.find_duplicates.Command.run}}\pysiglinewithargsret{\sphinxbfcode{\sphinxupquote{run}}}{\emph{cmd\_arguments}}{}
method called for command ‘sat find\_duplicates \textless{}options\textgreater{}’

\end{fulllineitems}


\end{fulllineitems}

\index{Progress\_bar (class in commands.find\_duplicates)}

\begin{fulllineitems}
\phantomsection\label{\detokenize{apidoc_commands/commands:commands.find_duplicates.Progress_bar}}\pysiglinewithargsret{\sphinxbfcode{\sphinxupquote{class }}\sphinxcode{\sphinxupquote{commands.find\_duplicates.}}\sphinxbfcode{\sphinxupquote{Progress\_bar}}}{\emph{name}, \emph{valMin}, \emph{valMax}, \emph{logger}, \emph{length=50}}{}
Create a progress bar in the terminal
\index{display\_value\_progression() (commands.find\_duplicates.Progress\_bar method)}

\begin{fulllineitems}
\phantomsection\label{\detokenize{apidoc_commands/commands:commands.find_duplicates.Progress_bar.display_value_progression}}\pysiglinewithargsret{\sphinxbfcode{\sphinxupquote{display\_value\_progression}}}{\emph{val}}{}
Display the progress bar.
\begin{quote}\begin{description}
\item[{Parameters}] \leavevmode
\sphinxstyleliteralstrong{\sphinxupquote{val}} \textendash{} (float) val must be between valMin and valMax.

\end{description}\end{quote}

\end{fulllineitems}


\end{fulllineitems}

\index{format\_list\_of\_str() (in module commands.find\_duplicates)}

\begin{fulllineitems}
\phantomsection\label{\detokenize{apidoc_commands/commands:commands.find_duplicates.format_list_of_str}}\pysiglinewithargsret{\sphinxcode{\sphinxupquote{commands.find\_duplicates.}}\sphinxbfcode{\sphinxupquote{format\_list\_of\_str}}}{\emph{l\_str}}{}
Make a list from a string
\begin{quote}\begin{description}
\item[{Parameters}] \leavevmode
\sphinxstyleliteralstrong{\sphinxupquote{l\_str}} \textendash{} (list or str) The variable to format

\item[{Returns}] \leavevmode
(list) the formatted variable

\end{description}\end{quote}

\end{fulllineitems}

\index{list\_directory() (in module commands.find\_duplicates)}

\begin{fulllineitems}
\phantomsection\label{\detokenize{apidoc_commands/commands:commands.find_duplicates.list_directory}}\pysiglinewithargsret{\sphinxcode{\sphinxupquote{commands.find\_duplicates.}}\sphinxbfcode{\sphinxupquote{list\_directory}}}{\emph{lpath}, \emph{extension\_ignored}, \emph{files\_ignored}, \emph{directories\_ignored}}{}
Make the list of all files and paths that are not filtered
\begin{quote}\begin{description}
\item[{Parameters}] \leavevmode\begin{itemize}
\item {} 
\sphinxstyleliteralstrong{\sphinxupquote{lpath}} \textendash{} (list) 
The list of path to of the directories where to  search for duplicates

\item {} 
\sphinxstyleliteralstrong{\sphinxupquote{extension\_ignored}} \textendash{} (list) The list of extensions to ignore

\item {} 
\sphinxstyleliteralstrong{\sphinxupquote{files\_ignored}} \textendash{} (list) The list of files to ignore

\item {} 
\sphinxstyleliteralstrong{\sphinxupquote{directories\_ignored}} \textendash{} (list) 
The list of directory paths to ignore

\end{itemize}

\item[{Returns}] \leavevmode
(list, list) 
files\_arb\_out is the list of {[}file, path{]} 
and files\_out is is the list of files

\end{description}\end{quote}

\end{fulllineitems}



\subsubsection{commands.generate module}
\label{\detokenize{apidoc_commands/commands:module-commands.generate}}\label{\detokenize{apidoc_commands/commands:commands-generate-module}}\index{commands.generate (module)}\index{Command (class in commands.generate)}

\begin{fulllineitems}
\phantomsection\label{\detokenize{apidoc_commands/commands:commands.generate.Command}}\pysiglinewithargsret{\sphinxbfcode{\sphinxupquote{class }}\sphinxcode{\sphinxupquote{commands.generate.}}\sphinxbfcode{\sphinxupquote{Command}}}{\emph{runner}}{}
Bases: \sphinxcode{\sphinxupquote{src.salomeTools.\_BaseCommand}}

The generate command generates SALOME modules from ‘pure cpp’ products.

\begin{DUlineblock}{0em}
\item[] warning: this command NEEDS YACSGEN to run.
\item[] 
\item[] examples:
\item[] \textgreater{}\textgreater{} sat generate SALOME \textendash{}products FLICACPP
\end{DUlineblock}
\index{getParser() (commands.generate.Command method)}

\begin{fulllineitems}
\phantomsection\label{\detokenize{apidoc_commands/commands:commands.generate.Command.getParser}}\pysiglinewithargsret{\sphinxbfcode{\sphinxupquote{getParser}}}{}{}
Define all options for command ‘sat generate \textless{}options\textgreater{}’

\end{fulllineitems}

\index{name (commands.generate.Command attribute)}

\begin{fulllineitems}
\phantomsection\label{\detokenize{apidoc_commands/commands:commands.generate.Command.name}}\pysigline{\sphinxbfcode{\sphinxupquote{name}}\sphinxbfcode{\sphinxupquote{ = 'generate'}}}
\end{fulllineitems}

\index{run() (commands.generate.Command method)}

\begin{fulllineitems}
\phantomsection\label{\detokenize{apidoc_commands/commands:commands.generate.Command.run}}\pysiglinewithargsret{\sphinxbfcode{\sphinxupquote{run}}}{\emph{cmd\_arguments}}{}
method called for command ‘sat generate \textless{}options\textgreater{}’

\end{fulllineitems}


\end{fulllineitems}

\index{build\_context() (in module commands.generate)}

\begin{fulllineitems}
\phantomsection\label{\detokenize{apidoc_commands/commands:commands.generate.build_context}}\pysiglinewithargsret{\sphinxcode{\sphinxupquote{commands.generate.}}\sphinxbfcode{\sphinxupquote{build\_context}}}{\emph{config}, \emph{logger}}{}
\end{fulllineitems}

\index{check\_module\_generator() (in module commands.generate)}

\begin{fulllineitems}
\phantomsection\label{\detokenize{apidoc_commands/commands:commands.generate.check_module_generator}}\pysiglinewithargsret{\sphinxcode{\sphinxupquote{commands.generate.}}\sphinxbfcode{\sphinxupquote{check\_module\_generator}}}{\emph{directory=None}}{}
Check if module\_generator is available.
\begin{quote}\begin{description}
\item[{Parameters}] \leavevmode
\sphinxstyleliteralstrong{\sphinxupquote{directory}} \textendash{} (str) The directory of YACSGEN.

\item[{Returns}] \leavevmode
(str) 
The YACSGEN path if the module\_generator is available, else None

\end{description}\end{quote}

\end{fulllineitems}

\index{check\_yacsgen() (in module commands.generate)}

\begin{fulllineitems}
\phantomsection\label{\detokenize{apidoc_commands/commands:commands.generate.check_yacsgen}}\pysiglinewithargsret{\sphinxcode{\sphinxupquote{commands.generate.}}\sphinxbfcode{\sphinxupquote{check\_yacsgen}}}{\emph{config}, \emph{directory}, \emph{logger}}{}
Check if YACSGEN is available.
\begin{quote}\begin{description}
\item[{Parameters}] \leavevmode\begin{itemize}
\item {} 
\sphinxstyleliteralstrong{\sphinxupquote{config}} \textendash{} (Config) The global configuration.

\item {} 
\sphinxstyleliteralstrong{\sphinxupquote{directory}} \textendash{} (str) The directory given by option \textendash{}yacsgen

\item {} 
\sphinxstyleliteralstrong{\sphinxupquote{logger}} \textendash{} (Logger) The logger instance

\end{itemize}

\item[{Returns}] \leavevmode
(RCO.ReturnCode) 
with value The path to yacsgen directory if ok

\end{description}\end{quote}

\end{fulllineitems}

\index{generate\_component() (in module commands.generate)}

\begin{fulllineitems}
\phantomsection\label{\detokenize{apidoc_commands/commands:commands.generate.generate_component}}\pysiglinewithargsret{\sphinxcode{\sphinxupquote{commands.generate.}}\sphinxbfcode{\sphinxupquote{generate\_component}}}{\emph{config}, \emph{compo}, \emph{product\_info}, \emph{context}, \emph{header}, \emph{logger}}{}
\end{fulllineitems}

\index{generate\_component\_list() (in module commands.generate)}

\begin{fulllineitems}
\phantomsection\label{\detokenize{apidoc_commands/commands:commands.generate.generate_component_list}}\pysiglinewithargsret{\sphinxcode{\sphinxupquote{commands.generate.}}\sphinxbfcode{\sphinxupquote{generate\_component\_list}}}{\emph{config}, \emph{product\_info}, \emph{context}, \emph{logger}}{}
\end{fulllineitems}



\subsubsection{commands.init module}
\label{\detokenize{apidoc_commands/commands:module-commands.init}}\label{\detokenize{apidoc_commands/commands:commands-init-module}}\index{commands.init (module)}\index{Command (class in commands.init)}

\begin{fulllineitems}
\phantomsection\label{\detokenize{apidoc_commands/commands:commands.init.Command}}\pysiglinewithargsret{\sphinxbfcode{\sphinxupquote{class }}\sphinxcode{\sphinxupquote{commands.init.}}\sphinxbfcode{\sphinxupquote{Command}}}{\emph{runner}}{}
Bases: \sphinxcode{\sphinxupquote{src.salomeTools.\_BaseCommand}}

The init command Changes the local settings of SAT
\index{getParser() (commands.init.Command method)}

\begin{fulllineitems}
\phantomsection\label{\detokenize{apidoc_commands/commands:commands.init.Command.getParser}}\pysiglinewithargsret{\sphinxbfcode{\sphinxupquote{getParser}}}{}{}
Define all options for command ‘sat init \textless{}options\textgreater{}’

\end{fulllineitems}

\index{name (commands.init.Command attribute)}

\begin{fulllineitems}
\phantomsection\label{\detokenize{apidoc_commands/commands:commands.init.Command.name}}\pysigline{\sphinxbfcode{\sphinxupquote{name}}\sphinxbfcode{\sphinxupquote{ = 'init'}}}
\end{fulllineitems}

\index{run() (commands.init.Command method)}

\begin{fulllineitems}
\phantomsection\label{\detokenize{apidoc_commands/commands:commands.init.Command.run}}\pysiglinewithargsret{\sphinxbfcode{\sphinxupquote{run}}}{\emph{cmd\_arguments}}{}
method called for command ‘sat init \textless{}options\textgreater{}’

\end{fulllineitems}


\end{fulllineitems}

\index{check\_path() (in module commands.init)}

\begin{fulllineitems}
\phantomsection\label{\detokenize{apidoc_commands/commands:commands.init.check_path}}\pysiglinewithargsret{\sphinxcode{\sphinxupquote{commands.init.}}\sphinxbfcode{\sphinxupquote{check\_path}}}{\emph{path\_to\_check}, \emph{logger}}{}
Verify that the given path is not a file and can be created.
\begin{quote}\begin{description}
\item[{Parameters}] \leavevmode\begin{itemize}
\item {} 
\sphinxstyleliteralstrong{\sphinxupquote{path\_to\_check}} \textendash{} (str) The path to check.

\item {} 
\sphinxstyleliteralstrong{\sphinxupquote{logger}} \textendash{} (Logger) The logger instance.

\end{itemize}

\end{description}\end{quote}

\end{fulllineitems}

\index{display\_local\_values() (in module commands.init)}

\begin{fulllineitems}
\phantomsection\label{\detokenize{apidoc_commands/commands:commands.init.display_local_values}}\pysiglinewithargsret{\sphinxcode{\sphinxupquote{commands.init.}}\sphinxbfcode{\sphinxupquote{display\_local\_values}}}{\emph{config}, \emph{logger}}{}
Display the base path
\begin{quote}\begin{description}
\item[{Parameters}] \leavevmode\begin{itemize}
\item {} 
\sphinxstyleliteralstrong{\sphinxupquote{config}} \textendash{} (Config) The global configuration.

\item {} 
\sphinxstyleliteralstrong{\sphinxupquote{key}} \textendash{} (str) The key from which to change the value.

\item {} 
\sphinxstyleliteralstrong{\sphinxupquote{logger}} \textendash{} (Logger) The logger instance.

\end{itemize}

\end{description}\end{quote}

\end{fulllineitems}

\index{set\_local\_value() (in module commands.init)}

\begin{fulllineitems}
\phantomsection\label{\detokenize{apidoc_commands/commands:commands.init.set_local_value}}\pysiglinewithargsret{\sphinxcode{\sphinxupquote{commands.init.}}\sphinxbfcode{\sphinxupquote{set\_local\_value}}}{\emph{config}, \emph{key}, \emph{value}, \emph{logger}}{}
Edit the site.pyconf file and change a value.
\begin{quote}\begin{description}
\item[{Parameters}] \leavevmode\begin{itemize}
\item {} 
\sphinxstyleliteralstrong{\sphinxupquote{config}} \textendash{} (Config) The global configuration.

\item {} 
\sphinxstyleliteralstrong{\sphinxupquote{key}} \textendash{} (str) The key from which to change the value.

\item {} 
\sphinxstyleliteralstrong{\sphinxupquote{value}} \textendash{} (str) The path to change.

\item {} 
\sphinxstyleliteralstrong{\sphinxupquote{logger}} \textendash{} (Logger) The logger instance.

\end{itemize}

\item[{Returns}] \leavevmode
(int) 0 if all is OK, else 1

\end{description}\end{quote}

\end{fulllineitems}



\subsubsection{commands.job module}
\label{\detokenize{apidoc_commands/commands:commands-job-module}}\label{\detokenize{apidoc_commands/commands:module-commands.job}}\index{commands.job (module)}\index{Command (class in commands.job)}

\begin{fulllineitems}
\phantomsection\label{\detokenize{apidoc_commands/commands:commands.job.Command}}\pysiglinewithargsret{\sphinxbfcode{\sphinxupquote{class }}\sphinxcode{\sphinxupquote{commands.job.}}\sphinxbfcode{\sphinxupquote{Command}}}{\emph{runner}}{}
Bases: \sphinxcode{\sphinxupquote{src.salomeTools.\_BaseCommand}}

The job command executes the commands of the job defined
in the jobs configuration file  
\textbar{} examples:
\textbar{} \textgreater{}\textgreater{} sat job \textendash{}jobs\_config my\_jobs \textendash{}name my\_job”
\index{getParser() (commands.job.Command method)}

\begin{fulllineitems}
\phantomsection\label{\detokenize{apidoc_commands/commands:commands.job.Command.getParser}}\pysiglinewithargsret{\sphinxbfcode{\sphinxupquote{getParser}}}{}{}
Define all options for command ‘sat job \textless{}options\textgreater{}’

\end{fulllineitems}

\index{name (commands.job.Command attribute)}

\begin{fulllineitems}
\phantomsection\label{\detokenize{apidoc_commands/commands:commands.job.Command.name}}\pysigline{\sphinxbfcode{\sphinxupquote{name}}\sphinxbfcode{\sphinxupquote{ = 'job'}}}
\end{fulllineitems}

\index{run() (commands.job.Command method)}

\begin{fulllineitems}
\phantomsection\label{\detokenize{apidoc_commands/commands:commands.job.Command.run}}\pysiglinewithargsret{\sphinxbfcode{\sphinxupquote{run}}}{\emph{cmd\_arguments}}{}
method called for command ‘sat job \textless{}options\textgreater{}’

\end{fulllineitems}


\end{fulllineitems}



\subsubsection{commands.jobs module}
\label{\detokenize{apidoc_commands/commands:commands-jobs-module}}\label{\detokenize{apidoc_commands/commands:module-commands.jobs}}\index{commands.jobs (module)}\index{Command (class in commands.jobs)}

\begin{fulllineitems}
\phantomsection\label{\detokenize{apidoc_commands/commands:commands.jobs.Command}}\pysiglinewithargsret{\sphinxbfcode{\sphinxupquote{class }}\sphinxcode{\sphinxupquote{commands.jobs.}}\sphinxbfcode{\sphinxupquote{Command}}}{\emph{runner}}{}
Bases: \sphinxcode{\sphinxupquote{src.salomeTools.\_BaseCommand}}

The jobs command command launches maintenances that are described in 
the dedicated jobs configuration file.

\begin{DUlineblock}{0em}
\item[] examples:
\item[] \textgreater{}\textgreater{} sat jobs \textendash{}name my\_jobs \textendash{}publish
\end{DUlineblock}
\index{getParser() (commands.jobs.Command method)}

\begin{fulllineitems}
\phantomsection\label{\detokenize{apidoc_commands/commands:commands.jobs.Command.getParser}}\pysiglinewithargsret{\sphinxbfcode{\sphinxupquote{getParser}}}{}{}
Define all options for command ‘sat jobs \textless{}options\textgreater{}’

\end{fulllineitems}

\index{name (commands.jobs.Command attribute)}

\begin{fulllineitems}
\phantomsection\label{\detokenize{apidoc_commands/commands:commands.jobs.Command.name}}\pysigline{\sphinxbfcode{\sphinxupquote{name}}\sphinxbfcode{\sphinxupquote{ = 'jobs'}}}
\end{fulllineitems}

\index{run() (commands.jobs.Command method)}

\begin{fulllineitems}
\phantomsection\label{\detokenize{apidoc_commands/commands:commands.jobs.Command.run}}\pysiglinewithargsret{\sphinxbfcode{\sphinxupquote{run}}}{\emph{cmd\_arguments}}{}
method called for command ‘sat jobs \textless{}options\textgreater{}’

\end{fulllineitems}


\end{fulllineitems}

\index{Gui (class in commands.jobs)}

\begin{fulllineitems}
\phantomsection\label{\detokenize{apidoc_commands/commands:commands.jobs.Gui}}\pysiglinewithargsret{\sphinxbfcode{\sphinxupquote{class }}\sphinxcode{\sphinxupquote{commands.jobs.}}\sphinxbfcode{\sphinxupquote{Gui}}}{\emph{xml\_dir\_path}, \emph{l\_jobs}, \emph{l\_jobs\_not\_today}, \emph{prefix}, \emph{logger}, \emph{file\_boards=''}}{}
Bases: \sphinxcode{\sphinxupquote{object}}

Class to manage the the xml data that can be displayed in a browser 
to see the jobs states
\index{add\_xml\_board() (commands.jobs.Gui method)}

\begin{fulllineitems}
\phantomsection\label{\detokenize{apidoc_commands/commands:commands.jobs.Gui.add_xml_board}}\pysiglinewithargsret{\sphinxbfcode{\sphinxupquote{add\_xml\_board}}}{\emph{name}}{}
Add a board to the board list
\begin{quote}\begin{description}
\item[{Parameters}] \leavevmode
\sphinxstyleliteralstrong{\sphinxupquote{name}} \textendash{} (str) the board name

\end{description}\end{quote}

\end{fulllineitems}

\index{find\_history() (commands.jobs.Gui method)}

\begin{fulllineitems}
\phantomsection\label{\detokenize{apidoc_commands/commands:commands.jobs.Gui.find_history}}\pysiglinewithargsret{\sphinxbfcode{\sphinxupquote{find\_history}}}{\emph{l\_jobs}, \emph{l\_jobs\_not\_today}}{}
find, for each job, in the existent xml boards the results for the job.
Store the results in the dictionary 
self.history = \{name\_job : list of (date, status, list links)\}
\begin{quote}\begin{description}
\item[{Parameters}] \leavevmode\begin{itemize}
\item {} 
\sphinxstyleliteralstrong{\sphinxupquote{l\_jobs}} \textendash{} (list) 
the list of jobs to run today

\item {} 
\sphinxstyleliteralstrong{\sphinxupquote{l\_jobs\_not\_today}} \textendash{} (list) 
the list of jobs that do not run today

\end{itemize}

\end{description}\end{quote}

\end{fulllineitems}

\index{find\_test\_log() (commands.jobs.Gui method)}

\begin{fulllineitems}
\phantomsection\label{\detokenize{apidoc_commands/commands:commands.jobs.Gui.find_test_log}}\pysiglinewithargsret{\sphinxbfcode{\sphinxupquote{find\_test\_log}}}{\emph{l\_remote\_log\_files}}{}
Find if there is a test log (board) in the remote log files and 
the path to it. There can be several test command, 
so the result is a list.
\begin{quote}\begin{description}
\item[{Parameters}] \leavevmode
\sphinxstyleliteralstrong{\sphinxupquote{l\_remote\_log\_files}} \textendash{} (list) the list of all remote log files

\item[{Returns}] \leavevmode
(list) 
the list of tuples (test log files path, res of the command)

\end{description}\end{quote}

\end{fulllineitems}

\index{initialize\_boards() (commands.jobs.Gui method)}

\begin{fulllineitems}
\phantomsection\label{\detokenize{apidoc_commands/commands:commands.jobs.Gui.initialize_boards}}\pysiglinewithargsret{\sphinxbfcode{\sphinxupquote{initialize\_boards}}}{\emph{l\_jobs}, \emph{l\_jobs\_not\_today}}{}
Get all the first information needed for each file and write the 
first version of the files
\begin{quote}\begin{description}
\item[{Parameters}] \leavevmode\begin{itemize}
\item {} 
\sphinxstyleliteralstrong{\sphinxupquote{l\_jobs}} \textendash{} (list) the list of jobs that run today

\item {} 
\sphinxstyleliteralstrong{\sphinxupquote{l\_jobs\_not\_today}} \textendash{} (list) the list of jobs that do not run today

\end{itemize}

\end{description}\end{quote}

\end{fulllineitems}

\index{last\_update() (commands.jobs.Gui method)}

\begin{fulllineitems}
\phantomsection\label{\detokenize{apidoc_commands/commands:commands.jobs.Gui.last_update}}\pysiglinewithargsret{\sphinxbfcode{\sphinxupquote{last\_update}}}{\emph{finish\_status='finished'}}{}
update information about the jobs for the file xml\_file
\begin{quote}\begin{description}
\item[{Parameters}] \leavevmode\begin{itemize}
\item {} 
\sphinxstyleliteralstrong{\sphinxupquote{l\_jobs}} \textendash{} (list) the list of jobs that run today

\item {} 
\sphinxstyleliteralstrong{\sphinxupquote{xml\_file}} \textendash{} (xmlManager.XmlLogFile) the xml instance to update

\end{itemize}

\end{description}\end{quote}

\end{fulllineitems}

\index{parse\_csv\_boards() (commands.jobs.Gui method)}

\begin{fulllineitems}
\phantomsection\label{\detokenize{apidoc_commands/commands:commands.jobs.Gui.parse_csv_boards}}\pysiglinewithargsret{\sphinxbfcode{\sphinxupquote{parse\_csv\_boards}}}{\emph{today}}{}
Parse the csv file that describes the boards to produce and fill 
the dict d\_input\_boards that contain the csv file contain
\begin{quote}\begin{description}
\item[{Parameters}] \leavevmode
\sphinxstyleliteralstrong{\sphinxupquote{today}} \textendash{} (int) the current day of the week

\end{description}\end{quote}

\end{fulllineitems}

\index{put\_jobs\_not\_today() (commands.jobs.Gui method)}

\begin{fulllineitems}
\phantomsection\label{\detokenize{apidoc_commands/commands:commands.jobs.Gui.put_jobs_not_today}}\pysiglinewithargsret{\sphinxbfcode{\sphinxupquote{put\_jobs\_not\_today}}}{\emph{l\_jobs\_not\_today}, \emph{xml\_node\_jobs}}{}
Get all the first information needed for each file and write the 
first version of the files
\begin{quote}\begin{description}
\item[{Parameters}] \leavevmode\begin{itemize}
\item {} 
\sphinxstyleliteralstrong{\sphinxupquote{xml\_node\_jobs}} \textendash{} (etree.Element) 
the node corresponding to a job

\item {} 
\sphinxstyleliteralstrong{\sphinxupquote{l\_jobs\_not\_today}} \textendash{} (list) 
the list of jobs that do not run today

\end{itemize}

\end{description}\end{quote}

\end{fulllineitems}

\index{update\_xml\_file() (commands.jobs.Gui method)}

\begin{fulllineitems}
\phantomsection\label{\detokenize{apidoc_commands/commands:commands.jobs.Gui.update_xml_file}}\pysiglinewithargsret{\sphinxbfcode{\sphinxupquote{update\_xml\_file}}}{\emph{l\_jobs}, \emph{xml\_file}}{}
update information about the jobs for the file xml\_file
\begin{quote}\begin{description}
\item[{Parameters}] \leavevmode\begin{itemize}
\item {} 
\sphinxstyleliteralstrong{\sphinxupquote{l\_jobs}} \textendash{} (list) the list of jobs that run today

\item {} 
\sphinxstyleliteralstrong{\sphinxupquote{xml\_file}} \textendash{} (xmlManager.XmlLogFile) 
the xml instance to update

\end{itemize}

\end{description}\end{quote}

\end{fulllineitems}

\index{update\_xml\_files() (commands.jobs.Gui method)}

\begin{fulllineitems}
\phantomsection\label{\detokenize{apidoc_commands/commands:commands.jobs.Gui.update_xml_files}}\pysiglinewithargsret{\sphinxbfcode{\sphinxupquote{update\_xml\_files}}}{\emph{l\_jobs}}{}
Write all the xml files with updated information about the jobs
\begin{quote}\begin{description}
\item[{Parameters}] \leavevmode
\sphinxstyleliteralstrong{\sphinxupquote{l\_jobs}} \textendash{} (list) the list of jobs that run today

\end{description}\end{quote}

\end{fulllineitems}

\index{write\_xml\_file() (commands.jobs.Gui method)}

\begin{fulllineitems}
\phantomsection\label{\detokenize{apidoc_commands/commands:commands.jobs.Gui.write_xml_file}}\pysiglinewithargsret{\sphinxbfcode{\sphinxupquote{write\_xml\_file}}}{\emph{xml\_file}, \emph{stylesheet}}{}
Write one xml file and the same file with prefix

\end{fulllineitems}

\index{write\_xml\_files() (commands.jobs.Gui method)}

\begin{fulllineitems}
\phantomsection\label{\detokenize{apidoc_commands/commands:commands.jobs.Gui.write_xml_files}}\pysiglinewithargsret{\sphinxbfcode{\sphinxupquote{write\_xml\_files}}}{}{}
Write the xml files

\end{fulllineitems}


\end{fulllineitems}

\index{Job (class in commands.jobs)}

\begin{fulllineitems}
\phantomsection\label{\detokenize{apidoc_commands/commands:commands.jobs.Job}}\pysiglinewithargsret{\sphinxbfcode{\sphinxupquote{class }}\sphinxcode{\sphinxupquote{commands.jobs.}}\sphinxbfcode{\sphinxupquote{Job}}}{\emph{name}, \emph{machine}, \emph{application}, \emph{board}, \emph{commands}, \emph{timeout}, \emph{config}, \emph{job\_file\_path}, \emph{logger}, \emph{after=None}, \emph{prefix=None}}{}
Bases: \sphinxcode{\sphinxupquote{object}}

Class to manage one job
\index{cancel() (commands.jobs.Job method)}

\begin{fulllineitems}
\phantomsection\label{\detokenize{apidoc_commands/commands:commands.jobs.Job.cancel}}\pysiglinewithargsret{\sphinxbfcode{\sphinxupquote{cancel}}}{}{}
In case of a failing job, one has to cancel every job that depend on it.
This method put the job as failed and will not be executed.

\end{fulllineitems}

\index{check\_time() (commands.jobs.Job method)}

\begin{fulllineitems}
\phantomsection\label{\detokenize{apidoc_commands/commands:commands.jobs.Job.check_time}}\pysiglinewithargsret{\sphinxbfcode{\sphinxupquote{check\_time}}}{}{}
Verify that the job has not exceeded its timeout.
If it has, kill the remote command and consider the job as finished.

\end{fulllineitems}

\index{get\_log\_files() (commands.jobs.Job method)}

\begin{fulllineitems}
\phantomsection\label{\detokenize{apidoc_commands/commands:commands.jobs.Job.get_log_files}}\pysiglinewithargsret{\sphinxbfcode{\sphinxupquote{get\_log\_files}}}{}{}
Get the log files produced by the command launched 
on the remote machine, and put it in the log directory of the user,
so they can be accessible from

\end{fulllineitems}

\index{get\_pids() (commands.jobs.Job method)}

\begin{fulllineitems}
\phantomsection\label{\detokenize{apidoc_commands/commands:commands.jobs.Job.get_pids}}\pysiglinewithargsret{\sphinxbfcode{\sphinxupquote{get\_pids}}}{}{}
Get the pid(s) corresponding to the command that have been launched
On the remote machine
\begin{quote}\begin{description}
\item[{Returns}] \leavevmode
(list) The list of integers corresponding to the found pids

\end{description}\end{quote}

\end{fulllineitems}

\index{get\_status() (commands.jobs.Job method)}

\begin{fulllineitems}
\phantomsection\label{\detokenize{apidoc_commands/commands:commands.jobs.Job.get_status}}\pysiglinewithargsret{\sphinxbfcode{\sphinxupquote{get\_status}}}{}{}
Get the status of the job (used by the Gui for xml display)
\begin{quote}\begin{description}
\item[{Returns}] \leavevmode
(str) The current status of the job

\end{description}\end{quote}

\end{fulllineitems}

\index{has\_begun() (commands.jobs.Job method)}

\begin{fulllineitems}
\phantomsection\label{\detokenize{apidoc_commands/commands:commands.jobs.Job.has_begun}}\pysiglinewithargsret{\sphinxbfcode{\sphinxupquote{has\_begun}}}{}{}
Returns True if the job has already begun
\begin{quote}\begin{description}
\item[{Returns}] \leavevmode
(bool) True if the job has already begun

\end{description}\end{quote}

\end{fulllineitems}

\index{has\_failed() (commands.jobs.Job method)}

\begin{fulllineitems}
\phantomsection\label{\detokenize{apidoc_commands/commands:commands.jobs.Job.has_failed}}\pysiglinewithargsret{\sphinxbfcode{\sphinxupquote{has\_failed}}}{}{}
Returns True if the job has failed. 
A job is considered as failed if the machine could not be reached,
if the remote command failed, 
or if the job finished with a time out.
\begin{quote}\begin{description}
\item[{Returns}] \leavevmode
(bool) True if the job has failed

\end{description}\end{quote}

\end{fulllineitems}

\index{has\_finished() (commands.jobs.Job method)}

\begin{fulllineitems}
\phantomsection\label{\detokenize{apidoc_commands/commands:commands.jobs.Job.has_finished}}\pysiglinewithargsret{\sphinxbfcode{\sphinxupquote{has\_finished}}}{}{}
Returns True if the job has already finished 
(i.e. all the commands have been executed)
If it is finished, the outputs are stored in the fields out and err.
\begin{quote}\begin{description}
\item[{Returns}] \leavevmode
(bool) True if the job has already finished

\end{description}\end{quote}

\end{fulllineitems}

\index{is\_running() (commands.jobs.Job method)}

\begin{fulllineitems}
\phantomsection\label{\detokenize{apidoc_commands/commands:commands.jobs.Job.is_running}}\pysiglinewithargsret{\sphinxbfcode{\sphinxupquote{is\_running}}}{}{}
Returns True if the job commands are running
\begin{quote}\begin{description}
\item[{Returns}] \leavevmode
(bool) True if the job is running

\end{description}\end{quote}

\end{fulllineitems}

\index{is\_timeout() (commands.jobs.Job method)}

\begin{fulllineitems}
\phantomsection\label{\detokenize{apidoc_commands/commands:commands.jobs.Job.is_timeout}}\pysiglinewithargsret{\sphinxbfcode{\sphinxupquote{is\_timeout}}}{}{}
Returns True if the job commands has finished with timeout
\begin{quote}\begin{description}
\item[{Returns}] \leavevmode
(bool) True if the job has finished with timeout

\end{description}\end{quote}

\end{fulllineitems}

\index{kill\_remote\_process() (commands.jobs.Job method)}

\begin{fulllineitems}
\phantomsection\label{\detokenize{apidoc_commands/commands:commands.jobs.Job.kill_remote_process}}\pysiglinewithargsret{\sphinxbfcode{\sphinxupquote{kill\_remote\_process}}}{\emph{wait=1}}{}
Kills the process on the remote machine.
\begin{quote}\begin{description}
\item[{Returns}] \leavevmode
(str, str) the output of the kill, the error of the kill

\end{description}\end{quote}

\end{fulllineitems}

\index{run() (commands.jobs.Job method)}

\begin{fulllineitems}
\phantomsection\label{\detokenize{apidoc_commands/commands:commands.jobs.Job.run}}\pysiglinewithargsret{\sphinxbfcode{\sphinxupquote{run}}}{}{}
Launch the job by executing the remote command.

\end{fulllineitems}

\index{time\_elapsed() (commands.jobs.Job method)}

\begin{fulllineitems}
\phantomsection\label{\detokenize{apidoc_commands/commands:commands.jobs.Job.time_elapsed}}\pysiglinewithargsret{\sphinxbfcode{\sphinxupquote{time\_elapsed}}}{}{}
Get the time elapsed since the job launching
\begin{quote}\begin{description}
\item[{Returns}] \leavevmode
The number of seconds

\item[{Return type}] \leavevmode
int

\end{description}\end{quote}

\end{fulllineitems}

\index{total\_duration() (commands.jobs.Job method)}

\begin{fulllineitems}
\phantomsection\label{\detokenize{apidoc_commands/commands:commands.jobs.Job.total_duration}}\pysiglinewithargsret{\sphinxbfcode{\sphinxupquote{total\_duration}}}{}{}
Gives the total duration of the job
\begin{quote}\begin{description}
\item[{Returns}] \leavevmode
(int) the total duration of the job in seconds

\end{description}\end{quote}

\end{fulllineitems}

\index{write\_results() (commands.jobs.Job method)}

\begin{fulllineitems}
\phantomsection\label{\detokenize{apidoc_commands/commands:commands.jobs.Job.write_results}}\pysiglinewithargsret{\sphinxbfcode{\sphinxupquote{write\_results}}}{}{}
Display on the terminal all the job’s information

\end{fulllineitems}


\end{fulllineitems}

\index{Jobs (class in commands.jobs)}

\begin{fulllineitems}
\phantomsection\label{\detokenize{apidoc_commands/commands:commands.jobs.Jobs}}\pysiglinewithargsret{\sphinxbfcode{\sphinxupquote{class }}\sphinxcode{\sphinxupquote{commands.jobs.}}\sphinxbfcode{\sphinxupquote{Jobs}}}{\emph{runner}, \emph{logger}, \emph{job\_file\_path}, \emph{config\_jobs}, \emph{lenght\_columns=20}}{}
Bases: \sphinxcode{\sphinxupquote{object}}

Class to manage the jobs to be run
\index{cancel\_dependencies\_of\_failing\_jobs() (commands.jobs.Jobs method)}

\begin{fulllineitems}
\phantomsection\label{\detokenize{apidoc_commands/commands:commands.jobs.Jobs.cancel_dependencies_of_failing_jobs}}\pysiglinewithargsret{\sphinxbfcode{\sphinxupquote{cancel\_dependencies\_of\_failing\_jobs}}}{}{}
Cancels all the jobs that depend on a failing one.
\begin{quote}\begin{description}
\item[{Returns}] \leavevmode
None

\end{description}\end{quote}

\end{fulllineitems}

\index{define\_job() (commands.jobs.Jobs method)}

\begin{fulllineitems}
\phantomsection\label{\detokenize{apidoc_commands/commands:commands.jobs.Jobs.define_job}}\pysiglinewithargsret{\sphinxbfcode{\sphinxupquote{define\_job}}}{\emph{job\_def}, \emph{machine}}{}
Takes a pyconf job definition and a machine (from class machine)
and returns the job instance corresponding to the definition.
\begin{quote}\begin{description}
\item[{Parameters}] \leavevmode\begin{itemize}
\item {} 
\sphinxstyleliteralstrong{\sphinxupquote{job\_def}} \textendash{} (Mapping a job definition

\item {} 
\sphinxstyleliteralstrong{\sphinxupquote{machine}} \textendash{} (Machine) the machine on which the job will run

\end{itemize}

\item[{Returns}] \leavevmode
(Job) The corresponding job in a job class instance

\end{description}\end{quote}

\end{fulllineitems}

\index{determine\_jobs\_and\_machines() (commands.jobs.Jobs method)}

\begin{fulllineitems}
\phantomsection\label{\detokenize{apidoc_commands/commands:commands.jobs.Jobs.determine_jobs_and_machines}}\pysiglinewithargsret{\sphinxbfcode{\sphinxupquote{determine\_jobs\_and\_machines}}}{}{}
Reads the pyconf jobs definition and instantiates all
the machines and jobs to be done today.
\begin{quote}\begin{description}
\item[{Returns}] \leavevmode
None

\end{description}\end{quote}

\end{fulllineitems}

\index{display\_status() (commands.jobs.Jobs method)}

\begin{fulllineitems}
\phantomsection\label{\detokenize{apidoc_commands/commands:commands.jobs.Jobs.display_status}}\pysiglinewithargsret{\sphinxbfcode{\sphinxupquote{display\_status}}}{\emph{len\_col}}{}
Takes a lenght and construct the display of the current status 
of the jobs in an array that has a column for each host.
It displays the job that is currently running on the host of the column.
\begin{quote}\begin{description}
\item[{Parameters}] \leavevmode
\sphinxstyleliteralstrong{\sphinxupquote{len\_col}} \textendash{} (int) the size of the column

\item[{Returns}] \leavevmode
None

\end{description}\end{quote}

\end{fulllineitems}

\index{find\_job\_that\_has\_name() (commands.jobs.Jobs method)}

\begin{fulllineitems}
\phantomsection\label{\detokenize{apidoc_commands/commands:commands.jobs.Jobs.find_job_that_has_name}}\pysiglinewithargsret{\sphinxbfcode{\sphinxupquote{find\_job\_that\_has\_name}}}{\emph{name}}{}
Returns the job by its name.
\begin{quote}\begin{description}
\item[{Parameters}] \leavevmode
\sphinxstyleliteralstrong{\sphinxupquote{name}} \textendash{} (str) a job name

\item[{Returns}] \leavevmode
(Job) the job that has the name.

\end{description}\end{quote}

\end{fulllineitems}

\index{is\_occupied() (commands.jobs.Jobs method)}

\begin{fulllineitems}
\phantomsection\label{\detokenize{apidoc_commands/commands:commands.jobs.Jobs.is_occupied}}\pysiglinewithargsret{\sphinxbfcode{\sphinxupquote{is\_occupied}}}{\emph{hostname}}{}
Returns True if a job is running on 
the machine defined by its host and its port.
\begin{quote}\begin{description}
\item[{Parameters}] \leavevmode
\sphinxstyleliteralstrong{\sphinxupquote{hostname}} \textendash{} (str, int) the pair (host, port)

\item[{Returns}] \leavevmode
(Job or bool) 
the job that is running on the host, 
or false if there is no job running on the host.

\end{description}\end{quote}

\end{fulllineitems}

\index{run\_jobs() (commands.jobs.Jobs method)}

\begin{fulllineitems}
\phantomsection\label{\detokenize{apidoc_commands/commands:commands.jobs.Jobs.run_jobs}}\pysiglinewithargsret{\sphinxbfcode{\sphinxupquote{run\_jobs}}}{}{}
The main method. Runs all the jobs on every host. 
For each host, at a given time, only one job can be running.
The jobs that have the field after (that contain the job that has
to be run before it) are run after the previous job.
This method stops when all the jobs are finished.
\begin{quote}\begin{description}
\item[{Returns}] \leavevmode
None

\end{description}\end{quote}

\end{fulllineitems}

\index{ssh\_connection\_all\_machines() (commands.jobs.Jobs method)}

\begin{fulllineitems}
\phantomsection\label{\detokenize{apidoc_commands/commands:commands.jobs.Jobs.ssh_connection_all_machines}}\pysiglinewithargsret{\sphinxbfcode{\sphinxupquote{ssh\_connection\_all\_machines}}}{\emph{pad=50}}{}
Do the ssh connection to every machine to be used today.
\begin{quote}\begin{description}
\item[{Returns}] \leavevmode
None

\end{description}\end{quote}

\end{fulllineitems}

\index{str\_of\_length() (commands.jobs.Jobs method)}

\begin{fulllineitems}
\phantomsection\label{\detokenize{apidoc_commands/commands:commands.jobs.Jobs.str_of_length}}\pysiglinewithargsret{\sphinxbfcode{\sphinxupquote{str\_of\_length}}}{\emph{text}, \emph{length}}{}
Takes a string text of any length and returns 
the most close string of length “length”.
\begin{quote}\begin{description}
\item[{Parameters}] \leavevmode\begin{itemize}
\item {} 
\sphinxstyleliteralstrong{\sphinxupquote{text}} \textendash{} (str) any string

\item {} 
\sphinxstyleliteralstrong{\sphinxupquote{length}} \textendash{} (int) a length for the returned string

\end{itemize}

\item[{Returns}] \leavevmode
(str) the most close string of length “length”

\end{description}\end{quote}

\end{fulllineitems}

\index{update\_jobs\_states\_list() (commands.jobs.Jobs method)}

\begin{fulllineitems}
\phantomsection\label{\detokenize{apidoc_commands/commands:commands.jobs.Jobs.update_jobs_states_list}}\pysiglinewithargsret{\sphinxbfcode{\sphinxupquote{update\_jobs\_states\_list}}}{}{}
Updates the lists that store the currently
running jobs and the jobs that have already finished.
\begin{quote}\begin{description}
\item[{Returns}] \leavevmode
None

\end{description}\end{quote}

\end{fulllineitems}

\index{write\_all\_results() (commands.jobs.Jobs method)}

\begin{fulllineitems}
\phantomsection\label{\detokenize{apidoc_commands/commands:commands.jobs.Jobs.write_all_results}}\pysiglinewithargsret{\sphinxbfcode{\sphinxupquote{write\_all\_results}}}{}{}
Display all the jobs outputs.
\begin{quote}\begin{description}
\item[{Returns}] \leavevmode
None

\end{description}\end{quote}

\end{fulllineitems}


\end{fulllineitems}

\index{Machine (class in commands.jobs)}

\begin{fulllineitems}
\phantomsection\label{\detokenize{apidoc_commands/commands:commands.jobs.Machine}}\pysiglinewithargsret{\sphinxbfcode{\sphinxupquote{class }}\sphinxcode{\sphinxupquote{commands.jobs.}}\sphinxbfcode{\sphinxupquote{Machine}}}{\emph{name}, \emph{host}, \emph{user}, \emph{port=22}, \emph{passwd=None}, \emph{sat\_path='salomeTools'}}{}
Bases: \sphinxcode{\sphinxupquote{object}}

Manage a ssh connection on a machine
\index{close() (commands.jobs.Machine method)}

\begin{fulllineitems}
\phantomsection\label{\detokenize{apidoc_commands/commands:commands.jobs.Machine.close}}\pysiglinewithargsret{\sphinxbfcode{\sphinxupquote{close}}}{}{}
Close the ssh connection

\end{fulllineitems}

\index{connect() (commands.jobs.Machine method)}

\begin{fulllineitems}
\phantomsection\label{\detokenize{apidoc_commands/commands:commands.jobs.Machine.connect}}\pysiglinewithargsret{\sphinxbfcode{\sphinxupquote{connect}}}{\emph{logger}}{}
Initiate the ssh connection to the remote machine
\begin{quote}\begin{description}
\item[{Parameters}] \leavevmode
\sphinxstyleliteralstrong{\sphinxupquote{logger}} \textendash{} The logger instance

\item[{Returns}] \leavevmode
None

\end{description}\end{quote}

\end{fulllineitems}

\index{copy\_sat() (commands.jobs.Machine method)}

\begin{fulllineitems}
\phantomsection\label{\detokenize{apidoc_commands/commands:commands.jobs.Machine.copy_sat}}\pysiglinewithargsret{\sphinxbfcode{\sphinxupquote{copy\_sat}}}{\emph{sat\_local\_path}, \emph{job\_file}}{}
Copy salomeTools to the remote machine in self.sat\_path

\end{fulllineitems}

\index{exec\_command() (commands.jobs.Machine method)}

\begin{fulllineitems}
\phantomsection\label{\detokenize{apidoc_commands/commands:commands.jobs.Machine.exec_command}}\pysiglinewithargsret{\sphinxbfcode{\sphinxupquote{exec\_command}}}{\emph{command}, \emph{logger}}{}
Execute the command on the remote machine
\begin{quote}\begin{description}
\item[{Parameters}] \leavevmode\begin{itemize}
\item {} 
\sphinxstyleliteralstrong{\sphinxupquote{command}} \textendash{} (str) The command to be run

\item {} 
\sphinxstyleliteralstrong{\sphinxupquote{logger}} \textendash{} The logger instance

\end{itemize}

\item[{Returns}] \leavevmode
(paramiko.channel.ChannelFile, etc) 
the stdin, stdout, and stderr of the executing command,
as a 3-tuple

\end{description}\end{quote}

\end{fulllineitems}

\index{mkdir() (commands.jobs.Machine method)}

\begin{fulllineitems}
\phantomsection\label{\detokenize{apidoc_commands/commands:commands.jobs.Machine.mkdir}}\pysiglinewithargsret{\sphinxbfcode{\sphinxupquote{mkdir}}}{\emph{path}, \emph{mode=511}, \emph{ignore\_existing=False}}{}
As mkdir by adding an option to not fail if the folder exists

\end{fulllineitems}

\index{put\_dir() (commands.jobs.Machine method)}

\begin{fulllineitems}
\phantomsection\label{\detokenize{apidoc_commands/commands:commands.jobs.Machine.put_dir}}\pysiglinewithargsret{\sphinxbfcode{\sphinxupquote{put\_dir}}}{\emph{source}, \emph{target}, \emph{filters={[}{]}}}{}
Uploads the contents of the source directory to the target path.
The target directory needs to exists. 
All sub-directories in source are created under target.

\end{fulllineitems}

\index{successfully\_connected() (commands.jobs.Machine method)}

\begin{fulllineitems}
\phantomsection\label{\detokenize{apidoc_commands/commands:commands.jobs.Machine.successfully_connected}}\pysiglinewithargsret{\sphinxbfcode{\sphinxupquote{successfully\_connected}}}{\emph{logger}}{}
Verify if the connection to the remote machine has succeed
\begin{quote}\begin{description}
\item[{Parameters}] \leavevmode
\sphinxstyleliteralstrong{\sphinxupquote{logger}} \textendash{} The logger instance

\item[{Returns}] \leavevmode
(bool) True if the connection has succeed, False if not

\end{description}\end{quote}

\end{fulllineitems}

\index{write\_info() (commands.jobs.Machine method)}

\begin{fulllineitems}
\phantomsection\label{\detokenize{apidoc_commands/commands:commands.jobs.Machine.write_info}}\pysiglinewithargsret{\sphinxbfcode{\sphinxupquote{write\_info}}}{\emph{logger}}{}
Prints the informations relative to the machine in the logger 
(terminal traces and log file)
\begin{quote}\begin{description}
\item[{Parameters}] \leavevmode
\sphinxstyleliteralstrong{\sphinxupquote{logger}} \textendash{} The logger instance

\item[{Returns}] \leavevmode
None

\end{description}\end{quote}

\end{fulllineitems}


\end{fulllineitems}

\index{develop\_factorized\_jobs() (in module commands.jobs)}

\begin{fulllineitems}
\phantomsection\label{\detokenize{apidoc_commands/commands:commands.jobs.develop_factorized_jobs}}\pysiglinewithargsret{\sphinxcode{\sphinxupquote{commands.jobs.}}\sphinxbfcode{\sphinxupquote{develop\_factorized\_jobs}}}{\emph{config\_jobs}}{}
update information about the jobs for the file xml\_file
\begin{quote}\begin{description}
\item[{Parameters}] \leavevmode
\sphinxstyleliteralstrong{\sphinxupquote{config\_jobs}} \textendash{} (Config) 
the config corresponding to the jos description

\end{description}\end{quote}

\end{fulllineitems}

\index{getParamiko() (in module commands.jobs)}

\begin{fulllineitems}
\phantomsection\label{\detokenize{apidoc_commands/commands:commands.jobs.getParamiko}}\pysiglinewithargsret{\sphinxcode{\sphinxupquote{commands.jobs.}}\sphinxbfcode{\sphinxupquote{getParamiko}}}{\emph{logger=None}}{}
\end{fulllineitems}

\index{get\_config\_file\_path() (in module commands.jobs)}

\begin{fulllineitems}
\phantomsection\label{\detokenize{apidoc_commands/commands:commands.jobs.get_config_file_path}}\pysiglinewithargsret{\sphinxcode{\sphinxupquote{commands.jobs.}}\sphinxbfcode{\sphinxupquote{get\_config\_file\_path}}}{\emph{job\_config\_name}, \emph{l\_cfg\_dir}}{}
\end{fulllineitems}



\subsubsection{commands.launcher module}
\label{\detokenize{apidoc_commands/commands:commands-launcher-module}}\label{\detokenize{apidoc_commands/commands:module-commands.launcher}}\index{commands.launcher (module)}\index{Command (class in commands.launcher)}

\begin{fulllineitems}
\phantomsection\label{\detokenize{apidoc_commands/commands:commands.launcher.Command}}\pysiglinewithargsret{\sphinxbfcode{\sphinxupquote{class }}\sphinxcode{\sphinxupquote{commands.launcher.}}\sphinxbfcode{\sphinxupquote{Command}}}{\emph{runner}}{}
Bases: \sphinxcode{\sphinxupquote{src.salomeTools.\_BaseCommand}}

The launcher command generates a SALOME launcher.

\begin{DUlineblock}{0em}
\item[] examples:
\item[] \textgreater{}\textgreater{} sat launcher SALOME 
\end{DUlineblock}
\index{getParser() (commands.launcher.Command method)}

\begin{fulllineitems}
\phantomsection\label{\detokenize{apidoc_commands/commands:commands.launcher.Command.getParser}}\pysiglinewithargsret{\sphinxbfcode{\sphinxupquote{getParser}}}{}{}
Define all possible options for command ‘sat launcher \textless{}options\textgreater{}’

\end{fulllineitems}

\index{name (commands.launcher.Command attribute)}

\begin{fulllineitems}
\phantomsection\label{\detokenize{apidoc_commands/commands:commands.launcher.Command.name}}\pysigline{\sphinxbfcode{\sphinxupquote{name}}\sphinxbfcode{\sphinxupquote{ = 'launcher'}}}
\end{fulllineitems}

\index{run() (commands.launcher.Command method)}

\begin{fulllineitems}
\phantomsection\label{\detokenize{apidoc_commands/commands:commands.launcher.Command.run}}\pysiglinewithargsret{\sphinxbfcode{\sphinxupquote{run}}}{\emph{cmd\_arguments}}{}
method called for command ‘sat launcher \textless{}options\textgreater{}’

\end{fulllineitems}


\end{fulllineitems}

\index{copy\_catalog() (in module commands.launcher)}

\begin{fulllineitems}
\phantomsection\label{\detokenize{apidoc_commands/commands:commands.launcher.copy_catalog}}\pysiglinewithargsret{\sphinxcode{\sphinxupquote{commands.launcher.}}\sphinxbfcode{\sphinxupquote{copy\_catalog}}}{\emph{config}, \emph{catalog\_path}}{}
Copy the xml catalog file into the right location
\begin{quote}\begin{description}
\item[{Parameters}] \leavevmode\begin{itemize}
\item {} 
\sphinxstyleliteralstrong{\sphinxupquote{config}} \textendash{} (Config) The global configuration

\item {} 
\sphinxstyleliteralstrong{\sphinxupquote{catalog\_path}} \textendash{} (str) the catalog file path

\end{itemize}

\item[{Returns}] \leavevmode
(dict) 
The environment dictionary corresponding to the file path.

\end{description}\end{quote}

\end{fulllineitems}

\index{generate\_catalog() (in module commands.launcher)}

\begin{fulllineitems}
\phantomsection\label{\detokenize{apidoc_commands/commands:commands.launcher.generate_catalog}}\pysiglinewithargsret{\sphinxcode{\sphinxupquote{commands.launcher.}}\sphinxbfcode{\sphinxupquote{generate\_catalog}}}{\emph{machines}, \emph{config}, \emph{logger}}{}
Generates an xml catalog file from a list of machines.
\begin{quote}\begin{description}
\item[{Parameters}] \leavevmode\begin{itemize}
\item {} 
\sphinxstyleliteralstrong{\sphinxupquote{machines}} \textendash{} (list) The list of machines to add in the catalog

\item {} 
\sphinxstyleliteralstrong{\sphinxupquote{config}} \textendash{} (Config) The global configuration

\item {} 
\sphinxstyleliteralstrong{\sphinxupquote{logger}} \textendash{} (Logger) 
The logger instance to use for the display and logging

\end{itemize}

\item[{Returns}] \leavevmode
(str) The catalog file path.

\end{description}\end{quote}

\end{fulllineitems}

\index{generate\_launch\_file() (in module commands.launcher)}

\begin{fulllineitems}
\phantomsection\label{\detokenize{apidoc_commands/commands:commands.launcher.generate_launch_file}}\pysiglinewithargsret{\sphinxcode{\sphinxupquote{commands.launcher.}}\sphinxbfcode{\sphinxupquote{generate\_launch\_file}}}{\emph{config}, \emph{logger}, \emph{launcher\_name}, \emph{pathlauncher}, \emph{display=True}, \emph{additional\_env=\{\}}}{}
Generates the launcher file.
\begin{quote}\begin{description}
\item[{Parameters}] \leavevmode\begin{itemize}
\item {} 
\sphinxstyleliteralstrong{\sphinxupquote{config}} \textendash{} (Config) The global configuration

\item {} 
\sphinxstyleliteralstrong{\sphinxupquote{logger}} \textendash{} (Logger) 
The logger instance to use for the display and logging

\item {} 
\sphinxstyleliteralstrong{\sphinxupquote{launcher\_name}} \textendash{} (str) The name of the launcher to generate

\item {} 
\sphinxstyleliteralstrong{\sphinxupquote{pathlauncher}} \textendash{} (str) The path to the launcher to generate

\item {} 
\sphinxstyleliteralstrong{\sphinxupquote{display}} \textendash{} (bool) If False, do not print anything in the terminal

\item {} 
\sphinxstyleliteralstrong{\sphinxupquote{additional\_env}} \textendash{} (dict) 
The dict giving additional environment variables

\end{itemize}

\item[{Returns}] \leavevmode
(str) The launcher file path.

\end{description}\end{quote}

\end{fulllineitems}



\subsubsection{commands.log module}
\label{\detokenize{apidoc_commands/commands:commands-log-module}}\label{\detokenize{apidoc_commands/commands:module-commands.log}}\index{commands.log (module)}\index{Command (class in commands.log)}

\begin{fulllineitems}
\phantomsection\label{\detokenize{apidoc_commands/commands:commands.log.Command}}\pysiglinewithargsret{\sphinxbfcode{\sphinxupquote{class }}\sphinxcode{\sphinxupquote{commands.log.}}\sphinxbfcode{\sphinxupquote{Command}}}{\emph{runner}}{}
Bases: \sphinxcode{\sphinxupquote{src.salomeTools.\_BaseCommand}}

The log command gives access to the logs produced by the salomeTools commands.

\begin{DUlineblock}{0em}
\item[] examples:
\item[] \textgreater{}\textgreater{} sat log
\end{DUlineblock}
\index{getParser() (commands.log.Command method)}

\begin{fulllineitems}
\phantomsection\label{\detokenize{apidoc_commands/commands:commands.log.Command.getParser}}\pysiglinewithargsret{\sphinxbfcode{\sphinxupquote{getParser}}}{}{}
Define all options for command ‘sat log \textless{}options\textgreater{}’

\end{fulllineitems}

\index{name (commands.log.Command attribute)}

\begin{fulllineitems}
\phantomsection\label{\detokenize{apidoc_commands/commands:commands.log.Command.name}}\pysigline{\sphinxbfcode{\sphinxupquote{name}}\sphinxbfcode{\sphinxupquote{ = 'log'}}}
\end{fulllineitems}

\index{run() (commands.log.Command method)}

\begin{fulllineitems}
\phantomsection\label{\detokenize{apidoc_commands/commands:commands.log.Command.run}}\pysiglinewithargsret{\sphinxbfcode{\sphinxupquote{run}}}{\emph{cmd\_arguments}}{}
method called for command ‘sat log \textless{}options\textgreater{}’

\end{fulllineitems}


\end{fulllineitems}

\index{ask\_value() (in module commands.log)}

\begin{fulllineitems}
\phantomsection\label{\detokenize{apidoc_commands/commands:commands.log.ask_value}}\pysiglinewithargsret{\sphinxcode{\sphinxupquote{commands.log.}}\sphinxbfcode{\sphinxupquote{ask\_value}}}{\emph{nb}}{}
Ask for an int n. 0\textless{}n\textless{}nb
\begin{quote}\begin{description}
\item[{Parameters}] \leavevmode
\sphinxstyleliteralstrong{\sphinxupquote{nb}} \textendash{} (int) The maximum value of the value to be returned by the user.

\item[{Returns}] \leavevmode
(int) 
the value entered by the user. Return -1 if it is not as expected

\end{description}\end{quote}

\end{fulllineitems}

\index{getMaxFormat() (in module commands.log)}

\begin{fulllineitems}
\phantomsection\label{\detokenize{apidoc_commands/commands:commands.log.getMaxFormat}}\pysiglinewithargsret{\sphinxcode{\sphinxupquote{commands.log.}}\sphinxbfcode{\sphinxupquote{getMaxFormat}}}{\emph{aListOfStr}, \emph{offset=1}}{}
returns format for columns width as ‘\%-30s”’ for example

\end{fulllineitems}

\index{get\_last\_log\_file() (in module commands.log)}

\begin{fulllineitems}
\phantomsection\label{\detokenize{apidoc_commands/commands:commands.log.get_last_log_file}}\pysiglinewithargsret{\sphinxcode{\sphinxupquote{commands.log.}}\sphinxbfcode{\sphinxupquote{get\_last\_log\_file}}}{\emph{logDir}, \emph{notShownCommands}}{}
Used in case of last option. 
Get the last log command file path.
\begin{quote}\begin{description}
\item[{Parameters}] \leavevmode\begin{itemize}
\item {} 
\sphinxstyleliteralstrong{\sphinxupquote{logDir}} \textendash{} (str) The directory where to search the log files

\item {} 
\sphinxstyleliteralstrong{\sphinxupquote{notShownCommands}} \textendash{} (list) the list of commands to ignore

\end{itemize}

\item[{Returns}] \leavevmode
(str) the path to the last log file

\end{description}\end{quote}

\end{fulllineitems}

\index{print\_log\_command\_in\_terminal() (in module commands.log)}

\begin{fulllineitems}
\phantomsection\label{\detokenize{apidoc_commands/commands:commands.log.print_log_command_in_terminal}}\pysiglinewithargsret{\sphinxcode{\sphinxupquote{commands.log.}}\sphinxbfcode{\sphinxupquote{print\_log\_command\_in\_terminal}}}{\emph{filePath}, \emph{logger}}{}
Print the contain of filePath. It contains a command log in xml format.
\begin{quote}\begin{description}
\item[{Parameters}] \leavevmode\begin{itemize}
\item {} 
\sphinxstyleliteralstrong{\sphinxupquote{filePath}} \textendash{} The command xml file from which extract the commands context and traces

\item {} 
\sphinxstyleliteralstrong{\sphinxupquote{logger}} \textendash{} (Logger) the logging instance to use in order to print.

\end{itemize}

\end{description}\end{quote}

\end{fulllineitems}

\index{remove\_log\_file() (in module commands.log)}

\begin{fulllineitems}
\phantomsection\label{\detokenize{apidoc_commands/commands:commands.log.remove_log_file}}\pysiglinewithargsret{\sphinxcode{\sphinxupquote{commands.log.}}\sphinxbfcode{\sphinxupquote{remove\_log\_file}}}{\emph{filePath}, \emph{logger}}{}
if it exists, print a warning and remove the input file
\begin{quote}\begin{description}
\item[{Parameters}] \leavevmode\begin{itemize}
\item {} 
\sphinxstyleliteralstrong{\sphinxupquote{filePath}} \textendash{} the path of the file to delete

\item {} 
\sphinxstyleliteralstrong{\sphinxupquote{logger}} \textendash{} (Logger) the logger instance to use for the print

\end{itemize}

\end{description}\end{quote}

\end{fulllineitems}

\index{show\_last\_logs() (in module commands.log)}

\begin{fulllineitems}
\phantomsection\label{\detokenize{apidoc_commands/commands:commands.log.show_last_logs}}\pysiglinewithargsret{\sphinxcode{\sphinxupquote{commands.log.}}\sphinxbfcode{\sphinxupquote{show\_last\_logs}}}{\emph{logger}, \emph{config}, \emph{log\_dirs}}{}
Show last compilation logs

\end{fulllineitems}

\index{show\_product\_last\_logs() (in module commands.log)}

\begin{fulllineitems}
\phantomsection\label{\detokenize{apidoc_commands/commands:commands.log.show_product_last_logs}}\pysiglinewithargsret{\sphinxcode{\sphinxupquote{commands.log.}}\sphinxbfcode{\sphinxupquote{show\_product\_last\_logs}}}{\emph{logger}, \emph{config}, \emph{product\_log\_dir}}{}
Show last compilation logs of a product

\end{fulllineitems}



\subsubsection{commands.make module}
\label{\detokenize{apidoc_commands/commands:module-commands.make}}\label{\detokenize{apidoc_commands/commands:commands-make-module}}\index{commands.make (module)}\index{Command (class in commands.make)}

\begin{fulllineitems}
\phantomsection\label{\detokenize{apidoc_commands/commands:commands.make.Command}}\pysiglinewithargsret{\sphinxbfcode{\sphinxupquote{class }}\sphinxcode{\sphinxupquote{commands.make.}}\sphinxbfcode{\sphinxupquote{Command}}}{\emph{runner}}{}
Bases: \sphinxcode{\sphinxupquote{src.salomeTools.\_BaseCommand}}

The make command executes the ‘make’ command in the build directory.

\begin{DUlineblock}{0em}
\item[] examples:
\item[] \textgreater{}\textgreater{} sat make SALOME \textendash{}products Python,KERNEL,GUI
\end{DUlineblock}
\index{getParser() (commands.make.Command method)}

\begin{fulllineitems}
\phantomsection\label{\detokenize{apidoc_commands/commands:commands.make.Command.getParser}}\pysiglinewithargsret{\sphinxbfcode{\sphinxupquote{getParser}}}{}{}
Define all options for the command ‘sat make \textless{}options\textgreater{}’

\end{fulllineitems}

\index{name (commands.make.Command attribute)}

\begin{fulllineitems}
\phantomsection\label{\detokenize{apidoc_commands/commands:commands.make.Command.name}}\pysigline{\sphinxbfcode{\sphinxupquote{name}}\sphinxbfcode{\sphinxupquote{ = 'make'}}}
\end{fulllineitems}

\index{run() (commands.make.Command method)}

\begin{fulllineitems}
\phantomsection\label{\detokenize{apidoc_commands/commands:commands.make.Command.run}}\pysiglinewithargsret{\sphinxbfcode{\sphinxupquote{run}}}{\emph{cmd\_arguments}}{}
method called for command ‘sat make \textless{}options\textgreater{}’

\end{fulllineitems}


\end{fulllineitems}

\index{get\_nb\_proc() (in module commands.make)}

\begin{fulllineitems}
\phantomsection\label{\detokenize{apidoc_commands/commands:commands.make.get_nb_proc}}\pysiglinewithargsret{\sphinxcode{\sphinxupquote{commands.make.}}\sphinxbfcode{\sphinxupquote{get\_nb\_proc}}}{\emph{product\_info}, \emph{config}, \emph{make\_option}}{}
\end{fulllineitems}

\index{get\_products\_list() (in module commands.make)}

\begin{fulllineitems}
\phantomsection\label{\detokenize{apidoc_commands/commands:commands.make.get_products_list}}\pysiglinewithargsret{\sphinxcode{\sphinxupquote{commands.make.}}\sphinxbfcode{\sphinxupquote{get\_products\_list}}}{\emph{options}, \emph{cfg}, \emph{logger}}{}
method that gives the product list with their informations from 
configuration regarding the passed options.
\begin{quote}\begin{description}
\item[{Parameters}] \leavevmode\begin{itemize}
\item {} 
\sphinxstyleliteralstrong{\sphinxupquote{options}} \textendash{} (Options) 
The Options instance that stores the commands arguments

\item {} 
\sphinxstyleliteralstrong{\sphinxupquote{cfg}} \textendash{} (Config) The global configuration

\item {} 
\sphinxstyleliteralstrong{\sphinxupquote{logger}} \textendash{} (Logger) 
The logger instance to use for the display and logging

\end{itemize}

\item[{Returns}] \leavevmode
(list) The list of tuples (product name, product\_informations).

\end{description}\end{quote}

\end{fulllineitems}

\index{make\_all\_products() (in module commands.make)}

\begin{fulllineitems}
\phantomsection\label{\detokenize{apidoc_commands/commands:commands.make.make_all_products}}\pysiglinewithargsret{\sphinxcode{\sphinxupquote{commands.make.}}\sphinxbfcode{\sphinxupquote{make\_all\_products}}}{\emph{config}, \emph{products\_infos}, \emph{make\_option}, \emph{logger}}{}
Execute the proper configuration commands 
in each product build directory.
\begin{quote}\begin{description}
\item[{Parameters}] \leavevmode\begin{itemize}
\item {} 
\sphinxstyleliteralstrong{\sphinxupquote{config}} \textendash{} (Config) The global configuration

\item {} 
\sphinxstyleliteralstrong{\sphinxupquote{products\_info}} \textendash{} (list) 
List of (str, Config) =\textgreater{} (product\_name, product\_info)

\item {} 
\sphinxstyleliteralstrong{\sphinxupquote{make\_option}} \textendash{} (str) The options to add to the command

\item {} 
\sphinxstyleliteralstrong{\sphinxupquote{logger}} \textendash{} (Logger) 
The logger instance to use for the display and logging

\end{itemize}

\item[{Returns}] \leavevmode
(int) the number of failing commands.

\end{description}\end{quote}

\end{fulllineitems}

\index{make\_product() (in module commands.make)}

\begin{fulllineitems}
\phantomsection\label{\detokenize{apidoc_commands/commands:commands.make.make_product}}\pysiglinewithargsret{\sphinxcode{\sphinxupquote{commands.make.}}\sphinxbfcode{\sphinxupquote{make\_product}}}{\emph{p\_name\_info}, \emph{make\_option}, \emph{config}, \emph{logger}}{}
Execute the proper configuration command(s) 
in the product build directory.
\begin{quote}\begin{description}
\item[{Parameters}] \leavevmode\begin{itemize}
\item {} 
\sphinxstyleliteralstrong{\sphinxupquote{p\_name\_info}} \textendash{} (tuple) (str, Config) =\textgreater{} (product\_name, product\_info)

\item {} 
\sphinxstyleliteralstrong{\sphinxupquote{make\_option}} \textendash{} (str) The options to add to the command

\item {} 
\sphinxstyleliteralstrong{\sphinxupquote{config}} \textendash{} (Config) The global configuration

\item {} 
\sphinxstyleliteralstrong{\sphinxupquote{logger}} \textendash{} (Logger) 
The logger instance to use for the display and logging

\end{itemize}

\item[{Returns}] \leavevmode
(int) 1 if it fails, else 0.

\end{description}\end{quote}

\end{fulllineitems}



\subsubsection{commands.makeinstall module}
\label{\detokenize{apidoc_commands/commands:commands-makeinstall-module}}\label{\detokenize{apidoc_commands/commands:module-commands.makeinstall}}\index{commands.makeinstall (module)}\index{Command (class in commands.makeinstall)}

\begin{fulllineitems}
\phantomsection\label{\detokenize{apidoc_commands/commands:commands.makeinstall.Command}}\pysiglinewithargsret{\sphinxbfcode{\sphinxupquote{class }}\sphinxcode{\sphinxupquote{commands.makeinstall.}}\sphinxbfcode{\sphinxupquote{Command}}}{\emph{runner}}{}
Bases: \sphinxcode{\sphinxupquote{src.salomeTools.\_BaseCommand}}

The makeinstall command executes the ‘make install’ command in the build directory.
In case of product constructed using a script (build\_source : ‘script’), 
then the makeinstall command do nothing.

\begin{DUlineblock}{0em}
\item[] examples:
\item[] \textgreater{}\textgreater{} sat makeinstall SALOME \textendash{}products KERNEL,GUI
\end{DUlineblock}
\index{getParser() (commands.makeinstall.Command method)}

\begin{fulllineitems}
\phantomsection\label{\detokenize{apidoc_commands/commands:commands.makeinstall.Command.getParser}}\pysiglinewithargsret{\sphinxbfcode{\sphinxupquote{getParser}}}{}{}
Define all options for the command ‘sat makeinstall \textless{}options\textgreater{}’

\end{fulllineitems}

\index{name (commands.makeinstall.Command attribute)}

\begin{fulllineitems}
\phantomsection\label{\detokenize{apidoc_commands/commands:commands.makeinstall.Command.name}}\pysigline{\sphinxbfcode{\sphinxupquote{name}}\sphinxbfcode{\sphinxupquote{ = 'makeinstall'}}}
\end{fulllineitems}

\index{run() (commands.makeinstall.Command method)}

\begin{fulllineitems}
\phantomsection\label{\detokenize{apidoc_commands/commands:commands.makeinstall.Command.run}}\pysiglinewithargsret{\sphinxbfcode{\sphinxupquote{run}}}{\emph{cmd\_arguments}}{}
method called for command ‘sat makeinstall \textless{}options\textgreater{}’

\end{fulllineitems}


\end{fulllineitems}

\index{get\_products\_list() (in module commands.makeinstall)}

\begin{fulllineitems}
\phantomsection\label{\detokenize{apidoc_commands/commands:commands.makeinstall.get_products_list}}\pysiglinewithargsret{\sphinxcode{\sphinxupquote{commands.makeinstall.}}\sphinxbfcode{\sphinxupquote{get\_products\_list}}}{\emph{options}, \emph{cfg}, \emph{logger}}{}
method that gives the product list with their informations from 
configuration regarding the passed options.
\begin{quote}\begin{description}
\item[{Parameters}] \leavevmode\begin{itemize}
\item {} 
\sphinxstyleliteralstrong{\sphinxupquote{options}} \textendash{} (Options) 
The Options instance that stores the commands arguments

\item {} 
\sphinxstyleliteralstrong{\sphinxupquote{cfg}} \textendash{} (Config) The global configuration

\item {} 
\sphinxstyleliteralstrong{\sphinxupquote{logger}} \textendash{} (Logger) 
The logger instance to use for the display and logging

\end{itemize}

\item[{Returns}] \leavevmode
(list) The list of (product name, product\_informations).

\end{description}\end{quote}

\end{fulllineitems}

\index{makeinstall\_all\_products() (in module commands.makeinstall)}

\begin{fulllineitems}
\phantomsection\label{\detokenize{apidoc_commands/commands:commands.makeinstall.makeinstall_all_products}}\pysiglinewithargsret{\sphinxcode{\sphinxupquote{commands.makeinstall.}}\sphinxbfcode{\sphinxupquote{makeinstall\_all\_products}}}{\emph{config}, \emph{products\_infos}, \emph{logger}}{}
Execute the proper configuration commands 
in each product build directory.
\begin{quote}\begin{description}
\item[{Parameters}] \leavevmode\begin{itemize}
\item {} 
\sphinxstyleliteralstrong{\sphinxupquote{config}} \textendash{} (Config) The global configuration

\item {} 
\sphinxstyleliteralstrong{\sphinxupquote{products\_info}} \textendash{} (list) 
List of (str, Config) =\textgreater{} (product\_name, product\_info)

\item {} 
\sphinxstyleliteralstrong{\sphinxupquote{logger}} \textendash{} (Logger) 
The logger instance to use for the display and logging

\end{itemize}

\item[{Returns}] \leavevmode
(int) the number of failing commands.

\end{description}\end{quote}

\end{fulllineitems}

\index{makeinstall\_product() (in module commands.makeinstall)}

\begin{fulllineitems}
\phantomsection\label{\detokenize{apidoc_commands/commands:commands.makeinstall.makeinstall_product}}\pysiglinewithargsret{\sphinxcode{\sphinxupquote{commands.makeinstall.}}\sphinxbfcode{\sphinxupquote{makeinstall\_product}}}{\emph{p\_name\_info}, \emph{config}, \emph{logger}}{}
Execute the proper configuration command(s) 
in the product build directory.
\begin{quote}\begin{description}
\item[{Parameters}] \leavevmode\begin{itemize}
\item {} 
\sphinxstyleliteralstrong{\sphinxupquote{p\_name\_info}} \textendash{} (tuple) 
(str, Config) =\textgreater{} (product\_name, product\_info)

\item {} 
\sphinxstyleliteralstrong{\sphinxupquote{config}} \textendash{} (Config) The global configuration

\item {} 
\sphinxstyleliteralstrong{\sphinxupquote{logger}} \textendash{} (Logger) 
The logger instance to use for the display and logging

\end{itemize}

\item[{Returns}] \leavevmode
(int) 1 if it fails, else 0.

\end{description}\end{quote}

\end{fulllineitems}



\subsubsection{commands.package module}
\label{\detokenize{apidoc_commands/commands:commands-package-module}}\label{\detokenize{apidoc_commands/commands:module-commands.package}}\index{commands.package (module)}\index{Command (class in commands.package)}

\begin{fulllineitems}
\phantomsection\label{\detokenize{apidoc_commands/commands:commands.package.Command}}\pysiglinewithargsret{\sphinxbfcode{\sphinxupquote{class }}\sphinxcode{\sphinxupquote{commands.package.}}\sphinxbfcode{\sphinxupquote{Command}}}{\emph{runner}}{}
Bases: \sphinxcode{\sphinxupquote{src.salomeTools.\_BaseCommand}}

The package command creates an archive.

\begin{DUlineblock}{0em}
\item[] There are 4 kinds of archive, which can be mixed:
\item[] 1- The binary archive. It contains all the product installation directories and a launcher.
\item[] 2- The sources archive. It contains the products archives, a project corresponding to the application and salomeTools.
\item[] 3- The project archive. It contains a project (give the project file path as argument).
\item[] 4- The salomeTools archive. It contains salomeTools.
\item[] 
\item[] examples:
\item[] \textgreater{}\textgreater{} sat package SALOME \textendash{}binaries \textendash{}sources
\end{DUlineblock}
\index{getParser() (commands.package.Command method)}

\begin{fulllineitems}
\phantomsection\label{\detokenize{apidoc_commands/commands:commands.package.Command.getParser}}\pysiglinewithargsret{\sphinxbfcode{\sphinxupquote{getParser}}}{}{}
Define all options for command ‘sat package \textless{}options\textgreater{}’

\end{fulllineitems}

\index{name (commands.package.Command attribute)}

\begin{fulllineitems}
\phantomsection\label{\detokenize{apidoc_commands/commands:commands.package.Command.name}}\pysigline{\sphinxbfcode{\sphinxupquote{name}}\sphinxbfcode{\sphinxupquote{ = 'package'}}}
\end{fulllineitems}

\index{run() (commands.package.Command method)}

\begin{fulllineitems}
\phantomsection\label{\detokenize{apidoc_commands/commands:commands.package.Command.run}}\pysiglinewithargsret{\sphinxbfcode{\sphinxupquote{run}}}{\emph{cmd\_arguments}}{}
method called for command ‘sat package \textless{}options\textgreater{}’

\end{fulllineitems}


\end{fulllineitems}

\index{add\_files() (in module commands.package)}

\begin{fulllineitems}
\phantomsection\label{\detokenize{apidoc_commands/commands:commands.package.add_files}}\pysiglinewithargsret{\sphinxcode{\sphinxupquote{commands.package.}}\sphinxbfcode{\sphinxupquote{add\_files}}}{\emph{tar}, \emph{name\_archive}, \emph{d\_content}, \emph{logger}, \emph{f\_exclude=None}}{}
Create an archive containing all directories and files that are given 
in the d\_content argument.
\begin{quote}\begin{description}
\item[{Parameters}] \leavevmode\begin{itemize}
\item {} 
\sphinxstyleliteralstrong{\sphinxupquote{tar}} \textendash{} (tarfile) The tarfile instance used to make the archive.

\item {} 
\sphinxstyleliteralstrong{\sphinxupquote{name\_archive}} \textendash{} (str) The name of the archive to make.

\item {} 
\sphinxstyleliteralstrong{\sphinxupquote{d\_content}} \textendash{} (dict) 
The dictionary that contain all directories and files to add in the archive.
d\_content{[}label{]} = (path\_on\_local\_machine, path\_in\_archive)

\item {} 
\sphinxstyleliteralstrong{\sphinxupquote{logger}} \textendash{} (Logger) the logging instance

\item {} 
\sphinxstyleliteralstrong{\sphinxupquote{f\_exclude}} \textendash{} (function) the function that filters

\end{itemize}

\item[{Returns}] \leavevmode
(int) 0 if success, 1 if not.

\end{description}\end{quote}

\end{fulllineitems}

\index{add\_readme() (in module commands.package)}

\begin{fulllineitems}
\phantomsection\label{\detokenize{apidoc_commands/commands:commands.package.add_readme}}\pysiglinewithargsret{\sphinxcode{\sphinxupquote{commands.package.}}\sphinxbfcode{\sphinxupquote{add\_readme}}}{\emph{config}, \emph{options}, \emph{where}}{}
\end{fulllineitems}

\index{add\_salomeTools() (in module commands.package)}

\begin{fulllineitems}
\phantomsection\label{\detokenize{apidoc_commands/commands:commands.package.add_salomeTools}}\pysiglinewithargsret{\sphinxcode{\sphinxupquote{commands.package.}}\sphinxbfcode{\sphinxupquote{add\_salomeTools}}}{\emph{config}, \emph{tmp\_working\_dir}}{}
Prepare a version of salomeTools that has a specific local.pyconf file 
configured for a source package.
\begin{quote}\begin{description}
\item[{Parameters}] \leavevmode\begin{itemize}
\item {} 
\sphinxstyleliteralstrong{\sphinxupquote{config}} \textendash{} (Config) The global configuration.

\item {} 
\sphinxstyleliteralstrong{\sphinxupquote{tmp\_working\_dir}} \textendash{} (str) 
The temporary local directory containing some specific directories
or files needed in the source package

\end{itemize}

\item[{Returns}] \leavevmode
(str) 
The path to the local salomeTools directory to add in the package

\end{description}\end{quote}

\end{fulllineitems}

\index{binary\_package() (in module commands.package)}

\begin{fulllineitems}
\phantomsection\label{\detokenize{apidoc_commands/commands:commands.package.binary_package}}\pysiglinewithargsret{\sphinxcode{\sphinxupquote{commands.package.}}\sphinxbfcode{\sphinxupquote{binary\_package}}}{\emph{config}, \emph{logger}, \emph{options}, \emph{tmp\_working\_dir}}{}
Prepare a dictionary that stores all the needed directories and files 
to add in a binary package.
\begin{quote}\begin{description}
\item[{Parameters}] \leavevmode\begin{itemize}
\item {} 
\sphinxstyleliteralstrong{\sphinxupquote{config}} \textendash{} (Config) The global configuration.

\item {} 
\sphinxstyleliteralstrong{\sphinxupquote{logger}} \textendash{} (Logger) the logging instance

\item {} 
\sphinxstyleliteralstrong{\sphinxupquote{options}} \textendash{} (OptResult) the options of the launched command

\item {} 
\sphinxstyleliteralstrong{\sphinxupquote{tmp\_working\_dir}} \textendash{} (str) 
The temporary local directory containing some specific directories
or files needed in the binary package

\end{itemize}

\item[{Returns}] \leavevmode
(dict) 
The dictionary that stores all the needed directories and files 
to add in a binary package.
\{label : (path\_on\_local\_machine, path\_in\_archive)\}

\end{description}\end{quote}

\end{fulllineitems}

\index{create\_project\_for\_src\_package() (in module commands.package)}

\begin{fulllineitems}
\phantomsection\label{\detokenize{apidoc_commands/commands:commands.package.create_project_for_src_package}}\pysiglinewithargsret{\sphinxcode{\sphinxupquote{commands.package.}}\sphinxbfcode{\sphinxupquote{create\_project\_for\_src\_package}}}{\emph{config}, \emph{tmp\_working\_dir}, \emph{with\_vcs}}{}
Create a specific project for a source package.
\begin{quote}\begin{description}
\item[{Parameters}] \leavevmode\begin{itemize}
\item {} 
\sphinxstyleliteralstrong{\sphinxupquote{config}} \textendash{} (Config) The global configuration.

\item {} 
\sphinxstyleliteralstrong{\sphinxupquote{tmp\_working\_dir}} \textendash{} (str)
The temporary local directory containing some specific directories
or files needed in the source package

\item {} 
\sphinxstyleliteralstrong{\sphinxupquote{with\_vcs}} \textendash{} (bool) 
True if the package is with vcs products 
(not transformed into archive products)

\end{itemize}

\item[{Returns}] \leavevmode
(dict)
The dictionary 
\{“project” : (produced project, project path in the archive)\}

\end{description}\end{quote}

\end{fulllineitems}

\index{exclude\_VCS\_and\_extensions() (in module commands.package)}

\begin{fulllineitems}
\phantomsection\label{\detokenize{apidoc_commands/commands:commands.package.exclude_VCS_and_extensions}}\pysiglinewithargsret{\sphinxcode{\sphinxupquote{commands.package.}}\sphinxbfcode{\sphinxupquote{exclude\_VCS\_and\_extensions}}}{\emph{filename}}{}
The function that is used to exclude from package the link to the 
VCS repositories (like .git)
\begin{quote}\begin{description}
\item[{Parameters}] \leavevmode
\sphinxstyleliteralstrong{\sphinxupquote{filename}} \textendash{} (str) The filname to exclude (or not).

\item[{Returns}] \leavevmode
(bool) True if the file has to be exclude

\end{description}\end{quote}

\end{fulllineitems}

\index{find\_application\_pyconf() (in module commands.package)}

\begin{fulllineitems}
\phantomsection\label{\detokenize{apidoc_commands/commands:commands.package.find_application_pyconf}}\pysiglinewithargsret{\sphinxcode{\sphinxupquote{commands.package.}}\sphinxbfcode{\sphinxupquote{find\_application\_pyconf}}}{\emph{config}, \emph{application\_tmp\_dir}}{}
Find the application pyconf file and put it in the specific temporary 
directory containing the specific project of a source package.
\begin{quote}\begin{description}
\item[{Parameters}] \leavevmode\begin{itemize}
\item {} 
\sphinxstyleliteralstrong{\sphinxupquote{config}} \textendash{} ‘Config) The global configuration.

\item {} 
\sphinxstyleliteralstrong{\sphinxupquote{application\_tmp\_dir}} \textendash{} (str) 
The path to the temporary application scripts directory of the project.

\end{itemize}

\end{description}\end{quote}

\end{fulllineitems}

\index{find\_product\_scripts\_and\_pyconf() (in module commands.package)}

\begin{fulllineitems}
\phantomsection\label{\detokenize{apidoc_commands/commands:commands.package.find_product_scripts_and_pyconf}}\pysiglinewithargsret{\sphinxcode{\sphinxupquote{commands.package.}}\sphinxbfcode{\sphinxupquote{find\_product\_scripts\_and\_pyconf}}}{\emph{p\_name}, \emph{p\_info}, \emph{config}, \emph{with\_vcs}, \emph{compil\_scripts\_tmp\_dir}, \emph{env\_scripts\_tmp\_dir}, \emph{patches\_tmp\_dir}, \emph{products\_pyconf\_tmp\_dir}}{}
Create a specific pyconf file for a given product. 
Get its environment script, its compilation script 
and patches and put it in the temporary working directory. 
This method is used in the source package in order to
construct the specific project.
\begin{quote}\begin{description}
\item[{Parameters}] \leavevmode\begin{itemize}
\item {} 
\sphinxstyleliteralstrong{\sphinxupquote{p\_name}} \textendash{} (str) The name of the product.

\item {} 
\sphinxstyleliteralstrong{\sphinxupquote{p\_info}} \textendash{} (Config) The specific configuration corresponding to the product

\item {} 
\sphinxstyleliteralstrong{\sphinxupquote{config}} \textendash{} (Config) The global configuration.

\item {} 
\sphinxstyleliteralstrong{\sphinxupquote{with\_vcs}} \textendash{} (bool) 
True if the package is with vcs products 
(not transformed into archive products)

\item {} 
\sphinxstyleliteralstrong{\sphinxupquote{compil\_scripts\_tmp\_dir}} \textendash{} (str) 
The path to the temporary compilation scripts directory of the project.

\item {} 
\sphinxstyleliteralstrong{\sphinxupquote{env\_scripts\_tmp\_dir}} \textendash{} (str) 
The path to the temporary environment script directory of the project.

\item {} 
\sphinxstyleliteralstrong{\sphinxupquote{patches\_tmp\_dir}} \textendash{} (str) 
The path to the temporary patch scripts directory of the project.

\item {} 
\sphinxstyleliteralstrong{\sphinxupquote{products\_pyconf\_tmp\_dir}} \textendash{} (str) 
The path to the temporary product scripts directory of the project.

\end{itemize}

\end{description}\end{quote}

\end{fulllineitems}

\index{get\_archives() (in module commands.package)}

\begin{fulllineitems}
\phantomsection\label{\detokenize{apidoc_commands/commands:commands.package.get_archives}}\pysiglinewithargsret{\sphinxcode{\sphinxupquote{commands.package.}}\sphinxbfcode{\sphinxupquote{get\_archives}}}{\emph{config}, \emph{logger}}{}
Find all the products from an archive and all the products
from a VCS (git, cvs, svn) repository.
\begin{quote}\begin{description}
\item[{Parameters}] \leavevmode\begin{itemize}
\item {} 
\sphinxstyleliteralstrong{\sphinxupquote{config}} \textendash{} (Config) The global configuration.

\item {} 
\sphinxstyleliteralstrong{\sphinxupquote{logger}} \textendash{} (Logger) The logging instance

\end{itemize}

\item[{Returns}] \leavevmode
(Dict, List)
The dictionary 
\{name\_product : (local path of its archive, path in the package of its archive )\}
and the list of specific configuration corresponding to the vcs products

\end{description}\end{quote}

\end{fulllineitems}

\index{get\_archives\_vcs() (in module commands.package)}

\begin{fulllineitems}
\phantomsection\label{\detokenize{apidoc_commands/commands:commands.package.get_archives_vcs}}\pysiglinewithargsret{\sphinxcode{\sphinxupquote{commands.package.}}\sphinxbfcode{\sphinxupquote{get\_archives\_vcs}}}{\emph{l\_pinfo\_vcs}, \emph{sat}, \emph{config}, \emph{logger}, \emph{tmp\_working\_dir}}{}
For sources package that require that all products from an archive, 
one has to create some archive for the vcs products.
So this method calls the clean and source command of sat 
and then create the archives.
\begin{quote}\begin{description}
\item[{Parameters}] \leavevmode\begin{itemize}
\item {} 
\sphinxstyleliteralstrong{\sphinxupquote{l\_pinfo\_vcs}} \textendash{} (list) 
The list of specific configuration corresponding to each vcs product

\item {} 
\sphinxstyleliteralstrong{\sphinxupquote{sat}} \textendash{} (Sat) 
The Sat instance that can be called to clean and source the products

\item {} 
\sphinxstyleliteralstrong{\sphinxupquote{config}} \textendash{} (Config) The global configuration.

\item {} 
\sphinxstyleliteralstrong{\sphinxupquote{logger}} \textendash{} (Logger) The logging instance

\item {} 
\sphinxstyleliteralstrong{\sphinxupquote{tmp\_working\_dir}} \textendash{} (str) 
The temporary local directory containing some specific directories
or files needed in the source package

\end{itemize}

\item[{Returns}] \leavevmode
(dict) 
The dictionary that stores all the archives to add in the sourcepackage.
\{label : (path\_on\_local\_machine, path\_in\_archive)\}

\end{description}\end{quote}

\end{fulllineitems}

\index{hack\_for\_distene\_licence() (in module commands.package)}

\begin{fulllineitems}
\phantomsection\label{\detokenize{apidoc_commands/commands:commands.package.hack_for_distene_licence}}\pysiglinewithargsret{\sphinxcode{\sphinxupquote{commands.package.}}\sphinxbfcode{\sphinxupquote{hack\_for\_distene\_licence}}}{\emph{filepath}}{}
Replace the distene licence env variable by a call to a file.
\begin{quote}\begin{description}
\item[{Parameters}] \leavevmode
\sphinxstyleliteralstrong{\sphinxupquote{filepath}} \textendash{} (str) The path to the launcher to modify.

\end{description}\end{quote}

\end{fulllineitems}

\index{make\_archive() (in module commands.package)}

\begin{fulllineitems}
\phantomsection\label{\detokenize{apidoc_commands/commands:commands.package.make_archive}}\pysiglinewithargsret{\sphinxcode{\sphinxupquote{commands.package.}}\sphinxbfcode{\sphinxupquote{make\_archive}}}{\emph{prod\_name}, \emph{prod\_info}, \emph{where}}{}
Create an archive of a product by searching its source directory.
\begin{quote}\begin{description}
\item[{Parameters}] \leavevmode\begin{itemize}
\item {} 
\sphinxstyleliteralstrong{\sphinxupquote{prod\_name}} \textendash{} (str) The name of the product.

\item {} 
\sphinxstyleliteralstrong{\sphinxupquote{prod\_info}} \textendash{} (Config)
The specific configuration corresponding to the product

\item {} 
\sphinxstyleliteralstrong{\sphinxupquote{where}} \textendash{} (str) 
The path of the repository where to put the resulting archive

\end{itemize}

\item[{Returns}] \leavevmode
(str) The path of the resulting archive

\end{description}\end{quote}

\end{fulllineitems}

\index{produce\_install\_bin\_file() (in module commands.package)}

\begin{fulllineitems}
\phantomsection\label{\detokenize{apidoc_commands/commands:commands.package.produce_install_bin_file}}\pysiglinewithargsret{\sphinxcode{\sphinxupquote{commands.package.}}\sphinxbfcode{\sphinxupquote{produce\_install\_bin\_file}}}{\emph{config}, \emph{logger}, \emph{file\_dir}, \emph{d\_sub}, \emph{file\_name}}{}
Create a bash shell script which do substitutions in BIRARIES dir 
in order to use it for extra compilations.
\begin{quote}\begin{description}
\item[{Parameters}] \leavevmode\begin{itemize}
\item {} 
\sphinxstyleliteralstrong{\sphinxupquote{config}} \textendash{} (Config) The global configuration.

\item {} 
\sphinxstyleliteralstrong{\sphinxupquote{logger}} \textendash{} (Logger) the logging instance

\item {} 
\sphinxstyleliteralstrong{\sphinxupquote{file\_dir}} \textendash{} (str) the directory where to put the files

\item {} 
\sphinxstyleliteralstrong{\sphinxupquote{d\_sub}} \textendash{} (dict) 
the dictionnary that contains the substitutions to be done

\item {} 
\sphinxstyleliteralstrong{\sphinxupquote{file\_name}} \textendash{} (str) the name of the install script file

\end{itemize}

\item[{Returns}] \leavevmode
(str) the produced file

\end{description}\end{quote}

\end{fulllineitems}

\index{produce\_relative\_env\_files() (in module commands.package)}

\begin{fulllineitems}
\phantomsection\label{\detokenize{apidoc_commands/commands:commands.package.produce_relative_env_files}}\pysiglinewithargsret{\sphinxcode{\sphinxupquote{commands.package.}}\sphinxbfcode{\sphinxupquote{produce\_relative\_env\_files}}}{\emph{config}, \emph{logger}, \emph{file\_dir}, \emph{binaries\_dir\_name}}{}
Create some specific environment files for the binary package.
These files use relative paths.
\begin{quote}\begin{description}
\item[{Parameters}] \leavevmode\begin{itemize}
\item {} 
\sphinxstyleliteralstrong{\sphinxupquote{config}} \textendash{} (Config) The global configuration.

\item {} 
\sphinxstyleliteralstrong{\sphinxupquote{logger}} \textendash{} (Logger) the logging instance

\item {} 
\sphinxstyleliteralstrong{\sphinxupquote{file\_dir}} \textendash{} (str) the directory where to put the files

\item {} 
\sphinxstyleliteralstrong{\sphinxupquote{binaries\_dir\_name}} \textendash{} (str) 
The name of the repository where the binaries are, in the archive.

\end{itemize}

\item[{Returns}] \leavevmode
(list) The list of path of the produced environment files

\end{description}\end{quote}

\end{fulllineitems}

\index{produce\_relative\_launcher() (in module commands.package)}

\begin{fulllineitems}
\phantomsection\label{\detokenize{apidoc_commands/commands:commands.package.produce_relative_launcher}}\pysiglinewithargsret{\sphinxcode{\sphinxupquote{commands.package.}}\sphinxbfcode{\sphinxupquote{produce\_relative\_launcher}}}{\emph{config}, \emph{logger}, \emph{file\_dir}, \emph{file\_name}, \emph{binaries\_dir\_name}, \emph{with\_commercial=True}}{}
Create a specific SALOME launcher for the binary package.
This launcher uses relative paths.
\begin{quote}\begin{description}
\item[{Parameters}] \leavevmode\begin{itemize}
\item {} 
\sphinxstyleliteralstrong{\sphinxupquote{config}} \textendash{} (Config) The global configuration.

\item {} 
\sphinxstyleliteralstrong{\sphinxupquote{logger}} \textendash{} (Logger) the logging instance

\item {} 
\sphinxstyleliteralstrong{\sphinxupquote{file\_dir}} \textendash{} (str) the directory where to put the launcher

\item {} 
\sphinxstyleliteralstrong{\sphinxupquote{file\_name}} \textendash{} (str) The launcher name

\item {} 
\sphinxstyleliteralstrong{\sphinxupquote{binaries\_dir\_name}} \textendash{} (str) 
the name of the repository where the binaries are, in the archive.

\end{itemize}

\item[{Returns}] \leavevmode
(str) the path of the produced launcher

\end{description}\end{quote}

\end{fulllineitems}

\index{product\_appli\_creation\_script() (in module commands.package)}

\begin{fulllineitems}
\phantomsection\label{\detokenize{apidoc_commands/commands:commands.package.product_appli_creation_script}}\pysiglinewithargsret{\sphinxcode{\sphinxupquote{commands.package.}}\sphinxbfcode{\sphinxupquote{product\_appli\_creation\_script}}}{\emph{config}, \emph{logger}, \emph{file\_dir}, \emph{binaries\_dir\_name}}{}
Create a script that can produce an application (EDF style) 
in the binary package.
\begin{quote}\begin{description}
\item[{Parameters}] \leavevmode\begin{itemize}
\item {} 
\sphinxstyleliteralstrong{\sphinxupquote{config}} \textendash{} (Config) The global configuration.

\item {} 
\sphinxstyleliteralstrong{\sphinxupquote{logger}} \textendash{} (Logger) the logging instance

\item {} 
\sphinxstyleliteralstrong{\sphinxupquote{file\_dir}} \textendash{} (str) the directory where to put the file

\item {} 
\sphinxstyleliteralstrong{\sphinxupquote{binaries\_dir\_name}} \textendash{} (str) 
The name of the repository where the binaries are, in the archive.

\end{itemize}

\item[{Returns}] \leavevmode
(str) The path of the produced script file

\end{description}\end{quote}

\end{fulllineitems}

\index{project\_package() (in module commands.package)}

\begin{fulllineitems}
\phantomsection\label{\detokenize{apidoc_commands/commands:commands.package.project_package}}\pysiglinewithargsret{\sphinxcode{\sphinxupquote{commands.package.}}\sphinxbfcode{\sphinxupquote{project\_package}}}{\emph{project\_file\_path}, \emph{tmp\_working\_dir}}{}
Prepare a dictionary that stores all the needed directories and files 
to add in a project package.
\begin{quote}\begin{description}
\item[{Parameters}] \leavevmode\begin{itemize}
\item {} 
\sphinxstyleliteralstrong{\sphinxupquote{project\_file\_path}} \textendash{} (str) The path to the local project.

\item {} 
\sphinxstyleliteralstrong{\sphinxupquote{tmp\_working\_dir}} \textendash{} (str) 
The temporary local directory containing some specific directories
or files needed in the project package

\end{itemize}

\item[{Returns}] \leavevmode
(dict)
The dictionary that stores all the needed directories and files
to add in a project package.
\{label : (path\_on\_local\_machine, path\_in\_archive)\}

\end{description}\end{quote}

\end{fulllineitems}

\index{source\_package() (in module commands.package)}

\begin{fulllineitems}
\phantomsection\label{\detokenize{apidoc_commands/commands:commands.package.source_package}}\pysiglinewithargsret{\sphinxcode{\sphinxupquote{commands.package.}}\sphinxbfcode{\sphinxupquote{source\_package}}}{\emph{sat}, \emph{config}, \emph{logger}, \emph{options}, \emph{tmp\_working\_dir}}{}
Prepare a dictionary that stores all the needed directories and files 
to add in a source package.
\begin{quote}\begin{description}
\item[{Parameters}] \leavevmode\begin{itemize}
\item {} 
\sphinxstyleliteralstrong{\sphinxupquote{config}} \textendash{} (Config) The global configuration.

\item {} 
\sphinxstyleliteralstrong{\sphinxupquote{logger}} \textendash{} (Logger) the logging instance

\item {} 
\sphinxstyleliteralstrong{\sphinxupquote{options}} \textendash{} (OptResult) the options of the launched command

\item {} 
\sphinxstyleliteralstrong{\sphinxupquote{tmp\_working\_dir}} \textendash{} (str)
The temporary local directory containing some specific directories
or files needed in the binary package

\end{itemize}

\item[{Returns}] \leavevmode
(dict)
the dictionary that stores all the needed directories and files
to add in a source package.
\{label : (path\_on\_local\_machine, path\_in\_archive)\}

\end{description}\end{quote}

\end{fulllineitems}

\index{update\_config() (in module commands.package)}

\begin{fulllineitems}
\phantomsection\label{\detokenize{apidoc_commands/commands:commands.package.update_config}}\pysiglinewithargsret{\sphinxcode{\sphinxupquote{commands.package.}}\sphinxbfcode{\sphinxupquote{update\_config}}}{\emph{config}, \emph{prop}, \emph{value}}{}
Remove from config.APPLICATION.products the products 
that have the property given as input.
\begin{quote}\begin{description}
\item[{Parameters}] \leavevmode\begin{itemize}
\item {} 
\sphinxstyleliteralstrong{\sphinxupquote{config}} \textendash{} (Config) The global config.

\item {} 
\sphinxstyleliteralstrong{\sphinxupquote{prop}} \textendash{} (str) The property to filter

\item {} 
\sphinxstyleliteralstrong{\sphinxupquote{value}} \textendash{} (str) The value of the property to filter

\end{itemize}

\end{description}\end{quote}

\end{fulllineitems}



\subsubsection{commands.patch module}
\label{\detokenize{apidoc_commands/commands:module-commands.patch}}\label{\detokenize{apidoc_commands/commands:commands-patch-module}}\index{commands.patch (module)}\index{Command (class in commands.patch)}

\begin{fulllineitems}
\phantomsection\label{\detokenize{apidoc_commands/commands:commands.patch.Command}}\pysiglinewithargsret{\sphinxbfcode{\sphinxupquote{class }}\sphinxcode{\sphinxupquote{commands.patch.}}\sphinxbfcode{\sphinxupquote{Command}}}{\emph{runner}}{}
Bases: \sphinxcode{\sphinxupquote{src.salomeTools.\_BaseCommand}}

The patch command apply the patches on the sources of the application products
if there is any.

\begin{DUlineblock}{0em}
\item[] examples:
\item[] \textgreater{}\textgreater{} sat patch SALOME \textendash{}products qt,boost
\end{DUlineblock}
\index{getParser() (commands.patch.Command method)}

\begin{fulllineitems}
\phantomsection\label{\detokenize{apidoc_commands/commands:commands.patch.Command.getParser}}\pysiglinewithargsret{\sphinxbfcode{\sphinxupquote{getParser}}}{}{}
Define all options for command ‘sat patch \textless{}options\textgreater{}’

\end{fulllineitems}

\index{name (commands.patch.Command attribute)}

\begin{fulllineitems}
\phantomsection\label{\detokenize{apidoc_commands/commands:commands.patch.Command.name}}\pysigline{\sphinxbfcode{\sphinxupquote{name}}\sphinxbfcode{\sphinxupquote{ = 'patch'}}}
\end{fulllineitems}

\index{run() (commands.patch.Command method)}

\begin{fulllineitems}
\phantomsection\label{\detokenize{apidoc_commands/commands:commands.patch.Command.run}}\pysiglinewithargsret{\sphinxbfcode{\sphinxupquote{run}}}{\emph{cmd\_arguments}}{}
method called for command ‘sat patch \textless{}options\textgreater{}’

\end{fulllineitems}


\end{fulllineitems}

\index{apply\_patch() (in module commands.patch)}

\begin{fulllineitems}
\phantomsection\label{\detokenize{apidoc_commands/commands:commands.patch.apply_patch}}\pysiglinewithargsret{\sphinxcode{\sphinxupquote{commands.patch.}}\sphinxbfcode{\sphinxupquote{apply\_patch}}}{\emph{config}, \emph{product\_info}, \emph{max\_product\_name\_len}, \emph{logger}}{}
The method called to apply patches on a product
\begin{quote}\begin{description}
\item[{Parameters}] \leavevmode\begin{itemize}
\item {} 
\sphinxstyleliteralstrong{\sphinxupquote{config}} \textendash{} (Config) The global configuration

\item {} 
\sphinxstyleliteralstrong{\sphinxupquote{product\_info}} \textendash{} (Config) 
The configuration specific to the product to be patched

\item {} 
\sphinxstyleliteralstrong{\sphinxupquote{logger}} \textendash{} (Logger: 
The logger instance to use for the display and logging

\end{itemize}

\item[{Returns}] \leavevmode
(RCO.ReturnCode)

\end{description}\end{quote}

\end{fulllineitems}



\subsubsection{commands.prepare module}
\label{\detokenize{apidoc_commands/commands:commands-prepare-module}}\label{\detokenize{apidoc_commands/commands:module-commands.prepare}}\index{commands.prepare (module)}\index{Command (class in commands.prepare)}

\begin{fulllineitems}
\phantomsection\label{\detokenize{apidoc_commands/commands:commands.prepare.Command}}\pysiglinewithargsret{\sphinxbfcode{\sphinxupquote{class }}\sphinxcode{\sphinxupquote{commands.prepare.}}\sphinxbfcode{\sphinxupquote{Command}}}{\emph{runner}}{}
Bases: \sphinxcode{\sphinxupquote{src.salomeTools.\_BaseCommand}}

The prepare command gets the sources of the application products 
and apply the patches if there is any.

\begin{DUlineblock}{0em}
\item[] examples:
\item[] \textgreater{}\textgreater{} sat prepare SALOME \textendash{}products KERNEL,GUI
\end{DUlineblock}
\index{getParser() (commands.prepare.Command method)}

\begin{fulllineitems}
\phantomsection\label{\detokenize{apidoc_commands/commands:commands.prepare.Command.getParser}}\pysiglinewithargsret{\sphinxbfcode{\sphinxupquote{getParser}}}{}{}
Define all options for command ‘sat prepare \textless{}options\textgreater{}’

\end{fulllineitems}

\index{name (commands.prepare.Command attribute)}

\begin{fulllineitems}
\phantomsection\label{\detokenize{apidoc_commands/commands:commands.prepare.Command.name}}\pysigline{\sphinxbfcode{\sphinxupquote{name}}\sphinxbfcode{\sphinxupquote{ = 'prepare'}}}
\end{fulllineitems}

\index{run() (commands.prepare.Command method)}

\begin{fulllineitems}
\phantomsection\label{\detokenize{apidoc_commands/commands:commands.prepare.Command.run}}\pysiglinewithargsret{\sphinxbfcode{\sphinxupquote{run}}}{\emph{cmd\_arguments}}{}
method called for command ‘sat prepare \textless{}options\textgreater{}’

\end{fulllineitems}


\end{fulllineitems}

\index{find\_products\_already\_getted() (in module commands.prepare)}

\begin{fulllineitems}
\phantomsection\label{\detokenize{apidoc_commands/commands:commands.prepare.find_products_already_getted}}\pysiglinewithargsret{\sphinxcode{\sphinxupquote{commands.prepare.}}\sphinxbfcode{\sphinxupquote{find\_products\_already\_getted}}}{\emph{l\_products}}{}
Returns the list of products that have an existing source directory.
\begin{quote}\begin{description}
\item[{Parameters}] \leavevmode
\sphinxstyleliteralstrong{\sphinxupquote{l\_products}} \textendash{} (list) The list of products to check

\item[{Returns}] \leavevmode
(list) 
The list of product configurations 
that have an existing source directory.

\end{description}\end{quote}

\end{fulllineitems}

\index{find\_products\_with\_patchs() (in module commands.prepare)}

\begin{fulllineitems}
\phantomsection\label{\detokenize{apidoc_commands/commands:commands.prepare.find_products_with_patchs}}\pysiglinewithargsret{\sphinxcode{\sphinxupquote{commands.prepare.}}\sphinxbfcode{\sphinxupquote{find\_products\_with\_patchs}}}{\emph{l\_products}}{}
Returns the list of products that have one or more patches.
\begin{quote}\begin{description}
\item[{Parameters}] \leavevmode
\sphinxstyleliteralstrong{\sphinxupquote{l\_products}} \textendash{} (list) The list of products to check

\item[{Returns}] \leavevmode
(list) 
The list of product configurations 
that have one or more patches.

\end{description}\end{quote}

\end{fulllineitems}

\index{remove\_products() (in module commands.prepare)}

\begin{fulllineitems}
\phantomsection\label{\detokenize{apidoc_commands/commands:commands.prepare.remove_products}}\pysiglinewithargsret{\sphinxcode{\sphinxupquote{commands.prepare.}}\sphinxbfcode{\sphinxupquote{remove\_products}}}{\emph{arguments}, \emph{l\_products\_info}, \emph{logger}}{}
Removes the products in l\_products\_info from arguments list.
\begin{quote}\begin{description}
\item[{Parameters}] \leavevmode\begin{itemize}
\item {} 
\sphinxstyleliteralstrong{\sphinxupquote{arguments}} \textendash{} (str) The arguments from which to remove products

\item {} 
\sphinxstyleliteralstrong{\sphinxupquote{l\_products\_info}} \textendash{} (list) 
List of (str, Config) =\textgreater{} (product\_name, product\_info)

\item {} 
\sphinxstyleliteralstrong{\sphinxupquote{logger}} \textendash{} (Logger) 
The logger instance to use for the display and logging

\end{itemize}

\item[{Returns}] \leavevmode
(str) The updated arguments.

\end{description}\end{quote}

\end{fulllineitems}



\subsubsection{commands.profile module}
\label{\detokenize{apidoc_commands/commands:commands-profile-module}}\label{\detokenize{apidoc_commands/commands:module-commands.profile}}\index{commands.profile (module)}\index{Command (class in commands.profile)}

\begin{fulllineitems}
\phantomsection\label{\detokenize{apidoc_commands/commands:commands.profile.Command}}\pysiglinewithargsret{\sphinxbfcode{\sphinxupquote{class }}\sphinxcode{\sphinxupquote{commands.profile.}}\sphinxbfcode{\sphinxupquote{Command}}}{\emph{runner}}{}
Bases: \sphinxcode{\sphinxupquote{src.salomeTools.\_BaseCommand}}

The profile command creates default profile.

\begin{DUlineblock}{0em}
\item[] examples: 
\item[] \textgreater{}\textgreater{} sat profile {[}PRODUCT{]} 
\item[] \textgreater{}\textgreater{} sat profile \textendash{}prefix (string)
\item[] \textgreater{}\textgreater{} sat profile \textendash{}name (string)
\item[] \textgreater{}\textgreater{} sat profile \textendash{}force
\item[] \textgreater{}\textgreater{} sat profile \textendash{}version (string) 
\item[] \textgreater{}\textgreater{} sat profile \textendash{}slogan (string) 
\end{DUlineblock}
\index{getParser() (commands.profile.Command method)}

\begin{fulllineitems}
\phantomsection\label{\detokenize{apidoc_commands/commands:commands.profile.Command.getParser}}\pysiglinewithargsret{\sphinxbfcode{\sphinxupquote{getParser}}}{}{}
Define all options for command ‘sat profile \textless{}options\textgreater{}’

\end{fulllineitems}

\index{name (commands.profile.Command attribute)}

\begin{fulllineitems}
\phantomsection\label{\detokenize{apidoc_commands/commands:commands.profile.Command.name}}\pysigline{\sphinxbfcode{\sphinxupquote{name}}\sphinxbfcode{\sphinxupquote{ = 'profile'}}}
\end{fulllineitems}

\index{run() (commands.profile.Command method)}

\begin{fulllineitems}
\phantomsection\label{\detokenize{apidoc_commands/commands:commands.profile.Command.run}}\pysiglinewithargsret{\sphinxbfcode{\sphinxupquote{run}}}{\emph{cmd\_arguments}}{}
method called for command ‘sat profile \textless{}options\textgreater{}’

\end{fulllineitems}


\end{fulllineitems}

\index{generate\_profile\_sources() (in module commands.profile)}

\begin{fulllineitems}
\phantomsection\label{\detokenize{apidoc_commands/commands:commands.profile.generate_profile_sources}}\pysiglinewithargsret{\sphinxcode{\sphinxupquote{commands.profile.}}\sphinxbfcode{\sphinxupquote{generate\_profile\_sources}}}{\emph{config}, \emph{options}, \emph{logger}}{}
Generates the sources of the profile

\end{fulllineitems}

\index{get\_profile\_name() (in module commands.profile)}

\begin{fulllineitems}
\phantomsection\label{\detokenize{apidoc_commands/commands:commands.profile.get_profile_name}}\pysiglinewithargsret{\sphinxcode{\sphinxupquote{commands.profile.}}\sphinxbfcode{\sphinxupquote{get\_profile\_name}}}{\emph{options}, \emph{config}}{}
\end{fulllineitems}

\index{profileConfigReader (class in commands.profile)}

\begin{fulllineitems}
\phantomsection\label{\detokenize{apidoc_commands/commands:commands.profile.profileConfigReader}}\pysiglinewithargsret{\sphinxbfcode{\sphinxupquote{class }}\sphinxcode{\sphinxupquote{commands.profile.}}\sphinxbfcode{\sphinxupquote{profileConfigReader}}}{\emph{config}}{}
Bases: {\hyperref[\detokenize{apidoc_src/src:src.pyconf.ConfigReader}]{\sphinxcrossref{\sphinxcode{\sphinxupquote{src.pyconf.ConfigReader}}}}} (\autopageref*{\detokenize{apidoc_src/src:src.pyconf.ConfigReader}})
\index{parseMapping() (commands.profile.profileConfigReader method)}

\begin{fulllineitems}
\phantomsection\label{\detokenize{apidoc_commands/commands:commands.profile.profileConfigReader.parseMapping}}\pysiglinewithargsret{\sphinxbfcode{\sphinxupquote{parseMapping}}}{\emph{parent}, \emph{suffix}}{}
Parse a mapping.

@param parent: The container to which the mapping will be added.
@type parent: A L\{Container\} instance.
@param suffix: The suffix for the value.
@type suffix: str
@return: a L\{Mapping\} instance representing the mapping.
@rtype: L\{Mapping\}
@raise ConfigFormatError: if a syntax error is found.

\end{fulllineitems}


\end{fulllineitems}

\index{profileReference (class in commands.profile)}

\begin{fulllineitems}
\phantomsection\label{\detokenize{apidoc_commands/commands:commands.profile.profileReference}}\pysiglinewithargsret{\sphinxbfcode{\sphinxupquote{class }}\sphinxcode{\sphinxupquote{commands.profile.}}\sphinxbfcode{\sphinxupquote{profileReference}}}{\emph{config}, \emph{type}, \emph{ident}}{}
Bases: {\hyperref[\detokenize{apidoc_src/src:src.pyconf.Reference}]{\sphinxcrossref{\sphinxcode{\sphinxupquote{src.pyconf.Reference}}}}} (\autopageref*{\detokenize{apidoc_src/src:src.pyconf.Reference}})

\end{fulllineitems}

\index{update\_pyconf() (in module commands.profile)}

\begin{fulllineitems}
\phantomsection\label{\detokenize{apidoc_commands/commands:commands.profile.update_pyconf}}\pysiglinewithargsret{\sphinxcode{\sphinxupquote{commands.profile.}}\sphinxbfcode{\sphinxupquote{update\_pyconf}}}{\emph{config}, \emph{options}, \emph{logger}}{}
Updates the pyconf

\end{fulllineitems}



\subsubsection{commands.run module}
\label{\detokenize{apidoc_commands/commands:commands-run-module}}\label{\detokenize{apidoc_commands/commands:module-commands.run}}\index{commands.run (module)}\index{Command (class in commands.run)}

\begin{fulllineitems}
\phantomsection\label{\detokenize{apidoc_commands/commands:commands.run.Command}}\pysiglinewithargsret{\sphinxbfcode{\sphinxupquote{class }}\sphinxcode{\sphinxupquote{commands.run.}}\sphinxbfcode{\sphinxupquote{Command}}}{\emph{runner}}{}
Bases: \sphinxcode{\sphinxupquote{src.salomeTools.\_BaseCommand}}

The run command runs the application launcher with the given arguments.

\begin{DUlineblock}{0em}
\item[] examples:
\item[] \textgreater{}\textgreater{} sat run SALOME
\end{DUlineblock}
\index{getParser() (commands.run.Command method)}

\begin{fulllineitems}
\phantomsection\label{\detokenize{apidoc_commands/commands:commands.run.Command.getParser}}\pysiglinewithargsret{\sphinxbfcode{\sphinxupquote{getParser}}}{}{}
Define all options for command ‘sat run \textless{}options\textgreater{}’

\end{fulllineitems}

\index{name (commands.run.Command attribute)}

\begin{fulllineitems}
\phantomsection\label{\detokenize{apidoc_commands/commands:commands.run.Command.name}}\pysigline{\sphinxbfcode{\sphinxupquote{name}}\sphinxbfcode{\sphinxupquote{ = 'run'}}}
\end{fulllineitems}

\index{run() (commands.run.Command method)}

\begin{fulllineitems}
\phantomsection\label{\detokenize{apidoc_commands/commands:commands.run.Command.run}}\pysiglinewithargsret{\sphinxbfcode{\sphinxupquote{run}}}{\emph{cmd\_arguments}}{}
method called for command ‘sat run \textless{}options\textgreater{}’

\end{fulllineitems}


\end{fulllineitems}



\subsubsection{commands.script module}
\label{\detokenize{apidoc_commands/commands:module-commands.script}}\label{\detokenize{apidoc_commands/commands:commands-script-module}}\index{commands.script (module)}\index{Command (class in commands.script)}

\begin{fulllineitems}
\phantomsection\label{\detokenize{apidoc_commands/commands:commands.script.Command}}\pysiglinewithargsret{\sphinxbfcode{\sphinxupquote{class }}\sphinxcode{\sphinxupquote{commands.script.}}\sphinxbfcode{\sphinxupquote{Command}}}{\emph{runner}}{}
Bases: \sphinxcode{\sphinxupquote{src.salomeTools.\_BaseCommand}}

The script command executes the script(s) of the the given products in the build directory.
This is done only for the products that are constructed using a script (build\_source : ‘script’).
Otherwise, nothing is done.

\begin{DUlineblock}{0em}
\item[] examples:
\item[]
\begin{DUlineblock}{\DUlineblockindent}
\item[] \textgreater{}\textgreater{} sat script SALOME \textendash{}products Python,numpy
\end{DUlineblock}
\end{DUlineblock}
\index{getParser() (commands.script.Command method)}

\begin{fulllineitems}
\phantomsection\label{\detokenize{apidoc_commands/commands:commands.script.Command.getParser}}\pysiglinewithargsret{\sphinxbfcode{\sphinxupquote{getParser}}}{}{}
Define all options for the command ‘sat script \textless{}options\textgreater{}’

\end{fulllineitems}

\index{name (commands.script.Command attribute)}

\begin{fulllineitems}
\phantomsection\label{\detokenize{apidoc_commands/commands:commands.script.Command.name}}\pysigline{\sphinxbfcode{\sphinxupquote{name}}\sphinxbfcode{\sphinxupquote{ = 'script'}}}
\end{fulllineitems}

\index{run() (commands.script.Command method)}

\begin{fulllineitems}
\phantomsection\label{\detokenize{apidoc_commands/commands:commands.script.Command.run}}\pysiglinewithargsret{\sphinxbfcode{\sphinxupquote{run}}}{\emph{cmd\_arguments}}{}
method called for command ‘sat script \textless{}options\textgreater{}’

\end{fulllineitems}


\end{fulllineitems}

\index{get\_products\_list() (in module commands.script)}

\begin{fulllineitems}
\phantomsection\label{\detokenize{apidoc_commands/commands:commands.script.get_products_list}}\pysiglinewithargsret{\sphinxcode{\sphinxupquote{commands.script.}}\sphinxbfcode{\sphinxupquote{get\_products\_list}}}{\emph{options}, \emph{cfg}, \emph{logger}}{}
Gives the product list with their informations from 
configuration regarding the passed options.
\begin{quote}\begin{description}
\item[{Parameters}] \leavevmode\begin{itemize}
\item {} 
\sphinxstyleliteralstrong{\sphinxupquote{options}} \textendash{} (Options) 
The Options instance that stores the commands arguments

\item {} 
\sphinxstyleliteralstrong{\sphinxupquote{cfg}} \textendash{} (Config) The global configuration

\item {} 
\sphinxstyleliteralstrong{\sphinxupquote{logger}} \textendash{} (Logger) 
The logger instance to use for the display and logging

\end{itemize}

\item[{Returns}] \leavevmode
(list) The list of (product name, product\_informations).

\end{description}\end{quote}

\end{fulllineitems}

\index{run\_script\_all\_products() (in module commands.script)}

\begin{fulllineitems}
\phantomsection\label{\detokenize{apidoc_commands/commands:commands.script.run_script_all_products}}\pysiglinewithargsret{\sphinxcode{\sphinxupquote{commands.script.}}\sphinxbfcode{\sphinxupquote{run\_script\_all\_products}}}{\emph{config}, \emph{products\_infos}, \emph{nb\_proc}, \emph{logger}}{}
Execute the script in each product build directory.
\begin{quote}\begin{description}
\item[{Parameters}] \leavevmode\begin{itemize}
\item {} 
\sphinxstyleliteralstrong{\sphinxupquote{config}} \textendash{} (Config) The global configuration

\item {} 
\sphinxstyleliteralstrong{\sphinxupquote{products\_info}} \textendash{} (list) 
List of (str, Config) =\textgreater{} (product\_name, product\_info)

\item {} 
\sphinxstyleliteralstrong{\sphinxupquote{nb\_proc}} \textendash{} (int) The number of processors to use

\item {} 
\sphinxstyleliteralstrong{\sphinxupquote{logger}} \textendash{} (Logger) 
The logger instance to use for the display and logging

\end{itemize}

\item[{Returns}] \leavevmode
(int) The number of failing commands.

\end{description}\end{quote}

\end{fulllineitems}

\index{run\_script\_of\_product() (in module commands.script)}

\begin{fulllineitems}
\phantomsection\label{\detokenize{apidoc_commands/commands:commands.script.run_script_of_product}}\pysiglinewithargsret{\sphinxcode{\sphinxupquote{commands.script.}}\sphinxbfcode{\sphinxupquote{run\_script\_of\_product}}}{\emph{p\_name\_info}, \emph{nb\_proc}, \emph{config}, \emph{logger}}{}
Execute the proper configuration command(s) 
in the product build directory.
\begin{quote}\begin{description}
\item[{Parameters}] \leavevmode\begin{itemize}
\item {} 
\sphinxstyleliteralstrong{\sphinxupquote{p\_name\_info}} \textendash{} (tuple) 
(str, Config) =\textgreater{} (product\_name, product\_info)

\item {} 
\sphinxstyleliteralstrong{\sphinxupquote{nb\_proc}} \textendash{} (int) The number of processors to use

\item {} 
\sphinxstyleliteralstrong{\sphinxupquote{config}} \textendash{} (Config) The global configuration

\item {} 
\sphinxstyleliteralstrong{\sphinxupquote{logger}} \textendash{} (Logger) 
The logger instance to use for the display and logging

\end{itemize}

\item[{Returns}] \leavevmode
(int) 1 if it fails, else 0.

\end{description}\end{quote}

\end{fulllineitems}



\subsubsection{commands.shell module}
\label{\detokenize{apidoc_commands/commands:module-commands.shell}}\label{\detokenize{apidoc_commands/commands:commands-shell-module}}\index{commands.shell (module)}\index{Command (class in commands.shell)}

\begin{fulllineitems}
\phantomsection\label{\detokenize{apidoc_commands/commands:commands.shell.Command}}\pysiglinewithargsret{\sphinxbfcode{\sphinxupquote{class }}\sphinxcode{\sphinxupquote{commands.shell.}}\sphinxbfcode{\sphinxupquote{Command}}}{\emph{runner}}{}
Bases: \sphinxcode{\sphinxupquote{src.salomeTools.\_BaseCommand}}

The shell command executes the shell command passed as argument.

\begin{DUlineblock}{0em}
\item[] examples:
\item[] \textgreater{}\textgreater{} sat shell \textendash{}command ‘ls -lt /tmp’
\end{DUlineblock}
\index{getParser() (commands.shell.Command method)}

\begin{fulllineitems}
\phantomsection\label{\detokenize{apidoc_commands/commands:commands.shell.Command.getParser}}\pysiglinewithargsret{\sphinxbfcode{\sphinxupquote{getParser}}}{}{}
Define all options for the command ‘sat shell \textless{}options\textgreater{}’

\end{fulllineitems}

\index{name (commands.shell.Command attribute)}

\begin{fulllineitems}
\phantomsection\label{\detokenize{apidoc_commands/commands:commands.shell.Command.name}}\pysigline{\sphinxbfcode{\sphinxupquote{name}}\sphinxbfcode{\sphinxupquote{ = 'shell'}}}
\end{fulllineitems}

\index{run() (commands.shell.Command method)}

\begin{fulllineitems}
\phantomsection\label{\detokenize{apidoc_commands/commands:commands.shell.Command.run}}\pysiglinewithargsret{\sphinxbfcode{\sphinxupquote{run}}}{\emph{cmd\_arguments}}{}
method called for command ‘sat shell \textless{}options\textgreater{}’

\end{fulllineitems}


\end{fulllineitems}



\subsubsection{commands.source module}
\label{\detokenize{apidoc_commands/commands:module-commands.source}}\label{\detokenize{apidoc_commands/commands:commands-source-module}}\index{commands.source (module)}\index{Command (class in commands.source)}

\begin{fulllineitems}
\phantomsection\label{\detokenize{apidoc_commands/commands:commands.source.Command}}\pysiglinewithargsret{\sphinxbfcode{\sphinxupquote{class }}\sphinxcode{\sphinxupquote{commands.source.}}\sphinxbfcode{\sphinxupquote{Command}}}{\emph{runner}}{}
Bases: \sphinxcode{\sphinxupquote{src.salomeTools.\_BaseCommand}}

The source command gets the sources of the application products
from cvs, git or an archive.

\begin{DUlineblock}{0em}
\item[] examples:
\item[] \textgreater{}\textgreater{} sat source SALOME \textendash{}products KERNEL,GUI
\end{DUlineblock}
\index{getParser() (commands.source.Command method)}

\begin{fulllineitems}
\phantomsection\label{\detokenize{apidoc_commands/commands:commands.source.Command.getParser}}\pysiglinewithargsret{\sphinxbfcode{\sphinxupquote{getParser}}}{}{}
Define all options for command ‘sat source \textless{}options\textgreater{}’

\end{fulllineitems}

\index{name (commands.source.Command attribute)}

\begin{fulllineitems}
\phantomsection\label{\detokenize{apidoc_commands/commands:commands.source.Command.name}}\pysigline{\sphinxbfcode{\sphinxupquote{name}}\sphinxbfcode{\sphinxupquote{ = 'source'}}}
\end{fulllineitems}

\index{run() (commands.source.Command method)}

\begin{fulllineitems}
\phantomsection\label{\detokenize{apidoc_commands/commands:commands.source.Command.run}}\pysiglinewithargsret{\sphinxbfcode{\sphinxupquote{run}}}{\emph{cmd\_arguments}}{}
method called for command ‘sat source \textless{}options\textgreater{}’

\end{fulllineitems}


\end{fulllineitems}

\index{check\_sources() (in module commands.source)}

\begin{fulllineitems}
\phantomsection\label{\detokenize{apidoc_commands/commands:commands.source.check_sources}}\pysiglinewithargsret{\sphinxcode{\sphinxupquote{commands.source.}}\sphinxbfcode{\sphinxupquote{check\_sources}}}{\emph{product\_info}, \emph{logger}}{}
Check that the sources are correctly get, 
using the files to be tested in product information
\begin{quote}\begin{description}
\item[{Parameters}] \leavevmode\begin{itemize}
\item {} 
\sphinxstyleliteralstrong{\sphinxupquote{product\_info}} \textendash{} (Config)
The configuration specific to the product to be prepared

\item {} 
\sphinxstyleliteralstrong{\sphinxupquote{logger}} \textendash{} (Logger) 
The logger instance to be used for the logging

\end{itemize}

\item[{Returns}] \leavevmode
(bool) 
True if the files exists (or no files to test is provided).

\end{description}\end{quote}

\end{fulllineitems}

\index{get\_all\_product\_sources() (in module commands.source)}

\begin{fulllineitems}
\phantomsection\label{\detokenize{apidoc_commands/commands:commands.source.get_all_product_sources}}\pysiglinewithargsret{\sphinxcode{\sphinxupquote{commands.source.}}\sphinxbfcode{\sphinxupquote{get\_all\_product\_sources}}}{\emph{config}, \emph{products}, \emph{logger}}{}
Get all the product sources.
\begin{quote}\begin{description}
\item[{Parameters}] \leavevmode\begin{itemize}
\item {} 
\sphinxstyleliteralstrong{\sphinxupquote{config}} \textendash{} (Config) The global configuration

\item {} 
\sphinxstyleliteralstrong{\sphinxupquote{products}} \textendash{} (list) 
The list of tuples (product name, product informations)

\item {} 
\sphinxstyleliteralstrong{\sphinxupquote{logger}} \textendash{} (Logger) 
The logger instance to be used for the logging

\end{itemize}

\item[{Returns}] \leavevmode
(int,dict) 
The tuple (number of success, dictionary product\_name/success\_fail)

\end{description}\end{quote}

\end{fulllineitems}

\index{get\_product\_sources() (in module commands.source)}

\begin{fulllineitems}
\phantomsection\label{\detokenize{apidoc_commands/commands:commands.source.get_product_sources}}\pysiglinewithargsret{\sphinxcode{\sphinxupquote{commands.source.}}\sphinxbfcode{\sphinxupquote{get\_product\_sources}}}{\emph{config}, \emph{product\_info}, \emph{is\_dev}, \emph{source\_dir}, \emph{logger}, \emph{pad}, \emph{checkout=False}}{}
Get the product sources.
\begin{quote}\begin{description}
\item[{Parameters}] \leavevmode\begin{itemize}
\item {} 
\sphinxstyleliteralstrong{\sphinxupquote{config}} \textendash{} (Config) The global configuration

\item {} 
\sphinxstyleliteralstrong{\sphinxupquote{product\_info}} \textendash{} (Config) 
The configuration specific to the product to be prepared

\item {} 
\sphinxstyleliteralstrong{\sphinxupquote{is\_dev}} \textendash{} (bool) True if the product is in development mode

\item {} 
\sphinxstyleliteralstrong{\sphinxupquote{source\_dir}} \textendash{} (Path) 
The Path instance corresponding to the directory 
where to put the sources

\item {} 
\sphinxstyleliteralstrong{\sphinxupquote{logger}} \textendash{} (Logger) 
The logger instance to use for the display and logging

\item {} 
\sphinxstyleliteralstrong{\sphinxupquote{pad}} \textendash{} (int) The gap to apply for the terminal display

\item {} 
\sphinxstyleliteralstrong{\sphinxupquote{checkout}} \textendash{} (bool) If True, get the source in checkout mode

\end{itemize}

\item[{Returns}] \leavevmode
(bool) True if it succeed, else False

\end{description}\end{quote}

\end{fulllineitems}

\index{get\_source\_for\_dev() (in module commands.source)}

\begin{fulllineitems}
\phantomsection\label{\detokenize{apidoc_commands/commands:commands.source.get_source_for_dev}}\pysiglinewithargsret{\sphinxcode{\sphinxupquote{commands.source.}}\sphinxbfcode{\sphinxupquote{get\_source\_for\_dev}}}{\emph{config}, \emph{product\_info}, \emph{source\_dir}, \emph{logger}, \emph{pad}}{}
Called if the product is in development mode
\begin{quote}\begin{description}
\item[{Parameters}] \leavevmode\begin{itemize}
\item {} 
\sphinxstyleliteralstrong{\sphinxupquote{config}} \textendash{} (Config) The global configuration

\item {} 
\sphinxstyleliteralstrong{\sphinxupquote{product\_info}} \textendash{} (Config) 
The configuration specific to the product to be prepared

\item {} 
\sphinxstyleliteralstrong{\sphinxupquote{source\_dir}} \textendash{} (Path)
The Path instance corresponding to the directory where to put the sources

\item {} 
\sphinxstyleliteralstrong{\sphinxupquote{logger}} \textendash{} (Logger)
The logger instance to use for the display and logging

\item {} 
\sphinxstyleliteralstrong{\sphinxupquote{pad}} \textendash{} (int) The gap to apply for the terminal display

\end{itemize}

\item[{Returns}] \leavevmode
(bool) True if it succeed, else False

\end{description}\end{quote}

\end{fulllineitems}

\index{get\_source\_from\_archive() (in module commands.source)}

\begin{fulllineitems}
\phantomsection\label{\detokenize{apidoc_commands/commands:commands.source.get_source_from_archive}}\pysiglinewithargsret{\sphinxcode{\sphinxupquote{commands.source.}}\sphinxbfcode{\sphinxupquote{get\_source\_from\_archive}}}{\emph{product\_info}, \emph{source\_dir}, \emph{logger}}{}
The method called if the product is to be get in archive mode
\begin{quote}\begin{description}
\item[{Parameters}] \leavevmode\begin{itemize}
\item {} 
\sphinxstyleliteralstrong{\sphinxupquote{product\_info}} \textendash{} (Config)
The configuration specific to 
the product to be prepared

\item {} 
\sphinxstyleliteralstrong{\sphinxupquote{source\_dir}} \textendash{} (Path)
The Path instance corresponding to the directory
where to put the sources

\item {} 
\sphinxstyleliteralstrong{\sphinxupquote{logger}} \textendash{} (Logger) 
The logger instance to use for the display and logging

\end{itemize}

\item[{Returns}] \leavevmode
(bool) True if it succeed, else False

\end{description}\end{quote}

\end{fulllineitems}

\index{get\_source\_from\_cvs() (in module commands.source)}

\begin{fulllineitems}
\phantomsection\label{\detokenize{apidoc_commands/commands:commands.source.get_source_from_cvs}}\pysiglinewithargsret{\sphinxcode{\sphinxupquote{commands.source.}}\sphinxbfcode{\sphinxupquote{get\_source\_from\_cvs}}}{\emph{user}, \emph{product\_info}, \emph{source\_dir}, \emph{checkout}, \emph{logger}, \emph{pad}, \emph{environ=None}}{}
The method called if the product is to be get in cvs mode
\begin{quote}\begin{description}
\item[{Parameters}] \leavevmode\begin{itemize}
\item {} 
\sphinxstyleliteralstrong{\sphinxupquote{user}} \textendash{} (str) The user to use in for the cvs command

\item {} 
\sphinxstyleliteralstrong{\sphinxupquote{product\_info}} \textendash{} (Config) 
The configuration specific to the product to be prepared

\item {} 
\sphinxstyleliteralstrong{\sphinxupquote{source\_dir}} \textendash{} (Path) 
The Path instance corresponding to the directory 
where to put the sources

\item {} 
\sphinxstyleliteralstrong{\sphinxupquote{checkout}} \textendash{} (bool) If True, get the source in checkout mode

\item {} 
\sphinxstyleliteralstrong{\sphinxupquote{logger}} \textendash{} (Logger) 
The logger instance to use for the display and logging

\item {} 
\sphinxstyleliteralstrong{\sphinxupquote{pad}} \textendash{} (int) The gap to apply for the terminal display

\item {} 
\sphinxstyleliteralstrong{\sphinxupquote{environ}} \textendash{} (src.environment.Environ) 
The environment to source when extracting.

\end{itemize}

\item[{Returns}] \leavevmode
(bool) True if it succeed, else False

\end{description}\end{quote}

\end{fulllineitems}

\index{get\_source\_from\_dir() (in module commands.source)}

\begin{fulllineitems}
\phantomsection\label{\detokenize{apidoc_commands/commands:commands.source.get_source_from_dir}}\pysiglinewithargsret{\sphinxcode{\sphinxupquote{commands.source.}}\sphinxbfcode{\sphinxupquote{get\_source\_from\_dir}}}{\emph{product\_info}, \emph{source\_dir}, \emph{logger}}{}
\end{fulllineitems}

\index{get\_source\_from\_git() (in module commands.source)}

\begin{fulllineitems}
\phantomsection\label{\detokenize{apidoc_commands/commands:commands.source.get_source_from_git}}\pysiglinewithargsret{\sphinxcode{\sphinxupquote{commands.source.}}\sphinxbfcode{\sphinxupquote{get\_source\_from\_git}}}{\emph{product\_info}, \emph{source\_dir}, \emph{logger}, \emph{pad}, \emph{is\_dev=False}, \emph{environ=None}}{}
Called if the product is to be get in git mode
\begin{quote}\begin{description}
\item[{Parameters}] \leavevmode\begin{itemize}
\item {} 
\sphinxstyleliteralstrong{\sphinxupquote{product\_info}} \textendash{} (Config) 
The configuration specific to the product to be prepared

\item {} 
\sphinxstyleliteralstrong{\sphinxupquote{source\_dir}} \textendash{} (Path)
The Path instance corresponding to the
directory where to put the sources

\item {} 
\sphinxstyleliteralstrong{\sphinxupquote{Logger}} (\sphinxstyleliteralemphasis{\sphinxupquote{logger}}) \textendash{} (Logger) 
The logger instance to use for the display and logging

\item {} 
\sphinxstyleliteralstrong{\sphinxupquote{pad}} \textendash{} (int) The gap to apply for the terminal display

\item {} 
\sphinxstyleliteralstrong{\sphinxupquote{is\_dev}} \textendash{} (bool) True if the product is in development mode

\item {} 
\sphinxstyleliteralstrong{\sphinxupquote{environ}} \textendash{} (src.environment.Environ)
The environment to source when extracting.

\end{itemize}

\item[{Returns}] \leavevmode
(bool) True if it succeed, else False

\end{description}\end{quote}

\end{fulllineitems}

\index{get\_source\_from\_svn() (in module commands.source)}

\begin{fulllineitems}
\phantomsection\label{\detokenize{apidoc_commands/commands:commands.source.get_source_from_svn}}\pysiglinewithargsret{\sphinxcode{\sphinxupquote{commands.source.}}\sphinxbfcode{\sphinxupquote{get\_source\_from\_svn}}}{\emph{user}, \emph{product\_info}, \emph{source\_dir}, \emph{checkout}, \emph{logger}, \emph{environ=None}}{}
The method called if the product is to be get in svn mode
\begin{quote}\begin{description}
\item[{Parameters}] \leavevmode\begin{itemize}
\item {} 
\sphinxstyleliteralstrong{\sphinxupquote{user}} \textendash{} (str) The user to use in for the svn command

\item {} 
\sphinxstyleliteralstrong{\sphinxupquote{product\_info}} \textendash{} (Config)
The configuration specific to the product to be prepared

\item {} 
\sphinxstyleliteralstrong{\sphinxupquote{source\_dir}} \textendash{} (Path)
The Path instance corresponding to the directory 
where to put the sources

\item {} 
\sphinxstyleliteralstrong{\sphinxupquote{checkout}} \textendash{} (boolean) 
If True, get the source in checkout mode

\item {} 
\sphinxstyleliteralstrong{\sphinxupquote{logger}} \textendash{} (Logger)
The logger instance to use for the display and logging

\item {} 
\sphinxstyleliteralstrong{\sphinxupquote{environ}} \textendash{} (src.environment.Environ)
The environment to source when extracting.

\end{itemize}

\item[{Returns}] \leavevmode
(bool) True if it succeed, else False

\end{description}\end{quote}

\end{fulllineitems}



\subsubsection{commands.template module}
\label{\detokenize{apidoc_commands/commands:module-commands.template}}\label{\detokenize{apidoc_commands/commands:commands-template-module}}\index{commands.template (module)}\index{Command (class in commands.template)}

\begin{fulllineitems}
\phantomsection\label{\detokenize{apidoc_commands/commands:commands.template.Command}}\pysiglinewithargsret{\sphinxbfcode{\sphinxupquote{class }}\sphinxcode{\sphinxupquote{commands.template.}}\sphinxbfcode{\sphinxupquote{Command}}}{\emph{runner}}{}
Bases: \sphinxcode{\sphinxupquote{src.salomeTools.\_BaseCommand}}

The template command creates the sources for a SALOME module from a template.

\begin{DUlineblock}{0em}
\item[] examples:
\item[] \textgreater{}\textgreater{} sat template \textendash{}name my\_product\_name \textendash{}template PythonComponent \textendash{}target /tmp
\end{DUlineblock}
\index{getParser() (commands.template.Command method)}

\begin{fulllineitems}
\phantomsection\label{\detokenize{apidoc_commands/commands:commands.template.Command.getParser}}\pysiglinewithargsret{\sphinxbfcode{\sphinxupquote{getParser}}}{}{}
Define all options for command ‘sat template \textless{}options\textgreater{}’

\end{fulllineitems}

\index{name (commands.template.Command attribute)}

\begin{fulllineitems}
\phantomsection\label{\detokenize{apidoc_commands/commands:commands.template.Command.name}}\pysigline{\sphinxbfcode{\sphinxupquote{name}}\sphinxbfcode{\sphinxupquote{ = 'template'}}}
\end{fulllineitems}

\index{run() (commands.template.Command method)}

\begin{fulllineitems}
\phantomsection\label{\detokenize{apidoc_commands/commands:commands.template.Command.run}}\pysiglinewithargsret{\sphinxbfcode{\sphinxupquote{run}}}{\emph{cmd\_arguments}}{}
method called for command ‘sat template \textless{}options\textgreater{}’

\end{fulllineitems}


\end{fulllineitems}

\index{TParam (class in commands.template)}

\begin{fulllineitems}
\phantomsection\label{\detokenize{apidoc_commands/commands:commands.template.TParam}}\pysiglinewithargsret{\sphinxbfcode{\sphinxupquote{class }}\sphinxcode{\sphinxupquote{commands.template.}}\sphinxbfcode{\sphinxupquote{TParam}}}{\emph{param\_def}, \emph{compo\_name}, \emph{dico=None}}{}~\index{check\_value() (commands.template.TParam method)}

\begin{fulllineitems}
\phantomsection\label{\detokenize{apidoc_commands/commands:commands.template.TParam.check_value}}\pysiglinewithargsret{\sphinxbfcode{\sphinxupquote{check\_value}}}{\emph{val}}{}
\end{fulllineitems}


\end{fulllineitems}

\index{TemplateSettings (class in commands.template)}

\begin{fulllineitems}
\phantomsection\label{\detokenize{apidoc_commands/commands:commands.template.TemplateSettings}}\pysiglinewithargsret{\sphinxbfcode{\sphinxupquote{class }}\sphinxcode{\sphinxupquote{commands.template.}}\sphinxbfcode{\sphinxupquote{TemplateSettings}}}{\emph{compo\_name}, \emph{settings\_file}, \emph{target}}{}~\index{check\_file\_for\_substitution() (commands.template.TemplateSettings method)}

\begin{fulllineitems}
\phantomsection\label{\detokenize{apidoc_commands/commands:commands.template.TemplateSettings.check_file_for_substitution}}\pysiglinewithargsret{\sphinxbfcode{\sphinxupquote{check\_file\_for\_substitution}}}{\emph{file\_}}{}
\end{fulllineitems}

\index{check\_user\_values() (commands.template.TemplateSettings method)}

\begin{fulllineitems}
\phantomsection\label{\detokenize{apidoc_commands/commands:commands.template.TemplateSettings.check_user_values}}\pysiglinewithargsret{\sphinxbfcode{\sphinxupquote{check\_user\_values}}}{\emph{values}}{}
\end{fulllineitems}

\index{get\_parameters() (commands.template.TemplateSettings method)}

\begin{fulllineitems}
\phantomsection\label{\detokenize{apidoc_commands/commands:commands.template.TemplateSettings.get_parameters}}\pysiglinewithargsret{\sphinxbfcode{\sphinxupquote{get\_parameters}}}{\emph{conf\_values=None}}{}
\end{fulllineitems}

\index{get\_pyconf\_parameters() (commands.template.TemplateSettings method)}

\begin{fulllineitems}
\phantomsection\label{\detokenize{apidoc_commands/commands:commands.template.TemplateSettings.get_pyconf_parameters}}\pysiglinewithargsret{\sphinxbfcode{\sphinxupquote{get\_pyconf\_parameters}}}{}{}
\end{fulllineitems}

\index{has\_pyconf() (commands.template.TemplateSettings method)}

\begin{fulllineitems}
\phantomsection\label{\detokenize{apidoc_commands/commands:commands.template.TemplateSettings.has_pyconf}}\pysiglinewithargsret{\sphinxbfcode{\sphinxupquote{has\_pyconf}}}{}{}
\end{fulllineitems}


\end{fulllineitems}

\index{get\_dico\_param() (in module commands.template)}

\begin{fulllineitems}
\phantomsection\label{\detokenize{apidoc_commands/commands:commands.template.get_dico_param}}\pysiglinewithargsret{\sphinxcode{\sphinxupquote{commands.template.}}\sphinxbfcode{\sphinxupquote{get\_dico\_param}}}{\emph{dico}, \emph{key}, \emph{default}}{}
\end{fulllineitems}

\index{get\_template\_info() (in module commands.template)}

\begin{fulllineitems}
\phantomsection\label{\detokenize{apidoc_commands/commands:commands.template.get_template_info}}\pysiglinewithargsret{\sphinxcode{\sphinxupquote{commands.template.}}\sphinxbfcode{\sphinxupquote{get\_template\_info}}}{\emph{config}, \emph{template\_name}, \emph{logger}}{}
\end{fulllineitems}

\index{prepare\_from\_template() (in module commands.template)}

\begin{fulllineitems}
\phantomsection\label{\detokenize{apidoc_commands/commands:commands.template.prepare_from_template}}\pysiglinewithargsret{\sphinxcode{\sphinxupquote{commands.template.}}\sphinxbfcode{\sphinxupquote{prepare\_from\_template}}}{\emph{config}, \emph{name}, \emph{template}, \emph{target\_dir}, \emph{conf\_values}, \emph{logger}}{}
Prepares a module from a template.

\end{fulllineitems}

\index{search\_template() (in module commands.template)}

\begin{fulllineitems}
\phantomsection\label{\detokenize{apidoc_commands/commands:commands.template.search_template}}\pysiglinewithargsret{\sphinxcode{\sphinxupquote{commands.template.}}\sphinxbfcode{\sphinxupquote{search\_template}}}{\emph{config}, \emph{template}}{}
\end{fulllineitems}



\subsubsection{commands.test module}
\label{\detokenize{apidoc_commands/commands:commands-test-module}}\label{\detokenize{apidoc_commands/commands:module-commands.test}}\index{commands.test (module)}\index{Command (class in commands.test)}

\begin{fulllineitems}
\phantomsection\label{\detokenize{apidoc_commands/commands:commands.test.Command}}\pysiglinewithargsret{\sphinxbfcode{\sphinxupquote{class }}\sphinxcode{\sphinxupquote{commands.test.}}\sphinxbfcode{\sphinxupquote{Command}}}{\emph{runner}}{}
Bases: \sphinxcode{\sphinxupquote{src.salomeTools.\_BaseCommand}}

The test command runs a test base on a SALOME installation.

\begin{DUlineblock}{0em}
\item[] examples:
\item[] \textgreater{}\textgreater{} sat test SALOME \textendash{}grid GEOM \textendash{}session light
\end{DUlineblock}
\index{check\_option() (commands.test.Command method)}

\begin{fulllineitems}
\phantomsection\label{\detokenize{apidoc_commands/commands:commands.test.Command.check_option}}\pysiglinewithargsret{\sphinxbfcode{\sphinxupquote{check\_option}}}{\emph{options}}{}
Check the options
\begin{quote}\begin{description}
\item[{Parameters}] \leavevmode
\sphinxstyleliteralstrong{\sphinxupquote{options}} \textendash{} (Options) The options

\item[{Returns}] \leavevmode
None

\end{description}\end{quote}

\end{fulllineitems}

\index{getParser() (commands.test.Command method)}

\begin{fulllineitems}
\phantomsection\label{\detokenize{apidoc_commands/commands:commands.test.Command.getParser}}\pysiglinewithargsret{\sphinxbfcode{\sphinxupquote{getParser}}}{}{}
Define all options for command ‘sat test \textless{}options\textgreater{}’

\end{fulllineitems}

\index{name (commands.test.Command attribute)}

\begin{fulllineitems}
\phantomsection\label{\detokenize{apidoc_commands/commands:commands.test.Command.name}}\pysigline{\sphinxbfcode{\sphinxupquote{name}}\sphinxbfcode{\sphinxupquote{ = 'test'}}}
\end{fulllineitems}

\index{run() (commands.test.Command method)}

\begin{fulllineitems}
\phantomsection\label{\detokenize{apidoc_commands/commands:commands.test.Command.run}}\pysiglinewithargsret{\sphinxbfcode{\sphinxupquote{run}}}{\emph{cmd\_arguments}}{}
method called for command ‘sat test \textless{}options\textgreater{}’

\end{fulllineitems}


\end{fulllineitems}

\index{ask\_a\_path() (in module commands.test)}

\begin{fulllineitems}
\phantomsection\label{\detokenize{apidoc_commands/commands:commands.test.ask_a_path}}\pysiglinewithargsret{\sphinxcode{\sphinxupquote{commands.test.}}\sphinxbfcode{\sphinxupquote{ask\_a\_path}}}{}{}
interactive as using ‘raw\_input’

\end{fulllineitems}

\index{check\_remote\_machine() (in module commands.test)}

\begin{fulllineitems}
\phantomsection\label{\detokenize{apidoc_commands/commands:commands.test.check_remote_machine}}\pysiglinewithargsret{\sphinxcode{\sphinxupquote{commands.test.}}\sphinxbfcode{\sphinxupquote{check\_remote\_machine}}}{\emph{machine\_name}, \emph{logger}}{}
\end{fulllineitems}

\index{create\_test\_report() (in module commands.test)}

\begin{fulllineitems}
\phantomsection\label{\detokenize{apidoc_commands/commands:commands.test.create_test_report}}\pysiglinewithargsret{\sphinxcode{\sphinxupquote{commands.test.}}\sphinxbfcode{\sphinxupquote{create\_test\_report}}}{\emph{config}, \emph{xml\_history\_path}, \emph{dest\_path}, \emph{retcode}, \emph{xmlname=''}}{}
Creates the XML report for a product.

\end{fulllineitems}

\index{generate\_history\_xml\_path() (in module commands.test)}

\begin{fulllineitems}
\phantomsection\label{\detokenize{apidoc_commands/commands:commands.test.generate_history_xml_path}}\pysiglinewithargsret{\sphinxcode{\sphinxupquote{commands.test.}}\sphinxbfcode{\sphinxupquote{generate\_history\_xml\_path}}}{\emph{config}, \emph{test\_base}}{}
Generate the name of the xml file that contain the history of the tests
on the machine with the current APPLICATION and the current test base.
\begin{quote}\begin{description}
\item[{Parameters}] \leavevmode\begin{itemize}
\item {} 
\sphinxstyleliteralstrong{\sphinxupquote{config}} \textendash{} (Config) The global configuration

\item {} 
\sphinxstyleliteralstrong{\sphinxupquote{test\_base}} \textendash{} (str) The test base name (or path)

\end{itemize}

\item[{Returns}] \leavevmode
(str) the full path of the history xml file

\end{description}\end{quote}

\end{fulllineitems}

\index{move\_test\_results() (in module commands.test)}

\begin{fulllineitems}
\phantomsection\label{\detokenize{apidoc_commands/commands:commands.test.move_test_results}}\pysiglinewithargsret{\sphinxcode{\sphinxupquote{commands.test.}}\sphinxbfcode{\sphinxupquote{move\_test\_results}}}{\emph{in\_dir}, \emph{what}, \emph{out\_dir}, \emph{logger}}{}
\end{fulllineitems}

\index{save\_file() (in module commands.test)}

\begin{fulllineitems}
\phantomsection\label{\detokenize{apidoc_commands/commands:commands.test.save_file}}\pysiglinewithargsret{\sphinxcode{\sphinxupquote{commands.test.}}\sphinxbfcode{\sphinxupquote{save\_file}}}{\emph{filename}, \emph{base}}{}
\end{fulllineitems}



\subsubsection{Module contents}
\label{\detokenize{apidoc_commands/commands:module-commands}}\label{\detokenize{apidoc_commands/commands:module-contents}}\index{commands (module)}

\chapter{Release Notes}
\label{\detokenize{index:release-notes}}

\section{Release notes}
\label{\detokenize{release_notes/release_notes_5.0.0:release-notes}}\label{\detokenize{release_notes/release_notes_5.0.0::doc}}
In construction.


\renewcommand{\indexname}{Python Module Index}
\begin{sphinxtheindex}
\def\bigletter#1{{\Large\sffamily#1}\nopagebreak\vspace{1mm}}
\bigletter{c}
\item {\sphinxstyleindexentry{commands}}\sphinxstyleindexpageref{apidoc_commands/commands:\detokenize{module-commands}}
\item {\sphinxstyleindexentry{commands.application}}\sphinxstyleindexpageref{apidoc_commands/commands:\detokenize{module-commands.application}}
\item {\sphinxstyleindexentry{commands.check}}\sphinxstyleindexpageref{apidoc_commands/commands:\detokenize{module-commands.check}}
\item {\sphinxstyleindexentry{commands.clean}}\sphinxstyleindexpageref{apidoc_commands/commands:\detokenize{module-commands.clean}}
\item {\sphinxstyleindexentry{commands.compile}}\sphinxstyleindexpageref{apidoc_commands/commands:\detokenize{module-commands.compile}}
\item {\sphinxstyleindexentry{commands.config}}\sphinxstyleindexpageref{apidoc_commands/commands:\detokenize{module-commands.config}}
\item {\sphinxstyleindexentry{commands.configure}}\sphinxstyleindexpageref{apidoc_commands/commands:\detokenize{module-commands.configure}}
\item {\sphinxstyleindexentry{commands.environ}}\sphinxstyleindexpageref{apidoc_commands/commands:\detokenize{module-commands.environ}}
\item {\sphinxstyleindexentry{commands.find\_duplicates}}\sphinxstyleindexpageref{apidoc_commands/commands:\detokenize{module-commands.find_duplicates}}
\item {\sphinxstyleindexentry{commands.generate}}\sphinxstyleindexpageref{apidoc_commands/commands:\detokenize{module-commands.generate}}
\item {\sphinxstyleindexentry{commands.init}}\sphinxstyleindexpageref{apidoc_commands/commands:\detokenize{module-commands.init}}
\item {\sphinxstyleindexentry{commands.job}}\sphinxstyleindexpageref{apidoc_commands/commands:\detokenize{module-commands.job}}
\item {\sphinxstyleindexentry{commands.jobs}}\sphinxstyleindexpageref{apidoc_commands/commands:\detokenize{module-commands.jobs}}
\item {\sphinxstyleindexentry{commands.launcher}}\sphinxstyleindexpageref{apidoc_commands/commands:\detokenize{module-commands.launcher}}
\item {\sphinxstyleindexentry{commands.log}}\sphinxstyleindexpageref{apidoc_commands/commands:\detokenize{module-commands.log}}
\item {\sphinxstyleindexentry{commands.make}}\sphinxstyleindexpageref{apidoc_commands/commands:\detokenize{module-commands.make}}
\item {\sphinxstyleindexentry{commands.makeinstall}}\sphinxstyleindexpageref{apidoc_commands/commands:\detokenize{module-commands.makeinstall}}
\item {\sphinxstyleindexentry{commands.package}}\sphinxstyleindexpageref{apidoc_commands/commands:\detokenize{module-commands.package}}
\item {\sphinxstyleindexentry{commands.patch}}\sphinxstyleindexpageref{apidoc_commands/commands:\detokenize{module-commands.patch}}
\item {\sphinxstyleindexentry{commands.prepare}}\sphinxstyleindexpageref{apidoc_commands/commands:\detokenize{module-commands.prepare}}
\item {\sphinxstyleindexentry{commands.profile}}\sphinxstyleindexpageref{apidoc_commands/commands:\detokenize{module-commands.profile}}
\item {\sphinxstyleindexentry{commands.run}}\sphinxstyleindexpageref{apidoc_commands/commands:\detokenize{module-commands.run}}
\item {\sphinxstyleindexentry{commands.script}}\sphinxstyleindexpageref{apidoc_commands/commands:\detokenize{module-commands.script}}
\item {\sphinxstyleindexentry{commands.shell}}\sphinxstyleindexpageref{apidoc_commands/commands:\detokenize{module-commands.shell}}
\item {\sphinxstyleindexentry{commands.source}}\sphinxstyleindexpageref{apidoc_commands/commands:\detokenize{module-commands.source}}
\item {\sphinxstyleindexentry{commands.template}}\sphinxstyleindexpageref{apidoc_commands/commands:\detokenize{module-commands.template}}
\item {\sphinxstyleindexentry{commands.test}}\sphinxstyleindexpageref{apidoc_commands/commands:\detokenize{module-commands.test}}
\indexspace
\bigletter{s}
\item {\sphinxstyleindexentry{src}}\sphinxstyleindexpageref{apidoc_src/src:\detokenize{module-src}}
\item {\sphinxstyleindexentry{src.architecture}}\sphinxstyleindexpageref{apidoc_src/src:\detokenize{module-src.architecture}}
\item {\sphinxstyleindexentry{src.catchAll}}\sphinxstyleindexpageref{apidoc_src/src:\detokenize{module-src.catchAll}}
\item {\sphinxstyleindexentry{src.colorama}}\sphinxstyleindexpageref{apidoc_src/src.colorama:\detokenize{module-src.colorama}}
\item {\sphinxstyleindexentry{src.colorama.ansi}}\sphinxstyleindexpageref{apidoc_src/src.colorama:\detokenize{module-src.colorama.ansi}}
\item {\sphinxstyleindexentry{src.colorama.ansitowin32}}\sphinxstyleindexpageref{apidoc_src/src.colorama:\detokenize{module-src.colorama.ansitowin32}}
\item {\sphinxstyleindexentry{src.colorama.initialise}}\sphinxstyleindexpageref{apidoc_src/src.colorama:\detokenize{module-src.colorama.initialise}}
\item {\sphinxstyleindexentry{src.colorama.win32}}\sphinxstyleindexpageref{apidoc_src/src.colorama:\detokenize{module-src.colorama.win32}}
\item {\sphinxstyleindexentry{src.colorama.winterm}}\sphinxstyleindexpageref{apidoc_src/src.colorama:\detokenize{module-src.colorama.winterm}}
\item {\sphinxstyleindexentry{src.coloringSat}}\sphinxstyleindexpageref{apidoc_src/src:\detokenize{module-src.coloringSat}}
\item {\sphinxstyleindexentry{src.compilation}}\sphinxstyleindexpageref{apidoc_src/src:\detokenize{module-src.compilation}}
\item {\sphinxstyleindexentry{src.configManager}}\sphinxstyleindexpageref{apidoc_src/src:\detokenize{module-src.configManager}}
\item {\sphinxstyleindexentry{src.debug}}\sphinxstyleindexpageref{apidoc_src/src:\detokenize{module-src.debug}}
\item {\sphinxstyleindexentry{src.ElementTree}}\sphinxstyleindexpageref{apidoc_src/src:\detokenize{module-src.ElementTree}}
\item {\sphinxstyleindexentry{src.environment}}\sphinxstyleindexpageref{apidoc_src/src:\detokenize{module-src.environment}}
\item {\sphinxstyleindexentry{src.environs}}\sphinxstyleindexpageref{apidoc_src/src:\detokenize{module-src.environs}}
\item {\sphinxstyleindexentry{src.example}}\sphinxstyleindexpageref{apidoc_src/src.example:\detokenize{module-src.example}}
\item {\sphinxstyleindexentry{src.example.essai\_logging\_1}}\sphinxstyleindexpageref{apidoc_src/src.example:\detokenize{module-src.example.essai_logging_1}}
\item {\sphinxstyleindexentry{src.example.essai\_logging\_2}}\sphinxstyleindexpageref{apidoc_src/src.example:\detokenize{module-src.example.essai_logging_2}}
\item {\sphinxstyleindexentry{src.exceptionSat}}\sphinxstyleindexpageref{apidoc_src/src:\detokenize{module-src.exceptionSat}}
\item {\sphinxstyleindexentry{src.fileEnviron}}\sphinxstyleindexpageref{apidoc_src/src:\detokenize{module-src.fileEnviron}}
\item {\sphinxstyleindexentry{src.fork}}\sphinxstyleindexpageref{apidoc_src/src:\detokenize{module-src.fork}}
\item {\sphinxstyleindexentry{src.loggingSat}}\sphinxstyleindexpageref{apidoc_src/src:\detokenize{module-src.loggingSat}}
\item {\sphinxstyleindexentry{src.options}}\sphinxstyleindexpageref{apidoc_src/src:\detokenize{module-src.options}}
\item {\sphinxstyleindexentry{src.product}}\sphinxstyleindexpageref{apidoc_src/src:\detokenize{module-src.product}}
\item {\sphinxstyleindexentry{src.pyconf}}\sphinxstyleindexpageref{apidoc_src/src:\detokenize{module-src.pyconf}}
\item {\sphinxstyleindexentry{src.returnCode}}\sphinxstyleindexpageref{apidoc_src/src:\detokenize{module-src.returnCode}}
\item {\sphinxstyleindexentry{src.salomeTools}}\sphinxstyleindexpageref{apidoc_src/src:\detokenize{module-src.salomeTools}}
\item {\sphinxstyleindexentry{src.system}}\sphinxstyleindexpageref{apidoc_src/src:\detokenize{module-src.system}}
\item {\sphinxstyleindexentry{src.template}}\sphinxstyleindexpageref{apidoc_src/src:\detokenize{module-src.template}}
\item {\sphinxstyleindexentry{src.test\_module}}\sphinxstyleindexpageref{apidoc_src/src:\detokenize{module-src.test_module}}
\item {\sphinxstyleindexentry{src.utilsSat}}\sphinxstyleindexpageref{apidoc_src/src:\detokenize{module-src.utilsSat}}
\item {\sphinxstyleindexentry{src.xmlManager}}\sphinxstyleindexpageref{apidoc_src/src:\detokenize{module-src.xmlManager}}
\end{sphinxtheindex}

\renewcommand{\indexname}{Index}
\printindex
\end{document}